% % vim: set textwidth=75 spell :
% Warning: don't use acronyms within definitions, they don't work properly.

\newacronym{RBL}{Real-time Black List}{description={
    Real-time Black Lists are a simple collaborative anti-spam technique
    used to reject or penalise email sent from mail servers reported to
    have sent large volumes of spam.  To use an RBL the mail server makes a
    DNS request incorporating the IP address of the currently connected
    client; if the requested hostname exists the client is on the RBL, and
    the mail server can decide what course of action to take.
}}

\newacronym{API}{Application Programming Interface}{description={
    One of the fundamental concepts when writing programs is the reuse of
    existing code, so that each program does not reinvent the wheel.  When
    a body of code is intended to be reused repeatedly, the user of this
    code needs to be informed of the functionality provided by the code.
    An API defines the interface provided to the user, and acts a contract
    between the user and the provider: if the user adheres to the API the
    provider guarantees it will work, while the provider is free to change
    the implementation as long as the API is preserved.
}}

\newacronym{SMTP}{Simple Mail Transfer Protocol}{description={
    SMTP is the protocol which transfers mail from the sender to the
    recipient across the Internet.  A brief introduction to SMTP is
    provided in section~\refwithpage{SMTP background}.
}}

\newacronym{MTA}{Mail Transfer Agent}{description={
    A Mail Transfer Agent sends and/or receives mail via SMTP\@.
}}

\newacronym{RFC}{Request For Comments}{description={
    The Request For Comments series is a series of proposals defining
    various protocols and file formats, e.g. SMTP\@.  The name is somewhat
    misleading: initially the authors were asking for peer review, but
    these documents are now the de facto standards the Internet runs on.
}}
