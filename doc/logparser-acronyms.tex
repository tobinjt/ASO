% % vim: set textwidth=75 spell :
% Warning: don't use acronyms within definitions, they don't work properly.

\newacronym{PLP}{Postfix Log Parser}{description={
    Postfix Log Parser is the implementation of the algorithm described in
    this paper.
}}

\newacronym{RBL}{Real-time Black List}{description={
    Real-time Black Lists are a simple collaborative anti-spam technique
    used to reject or penalise email sent from mail servers reported to
    have sent large volumes of spam.  To use an RBL the mail server makes a
    DNS request incorporating the IP address of the currently connected
    client; if the requested hostname exists the client is on the RBL, and
    the mail server can decide what course of action to take.
}}

\newacronym{API}{Application Programming Interface}{description={
    One of the fundamental concepts when writing programs is the reuse of
    existing code, so that each program does not reinvent the wheel.  When
    a body of code is intended to be reused repeatedly, the user of this
    code needs to be informed of the functionality provided by the code.
    An API defines the interface provided to the user, and acts as a
    contract between the user and the provider: if the user adheres to the
    API the provider guarantees it will work, while the provider is free to
    change the implementation as long as the API is preserved.
}}

\newacronym{SMTP}{Simple Mail Transfer Protocol}{description={
    SMTP is the protocol which transfers mail from the sender to the
    recipient across the Internet.  A brief introduction to SMTP is
    provided in section~\refwithpage{SMTP background}.
}}

\newacronym{MTA}{Mail Transfer Agent}{description={
    A Mail Transfer Agent sends and/or receives mail via SMTP.
}}

\newacronym{RFC}{Request For Comments}{description={
    The Request For Comments series is a series of proposals defining
    various protocols and file formats, e.g. SMTP.  The name is somewhat
    misleading: initially the authors were asking for peer review, but
    these documents are now the de facto standards the Internet runs on.
}}

\newacronym{SQL}{Structured Query Language}{description={
    SQL is the standard language used for database querying, modification
    and maintenance.  Some information about SQL, including its history,
    can be found at \cite{Wikipedia-sql}.
}}

\newacronym{SPF}{Sender Policy Framework}{description={
    SPF is introduced in section~\refwithpage{spf introduction} and
    explained fully in~\cite{openspf, Wikipedia-spf}.
}}

\newacronym{ISP}{Internet Service Provider}{description={
    An ISP is a company which sells Internet access to consumers.
}}

\newacronym{LMA}{Log Mail Analyzer}{description={
    One of the other Postfix log parsers reviewed and the only example of
    published prior art~\cite{log-mail-analyser} found by the author.
}}

\newacronym{CSV}{Comma-Separated Value}{description={
    The most basic form of database available, each record is a single line
    in the file and the fields are separated by a special character,
    typically a comma or colon.
}}


\newacronym{FQDN}{Fully Qualified Domain Name}{description={
    An FQDN is a hostname plus domain name, e.g.  \newline
        example.com     is a domain name          \newline
        www             is a hostname             \newline
        www.example.com is a FQDN
}}

\newacronym{DNS}{Domain Name System}{description={
    The DNS converts between hostnames (www.example.com) and IP addresses
    (10.1.2.3).
}}

\newacronym{pid}{Process Identifier}{description={
    There may be multiple copies of any program executing at any one time,
    so the program's name is not suitable as a distinguishing identifier;
    instead each process executing is given a pid which is guaranteed to be
    unique for the lifetime of the process.  Once the process has
    completed, the pid may be reused, as they are drawn from a finite pool.
}}

\newacronym{UCE}{Unsolicited Commercial Email}{description={
    UCE is a more restrictive definition of spam than most people would
    use: it only covers mail that is explicitly commercial, thus excluding
    viruses, Bayesian poisoning mails, backscatter, and those annoying
    chain letters you get from friends.
}}

\newacronym{regex}{Regular Expression}{description={
    Regular expressions are a method of matching patterns in text that are
    explained in detail in~\cite{Wikipedia-regex, perlre} .
}}

% I think this is better than having both regex and regexes in the acronym
% list.  Likewise for pids.
\newcommand{\regexes}{\regex{}es}
\newcommand{\pids}{\pid{}s}

\newacronym{ESMTP}{Extended SMTP}{description={
    ESMTP is Extended SMTP, defined in RFC~1869~\cite{RFC1869}.
}}

\newacronym{HTML}{Hypertext Markup Language}{description={
    HTML is the markup language designed for writing web
    pages~\cite{Wikipedia-html, w3schools-html}.
}}

\newacronym{IP}{Internet Protocol}{description={
    The Internet Protocol~\cite{Wikipedia-ip} is the protocol used to
    communicate between computers on the Internet.  An IP address is a
    unique address assigned to a computer on the Internet, allowing it to
    communicate with other computers on the Internet.  For more information
    see~\cite{Wikipedia-ip-address}.
}}

\newacronym{PDF}{Portable Document Format}{description={
    PDF is the file format created by Adobe Systems in 1993 for document
    exchange.  It aims to be device independent, so documents should look
    the same whether viewed on screen or printed.
}}

\newacronym{LCD}{Lowest Common Denominator}{description={
    LCD is a mathematical term which is used figuratively to mean the least
    useful member of a set of alternatives.
}}

\newacronym{SLCT}{Simple Logfile Clustering Tool}{description={
    SLCT~\cite{slct-paper} is a tool implementing an algorithm designed by
    Risto Vaarandi~\cite{risto-vaarandi} for identifying and grouping
    similar log files, then producing a regex for each group of lines.
}}
