% vim: set filetype=tex :
% Note: every acronym is listed twice: once so it can be expanded, and once
% so it will appear with a long description in the glossary.
% Warning: do not use acronyms within definitions, they do not work
% properly.

\newglossaryentry{PLP}{
    type=\acronymtype,
    name={PLP},
    description={Postfix Log Parser},
    first={Postfix Log Parser (PLP)}
}
\newglossaryentry{PLP glossary}{
    name={Postfix Log Parser},
    description={
        Postfix Log Parser is the parser implementing the algorithm
        described in this thesis.
    }
}

\newglossaryentry{DNSBL}{
    type=\acronymtype,
    name={DNSBL},
    description={DNS Blacklist},
    first={DNS Blacklist (DNSBL)}
}
\newglossaryentry{DNSBL glossary}{
    name={DNS Blacklist},
    description={
        A DNS Blacklist~\cite{Wikipedia-DNSBL} is a simple collaborative
        anti-spam technique used to reject or penalise mail sent from mail
        servers believed to be the source of large volumes of spam.  The
        criteria used when deciding whether or not to include an IP address
        on a DNSBL vary widely between DNSBLs, so before using one it is
        very important to check their listing policies.  To use a DNSBL the
        mail server makes a DNS request incorporating the IP address of the
        client; if the requested hostname exists the client is on the
        DNSBL, and the mail server can decide what course of action to
        take.
    }
}

\newglossaryentry{API}{
    type=\acronymtype,
    name={API},
    description={Application Programming Interface},
    first={Application Programming Interface (API)}
}
\newglossaryentry{API glossary}{
    name={Application Programming Interface},
    description={
        One of the fundamental concepts when writing programs is the reuse
        of existing code, so that each program does not reinvent the wheel.
        When a body of code is intended to be reused repeatedly, the user
        of this code needs to be informed of the functionality provided by
        the code.  An API defines the interface provided to the user, and
        acts as a contract between the user and the provider: if the user
        adheres to the API the provider guarantees it will work, but is
        free to change the implementation provided the API is preserved.
    }
}

\newglossaryentry{SMTP}{
    type=\acronymtype,
    name={SMTP},
    description={Simple Mail Transfer Protocol},
    first={Simple Mail Transfer Protocol (SMTP)}
}
\newglossaryentry{SMTP glossary}{
    name={Simple Mail Transfer Protocol},
    description={
        SMTP is the protocol that transfers mail from the sender to the
        recipient across the Internet.  It is a simple, human readable,
        plain text protocol, making it quite simple to test and debug
        problems with it.  A detailed description of SMTP is beyond the
        scope of this document: the original protocol definition is
        RFC~821~\cite{RFC821}, updated in RFC~2821~\cite{RFC2821};
        introductory guides can be found
        at~\cite{smtp-intro-01,smtp-intro-02}.
    }
}

\newglossaryentry{LMTP}{
    type=\acronymtype,
    name={LMTP},
    description={Local Mail Transfer Protocol},
    first={Local Mail Transfer Protocol (LMTP)}
}
\newglossaryentry{LMTP glossary}{
    name={Local Mail Transfer Protocol},
    description={
        LMTP is a protocol derived from SMTP that removes the need for the
        server to maintain a mail delivery queue, instead relying on the
        client to maintain it.  Typically the client is an MTA, and the
        server a delivery agent or a mail store.  Full details are
        available in~\cite{lmtp-rfc}.
    }
}

\newglossaryentry{MTA}{
    type=\acronymtype,
    name={MTA},
    description={Mail Transfer Agent},
    first={Mail Transfer Agent (MTA)}
}
\newglossaryentry{MTA glossary}{
    name={Mail Transfer Agent},
    description={
        A Mail Transfer Agent sends and receives mail via SMTP\@.  Users
        submit mail to an MTA via their mail client (e.g.\ Microsoft
        Outlook, Thunderbird, webmail services); the sending MTA transfers
        the mail to the receiving MTA (which may in turn forward it to
        another MTA).
    }
}

\newglossaryentry{RFC}{
    type=\acronymtype,
    name={RFC},
    description={Request For Comments},
    first={Request For Comments (RFC)}
}
\newglossaryentry{RFC glossary}{
    name={Request For Comments},
    description={
        The Request For Comments series is a series of proposals defining
        various protocols and file formats, e.g.\ SMTP\@.  The name is
        somewhat misleading: initially the authors were asking for peer
        review, but these documents are now the de facto standards the
        Internet runs on.
    }
}

\newglossaryentry{SQL}{
    type=\acronymtype,
    name={SQL},
    description={Structured Query Language},
    first={Structured Query Language (SQL)}
}
\newglossaryentry{SQL glossary}{
    name={Structured Query Language},
    description={
        The standard language used for database querying, modification, and
        maintenance.  Some information and history can be found at
        \urlLastChecked{http://en.wikipedia.org/wiki/SQL}{XXX}; a good
        introduction can be found at
        \urlLastChecked{http://philip.greenspun.com/sql/}{XXX}.
    }
}

\newglossaryentry{LMA}{
    type=\acronymtype,
    name={LMA},
    description={Log Mail Analyzer},
    first={Log Mail Analyzer (LMA)}
}
\newglossaryentry{LMA glossary}{
    name={Log Mail Analyzer},
    description={
        One of the other Postfix log file parsers reviewed and the only
        example of prior published paper~\cite{log-mail-analyser} found
        that specifically deals with parsing Postfix log files.
    }
}

\newglossaryentry{CSV}{
    type=\acronymtype,
    name={CSV},
    description={Comma-Separated Value},
    first={Comma-Separated Value (CSV)}
}
\newglossaryentry{CSV glossary}{
    name={Comma-Separated Value},
    description={
        The most basic form of database available, each record is a single
        line in the file and the fields are separated by a special
        character, typically a comma or colon.
    }
}


\newglossaryentry{FQDN}{
    type=\acronymtype,
    name={FQDN},
    description={Fully Qualified Domain Name},
    first={Fully Qualified Domain Name (FQDN)}
}
\newglossaryentry{FQDN glossary}{
    name={Fully Qualified Domain Name},
    description={
        An FQDN is a hostname plus domain name, e.g.\ \newline{}
            example.com     is a domain name          \newline{}
            www             is a hostname             \newline{}
            www.example.com is a FQDN
    }
}

\newglossaryentry{pid}{
    type=\acronymtype,
    name={pid},
    description={Process Identifier},
    first={Process Identifier (pid)}
}
\newglossaryentry{pid glossary}{
    name={Process Identifier},
    description={
        There may be multiple copies of any program executing at any one
        time, so the program's name is not suitable as a distinguishing
        identifier; instead each process executing is given a pid that is
        guaranteed to be unique for the lifetime of the process.  Once the
        process has completed, the pid may be reused, as they are drawn
        from a finite pool.
    }
}

\newglossaryentry{UCE}{
    type=\acronymtype,
    name={UCE},
    description={Unsolicited Commercial Email},
    first={Unsolicited Commercial Email (UCE)}
}
\newglossaryentry{UCE glossary}{
    name={Unsolicited Commercial Email},
    description={
        UCE is a more restrictive definition of spam than most people use:
        it only covers mail that is explicitly commercial, thus excluding
        viruses, Bayesian poisoning mails, backscatter, and those annoying
        chain letters you get from friends.
    }
}

\newglossaryentry{ESMTP}{
    type=\acronymtype,
    name={ESMTP},
    description={Extended SMTP},
    first={Extended SMTP (ESMTP)}
}
\newglossaryentry{ESMTP glossary}{
    name={Extended SMTP},
    description={
        ESMTP is Extended SMTP, defined in RFC~1869~\cite{RFC1869}.
    }
}

\newglossaryentry{IP}{
    type=\acronymtype,
    name={IP},
    description={Internet Protocol},
    first={Internet Protocol (IP)}
}
\newglossaryentry{IP glossary}{
    name={Internet Protocol},
    description={
        The Internet Protocol is the protocol used to communicate between
        computers on the Internet.  An IP address is a unique address
        assigned to a computer on the Internet, allowing it to communicate
        with other computers on the Internet.
    }
}

\newglossaryentry{SLCT}{
    type=\acronymtype,
    name={SLCT},
    description={Simple Logfile Clustering Tool},
    first={Simple Logfile Clustering Tool (SLCT)}
}
\newglossaryentry{SLCT glossary}{
    name={Simple Logfile Clustering Tool},
    description={
        SLCT~\cite{slct-paper} is a tool implementing an algorithm designed
        by Risto Vaarandi for identifying, grouping, and producing a regex
        to match similar log lines.
    }
}

\newglossaryentry{ATN}{
    type=\acronymtype,
    name={ATN},
    description={Augmented Transition Networks},
    first={Augmented Transition Networks (ATN)}
}
\newglossaryentry{ATN glossary}{
    name={Augmented Transition Networks},
    description={
        Originally described in~\cite{atns} and further in~\cite{nlpip},
        Augmented Transition Networks are a tool used in Computational
        Linguistics for creating grammars to parse or generate sentences,
    }
}

\newglossaryentry{CLI}{
    type=\acronymtype,
    name={CLI},
    description={Command Line Interface},
    first={Command Line Interface (CLI)}
}
\newglossaryentry{CLI glossary}{
    name={Command Line Interface},
    description={
        An computer interface based on typing commands, rather than using a
        mouse.
    }
}

\newglossaryentry{NLP}{
    type=\acronymtype,
    name={NLP},
    description={Natural Language Processing},
    first={Natural Language Processing (NLP)}
}
\newglossaryentry{NLP glossary}{
    name={Natural Language Processing},
    description={
        Natural Language Processing~\cite{Wikipedia-nlp} is an area of
        research that attempts to increase our understanding of the
        languages normally used by humans (e.g.\ English, Spanish,
        Japanese), with the goal of writing programs that can understand
        human languages.
    }
}

\newglossaryentry{FA}{
    type=\acronymtype,
    name={FA},
    description={Finite Automata},
    first={Finite Automata (FA)}
}
\newglossaryentry{FA glossary}{
    name={Finite Automata},
    description={
        Finite automata are computing devices that accept or recognize
        regular languages.  They lack any form of storage, so while they
        can recognise languages such as ${(ab)}^{*}$ or
        $a^{*}{(b^{*}|c)}b$, they cannot count and so cannot recognise
        languages such as $a^{n}b^{n}$.
    }
}

\newglossaryentry{CFL}{
    type=\acronymtype,
    name={CFL},
    description={Context Free Language},
    first={Context Free Language (CFL)}
}
\newglossaryentry{CFL glossary}{
    name={Context Free Language},
    description={
        XXX WRITE THIS DESCRIPTION
    }
}

\newglossaryentry{PDA}{
    type=\acronymtype,
    name={PDA},
    description={Push-Down Automata},
    first={Push-Down Automata (PDA)}
}
\newglossaryentry{PDA glossary}{
    name={Push-Down Automata},
    description={
        XXX WRITE THIS DESCRIPTION
    }
}

%\newglossaryentry{}{
%    type=\acronymtype,
%    name={},
%    description={},
%    first={}
%}
%\newglossaryentry{}{
%    name={},
%    description={
%    }
%}
