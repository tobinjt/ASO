% vim: set filetype=tex :
% Note: every acronym is listed twice: once so it can be expanded, and once
% so it will appear with a long description in the glossary.
% Warning: do not use acronyms within definitions, they do not work
% properly.

\newglossaryentry{PLP}{
    type=\acronymtype,
    name={PLP},
    description={Postfix Log Parser},
    first={Postfix Log Parser (PLP)}
}

\newglossaryentry{DNSBL}{
    type=\acronymtype,
    name={DNSBL},
    description={DNS Blacklist},
    first={DNS Blacklist (DNSBL)}
}
\newglossaryentry{DNSBL glossary}{
    name={DNS Blacklist},
    description={
        A DNS Blacklist is a simple collaborative anti-spam technique used
        to reject or penalise mail sent from mail servers believed to be
        the source of large volumes of spam.  The criteria used when
        deciding whether or not to include an IP address on a DNSBL vary
        widely between DNSBLs, so before using one it is very important to
        check their listing policies.  To use a DNSBL the mail server makes
        a DNS request incorporating the IP address of the client; if the
        requested hostname exists the client is on the DNSBL, and the mail
        server can decide what course of action to take.
    }
}

\newglossaryentry{API}{
    type=\acronymtype,
    name={API},
    description={Application Programming Interface},
    first={Application Programming Interface (API)}
}
\newglossaryentry{API glossary}{
    name={Application Programming Interface},
    description={
        One of the fundamental concepts when writing programs is the reuse
        of existing code, so that each new program does not reinvent
        existing wheels.  An API defines the interface provided to the user
        of the existing code, and acts as a contract between the user and
        the provider: if the user adheres to the API the provider
        guarantees it will work, but is free to change the implementation
        as long as the API is preserved.
    }
}

\newglossaryentry{SMTP}{
    type=\acronymtype,
    name={SMTP},
    description={Simple Mail Transfer Protocol},
    first={Simple Mail Transfer Protocol (SMTP)}
}
\newcommand{\SMTPglossaryDescription}[0]{%
    \acronym{SMTP} is the protocol used for transferring mail between the
    sending and receiving \acronym{MTA}\@.  It is a simple, human readable,
    plain text protocol, making it quite simple to test and debug problems
    with it.  A detailed description of SMTP is beyond the scope of this
    thesis: the original protocol definition is in RFC~821~\cite{RFC821},
    updated in RFC~2821~\cite{RFC2821}; an introductory guide can be found
    at
    \urlLastChecked{http://db.glug-bom.org/lug-authors/philip/docs/mail-stuff/smtp-intro.html}{2009/02/28}
    if more information is needed.%
}
\newglossaryentry{SMTP glossary}{
    name={Simple Mail Transfer Protocol},
    description={
        \SMTPglossaryDescription{}
    }
}

\newglossaryentry{LMTP}{
    type=\acronymtype,
    name={LMTP},
    description={Local Mail Transfer Protocol},
    first={Local Mail Transfer Protocol (LMTP)}
}
\newglossaryentry{LMTP glossary}{
    name={Local Mail Transfer Protocol},
    description={
        LMTP is a protocol derived from SMTP that removes the need for the
        server to maintain a mail delivery queue, instead relying on the
        client to maintain it.  Typically the client is an MTA, and the
        server a delivery agent or a mail store.  Full details are
        available in~\cite{lmtp-rfc}.
    }
}

\newglossaryentry{MTA}{
    type=\acronymtype,
    name={MTA},
    description={Mail Transfer Agent},
    first={Mail Transfer Agent (MTA)}
}
\newglossaryentry{MTA glossary}{
    name={Mail Transfer Agent},
    description={
        A Mail Transfer Agent sends and receives mail via SMTP\@.  Users
        submit mail to an MTA via their mail client (e.g.\ Microsoft
        Outlook, Thunderbird, webmail services); the sending MTA transfers
        the mail to the receiving MTA (which may in turn forward it to
        another MTA).
    }
}

\newglossaryentry{RFC}{
    type=\acronymtype,
    name={RFC},
    description={Request For Comments},
    first={Request For Comments (RFC)}
}
\newglossaryentry{RFC glossary}{
    name={Request For Comments},
    description={
        The Request For Comments series is a series of proposals defining
        various protocols and file formats, e.g.\ SMTP\@.  The name is
        somewhat misleading nowadays: initially the authors were asking for
        peer review, but these documents are now the de facto standards the
        Internet runs on.
    }
}

\newglossaryentry{SQL}{
    type=\acronymtype,
    name={SQL},
    description={Structured Query Language},
    first={Structured Query Language (SQL)}
}

\newglossaryentry{LMA}{
    type=\acronymtype,
    name={LMA},
    description={Log Mail Analyzer},
    first={Log Mail Analyzer (LMA)}
}

\newglossaryentry{CSV}{
    type=\acronymtype,
    name={CSV},
    description={Comma-Separated Value},
    first={Comma-Separated Value (CSV)}
}

\newglossaryentry{FQDN}{
    type=\acronymtype,
    name={FQDN},
    description={Fully Qualified Domain Name},
    first={Fully Qualified Domain Name (FQDN)}
}
\newglossaryentry{FQDN glossary}{
    name={Fully Qualified Domain Name},
    description={
        An FQDN is a hostname plus domain name, e.g.\ \newline{}
            example.com     is a domain name          \newline{}
            www             is a hostname             \newline{}
            www.example.com is a FQDN
    }
}

\newglossaryentry{pid}{
    type=\acronymtype,
    name={pid},
    description={Process Identifier},
    first={Process Identifier (pid)}
}
\newglossaryentry{pid glossary}{
    name={Process Identifier},
    description={
        There may be multiple copies of any program executing at any one
        time, so the program's name is not suitable as a distinguishing
        identifier; instead each process executing is given a pid that is
        guaranteed to be unique for the lifetime of the process.  Once the
        process has completed, the pid may be reused, as they are drawn
        from a finite pool.
    }
}

\newglossaryentry{UCE}{
    type=\acronymtype,
    name={UCE},
    description={Unsolicited Commercial Email},
    first={Unsolicited Commercial Email (UCE)}
}
\newglossaryentry{UCE glossary}{
    name={Unsolicited Commercial Email},
    description={
        UCE is a more restrictive definition of spam than most people use:
        it only covers mail that is explicitly commercial, thus excluding
        viruses, Bayesian poisoning mails, backscatter\glsadd{Backscatter},
        and those annoying chain letters you get from friends.
    }
}

\newglossaryentry{ESMTP}{
    type=\acronymtype,
    name={ESMTP},
    description={Extended SMTP},
    first={Extended SMTP (ESMTP)}
}
\newglossaryentry{ESMTP glossary}{
    name={Extended SMTP},
    description={
        \acronym{ESMTP}~\cite{RFC1869}, provides a flexible mechanism for
        \acronym{SMTP} to be extended with new functionality, allowing new
        features to be tested without having to be included in the standard
        protocol.  \acronym{ESMTP} is backwards compatible with
        \acronym{SMTP}: \acronym{ESMTP} clients and servers can interact
        with \acronym{SMTP} clients and servers without difficulty.
    }
}

\newglossaryentry{IP}{
    type=\acronymtype,
    name={IP},
    description={Internet Protocol},
    first={Internet Protocol (IP)}
}

\newglossaryentry{SLCT}{
    type=\acronymtype,
    name={SLCT},
    description={Simple Logfile Clustering Tool},
    first={Simple Logfile Clustering Tool (SLCT)}
}

\newglossaryentry{ATN}{
    type=\acronymtype,
    name={ATN},
    description={Augmented Transition Networks},
    first={Augmented Transition Networks (ATN)}
}

\newglossaryentry{CLI}{
    type=\acronymtype,
    name={CLI},
    description={Command Line Interface},
    first={Command Line Interface (CLI)}
}

\newglossaryentry{NLP}{
    type=\acronymtype,
    name={NLP},
    description={Natural Language Processing},
    first={Natural Language Processing (NLP)}
}
\newglossaryentry{NLP glossary}{
    name={Natural Language Processing},
    description={
        Natural Language Processing is an area of research that attempts to
        increase our understanding of the languages normally used by humans
        (e.g.\ English, Spanish, Japanese), with the goal of writing
        programs that can understand human languages.
    }
}

\newglossaryentry{FA}{
    type=\acronymtype,
    name={FA},
    description={Finite Automata},
    first={Finite Automata (FA)}
}
\newglossaryentry{FA glossary}{
    name={Finite Automata},
    description={
        Finite automata are computing devices that accept or recognize
        regular languages.  They lack any form of storage, so while they
        can recognise languages such as ${(ab)}^{*}$ or
        $a^{*}{(b^{*}|c)}b$, they cannot count and so cannot recognise
        languages such as $a^{n}b^{n}$.
    }
}

\newglossaryentry{CFG}{
    type=\acronymtype,
    name={CFG},
    description={Context Free Grammar},
    first={Context Free Grammar (CFG)}
}
\newglossaryentry{CFG glossary}{
    name={Context Free Grammar},
    description={
        See Context Free Language.
    }
}

\newglossaryentry{CFL}{
    type=\acronymtype,
    name={CFL},
    description={Context Free Language},
    first={Context Free Language (CFL)}
}
\newglossaryentry{CFL glossary}{
    name={Context Free Language},
    description={
        A Context Free Language is a language generated or parsed by a
        \acronym{CFG}.  Every transition in a CFG takes the form
        $L\rightarrow{}R$, where $L$ is a single non-terminal symbol, and
        $R$ is a (possibly empty) string of terminal and/or non-terminal
        symbols.  Every CFG can be recognised by a \acronym{PDA}, and
        \acronyms{PDA} cannot recognise more complicated grammars, so
        \acronyms{CFL}, \acronyms{CFG}, and \acronyms{PDA} are equivalent.
    }
}

\newglossaryentry{PDA}{
    type=\acronymtype,
    name={PDA},
    description={Push-Down Automata},
    first={Push-Down Automata (PDA)}
}
\newglossaryentry{PDA glossary}{
    name={Push-Down Automata},
    description={
        A Push-Down Automation is a computational device similar to a
        \acronym{FA}, but it can additionally use a stack to store data.  A
        \acronym{PDA} can manipulate the stack during state transitions
        (adding or removing a single piece of data), and when determining
        which state transition to take it can use the piece of data on the
        top of the stack in addition to the input.  \acronyms{PDA} can
        recognise languages of the form $a^{n}b^{n}$, i.e.\ any number of
        one symbol followed by the same number of another symbol.
    }
}

\newglossaryentry{SPF}{
    type=\acronymtype,
    name={SPF},
    description={Sender Policy Framework},
    first={Sender Policy Framework (SPF)}
}

%\newglossaryentry{}{
%    type=\acronymtype,
%    name={},
%    description={},
%    first={}
%}
%\newglossaryentry{}{
%    name={},
%    description={
%    }
%}
