\section{Parser Architecture}

XXX REWRITE\@: MERGE CONTENT FROM ELSEWHERE IN THE THESIS AND PAPER\@.

XXX NEEDS AN INTRODUCTION\@.

XXX MERGE SOME OF THE PAPER'S CONTENT IN HERE\@.

\subsection{Parser architecture and design}

\label{parser design}

It should be clear from the earlier Postfix background (\sectionref{postfix
background}) that log files produced by Postfix vary widely from host to
host, depending on the set of restrictions chosen by the administrator.
With this in mind, one of the parser's design aims was to make adding new
rules as easy as possible, to enable administrators to properly parse their
own log files.  To enable this the architecture is divided into three
parts:

XXX UPDATE SECTION REFERENCES BELOW\@: THEY NEED TO REFER TO BOTH THE
ARCHITECTURE AND THE IMPLEMENTATION\@.

\begin{description}

    \item [framework] The framework is responsible for loading rules,
        managing data structures, finding the rule that matches each log
        line, invoking the correct action, etc\@.  The framework is
        described in detail in \sectionref{framework}.  XXX EXTEND
        FRAMEWORK SECTION IF NECESSARY\@.

    \item [rules] Rules match individual log lines and determine which
        actions will be executed.  Rules provide an easily extensible
        method of associating log lines with actions, and are described in
        detail in \sectionref{rules}.

    \item [actions] Actions are invoked to deal with a log line once it has
        been identified by the rules: actions modify data structures,
        handle complications, and cause data to be saved to the database.
        The actions do the work of reconstructing the journey a mail takes
        through Postfix.  Details of the actions available in the Postfix
        parser can be found in \sectionref{actions-in-detail}, and
        \sectionref{adding new actions} describes the process of adding new
        actions.

\end{description}

It may help to think of the rules and actions as components that plug into
the framework.  The framework tries each rule in turn until is finds one
which matches the log line, then invokes the action specified by the rule.

\label{why separate rules, actions, and framework?}

XXX MERGE THE NEXT TWO PARAGRAPHS\@.  PULL CONTENT FROM THE PAPER\@.

Decoupling the parsing rules from the associated actions and framework
allows new rules to be written and tested without requiring modifications
to the parser source code (significantly lowering the barrier to entry for
new or casual users who need to parse new log lines), and greatly
simplifies framework, actions, and rules.  Decoupling also creates a clear
separation of functionality: rules handle low level details of identifying
log lines and extracting data from a log line; actions handle the higher
level details of following the path a mail takes through Postfix,
assembling the required data before storing it, dealing with complications
arising, etc; the framework provides services to actions and stores data.

Decoupling the actions from the framework simplifies both framework and
actions: the framework provides services to the actions, and does not need
to deal with the complications that arise, or the task of reconstructing a
mail's journey through Postfix; actions benefit from having services
provided by the framework, freeing them to concentrate on the task of
accurately reconstructing each mail's journey through Postfix and dealing
with the complications described in \sectionref{complications}.

Separating the rules from the actions and framework makes it possible to
parse new log lines without modifying the core parsing algorithm.  Adding a
new rule with the action to invoke and a regex to match the log lines is
trivial in comparison to understanding a program's entire parsing
algorithm, identifying the correct location to change and making the
appropriate changes.  Bear in mind that the changes must be made without
adversely affecting existing parsing, particularly as there may be edge
cases that are not immediately obvious ---
\sectionref{yet-more-aborted-delivery-attempts} describes a complication
that occurs only four times in \numberOFlogFILES{} log files tested.
Requiring changes to the parser's code also complicates upgrades, as the
changes must be preserved during the upgrade, and may clash with changes
made by the developer (\sectionref{adding new actions} describes safely
adding new actions).  \parsername{} allows the user to add new rules to the
database without changing the parsing algorithm, unless the new log lines
to be parsed require functionality not already provided by the algorithm.
If the new log lines do require new functionality, new actions can be added
to the parser without modifying existing actions or other parts of the
algorithm; only in the rare event that the new actions require support from
other sections of the code will more extensive changes be required.

There is some similarity between the parser's design and William Wood's
\gls{ATN}~\cite{atns,nlpip}, a tool used in Computational Linguistics for
creating grammars to parse or generate sentences.  The resemblance between
\gls{ATN} and the parser is accidental, but it is obvious that the two
different approaches share an underlying separation of concerns; this
appears to be a natural division of responsibility and functionality.
\Tableref{Similarities between ATN and this architecture} shows the
similarities.

% Do Not Reformat!

\begin{table}[ht]
    \caption{Similarities between ATN and this architecture}
    \empty{}\label{Similarities between ATN and this architecture}
    \begin{tabular}[]{lll}
        \tableline{}
        \gls{ATN}          & Architecture  & Similarity                \\
        \tableline{}
        Networks        & Algorithm     & Determines the sequence 
                                        of transitions              \\
                        &               & or actions that 
                                        constitutes a valid  input. \\
        Transitions     & Actions       & Save data and impose
                                        conditions the              \\
                        &               & input must meet to be
                                        accepted as valid.          \\
        Abbreviations   & Rules         & Responsible for 
                                        classifying input.          \\
        \tableline{}
    \end{tabular}
\end{table}

\subsection{Rules}

\label{rule in architecture}

XXX TO BE WRITTEN\@.  TAKE CONTENT FROM THE PAPER AND POSSIBLY FROM THE
IMPLEMENTATION SECTION\@.

\subsubsection{Rule characteristics}

\label{rule characteristics}

Rule have certain characteristics that may help in understanding the
parser:

\begin{itemize}

    \item Rules are annotated with the name of a Postfix program, and will
        only be used when parsing log lines produced by that
        program.\footnote{There are also generic rules which are used when
        parsing log lines produced by any Postfix program, but only if
        there are also rules specific to that program, and those rules have
        already have been tried and have failed on the current log line.}
        Any given rule will only be used to parse a subset of the log
        lines, and any given log line will only have a subset of the rules
        tried against it.

    \item The first matching rule wins: no further rules are tried against
        that log line, but there is a mechanism for prioritising the rules
        so that more specific rules can be tried first.

    \item Rules are completely self-contained and can be understood in
        isolation, without reference to any other rules.

    \item Rule processing time is a linear function of the number of rules.
        XXX IMPROVE THIS\@.

\end{itemize}

\label{comparison against context-free grammars}

In context-free grammar terms the parser rules could be described as:

$\text{\textless{}log-line\textgreater{}} \mapsto \text{rule-1} |
\text{rule-2} | \text{rule-3} | \dots | \text{rule-n}$


\subsubsection{Overlapping rules}

\label{overlapping rules}

XXX IS THERE ANY PREVIOUS RESEARCH IN THIS AREA\@?

XXX DFA COMPARISON\@.

The parser does not try to detect overlapping rules; that responsibility is
left to the author of the rules.  Unintentionally overlapping rules lead to
inconsistent parsing and data extraction because the order in which rules
are tried against each line may change between log files, and the first
matching rule wins.  Overlapping rules are frequently a requirement,
allowing a more specific rule to match some log lines and a more general
rule to match the majority, e.g.\ separating \gls{SMTP} delivery to
specific sites from \gls{SMTP} delivery to the rest of the world.  The
algorithm provides a mechanism for ordering overlapping rules: the priority
field in each rule (defaults to zero).  Negative priorities may be useful
for catchall rules.

Detecting overlapping rules is difficult, but the following approaches may
be helpful:

\begin{itemize}

    \item Sort by regex and visually inspect the list, e.g.\ with \gls{SQL}
        similar to: \textbf{select regex from rules order by regex;}

    \item Compare the results of parsing using sorted, shuffled, and
        reversed rules (described in \sectionref{rule ordering for
        efficiency}).  Parse several log files using optimal ordering, then
        dump a textual representation of the rules, connections, and
        results tables.  Repeat with shuffled and reversed ordering,
        starting with a fresh database.  If there are no overlapping rules
        the tables from each run will be identical; differences indicate
        overlapping rules.  The rules that overlap can be determined by
        examining the differences in the tables: each result contains a
        reference to the rule which created it, if the references differ
        between runs the two rules referenced in the differing results
        overlap.  Unfortunately this method cannot prove the absence of
        overlapping rules; it can detect overlapping rules, but only if
        there are log lines in the input files that match more than one
        rule.

\end{itemize}


\subsection{Framework}

XXX PULL CONTENT FROM THE PAPER FOR HERE\@; IT'S BETTER WRITTEN\@.

\label{framework}

The intermingling of log entries from different mails immediately rules out
the possibility of handling each mail in isolation; the parser must be able
to handle multiple mails in parallel, each potentially at a different stage
in its journey, without any interference between mails --- except in the
minority of cases where intra-mail interference is required.  The best way
to implement this is to maintain state information for every unfinished
mail and manipulate the appropriate mail correctly for each log line
encountered.

This functionality is provided by the framework, which both drives parsing
overall and provides services to the actions it invokes.  The framework is
responsible for loading rules from the database and sanity checking them,
reading log files, matching each rule against the current log line,
invoking the correct action, maintaining state tables, loading and saving
state, displaying a progress indicator, and miscellaneous other tasks.
Actions use services provided by the framework, including storing and
retrieving data in the state tables, extracting and saving data captured by
rules, storage of global data, debugging functions, preparing a mail for
entry in the database, and entering mails in the database.

There is a similarity between this design and the event-driven programming
paradigm commonly used in GUI programs, where one part of the program
responds to events (mouse clicks in a GUI program, log lines being matched
in the parser) and invokes the correct action.

The framework's high level overview could be expressed as (indentation
denoting flow of control):

\begin{verbatim}
for each line in the input files:
    for each rule defined by the user:
        if this rule matches the input line:
            perform the action specified by the rule
            skip the remaining rules
            process the next input line
    warn the user that the input line was not parsed
\end{verbatim}


\subsection{Actions}

\label{actions in architecture}

XXX TO BE WRITTEN\@.  START WITH THE CONTENT FROM THE PAPER, THEN CHECK THE
IMPLEMENTATION SECTION\@.

