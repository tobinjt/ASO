\section{Background}

\label{background}

\subsection{Introduction}

This section provides background information helpful in understanding the
remainder of the document.  It begins with a discussion of the motivation
underlying the project, followed by some technical information: the use of
a database as an \API{}\@; a brief introduction to \SMTP{}\@; a longer
introduction to Postfix, concentrating on the topics most relevant to this
document, namely Postfix anti-spam restrictions and policy servers.  The
assumptions made in designing and implementing the parser are explained, as
are the conventions used in this document.  Other projects which attempt to
parse Postfix log files are summarised (full details are available in
appendix~\refwithpage{other-parsers}), finishing with a review of
previously published research in this area.

\subsection{Motivation}

\label{motivation}

This document and the program it describes are part of a larger project to
optimise a mail server's Postfix restrictions, generate statistics and
graphs, and provide a platform on which new restrictions can be trialled
and evaluated to determine if they are beneficial in the fight against
spam.  The program parses Postfix log files and populates a database with
the data gleaned from those log files, providing a consistent and simple
view of the log files which future tools can utilise.  The gathered data
can then be used to optimise current anti-spam measures, provide a baseline
to test new anti-spam measures against, or to produce statistics showing
how effective those measures are.

A snippet of \SQL{} provides a short example of the optimisation possible
using data from the database is determining which Postfix restrictions
reject the highest number of mails:

\begin{verbatim}
SELECT name, description, restriction_name, hits_total
    FROM rules
    WHERE postfix_action = 'REJECTED'
    ORDER BY hits_total DESC;
\end{verbatim}

If the database supports sub-selects percentages can be obtained:
\footnote{\SQL{} note: $||$ is the concatenation operator in SQLite3; if
the database containing the extracted data does not support this syntax,
then simply remove `` $||$ '$\%$'\hspace{1ex}'' from the query --- the
results will be the same, just slightly less visually pleasing.}

\begin{verbatim}
SELECT name, description, restriction_name, hits_total,
        (hits_total * 100.0 /
            (SELECT SUM(hits_total)
                FROM rules
                WHERE postfix_action = 'REJECTED'
            )
        ) || '%' AS percentage
    FROM rules
    WHERE postfix_action = 'REJECTED'
    ORDER BY hits_total DESC;
\end{verbatim}

XXX GIVE SAMPLE OUTPUT FOR THE SNIPPET ABOVE\@.

Another example is determining which restrictions are not effective: this
shows which restrictions had fewer than 100 hits on the last log file
parsed, and the percentage of total rejections each restriction represents.

\begin{verbatim}
SELECT name, description, restriction_name, hits,
        (hits * 100.0 /
            (SELECT SUM(hits)
                FROM rules
                WHERE postfix_action = 'REJECTED'
            )
        ) || '%' AS percentage
    FROM rules
    WHERE postfix_action = 'REJECTED'
        AND hits < 100
    ORDER BY hits ASC;
\end{verbatim}

These database queries yield summary statistics about the efficiency of
spam avoidance techniques; statistics that are far less feasible to assess
directly from log files without prior pre-processing into a database in the
fashion proposed, implemented and tested herein.

\subsection{Database as Application Programming Interface}

\label{database as API}

The database populated by this program provides a simple interface to
Postfix log files.  Although the interface is a database schema, it is in
effect quite similar to any other \API{} provided by shared code: it
insulates both user and provider of the \API{} from changes in the
implementation of the \API{}\@.  The algorithm implemented by the parser
can be improved; support can be added for earlier or later releases of
Postfix; bugs can be fixed or limitations removed from the parser; these
changes will not cause the user to be negatively impacted.  Statistics
and/or graphs can be generated from the database; new restrictions can be
tested and the results inspected; trends in the fight against spam can
emerge from historical data saved in the database; the parser remains the
same as the usage adapts.  Using a database simplifies writing programs
which need to interact with the data in several ways:

\begin{enumerate}

    \item XXX REWRITE THIS PARAGRAPH\@.  The majority of programming
        languages have existing code available providing access to
        databases, so the majority of languages will be able to access the
        results obtained by the parser.  If this parser was written to be
        utilised as shared code there would be little or no difficulty in
        using it when writing in the programing language it is implemented
        in, but every other programing language wishing to utilise it would
        require an interface layer to be written.  This requirement would
        drastically reduce the viability of any project wishing to build
        upon the work of this project.

    \item Databases provide complex querying and sorting functionality to
        the user without requiring large amounts of programming.  All
        databases provide a program, of varying complexity and
        sophistication, which can be used for ad hoc queries with minimal
        investment of time.

    \item Databases are easily extensible, e.g.:

        \begin{itemize}

            \item Other tables can be added to the database, e.g.\ to cache
                historical, summary or computed data.

            \item New columns can be added to the tables used by the
                program, with sufficient DEFAULT clauses or a clever
                TRIGGER or two.\footnote{Please refer to an \SQL{} guide
                for explanations of these terms,
                e.g.~\cite{sql-for-web-nerds}}

            \item A VIEW gives a custom arrangement of data with very
                little effort.

            \item If the database supports it, access can be granted on a
                fine-grained basis, e.g.\ allowing the finance department
                to produce invoices, the helpdesk to run limited queries as
                part of dealing with support calls, and the administrators
                to have full access to the data.

            \item Triggers can be written to perform actions when certain
                events occur.  In pseudo-\SQL{}\@:

\begin{verbatim}
CREATE TRIGGER ON INSERT INTO results
    WHERE sender = 'boss@example.com'
        AND postfix_action = 'REJECTED'
    SEND PANIC EMAIL TO 'postmaster@example.com';
\end{verbatim}

        \end{itemize}


    \item \SQL{} is reasonably standard and many people will already be
        familiar with it; for those unfamiliar with it there are lots of
        readily available resources from which to learn (a good
        introduction to \SQL{} can be found at~\cite{sql-for-web-nerds},
        others are~\cite{w3schools-sql-tutorial, sqlcourse.com}).  Although
        every vendor implements a different dialect of \SQL{}, the basics
        are the same everywhere (analogous to the overall similarities and
        minor differences between Irish English, British English, American
        English and Australian English).  Depending on the database in use
        there may be tools available which reduce or remove the requirement
        to know \SQL{}; for SQLite (the default database used by the
        implementation) there are several available~\cite{sqlite-guis}.

\end{enumerate}

Storing the results in a database will also increase the efficiency of
using those results, as the log files need only be parsed once; indeed the
results may be used by someone with no access to the original log files.



\subsection{Simple Mail Transfer Protocol}
\label{SMTP background}

The \SMTPlong{}, originally defined in \RFC{}~821~\cite{RFC821} and updated
in \RFC{}~2821~\cite{RFC2821}, is used for transferring mail between the
sending and receiving \MTA{}\@.  It is a simple, human readable, plain text
protocol, making it quite easy to test and debug problems with it.  Despite
the simplicity of the protocol many virus and/or spam sending programs fail
to implement it properly, so requiring strict adherence to the protocol
specification is beneficial in protecting against spam and
viruses.\footnote{\label{footnote:rfc760}Originally all mail servers
adhered to the principle of \textit{Be liberal in what you accept, and
conservative in what you send\/} from \RFC{}~760~\cite{rfc760}, but
unfortunately that principle was written in a friendlier time.  Given the
deluge of spam that mail servers are subjected to daily, a more appropriate
maxim could be: \textit{Require strict adherence to \RFC{}~2821; implement
the strongest restrictions you can; relax the restrictions and adherence
only when legitimate mail is impeded.\/}  It's not as friendly, nor as
catchy, but it more accurately reflects the current situation.} A typical
\SMTP{} conversation resembles the following (the lines starting with a
three digit number are sent by the server, all other lines are sent by the
client):

\begin{verbatim}
220 smtp.example.com ESMTP
HELO client.example.com
250 smtp.example.com
MAIL FROM: <alice@example.com>
250 2.1.0 Ok
RCPT TO: <bob@example.com>
250 2.1.5 Ok
DATA
354 End data with <CR><LF>.<CR><LF>
Message headers and body sent here.
.
250 2.0.0 Ok: queued as D7AFA38BA
QUIT
221 2.0.0 Bye
\end{verbatim}

An example deviation from the protocol:

\begin{verbatim}
220 smtp.example.com ESMTP
HELO client.example.com
250 smtp.example.com
MAIL FROM: Alice N. Other alice@example.com
501 5.1.7 Bad sender address syntax
RCPT TO: Bob in Sales/Marketing bob@example.com
503 5.5.1 Error: need MAIL command
DATA
503 5.5.1 Error: need RCPT command
Message headers and body sent here.
.
502 5.5.2 Error: command not recognized
QUIT
221 2.0.0 Bye
\end{verbatim}

This client is so poorly written that not only does it present the sender
and recipient addresses improperly, it ignores the error messages returned
by the server and carries on regardless.  There are many spam and virus
sending programs which are this deficient --- unfortunately others
(particularly newer programs) were written by competent programmers, or
utilise competently written programs (e.g.\ Postfix or Sendmail on Unix
hosts, Microsoft Outlook on Windows hosts).  Traditionally a mail server
would have done its best to deal with deficient clients, with the intention
of accepting as much mail destined for its users as
possible,\footref{footnote:rfc760} e.g.\ by ignoring the absence of a HELO
command, or accepting sender or recipient addresses which were not enclosed
in \texttt{<>}.  

A detailed description of \SMTP{} is beyond the scope of this document:
introductory guides can be found at~\cite{smtp-intro-01, smtp-intro-02},
the definitive references are~\cite{RFC821, RFC2821}.

\subsection{Postfix}

\label{postfix background}

Postfix is a \MTA{} with the following design aims (in order of
importance): security, flexibility of configuration, scalability, and high
performance.  It features extensive, extensible, optional anti-spam
restrictions, allowing an administrator to deploy those restrictions which
they judge suitable for their site's needs, rather than a fixed set chosen
by Postfix's author.  These restrictions can be selectively applied,
combined and bypassed on a per-client, per-recipient or per-sender basis,
allowing varying levels of stricture and/or permissiveness.  Postfix
leverages simple lookup tables to support arbitrarily complicated
user-defined sequences of restrictions and exceptions, with policy
servers\footnote{Policy servers will be explained in \sectionref{policy
servers}.} as the ultimate in flexibility.  Administrators can also supply
their own rejection messages to make it clear to senders why exactly their
mail was rejected.  Unfortunately this flexibility has a cost: complexity
in the log files generated.  While it is easy to use standard Unix text
processing utilities to determine the fate of an individual email,
following the journey an email takes through Postfix can be quite
difficult.  The log files tend to follow a 90\%-10\% pattern: 90\% of the
time the journey is simple, but the other 10\% of the time accounts for
90\% of the code in the parser.\footnote{These numbers don't have a solid
scientific basis, they're based on gut feeling from writing and debugging
the software.}

Postfix's design follows the Unix philosophy of \textit{Write programs that
do one thing and do it well\/}~\cite{unix-philosophy},\footnote{This
principle was also followed when designing the parser (\sectionref{parser
design}), notably in having separate actions rather than one huge action.}
and is separated into various component programs to perform the tasks
required of an \MTA{}\@: receive mail, send mail, local delivery of mail,
etc. --- full details can be found at~\cite{postfix-overview}.  Each log
line contains the name of the Postfix component which produced it, and this
information is used when determining which rules should be used to parse
each log line (see \sectionref{rule characteristics} for details).  This
design has positive security implications also: those components which
interact with other hosts are not privileged,\footnote{Privilege in this
case means the power to perform actions that are limited to the
administrator, and not available to ordinary users.} so bugs in those
components will not give an attacker extra privileges; those components
which are privileged do not interact with other hosts, making it much more
difficult for an attacker to exploit any bugs which may exist in those
components.

\subsubsection{Mixing and matching Postfix restrictions}

\label{Mixing and matching Postfix restrictions}

Postfix restrictions are documented fully in~\cite{smtpd_access_readme,
smtpd_per_user_control, policy-servers}; the following is a brief overview
only.

Postfix uses one restriction list (containing zero or more restrictions)
for each stage of the \SMTP{} conversation: client connection, HELO
command, MAIL FROM command, RCPT TO commands, DATA command, and end of
data.  The appropriate restriction list is evaluated for each stage
(evaluation will be explained shortly), though by default the restriction
lists for client connection, HELO and MAIL FROM commands will not be
evaluated until the first RCPT TO command is received, because some clients
don't deal properly with rejections before RCPT TO\@; a benefit of this
delay is that Postfix has more information available when logging
rejections.

Each restriction is evaluated to produce a result of \textit{reject},
\textit{permit}, \textit{dunno\/} or the name of another restriction to be
evaluated.\footnote{Other results are possible as described
in~\cite{smtpd_access_readme, smtpd_per_user_control, policy-servers}.} The
meaning of \textit{permit\/} and \textit{reject\/} is fairly obvious;
\textit{dunno\/} means to stop evaluating the current restriction and
continue processing the remainder of the restriction list, allowing more
specific cases to be used as exceptions to more general cases.  When the
result is the name of another restriction Postfix will evaluate the new
restriction, allowing restrictions to be chosen based on the client \IP{}
address, HELO hostname, sender address, recipient address,
etc.\footnote{E.g.\ the administrator may require that all clients on the
local network have valid \DNS{} entries, to prevent people sending mail
from unknown machines.}  The administrator can define new restrictions as a
list of existing restrictions, allowing arbitrarily long and complex
sequences of lookups, restrictions and exceptions.  Postfix tries to
protect the administrator in as far as is reasonable, e.g.\ the restriction
\texttt{check\_helo\_mx\_access} cannot cause a mail to be accepted,
because the parameter it checks (the hostname given in the HELO command) is
under the control of the remote client.  Despite this, it is possible for
the administrator to make catastrophic mistakes, e.g.\ rejecting mail for
all users --- the administrator must remember that \textit{With great power
comes great responsibility}.

Postfix uses simple lookup tables as the deciding factor when evaluating
some restrictions, e.g.\ in the example restriction below\newline
\tab{}\texttt{check\_client\_access~cidr:/etc/postfix/client\_access}
\newline The line above can be broken down as follows:

\begin{description}

    \item [check\_client\_access] The name of the restriction to evaluate.

    \item [cidr] The type of the lookup table.

    \item [/etc/postfix/client\_access] The file containing the lookup
        table.

\end{description}

The restriction \texttt{check\_client\_access} checks whether the \IP{}
address of the connected client is found in the file and returned the
associated action if found; the method of searching the file is dependant
on the type of the file (\texttt{cidr} in the example) --- see
\cite{postfix-lookup-tables} for more details.  Other restrictions
determine their result by consulting external sources, e.g.\
\texttt{reject\_rbl\_client rbl.example.com} checks whether the \IP{}
address of the client is present in the \RBL{} \texttt{rbl.example.com},
rejecting the command if the client is listed.
This description is necessarily brief, for further details
see~\cite{smtpd_access_readme, smtpd_per_user_control, policy-servers}.


\subsubsection{Policy servers}

\label{policy servers}

A \textit{policy server\/}~\cite{policy-servers} is an external program
consulted by Postfix to determine the fate of an \SMTP{} command.  The
policy server is given state information\footnote{Sample state information
is shown in table~\refwithpage{Example attributes sent to policy servers}}
and returns a verdict as described in \sectionref{Mixing and matching
Postfix restrictions}.  The policy server can perform more complex checks
than those provided by Postfix: a trivial example is restricting mail from
addresses associated with the payroll system to sending mail on the third
Tuesday after pay day only, to help prevent problems from spam or (worse)
phishing mails with faked sender addresses.\footnote{A
phishing mail might claim that the payroll system had a disastrous disk
failure; until the server is replaced and restored all salary payments will
have to be processed manually, so please reply to this mail with your name,
address and back account details.}

Some widely deployed policy servers:

\begin{itemize}

    \item Checking \SPF{} records~\cite{openspf, Wikipedia-spf}.
        \SPF{}\label{spf introduction} records specify which mail servers
        are allowed to send mail claiming to be from a particular domain.
        The intention is to reduce spam from faked sender addresses,
        backscatter~\cite{postfix-backscatter} and
        joe~jobs~\cite{Wikipedia-joe-job}; however there has been a lot of
        resistance to the proposal because it breaks or vastly complicates
        some features of \SMTP{}, e.g.\ forwarding mail from one company or
        university to another when a user changes jobs.

    \item Greylisting~\cite{greylisting} is a technique that temporarily
        rejects mail when the triple of (sender address, recipient address,
        remote \IP{} address) is unknown; on second and subsequent delivery
        attempts from that triple the mail will be accepted.  The
        assumption is that maintaining a list of failed addresses and
        retrying after a temporary failure is uneconomical for a spam
        sender, but that a legitimate mail server must retry.  Sadly spam
        senders are using increasingly complex and well written programs to
        distribute spam, frequently using an \ISP{} provided \SMTP{} server
        from a compromised machine on the \ISP{}'s network.  Greylisting
        will slowly become less useful, but it does block a large
        percentage of spam mail at the moment; the most effective
        restrictions over the \numberOFlogFILES{} log files used in testing
        the parser are shown in table~\refwithpage{Summary of rejections}.
        Greylisting is obviously worth using, at least at the moment,
        particularly when you factor in Greylisting's position as the final
        restriction which a mail must overcome:\footnote{Greylisting is the
        final restriction a mail must overcome in the configuration used on
        the mail server the log files were obtained from; an administrator
        is free to use Greylisting at whichever position in the restriction
        list they feel is most appropriate for their mail system.}
        Greylisting only takes effect for mails which have passed every
        other restriction.

        \begin{table}[ht]
            \caption{Summary of rejections}\label{Summary of rejections}
            \input{build/restriction-table-include.tex}
        \end{table}

    \item Using a scoring system such as
        Policyd-weight~\cite{policyd-weight} where tests accumulate points
        against the sending system --- if the eventual score is higher than
        a threshold the mail is rejected.  Postfix's restrictions are
        binary, and the administrator must manually whitelist clients if
        they are to bypass a restriction; using a threshold which requires
        hitting several restrictions frees the administrator from
        whitelisting clients which fall foul of one restriction only.

    \item Rate limiting on a per-sender, per-client or per-recipient basis
        as performed by Policyd~\cite{policyd}.

\end{itemize}

Example attributes taken from~\cite{policy-servers}:

\begin{table}[ht]

    \caption{Example attributes sent to policy servers}\label{Example
    attributes sent to policy servers}

    \begin{tabular}[]{ll}

        request                 & smtpd\_access\_policy     \\
        protocol\_state         & RCPT                      \\
        protocol\_name          & SMTP                      \\
        helo\_name              & some.domain.tld           \\
        queue\_id               & 8045F2AB23                \\
        sender                  & foo@bar.tld               \\
        recipient               & bar@foo.tld               \\
        recipient\_count        & 0                         \\
        client\_address         & 1.2.3.4                   \\
        client\_name            & another.domain.tld        \\
        reverse\_client\_name   & another.domain.tld        \\
        instance                & 123.456.7                 \\

    \end{tabular}
\end{table}



\subsection{Assumptions}

The algorithm described and the program implementing it make a small number
of (hopefully safe and reasonable) assumptions:

\begin{itemize}

    \item The log files are whole and complete: nothing has been removed,
        either deliberately or accidentally (e.g.\ log rotation gone awry,
        file system filling up, logging system unable to cope with the
        volume of log messages).  On a well run system it is extremely
        unlikely that any of these problems will arise, though it is of
        course possible, particularly when suffering a deluge of spam or a
        mail loop.

    \item Postfix logs sufficient information to make it possible to
        accurately reconstruct the actions it has taken.  There are a
        number of heuristics used when parsing; see
        \sectionref{identifying-bounce-notifications},
        \sectionref{aborted-delivery-attempts} and \sectionref{pickup
        logging after cleanup} for details.

    \item The Postfix queue has not been tampered with, causing unexplained
        appearance or disappearance of mail.  This may happen if the
        administrator deletes mail from the queue without using 
        \daemon{postsuper}, or if there is filesystem corruption.

\end{itemize}

In some ways this task is similar to reverse engineering or replicating a
black box program based solely on its inputs and outputs.  Although the
source code is available,\footnote{Reading and understanding the source
code would require a significant investment of time; the source code for
Postfix 2.5.1 is 17MB.} there are advantages to treating Postfix as a black
box while developing the parser:

\begin{itemize}

    \item The parser is developed using real world log files rather than
        the idealised log files someone would naturally envisage reading
        the source code.

    \item The source code cannot accurately communicate the variety of
        orderings in which log lines are written to the log file, as
        process scheduling independently interferes with logging and other
        processing.

    \item The parser acts as a second source of information, with the
        information gathered from empirical evidence.  An interesting
        project would be to compare the empirical knowledge inherent in the
        parsing algorithm with the documentation and source code of
        Postfix.

\end{itemize}


\subsection{Parser design}

\label{parser design}

It should be clear from the earlier Postfix background (\sectionref{postfix
background}) that log files produced by Postfix are not fixed; they vary
widely from host to host, depending on the set of restrictions chosen by
the administrator.  With this in mind, one of the parser's design aims was
to make adding new rules as easy as possible, to enable administrators to
properly parse their log files.  To enable this the parser is divided into
three parts:

\begin{description}

    \item [rules] Rules match individual log lines and determine which
        actions will be executed.  Rules provide an easily extensible
        method of associating log lines with actions, and are described in
        detail in \sectionref{rules}.

    \item [actions] Actions are invoked to deal with a log line once it has
        been identified by the rules: actions modify data structures,
        handle complications, and cause data to be saved to the database.
        The actions perform the work of reconstructing the journey a mail
        takes through Postfix.  Full details of actions can be found in
        \sectionref{actions-in-detail} and \sectionref{adding new actions}.

    \item [framework] The framework is responsible for loading rules,
        managing data structures, finding the rule which matches each log
        line, invoking the correct action, etc.  The framework provides the
        structure which actions and rules plug into.  The framework is
        described in detail in \sectionref{framework}.  XXX EXTEND
        FRAMEWORK SECTION IF NECESSARY\@.

\end{description}

It may help to think of the rules and actions as components which plug into
the framework.  

\label{why separate rules, actions and framework?}

XXX MERGE THE NEXT TWO PARAGRAPHS\@.

Decoupling the parsing rules from the associated actions and framework
allows new rules to be written and tested without requiring modifications
to the parser source code (significantly lowering the barrier to entry for
new or casual users who need to parse new log lines), and greatly
simplifies framework, actions and rules.  Decoupling also creates a clear
separation of functionality: rules handle low level details of identifying
log lines and extracting data from a log line; actions handle
the higher level details of following the path a mail takes through
Postfix, assembling the required data before storing it, dealing with
complications arising, etc; the framework provides services to actions and
stores data.

Decoupling the actions from the framework simplifies both framework and
actions: the framework provides services to the actions, and doesn't need
to deal with the complications which arise, or the task of reconstructing a
mail's journey through Postfix; actions benefit from having services
provided by the framework, freeing them to concentrate on the task of
accurately reconstructing each mail's journey through Postfix and dealing
with the complications described in \sectionref{additional complications}.

XXX CLARIFY THE SENTENCE STARTING WITH "Although this may seem" IN THE NEXT
PARAGRAPH\@.

Separating the rules from the actions and framework makes it possible to
parse new log lines without modifying the core parsing algorithm.  Although
this may seem like a trivial point, is it substantially more difficult to
understand a program's entire parsing algorithm, identify the correct
location to change and make the appropriate changes, versus adding a new
rule with the action to invoke, a \regex{} to match the log lines, and the
specification of the data to extract.  Bear in mind that the changes must
be made without adversely affecting existing parsing, particularly as there
may be edge cases which are not immediately obvious.\footnote{See
\sectionref{yet-more-aborted-delivery-attempts} for a complication which
occurs only four times in \numberOFlogFILES{} log files tested.}  Requiring
changes to the parser's code also complicates upgrades, as the changes
must be preserved during the upgrade, and may clash with changes made by
the developer.\footnote{See \sectionref{adding new actions} for how to add
new actions}  \parsername{} allows the user to add new rules to the
database without changing the parsing algorithm, unless the new log lines
to be parsed require functionality not already provided by the algorithm.
If the new log lines do require new functionality, new actions can be added
to the parser without modifying existing actions or other parts of the
algorithm; only in the rare case that the new actions require support from
other sections of the code will more extensive changes be required.

XXX CHANGE THIS; THE PARSER DESIGN IS NOVEL, DON'T STRESS THE SIMILARITY
TOO MUCH\@.

There is some similarity between the parser's design and William Wood's
\ATN{}~\cite{atns, nlpip}, a tool used in Computational Linguistics for
creating grammars to parse or generate sentences.  The resemblance between
\ATN{} and the parser is accidental, but it is interesting how two
apparently different approaches share an underlying separation of concerns;
this appears to be a natural division of responsibility and functionality.

% Do Not Reformat!

\begin{tabular}[]{lll}
    \textit{\ATN{}\/}   & \textit{Parser\/} & \textit{Similarity\/}     \\
    Networks            & Algorithm         & Determines the sequence 
                                              of transitions            \\
                        &                   & or actions which 
                                              constitutes a valid       \\
                        &                   & input.                    \\
    Transitions         & Actions           & Save data and impose
                                              conditions the            \\
                        &                   & input must meet to be
                                              considered valid.         \\
    Abbreviations       & Rules             & Responsible for 
                                              classifying input.        \\
\end{tabular}

\subsection{Conventions used in the document}

The words \textit{connection\/} and \textit{mail\/} are often used
interchangeably in this document; in general the word used was chosen based
on the context it appears in.

\subsection{Other Postfix log parsers}

Ten other parsers have been reviewed in
appendix~\refwithpage{other-parsers} as part of the background research for
this project.  None of the reviewed parsers perform the type of advanced
parsing and log correlation described here; all are intended to perform a
specific parsing and reporting task, rather than be a generic parser,
extracting data and leaving generation of reports from the data to other
programs.  Some parsers save the data extracted to a data store, but the
many discard all data once they have finished running, making historical
analysis impossible.  The other parsers reviewed all produce a report of
varying complexity and detail, whereas the program described here doesn't
attempt to produce a report at all; that responsibility is deferred to a
separate program, to be developed later.  The parsing algorithm and program
described here are designed to enable much more detailed log analysis by
providing a stable platform for subsequent programs to develop upon.

XXX EXTEND THIS\@.


\subsection{Previous research in this area}

\label{prior art}

There only appears to be one prior paper published about parsing Postfix
log files: \textit{Log Mail Analyzer: Architecture and Practical
Utilizations\/}~\cite{log-mail-analyser}.  The aim of \LMA{} is quite
different from this parser: it attempts to present correlated log data in a
form suitable for a systems administrator to search using the myriad of
standard Unix text processing utilities already available, producing both a
\CSV{} file and either a MySQL or Berkeley DB database.  The decision to
support both \CSV{} and Berkeley DB appears to have been a limiting factor:
\CSV{} is a very simple format where a record is stored in a single line,
with fields separated by a comma or other punctuation symbol.  Problems
with \CSV{} files include the difficulty in escaping separators, providing
multiple values for a field (e.g.\ multiple recipients) and adding new
fields.  There is no facility to include documentation of the fields,
separator, etc., unlike \SQL{} databases where every database includes a
schema naming the fields and the type of data they store (integer, text,
timestamp, etc.).

Berkeley DB only supports storing simple \textbf{(key, value)} pairs,
unlike \SQL{} databases which store arbitrary tuples; in \LMA{}'s main
table the key is an integer referred to by secondary tables, and the value
used is a \CSV{} line containing all of the data for that row.  The
secondary by-sender, by-recipient, by-date and by-\IP{} tables use the
sender/recipient/date/\IP{} as the key, and a \CSV{} list of integers
referring to the main table is the value.  In effect this re-implements
\SQL{} foreign keys, but without the functionality offered by even the most
basic of \SQL{} databases (joins, ordering, searches, etc.).  It also
requires custom code to search on some combination of the above, though the
authors of \LMA{} did provide some queries: IP-STORY, FROM-STORY,
DAILY-EMAIL and DAILY-REJECT\@.  Berkeley DB appears to be the least useful
of the three output formats: it doesn't provide the functionality of a
basic \SQL{} database, not can it be used with standard Unix text
processing tools.

The data stored is limited to time and date, hostname and \IP{} address of
client, \IP{} of server, sender and recipient addresses, \SMTP{} code, and
size (for accepted mails only).  Handling of multiple recipients, \SMTP{}
codes or remote servers\footnote{A single mail may be sent to multiple
remote servers if it was addressed to recipients in different domains, or
Postfix needs to try multiple servers for one or more recipients.} is not
explained in the paper; experimental observation shows that multiple
records are added when there are multiple recipients (sadly the records are
not associated or linked in any way), and presumably the same approach is
taken when there are multiple destination servers.

The schema used with the MySQL database is undocumented, but at least it's
possible to discover the schema with an \SQL{} database, unlike with
Berkeley DB\@; all \SQL{} databases embed the schema into the database and
provide commands for displaying it.  Berkeley DB does not embed a schema,
as there is neither requirement nor benefit; it only provides \textbf{(key,
value)} pairs, so any additional structuring of the data is imposed by the
application, thus it behoves the application to document this structure.

Parsing in \LMA{} requires major changes to the code to parse new log lines
or extract additional data.  It does not appear to deal with any of the
complications discussed in \sectionref{complications} (initial
complications) and \sectionref{additional complications} (additional
complications).  It does not differentiate between different types of
rejections, so it is not suitable for the purposes of this project; the
data about which restriction caused the rejection is discarded, whereas an
important goal of this project is to retain that data to aid optimisation
and evaluation of Postfix restrictions.  \LMA{} fails to parse Postfix log
files generated on Solaris hosts because the fields added automatically
differ from those added on Linux hosts; log files from Solaris hosts (and
possibly other operating systems) thus require preprocessing before parsing
by \LMA{}.  Once the preprocessing has been performed on the
\numberOFlogFILES{} test log files \LMA{} parses the log files without
complaint, though it does produce 32 entries in its output file for every
rejection in the input log file; it also missed some 40\% of delivered
mail.  Once these glaring deficiencies were discovered the author did not
spend more time checking the results.

\LMA{} does provide some simple reports: IP-STORY, FROM-STORY, DAILY-EMAIL
and DAILY-REJECT\@.  IP-STORY shows the \CSV{} lines with the specified
client \IP{} address; FROM-STORY shows the \CSV{} lines with the specified
sender address; DAILY-EMAIL shows the \CSV{} lines for the specified day.
DAILY-REJECT initially failed with an error message from the Perl
interpreter; after making some corrections to the code it worked, and
produced the \CSV{} lines for the specified day where the \SMTP{} code
signifies a rejection.  All of these reports are trivially simple to
produce from the \CSV{} file using the standard Unix tool \texttt{awk},
e.g.\ the most complicated, DAILY-REJECT, is merely:

\begin{verbatim}
awk -F\| '$1 ~ /2007-01-26/ && $7 ~ /^[450]|deferred/
    { print $0; print " "; }' lma_output.txt
\end{verbatim}

Notes about the command above:

\begin{enumerate}

    \item It outputs an unnecessary line containing only a single space
        after each matching line, for compatibility with the output of
        DAILY-REJECT\@.

    \item Some records have an \SMTP{} code of \textit{0}, though this is
        not a valid \SMTP{} code; presumably this is due to \LMA{} not
        parsing a line properly.  \LMA{} considers this code to be a
        rejection, so the command above must, too.

    \item In a tiny proportion of records the \SMTP{} code is
        \textit{deferred\/} (again, not a valid code, and most likely due
        to mis-parsing by \LMA{}); it is treated as a rejection by \LMA{},
        so the equivalent \texttt{awk} command must do so, too.

\end{enumerate}

There are a number of differences in the output from DAILY-REJECT and the
\texttt{awk} command:

\begin{itemize}

    \item \LMA{} does not escape or replace the separators used in the
        \CSV{} records when they are present in sender or recipient
        addresses, leading to some mangled records; these records are not
        present in the output from \texttt{awk}.

    \item Where there are multiple records in the \CSV{} file, DAILY-REJECT
        produces only one output line, whereas the \texttt{awk} command
        produces all matching lines.

\end{itemize}

In summary, \LMA{} appears to be a proof of concept, written to demonstrate
the point of their paper (that having this information in an accessible
fashion is useful to systems administrators), rather than a program
designed to be extensible and useful in a production environment.

% Literature review notes:
%
% Hard-coded parsing, requiring code changes to add more.  Attempts to
% correlate log lines, saves data to database for data mining purposes.
% Hard to extend/expand/understand.  Appears to only save: date and hour,
% \DNS{} name and \IP{} address host, mail server \IP{} address, sender,
% receiver and e-mail status (sent, rejected).  Undocumented schema.
% Design decision to use \CSV{} as an intermediate format between the log
% file and the database seems to have been restrictive.  Appears to require
% a queueid but majority of log entries (e.g. rejections) lack a queueid.
% Supports whitelisting \IP{} addresses when parsing logs, but whitelisting
% when generating reports/data mining would be preferable.  Supporting
% Berkeley DB is probably limiting the software - an example is the
% difficulty in searching a pipe-delimited string, so they have
% re-implemented foreign keys with tables keyed by ip address etc. pointing
% at the main table - this also won't scale well.  There doesn't appear to
% be any attempt to deal with the complications I've encountered: their
% parsing isn't detailed enough to encounter them.  It doesn't run
% properly; doesn't create any output; throws up errors.

\subsection{Conclusion}

This section has provided background information on several topics relevant
to the remainder of the document.  It started with the motivation behind
the project, continuing with explanations of:

\begin{itemize}

    \item Using a database as an \API{}.

    \item \SMTP{}.

    \item Postfix restrictions and policy servers.

    \item Assumptions made when designing and developing the parser.

    \item A description of the parser's novel design.

    \item Conventions used in this document.

    \item An in-depth comparison of this parser with \LMA{}, the parser
        described in previously published research.

    \item A review of the literature previously published in this area.

\end{itemize}

