\chapter{Background}

\label{background}

This section provides background information helpful in understanding the
remainder of this thesis.  It begins with a discussion of the motivation
underlying the project, followed by a brief introduction to \acronym{SMTP},
and finishes with a longer introduction to Postfix, concentrating on
anti-spam restrictions and policy servers.

\section{Motivation}

\label{motivation}

This work is part of a larger project to optimise a mail server's
Postfix-based anti-spam restrictions, generate statistics and graphs, and
provide a platform on which new restrictions can be evaluated to determine
if they are beneficial in the fight against spam.  The program written for
this project, \parsername{}, parses Postfix log files and populates a
database with the data gleaned from those log files, providing a consistent
and simple view of the log files that future tools can utilise.  The
gathered data can be used to optimise current anti-spam measures, to
provide a baseline to test new anti-spam measures against, or to produce
statistics showing how effective those measures are.

A short example of the optimisation possible using data from the database
is determining which Postfix restrictions reject the highest number of
mails:

\begin{verbatim}
SELECT name, description, restriction_name, hits_total
    FROM rules
    WHERE action = 'DELIVERY_REJECTED'
    ORDER BY hits_total DESC;
\end{verbatim}

If the database supports sub-selects (where the results of one query are
used as a parameter in another), percentages can be obtained for the top
ten restrictions (sample output shown in \tableref{Sample output from SQL
query}):

\input{build/include-sample-sql-query}

\begin{table}[ht]
    \caption{Sample output from SQL query}
    \empty{}\label{Sample output from SQL query}
    \begin{tabular}[]{lrr}
        \tabletopline{}%
        Restriction & Number of  & Percentage of    \\
                    & rejections & total rejections \\
        \tablemiddleline{}%
        \input{build/include-sample-sql-output}
        \tablebottomline{}%
    \end{tabular}
\end{table}


Another example is determining which restrictions are not effective: this
sample query shows which restrictions had fewer than 100 rejections in the
last log file parsed, and the percentage of total rejections each of those
restrictions represents.

\begin{verbatim}
SELECT name, description, restriction_name, hits,
        (hits * 100.0 /
            (SELECT SUM(hits)
                FROM rules
                WHERE action = 'DELIVERY_REJECTED'
            )
        ) || '%' AS percentage
    FROM rules
    WHERE action = 'DELIVERY_REJECTED'
        AND hits < 100
    ORDER BY hits ASC;
\end{verbatim}

The database queries above yield summary statistics about the efficiency of
anti-spam techniques.  These statistics are far less feasible to assess
directly from log files without prior pre-processing into a database in the
fashion proposed, implemented, and tested herein.

\section{Simple Mail Transfer Protocol (SMTP)}

\label{SMTP background}

\SMTPglossaryDescription{} Despite the simplicity of the protocol many
virus and spam sending programs fail to implement it properly, so requiring
strict adherence to the protocol specification is beneficial in protecting
against spam and viruses.\footnote{\label{footnote:rfc760}Originally all
mail servers adhered to the principle of \textit{Be liberal in what you
accept, and conservative in what you send\/} from
\acronym{RFC}~760~\cite{rfc760}, but unfortunately that principle was
written in a friendlier time.  Given the deluge of spam that mail servers
are subjected to daily, a more appropriate maxim could be: \textit{Require
strict adherence to relevant \acronyms{RFC}; implement the strongest
restrictions you can; relax the restrictions and adherence only when
legitimate mail is impeded.\/}  It is neither as friendly nor as catchy,
but it more accurately reflects the current situation.} A typical
\acronym{SMTP} conversation resembles the following (the lines starting
with a three digit number are sent by the server, all other lines are sent
by the client):

\begin{verbatim}
220 smtp.example.com ESMTP
HELO client.example.com
250 smtp.example.com
MAIL FROM: <alice@example.com>
250 2.1.0 Ok
RCPT TO: <bob@example.com>
250 2.1.5 Ok
DATA
354 End data with <CR><LF>.<CR><LF>
Message headers and body sent here.
.
250 2.0.0 Ok: queued as D7AFA38BA
QUIT
221 2.0.0 Bye
\end{verbatim}

An example deviation from the protocol:

\begin{verbatim}
220 smtp.example.com ESMTP
HELO client.example.com
250 smtp.example.com
MAIL FROM: Alice in Distribution alice@example.com
501 5.1.7 Bad sender address syntax
RCPT TO: Bob in Sales/Marketing bob@example.com
503 5.5.1 Error: need MAIL command
DATA
503 5.5.1 Error: need RCPT command
Message headers and body sent here.
.
502 5.5.2 Error: command not recognized
QUIT
221 2.0.0 Bye
\end{verbatim}

This example client is so poorly written that not only does it present the
sender and recipient addresses improperly, it ignores the error messages
returned by the server and carries on regardless.  Many spam and virus
sending programs have serious deficiencies; unfortunately other programs,
particularly newer programs, were written by competent programmers or send
mail using well written programs (e.g.\ Postfix or Sendmail on Unix hosts,
Microsoft Outlook on Windows hosts).  Traditionally a mail server would
have done its best to deal with deficient clients, with the intention of
accepting as much mail sent to its users as possible, e.g.\ by ignoring the
absence of a HELO command, or by accepting sender or recipient addresses
that were not enclosed in \texttt{<>}.

\section{Postfix}

\label{postfix background}

Postfix is a \glsfirst{MTA} with the following design aims (in order of
importance): security, flexibility of configuration, scalability, and high
performance.  It features extensive optional anti-spam restrictions,
allowing an administrator to employ those restrictions which they judge
suitable for their site's needs, rather than a fixed set chosen by
Postfix's author.  These restrictions can be selectively applied, combined,
and bypassed on a per-client, per-recipient, or per-sender basis, allowing
different levels of stricture and permissiveness.  Administrators can
supply their own rejection messages to make it clear to senders exactly why
their mail was rejected.  Policy servers (\sectionref{policy servers})
provide a simple way to write new restrictions without having to modify
Postfix's source code.  Unfortunately this flexibility has a cost:
complexity in the log files generated.  Although it is relatively simple in
most cases to use standard Unix text processing utilities to determine the
fate of an individual mail, in some cases it can be quite difficult.  For
most mails the journey is simple and brief, but the remainder can have
quite complex journeys (see \sectionref{complications} for details).

Postfix's design follows the Unix philosophy of \textit{``Write programs
that do one thing and do it well''\/}~\cite{unix-philosophy}, and it is
separated into multiple components that each perform one of the tasks
required of an \acronym{MTA}\@: receive mail, send mail, local delivery of
mail, etc.;\ full details can be found in~\cite{postfix-overview}.
Postfix's design is strongly influenced by security concerns: those
components that interact with other hosts are not
privileged,\footnote{Privilege means the power to perform actions that are
limited to the administrator, and not available to ordinary users.} so bugs
in those components will not give an attacker extra privileges; those
components that are privileged do not interact with other hosts, making it
much more difficult for an attacker to exploit any bugs that may exist in
those components.

\subsection{Mixing and Matching Postfix Restrictions}

\label{Mixing and matching Postfix restrictions}

Postfix restrictions are documented fully in~\cite{smtpd_access_readme,
smtpd_per_user_control, policy-servers}; the following is only a brief
introduction.

Postfix uses one restriction list (containing zero or more restrictions)
for each stage of the \acronym{SMTP} conversation: client connection, HELO
command, MAIL FROM command, RCPT TO commands, DATA command, and end of
data.  The appropriate restriction list is evaluated for each stage, though
by default the restriction lists for client connection, HELO, and MAIL FROM
commands will not be evaluated until the first RCPT TO command is received,
because some clients do not deal properly with rejections before this
stage; a benefit of this delay is that Postfix has more information
available when logging rejections.

Postfix uses simple lookup tables to make decisions when evaluating some
restrictions, e.g.\newline{}
\tab{}\texttt{check\_client\_access~cidr:/etc/postfix/client\_access}

\begin{eqlist}

    \item [check\_client\_access] The name of the restriction to evaluate.

    \item [cidr] The type of the lookup table.

    \item [/etc/postfix/client\_access] The file containing the lookup
        table.

\end{eqlist}

The restriction \texttt{check\_client\_access} checks if the \acronym{IP}
address of the connected client is found in the specified table and returns
the associated action if found; the method of searching the file is
dependant on the type of the file (\texttt{cidr} in the example) --- see
\cite{postfix-lookup-tables} for more details.  Other restrictions
determine their result by consulting external sources, e.g.\newline{}
\tab{}\texttt{reject\_rbl\_client dnsbl.example.com}\newline{} checks the
\acronym{DNSBL} \texttt{dnsbl.example.com} and rejects the command if the
client's \acronym{IP} address is listed.

Each restriction is evaluated to produce a result of \textit{reject},
\textit{permit}, \textit{dunno}, or the name of another restriction to be
evaluated; other less commonly used results are possible as described
in~\cite{smtpd_access_readme,smtpd_per_user_control,policy-servers}.  The
meaning of \textit{permit\/} and \textit{reject\/} is fairly obvious;
\textit{dunno\/} means to stop evaluating the current restriction and
continue processing the remainder of the restriction list, allowing
exceptions to more general rules.  E.g.\ the general rule is that machines
on the local network must authenticate before sending mail, except for the
web server, because the legacy applications running on it lack
authentication support; this configuration example shows how:

\begin{verbatim}
  main.cf:
  smtpd_client_restrictions =
    ...
    permit_sasl_authenticated,
    check_client_access /etc/postfix/allow_webserver.cidr,
    ...

  /etc/postfix/allow_webserver.cidr:
    192.0.2.80/32   dunno
    192.0.2.0/24    reject "Please authenticate to send mail"
\end{verbatim}

That example also shows how easy it is to supply a custom rejection
message.  When the result is the name of another restriction Postfix will
evaluate that restriction, allowing restrictions to be chosen based on the
client \acronym{IP} address, client hostname, HELO hostname, sender
address, or recipient address.  E.g.\ the administrator may require that
all clients on the local network have valid DNS entries, to prevent people
sending mail from unknown machines; one example of how to achieve this is:

\begin{verbatim}
  main.cf:
  smtpd_client_restrictions =
    ...
    check_client_access /etc/postfix/require_dns_entries.cidr,
    ...

  /etc/postfix/require_dns_entries.cidr:
    192.0.2.0/24    reject_unknown_client_hostname
\end{verbatim}

The administrator can define new restrictions as a list of existing
restrictions, allowing arbitrarily long and complex user-defined sequences
of lookups, restrictions, and exceptions.  Postfix tries to protect the
administrator from misconfiguration in as far as is reasonable, e.g.\ the
restriction \texttt{check\_helo\_mx\_access} cannot cause a mail to be
accepted, because the parameter it checks (the hostname given in the HELO
command) is under the control of the remote client.  Despite this, it is
possible for the administrator to make catastrophic mistakes, e.g.\
rejecting all mail --- the administrator must be cognisant of the effects
their configuration changes will have.  This is similar to one of UNIX's
design philosophies: \textit{``UNIX was not designed to stop its users from
doing stupid things, as that would also stop them from doing clever
things''\/}~\cite{unix-philosophy}.

\subsection{Policy Servers}

\label{policy servers}

A policy server~\cite{policy-servers} is an external program consulted by
Postfix to determine the fate of an \acronym{SMTP} command.  The policy
server is given state information by Postfix (sample state information is
shown in \tableref{Example attributes sent to policy servers}) and returns
a result (reject, permit, dunno, a rejection name) as described in
\sectionref{Mixing and matching Postfix restrictions}.  A policy server can
perform more complex checks than those provided by Postfix: a trivial
example is allowing addresses associated with the payroll system to send
mail on the third Tuesday after pay day only, to help prevent problems from
phishing mails using faked sender addresses.  E.g.\ a phishing mail might
claim that the payroll system had a disastrous disk failure, and until the
server is replaced all salary payments will have to be processed manually,
so please reply to this mail with your name, address, and bank account
details; the criminal can then use any details sent to him to help with
identity theft.

\begin{table}[ht]

    \caption{Example attributes sent to policy servers, taken
    from~\cite{policy-servers}}
    \empty{}\label{Example attributes sent to policy servers}

    \centering{}

    \begin{tabular}[]{ll}

        request                 & smtpd\_access\_policy     \\
        protocol\_state         & RCPT                      \\
        protocol\_name          & SMTP                      \\
        helo\_name              & some.domain.tld           \\
        queue\_id               & 8045F2AB23                \\
        sender                  & foo@bar.tld               \\
        recipient               & bar@foo.tld               \\
        recipient\_count        & 0                         \\
        client\_address         & 1.2.3.4                   \\
        client\_name            & another.domain.tld        \\
        reverse\_client\_name   & another.domain.tld        \\
        instance                & 123.456.7                 \\

    \end{tabular}

\end{table}

Some widely deployed policy servers:

\begin{itemize}

    \item \acronym{SPF}~\cite{openspf}.  \acronym{SPF}\label{spf
        introduction} records specify which mail servers are allowed to
        send mail using sender addresses from a particular domain.  The
        intention is to reduce spam from faked sender addresses,
        backscatter~\cite{postfix-backscatter}, and
        Joe~jobs~\glsadd{Joe-job}.  There has been considerable resistance
        to \acronym{SPF} because it breaks or vastly complicates some
        commonly-used features of \acronym{SMTP}, e.g.\ forwarding mail
        from one domain to another when a user moves.

    \item Greylisting~\cite{greylisting} is a technique that temporarily
        rejects a delivery attempt when the triple of \newline{}
        \tab{}\texttt{(sender address, recipient address, remote
        \acronym{IP} address)}\newline{} has not been seen before; on
        second and subsequent delivery attempts from that triple the mail
        will be accepted.  This blocks spam from some senders because
        maintaining a list of failed addresses and retrying after a
        temporary failure is uneconomical for a spam sender, but a
        legitimate mail server must retry deliveries that failed
        temporarily.  Sadly spam senders are using increasingly complex and
        well written programs to distribute spam, frequently using an ISP
        provided \acronym{SMTP} server from a compromised machine on the
        ISP's network.  Greylisting will slowly become less effective as
        spam senders adapt, but it does block a large percentage of spam
        mail at the moment; the most effective restrictions from the
        \numberOFlogFILES{} log files used when generating the results in
        \sectionref{Results} are shown in \tableref{Summary of rejections}.
        The table shows that greylisting is worth using at the moment,
        particularly when you take into account its position as the final
        restriction that a mail must overcome in the configuration used on
        the mail server that generated the log files: on that server
        greylisting only takes effect for mails that have passed all other
        restrictions.  Some problems may be encountered when using
        greylisting: some servers fail to retry after a temporary failure,
        or legitimate mail may be delayed, particularly when coming from a
        pool of servers.

        \begin{table}[ht]
            \caption{Summary of rejections}\label{Summary of rejections}
            \input{build/include-restriction-summary-table}
        \end{table}

    \item Scoring systems such as Policyd-weight~\cite{policyd-weight}
        perform tests on features of the delivery attempt (e.g.\
        \acronym{IP} address, sender address), incrementing or decrementing
        a score based on the results; if the eventual score is higher than
        a threshold the mail is rejected.  The administrator must manually
        whitelist clients if they are to bypass a Postfix restriction,
        whereas using a threshold that requires a delivery attempt to hit
        several scored restrictions will allow mail from clients that would
        fall foul of one binary restriction only.

\end{itemize}

\section{Summary}

This section has provided background information useful in understanding
this thesis, starting with the motivation behind the project, continuing
with an introduction to \acronym{SMTP}, and finishing with Postfix, its
anti-spam restrictions, and its support for policy servers.
