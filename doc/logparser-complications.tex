\section{Complications Encountered While Writing The \parsernamelong{}}

\label{complications}

It was initially expected that parsing Postfix log files would be a
relatively simple task, requiring a couple of months of work.  The author
had found Postfix log files useful when investigating problems reported by
users, and an examination of several log files gave the impression that
they would be straightforward to parse, process, and understand.  The large
variation in log lines was not apparent, because most log lines are
recognised by a small set of rules, as shown in \figureref{rule hits
graph}.  Most of the myriad complications and difficulties documented in
this section were discovered during \parsernames{} development, but the
first three complications were identified during the planning and design
phase of this project, influencing the architecture's design.

Each of these complications caused \parsername{} to operate incorrectly,
generate warning messages, or leave mails in the state table.  The
complications are listed in the order in which they were overcome during
development of \parsername{}, with the first complication occurring several
orders of magnitude more frequently than the last.  When deciding which
problem to tackle next, the problem causing the highest number of warning
messages or mails incorrectly remaining in the state tables was always
chosen, because that approach yielded the biggest improvement in the
parser, and made the remaining problems more apparent.  Several of the
complications were discovered because \parsername{} is very careful to
check the origin of each mail; if it does not know the origin of a mail,
that mail will be tagged as fake, and a warning issued if the mail's origin
has not been determined before trying to save it to the database.  Some of
the solutions to these complications require discarding log lines because
they are out of order, and, less frequently, discarding a data structure
containing data gathered about a connection or a mail, because of the
paucity of information contained in the data structure.

The fifteen complications documented in this section can be divided into
three broad classes:

\begin{description}

    \item [Recreating Postfix behaviour] Three complications needed to be
        solved for \parsername{} to accurately recreate Postfix's
        behaviour.

    \item [Lack of information in the log files] Seven complications are
        caused by lack of information in the log files.  In general, some
        extra logging by Postfix could remove these complications, and for
        one complication (\sectionref{identifying bounce notifications}),
        extra logging in later versions of Postfix does remove it; three of
        the complications would be removed if Postfix generated a log line
        whenever it discarded a mail.

    \item [The order that log lines are found in log files] Five
        complications are caused by out of order log lines.  Postfix is not
        to blame for these: they are caused by process scheduling,
        inter-process communication delays, and are more likely to occur
        when the mail server is heavily loaded.

\end{description}

\subsection{Queueid Vs Pid}

\label{queueid vs pid}

A delivery attempt lacks a queueid until the first recipient has been
accepted, so log lines must first be correlated by \daemon{smtpd}
\acronym{pid}, then transition to being correlated by their queueid.  This
is relatively minor, but does require:

\begin{itemize}

    \squeezeitems{}

    \item Two versions of several functions: \texttt{by\_pid} and
        \texttt{by\_queueid}.

    \item Two state tables to hold data structures.

    \item Each action needs to know whether it deals with accepted mails,
        and should find the mail in the \texttt{queueid} state table, or
        deals with delivery attempts, and thus will find the connection in
        the \texttt{connections} state table.  Some actions, e.g.\
        \action{DELIVERY\_REJECTED}, need to query both tables.  Support
        functions fall into two groups: one group deals with state tables,
        and so need to know which one to operate on; the functions in the
        other group are passed a specific mail or connection to work with,
        and do not need to know about state tables at all.

\end{itemize}

The \action{CLONE} action is responsible for copying an entry from the
\texttt{connections} state table to the \texttt{queueids} state table.

\subsection{Connection Reuse}

\label{connection reuse}

Multiple independent mails may be delivered across one connection: this
requires \parsername{} to clone the connection's data as soon as a mail is
accepted, so that subsequent mails will not overwrite each other's data.
This must be done every time a mail is accepted, because it is impossible
to tell in advance which connections will accept multiple mails.  Once a
mail has been accepted its log lines will not be correlated by
\acronym{pid} any more: its queueid will be used instead, as described in
\sectionref{queueid vs pid}.  If the original connection has any useful
data (e.g.\ rejections) it will be saved to the database when the client
disconnects.  One unsolved difficulty is distinguishing between different
groups of rejections, e.g.\ when dealing with the following sequence:

\begin{enumerate}

    \squeezeitems{}

    \item The client attempts to deliver a mail, but it is rejected.

    \item The client issues the RSET command to reset the \acronym{SMTP}
        session.

    \item The client attempts to deliver another mail, likewise rejected.

\end{enumerate}

Ideally, the sequence above would result in two separate entries in the
connections table, but only one will be created.



\subsection{Re-injected Mails}

\label{Re-injected mails}

\label{tracking re-injected mail}

Mails sent to local addresses are not always delivered directly to a
mailbox: sometimes they are sent to and accepted for a local address, but
need to be delivered to one or more remote addresses because of aliasing.
When this occurs, a child mail will be injected into the Postfix queue, but
without the explicit logging that mails injected by \daemon{smtpd} or
\daemon{postdrop} have.  Thus the source of the mail is not immediately
discernible from the log line in which the mail's queueid first appears:
from a strictly chronological reading of the log lines it usually appears
as if the child mail does not have an origin.  Subsequently, the parent
mail will log the creation of the child mail, e.g.\ parent mail
\texttt{3FF7C4317} creates child mail \texttt{56F5B43FD}\@:

\texttt{\tab{} 3FF7C4317: to=<username@example.com>, relay=local, \hfill{}
\newline{} \tab{} \tab{} \tab{} delay=0, status=sent (forwarded as
56F5B43FD)}

The sample log line above would be processed by the \action{TRACK} action,
which creates the child mail if it does not exist in the state tables,
links it to the parent mail, and checks that the child is not being tracked
for a second time.

Unfortunately, although all log lines from an individual process appear in
chronological order, the order in which log lines from different processes
are interleaved in a log file is subject to the vagaries of process
scheduling.  In addition, the first log line belonging to the child mail
(the example log line above belongs to the parent mail) is logged by either
\daemon{qmgr} or \daemon{cleanup}, so the order also depends on how soon
they process the new mail.

Because of the uncertain order that log lines can appear in, \parsername{}
cannot complain when it encounters a log line from either \daemon{qmgr} or
\daemon{cleanup} for a mail that does not exist in the state tables;
instead it must create the state table entry and flag the mail as coming
from an unknown origin.  Subsequently invoked actions will clear that flag
if the origin of the mail becomes clear.  The parser could omit checking
where mails originate from, but requiring an explicit source helps to
expose bugs in the parser; such checks helped to identify the complications
described in \sectionref{discarding cleanup log lines} and
\sectionref{pickup logging after cleanup}.

Process scheduling can have a still more confusing effect: the child mail
will often be created, delivered, and entirely finished with,
\textit{before\/} the parent mail logs its creation!  Thus, mails flagged
as coming from an unknown origin cannot be entered into the database when
their final log line is processed, and \parsername{} cannot warn the user;
instead they must be marked as ready for entry and subsequently entered
once their origin has been identified.  Unfortunately, it is not possible
to distinguish between child mails waiting to be tracked and other mails
whose origin is unknown, except for bounce notifications, as described in
\sectionref{identifying bounce notifications}.  Mails whose origin is
unknown can remain in the state tables indefinitely if their origin is not
determined at some point; they will cause a queueid clash if the queueid is
reused, and, most importantly, \parsernames{} understanding of such mails
is incorrect.  This problem has only been observed when a state table entry
is missing its initial log line, usually because it is in an earlier log
file; this specific instance is not a serious problem, because
\parsername{} cannot be expected to fully understand and reconstruct a mail
when some of that mail's log lines are missing.

Tracking re-injected mail requires \parsername{} to do the following in the
\action{COMMIT} action:

\begin{enumerate}

    \item If a mail is tagged with the unknown origin flag, it is assumed
        to be a child mail whose parent has not yet been identified.  The
        mail is tagged as ready to be entered in the database, but entry is
        postponed until the parent is identified.  The child mail will not
        have any subsequent log lines: only its parent will refer to it.

    \item If the mail is a child mail whose parent has been identified, it
        is entered in the database as usual, then removed from its parent's
        list of children.  If this child is the last mail on that list, and
        the parent has already been entered in the database, the parent
        will be removed from the state tables.

    \item If the mail is a parent, it is entered in the database as usual
        because there will be no further log lines for it.  There may be
        child mails waiting to be entered in the database; these are
        entered as usual, and removed from the state tables.  If the state
        tables contain incomplete child mails, the parent's removal from
        the state tables will be postponed until the last child has been
        entered.

\end{enumerate}

\subsection{Identifying Bounce Notifications}

\label{identifying bounce notifications}

Postfix 2.2.x (and presumably previous versions) does not generate a log
line when it generates a bounce notification; the log file will have log
lines for a mail whose origin is unknown.  Similarities exist to the
problem of identifying re-injected mails discussed in \sectionref{tracking
re-injected mail}, but unlike the solution described therein bounce
notifications do not eventually have a log line that identifies their
origin.  Heuristics must be used to identify bounce notifications:

\begin{enumerate}

    \item The sender address is \verb!<>!.\glsadd{<>}

    \item Neither \daemon{smtpd} nor \daemon{pickup} have logged any
        messages associated with the mail, indicating that it was generated
        internally by Postfix, rather than accepted via \acronym{SMTP} or
        submitted locally by \daemon{postdrop}.

    \item The message-id has a specific format: \newline{}
        \tab{} \texttt{YYYYMMDDhhmmss.queueid@server\_hostname} \newline{}
        e.g.\ \texttt{20070321125732.D168138A1@smtp.example.com}

    \item The queueid embedded in the message-id must be the same as the
        queueid of the mail: this is what distinguishes a new bounce
        notification from a bounce notification that is being re-injected
        as a result of aliasing.  For the latter, the message-id will be
        unchanged from the original bounce notification, and so even if it
        happens to be in the correct format, i.e.\ if it was generated by
        Postfix on this or another server, it will not equal the queueid of
        the mail.

\end{enumerate}

Once a mail has been identified as a bounce notification, the unknown
origin flag is cleared and the mail can be entered in the database.

A small chance exists that a mail will be incorrectly identified as a
bounce notification, because the heuristics used may be too broad.  For this to
occur the following conditions would have to be met:

\begin{enumerate}

    \squeezeitems{}

    \item The mail must have been generated internally by Postfix.

    \item The sender address must be \verb!<>!.\glsadd{<>}

    \item The message-id must have the correct format and contain the
        queueid of the mail.  Although a mail sent from elsewhere could
        easily have the correct message-id format, the chance that the
        queueid in the message-id would correspond with the queueid of the
        mail is extremely small.

\end{enumerate}

If a mail is misclassified as a bounce message it will almost certainly
have been generated internally by Postfix; arguably, misclassification of
this kind is a benefit rather than a drawback, because other mails
generated internally by Postfix will be handled correctly.  Postfix 2.3 and
subsequent versions log the creation of a bounce message, so this
complication does not arise in their log files.  The solution to this
complication will help with solving the complication in \sectionref{Bounce
notification mails delivered before their creation is logged}.

This check is performed during the \action{COMMIT} action.

\subsection{Aborted Delivery Attempts}

\label{aborted delivery attempts}

Some mail clients behave strangely during the \acronym{SMTP} dialogue: the
client aborts the first delivery attempt after the first recipient is
accepted, then makes a second delivery attempt for the same recipient that
it continues with until delivery is complete.  Microsoft Outlook is one
client that behaves in this fashion; other clients may act in a similar
way.  An example dialogue exhibiting this behaviour is presented below
(lines starting with a three digit number are sent by the server, the other
lines are sent by the client):

\begin{verbatim}
220 smtp.example.com ESMTP
EHLO client.example.com
250-smtp.example.com
250-PIPELINING
250-SIZE 15240000
250-ENHANCEDSTATUSCODES
250-8BITMIME
250 DSN
MAIL FROM: <sender@example.com>
250 2.1.0 Ok
RCPT TO: <recipient@example.net>
250 2.1.5 Ok
RSET
250 2.0.0 Ok
RSET
250 2.0.0 Ok
MAIL FROM: <sender@example.com>
250 2.1.0 Ok
RCPT TO: <recipient@example.net>
250 2.1.5 Ok
DATA
354 End data with <CR><LF>.<CR><LF>
The mail transfer is not shown.
250 2.0.0 Ok: queued as 880223FA69
QUIT
221 2.0.0 Bye
\end{verbatim}

Postfix does not log a message making the client's behaviour clear, so
heuristics are required to identify when a delivery attempt is aborted in
this way.  A list of all mails accepted during a connection is saved in the
connection's state table entry, and the accepted mails are examined when
the disconnection action is invoked.  Each accepted mail is checked for the
following:

\begin{itemize}

    \item Was the second result processed by the \action{CLONE} action?
        The first two \daemon{smtpd} log lines will be a connection log
        line and a mail acceptance log line.

    \item Is \daemon{smtpd} the only Postfix component that produced a log
        line for this mail?  Every mail that passes normally through
        Postfix will have a \daemon{cleanup} log line, and later a
        \daemon{qmgr} log line; lack of a \daemon{cleanup} log line is a
        sure sign the mail did not make it too far.

    \item Can the mail be found in the state tables?  If not it cannot be
        an aborted delivery attempt.

    \item If third and subsequent results exist, were those log lines
        processed by the \action{SAVE\_DATA} action?  Any log lines after
        the first two should be informational only.

\end{itemize}

If all the checks above are successful the mail is assumed to be an aborted
delivery attempt and is removed from the state tables.  There will be no
further log lines for such mails, so without identifying and discarding
them they accumulate in the state table and will cause clashes if the
queueid is reused.  Such mails cannot be entered in the database because
the only data they contain is the client hostname and IP address, but the
database schema requires many more fields --- see \sectionref{connections
table} and \sectionref{results table}.  These heuristics are quite
restrictive, and appear to have little scope for false positives; any false
positives would cause a warning when the next log line for such a mail is
parsed.  False negatives are less likely to be detected: there may be
queueid clashes (and warnings) if mails remain in the state tables after
they should have been removed; if the queueid is not reused, the only way
to detect false negatives is to examine the state tables after each parsing
run.

This check is performed in the \action{DISCONNECT} action; it requires
support in the \action{CLONE} action where the list of accepted mails is
maintained.


\subsection{Further Aborted Delivery Attempts}

Some mail clients disconnect abruptly if a second or subsequent recipient
is rejected; they may also disconnect after other errors, but such
disconnections are handled elsewhere in the parser, e.g.\
\sectionref{timeouts during data phase}.  Postfix does not log a message
saying the mail has been discarded, as should be expected by now.  The
checks to identify this happening are:

\begin{itemize}

    \item Is the mail missing its \daemon{cleanup} log line?  Every mail
        that passes normally through Postfix will have a \daemon{cleanup}
        log line; lack of one is a sure sign the mail did not make it too
        far.

    \item Were there three or more \daemon{smtpd} log lines for the mail?
        There should be a connection log line and a mail acceptance log
        line, followed by one or more delivery attempt rejected log lines.

\end{itemize}

If both checks are successful the mail is assumed to have been discarded by
Postfix when the client disconnected; \parsername{} will remove it from the
state tables.  There will be no further log lines for such mails, so if
\parsername{} does not deal with them immediately they accumulate in the
state table and will cause clashes if the queueid is reused.

These checks are made during the \action{DISCONNECT} action.

\subsection{Timeouts During DATA Phase}

\label{timeouts during data phase}

The DATA phase of the \acronym{SMTP} conversation is where the headers and
body of the mail are transferred.  Sometimes a timeout occurs or the
connection is lost during the DATA phase;\footnote{For the sake of brevity
\textit{timeout\/} will be used throughout this section, but everything
applies equally to lost connections.} when this occurs Postfix will discard
the mail and \parsername{} needs to discard the data associated with that
mail.  It seems more appropriate to save the mail's data to the database,
but if a timeout occurs no data will be available to save; the timeout is
recorded and saved with the connection instead.  To deal properly with
timeouts the \action{TIMEOUT} action does the following:

\begin{enumerate}

    \item Record the timeout and data extracted from the log line in the
        connection's results.

    \item If no mails have been accepted over the connection, nothing needs
        to be done; the \action{TIMEOUT} action ends.

    \item If one or more recipients have been accepted, Postfix will have
        allocated a queueid for the incoming mail, and there may be a mail
        in the state tables that needs to be discarded by \parsername{}.
        The timeout may have interrupted transfer of an accepted delivery
        attempt, or it may have occurred after a mail delivery attempt was
        rejected.  If a mail needs to be discarded, the following checks
        will all pass:

        \begin{itemize}

            \item The timestamp of the log line preceding the timeout log
                line must be earlier than the timestamp of the last
                accepted delivery attempt, i.e.\ there have not been any
                rejections since then the delivery attempt was accepted.

            \item The mail must exist in the state tables.

            \item The mail must not have a \daemon{qmgr} log line.

        \end{itemize}

        If all checks pass the mail will be discarded from the state tables
        and will not be entered in the database.  If one or more checks do
        not pass, \parsername{} assumes that the timeout happened after a
        rejected delivery attempt.  This assumption is not necessarily
        correct, because Postfix could have accepted an earlier recipient,
        rejected a later one, and continued to accept delivery of the mail
        for the first recipient.  In that case the timeout applies to the
        partially accepted mail, which will be discarded by Postfix and
        should be discarded by \parsername{}; however, this has not
        occurred in practice.  Processing timeouts is further complicated
        by the presence of out of order \daemon{cleanup} log lines: see
        \sectionref{discarding cleanup log lines} for details.

\end{enumerate}

This complication is dealt with by the \action{TIMEOUT} action, with help
from the \action{CLONE} action.  When a client tries to delivery a mail
larger than the sever will accept, the \action{MAIL\_TOO\_LARGE} action
will perform the same processing as the \action{TIMEOUT} action.

\subsection{Discarding Cleanup Log Lines}

\label{discarding cleanup log lines}

The author has only observed this complication occurring after a timeout,
though there may be other circumstances that trigger it.  Sometimes the
\daemon{cleanup} log line for a mail being accepted is logged after the
timeout log line, by which time \parsername{} has discarded the mail
because it did not have enough data to satisfy the database schema (see
\sectionref{timeouts during data phase}) and was not expected to have any
more log lines; parsing the \daemon{cleanup} log line causes the
\action{CLEANUP\_PROCESSING} action to create a new state table entry, to
help deal with re-injected mails (\sectionref{Re-injected mails}).  This is
incorrect because the log line actually belongs to the mail that has just
been discarded; if the queueid is reused there will be a queueid clash,
otherwise the new mail will just remain in the state tables.

During the \action{TIMEOUT} action, if the mail's \daemon{cleanup} log line
is still pending, the \action{TIMEOUT} action updates the
\texttt{timeout\_queueids} state table, adding the queueid and the
timestamp from the log line.  To deal with this complication, the following
checks will be performed for each \daemon{cleanup} log line that is
processed:

\begin{itemize}

    \item Does the \texttt{timeout\_queueids} state table have an entry for
        the queueid in the log line?  If an entry is found it will be
        removed, regardless of whether the remaining criteria are
        satisfied.  If the queueid does not exist in the
        \texttt{timeout\_queueids} state table, the log line being
        processed cannot belong to a discarded mail.

    \item Has the queueid been reused yet, i.e.\ does it have an entry in
        the \texttt{queueids} state table?  If it has an entry in the
        \texttt{queueids} state table, the log line being processed belongs
        to that mail, not to a previously discarded mail.

    \item The timestamp of the \daemon{cleanup} log line must be within ten
        minutes of the mail acceptance timestamp.  Timeouts happen after
        five minutes, but some data may have been transferred slowly, and
        empirical evidence shows that ten minutes is not unreasonable; it
        appears to be a good compromise between false positives (log lines
        incorrectly discarded) and false negatives (new state table entries
        incorrectly created).

\end{itemize}

The \daemon{cleanup} log line must pass the checks above for it to be
discarded because some, but not all, connections where a timeout occurs
will have an associated \daemon{cleanup} log line; if the
\action{CLEANUP\_PROCESSING} action blindly discarded the next
\daemon{cleanup} log line after a timeout it would sometimes be mistaken.

This complication is handled by the \action{CLEANUP\_PROCESSING} and
\action{TIMEOUT} actions.

\subsection{Pickup Logging After Cleanup}

\label{pickup logging after cleanup}

When mail is submitted locally, \daemon{pickup} processes the new mail and
generates a log line.  Occasionally, this log line will occur later in the
log file than the \daemon{cleanup} log line, so the \action{PICKUP} action
will find that a state table entry already exists for that queueid.
Normally when this happens a warning is generated by the \action{PICKUP}
action, but if the following conditions are met it is assumed that the log
lines were out of order:

\begin{itemize}

    \squeezeitems{}

    \item The only program that has logged anything so far for the mail is
        \daemon{cleanup}.

    \item The difference between the timestamps of the \daemon{cleanup} and
        \daemon{pickup} log lines is less than five seconds.

\end{itemize}

As always with heuristics, there may be circumstances in which these
heuristics match incorrectly, but none have been identified so far.  This
complication only seems to occur during periods of particularly heavy load,
so is most likely caused by process scheduling vagaries.

This complication is dealt with by the \action{PICKUP} action.

\subsection{Smtpd Stops Logging}

\label{smtpd stops logging}

Occasionally a \daemon{smtpd} will stop logging without an immediately
obvious reason.  After poring over log files for some time, several reasons
have been found for this rare event:

\begin{enumerate}

    \item Postfix is stopped or its configuration is reloaded.  When this
        happens all \daemon{smtpd} processes exit, so all entries in the
        connections state table must be cleaned up, entered in the database
        if they have enough data, and deleted.

    \item Sometimes a \daemon{smtpd} is killed by a signal (sent by
        Postfix, by the administrator, or by the OS), so its active
        connection must be cleaned up, entered in the database if it has
        enough data, and deleted from the connections state table.

    \item Occasionally a \daemon{smtpd} will exit with an error, so the
        active connection must be cleaned up, entered in the database if it
        has enough data, and deleted from the connections state table.

    \item Every Postfix process uses a watchdog timer that kills the
        process if it is not reset for a considerable period of time (five
        hours by default).  This safeguard prevents Postfix processes from
        running indefinitely and consuming resources if a failure causes
        them to enter a stuck state.

\end{enumerate}

The circumstances above account for all occasions where a \daemon{smtpd}
suddenly stops logging.  In addition to removing an active connection from
the state tables, the last accepted mail may need to be discarded, as
described in \sectionref{timeouts during data phase}; otherwise the queueid
state table is untouched.

This complication is handled by several actions: \action{POSTFIX\_RELOAD} (1),
\action{SMTPD\_DIED} (2 \& 3), and \action{SMTPD\_WATCHDOG} (4).

% \newpage{} % There's always a fucking varioref problem here

\subsection{Out Of Order Log Lines}

\label{out of order log lines}

Occasionally, a log file will have out of order log lines that cannot be
dealt with by the various techniques described in \sectionref{tracking
re-injected mail}, \sectionref{discarding cleanup log lines}, or
\sectionref{pickup logging after cleanup}.  In the \numberOFlogFILES{} log
files used in \sectionref{Evaluation} this problem occurs only five times
in 60,721,709 log lines.  All five occurrences have the same
characteristics: the \daemon{local} log line showing delivery to a user's
mailbox comes after the \daemon{qmgr} log line showing removal of the mail
from the queue because delivery is complete.  This causes several problems:
the data in the state tables for the mail is not complete, so it cannot be
entered into the database; a new mail is created when the \daemon{local}
log line is processed and remains in the state tables; four warnings are
issued per pair of out of order log lines.

The \action{COMMIT} action examines the list of programs that have produced
log lines for each mail, checking the list against a table of known-good
Postfix component combinations.  If the mail's combination is found in the
table it can be entered in the database; if the combination is not found
entry must be postponed and the mail flagged for later entry.  The
\action{MAIL\_DELIVERED} action checks for that flag; if the log line is
has just processed has caused the mail to reach a valid combination then
entry of the mail into the database will proceed, otherwise it must be
postponed once more.

The valid combinations are explained below.  In addition to the components
shown in each combination, every mail will have log lines from both
\daemon{cleanup} and \daemon{qmgr}, and any mail may also have a log line
from \daemon{bounce}, \daemon{postsuper}, or both.

% This will put the text on the line following the item name, if the
% enumitem package is loaded.
%\begin{description}[style=nextline]
\begin{description}

    \item [\daemon{local}:] Local delivery of a bounce notification, or
        local delivery of a re-injected mail.

    \item [\daemon{local}, \daemon{pickup}:] Mail submitted locally on the
        server, delivered locally on the server.

    \item [\daemon{local}, \daemon{pickup}, \daemon{smtp}:] Mail submitted
        locally \newline{} on the server, for both local and remote
        delivery.

    \item [\daemon{local}, \daemon{smtp}, \daemon{smtpd}:] Mail accepted
        from a remote client, for both local and remote delivery.

    \item [\daemon{local}, \daemon{smtpd}:] Mail accepted from a remote
        client, for local delivery only.

    \item [\daemon{pickup}, \daemon{smtp}:] Mail submitted locally on the
        server, for remote delivery only.

    \item [\daemon{smtp}:] Remote delivery of either a re-injected mail or
        a bounce notification.

    \item [\daemon{smtp}, \daemon{smtpd}:] Mail accepted from a remote
        client, then remotely delivered.  Typically this is a mail server
        relaying mail for clients on the local network to addresses outside
        the local network.

    \item [\daemon{smtpd}, \daemon{postsuper}:] Mail accepted from a remote
        client, then deleted by the administrator before any delivery
        attempt was made.  Note that \daemon{postsuper} is required, not
        optional, for this combination.

\end{description}

This check applies to accepted mails only, not to rejected mails.  This
check is performed during the \action{COMMIT} action, with support from the
\action{MAIL\_DELIVERED} action.

\subsection{Yet More Aborted Delivery Attempts}

\label{yet more aborted delivery attempts}

The aborted delivery attempts described in \sectionref{aborted delivery
attempts} occur frequently, but those described in this section only occur
four times in the \numberOFlogFILES{} log files used in
\sectionref{Evaluation}.  The symptoms are the same as in
\sectionref{aborted delivery attempts}, except that the mail \textit{has\/}
a \daemon{cleanup} log line; nothing can be found in the log files to
explain why this mail does not have any further log lines.  The only way to
detect these mails is to periodically scan all mails in the state tables,
deleting any mails with the following characteristics:

\begin{itemize}

    \item The timestamp of the last log line for the mail must be 12 hours
        or more earlier than the timestamp of the last log line parsed from
        the current log file.

    \item There must be exactly two \daemon{smtpd} and one \daemon{cleanup}
        log lines for the mail, with no additional log lines.

\end{itemize}

12 hours is a somewhat arbitrary time period, but it is far longer than
Postfix would delay delivery of a mail in the queue, unless it was not
running for an extended period of time.  Each time the end of a log file is
reached, the state tables are scanned for mails matching the
characteristics above, and any mails found are deleted.

\subsection{Mail Deleted Before Delivery Is Attempted}

\label{Mail deleted before delivery is attempted}

Postfix logs the recipient address when delivery of a mail is attempted, so
if delivery has yet to be attempted \parsername{} cannot determine the
recipient address or addresses.  This can occur when mail is accepted
faster than Postfix can attempt delivery, and the administrator deletes
some of the mail before Postfix has had a chance to attempt delivery.  The
deleted mail's recipient address or addresses will not have been logged
yet, and the deleted mails will not have any more log lines.  A dummy
recipient address needs to be added by \parsername{}, because every mail is
required by the database schema (\sectionref{results table}) to have at
least one recipient.  When this complication occurs the log file will
typically show many instances of it, closely grouped.  Generally, this
problem arises because the administrator has deleted some mails from
Postfix's mail queue to stop a mail loop.

This lack of information cannot easily be overcome: it is simple to
configure Postfix to log a warning for every accepted recipient, but
Postfix will not yet have allocated a queueid for the mail when the warning
for the first recipient is logged, so the warning will be associated with
the connection rather than the accepted mail.  A queueid will be allocated
after Postfix accepts the MAIL FROM command if
\texttt{smtpd\_delay\_open\_until\_valid\_rcpt} is set to ``no'', but that
setting will cause disk IO for every delivery attempt, instead of just for
delivery attempts where recipients are accepted, and consequently a drastic
reduction in the performance of the mail server.

The \action{DELETE} action is responsible for handling this complication.

\subsection[Bounce Notification Mails Delivered Before Their Creation Is
Logged]{Bounce Notification Mails Delivered Before \newline{} Their
Creation Is Logged}

\label{Bounce notification mails delivered before their creation is logged}

This is yet another complication that only occurs during periods of
extremely high load, when process scheduling and even hard disk access
times cause strange behaviour.  In this complication, bounce notification
mails are created, delivered, and deleted from the queue, \textit{before\/}
the log line from \daemon{bounce} that explains their origin.  The origin
of the mail is unknown when it is created, but when the time comes to enter
it into the database its origin is correctly identified by the heuristics
used to identify Postfix 2.2 bounce notifications, described in
\sectionref{identifying bounce notifications}.  To avoid incorrectly
creating a new mail when the out of order bounce notification log line is
processed, the \action{COMMIT} action maintains a cache of recently
committed bounce notification mails named \texttt{bounce\_queueids}, which
the \action{BOUNCE\_CREATED} action subsequently checks when processing the
bounce creation log line.  If the queueid exists in the cache, and its
start time is less than ten seconds before the timestamp of the bounce log
line, it is assumed that the bounce notification mail has already been
processed and the \action{BOUNCE\_CREATED} action does not create one.  If
the queueid exists in the cache it is removed, because it has either just
been used or the problem did not occur for the new mail.  Whether the
\action{BOUNCE\_CREATED} action creates a new mail or finds an existing
mail in the \texttt{queueids} state table (not the
\texttt{bounce\_queueids} cache), it flags the mail as having been seen by
the \action{BOUNCE\_CREATED} action; if this flag is present the
\action{COMMIT} action will not add the mail to the
\texttt{bounce\_queueids} cache.  This prevents a bounce notification log
line being incorrectly discarded if a queueid is reused within 10 seconds.

\subsection{Mails Deleted During Delivery}

\label{Mails deleted during delivery}

The administrator can delete mails using \daemon{postsuper}; occasionally,
mails that are in the process of being delivered will be deleted by the
administrator.  This results in the log lines from the delivery agent
(\daemon{local}, \daemon{smtp}, or \daemon{virtual}) appearing in the log
file \textit{after\/} the mail has been removed from the state tables and
saved in the database.  The \action{DELETE} action adds deleted mails to a
cache named \textit{postsuper\_deleted\_queueids}, which is checked by the
\action{MAIL\_DELIVERED} action, and the current log line discarded if the
following conditions are met:

\begin{enumerate}

    \item The queueid is not found in the state tables.

    \item The queueid is found in the cache of deleted mails.

    \item The timestamp of the log line is within 5 minutes of the final
        timestamp of the mail.

\end{enumerate}

Sadly, this solution involves discarding some data, but the complication
only arises eight times in quick succession in one log file, which is not
in the \numberOFlogFILES{} log files used for evaluating the parser; if
this complication occurred more frequently it might be desirable to find
the mail in the database and add the log line to it.

\subsection{Summary}

This section has described the complications encountered while implementing
\parsername{}.  Five of the fifteen complications are caused by log lines
appearing in an unexpected or abnormal order in log files, often because of
heavy system load: these complications typically caused warnings and a new
state table entry.  Three of the fifteen complications must be dealt with
to correctly recreate Postfix's behaviour.  The remaining seven
complications are caused by deficiencies in Postfix's logging, and some
could easily be resolved by adding additional logging to Postfix; these
seven are dealt with by applying heuristics to specific log lines or mails,
sometimes in conjunction with information cached when mails are removed
from the state tables.
