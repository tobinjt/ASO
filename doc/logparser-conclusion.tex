\chapter{Conclusion}

\label{conclusion}

XXX THIS NEEDS TO BE EXTENDED\@.

XXX THERE IS NO MENTION OF RESULTS/EFFICIENCY\@.

Parsing Postfix log files appears at first sight to be an almost trivial
task, especially if one has previous experience in parsing log files, but
it turns out to be a much more taxing project than initially expected.  The
variety and breadth of log lines produced by Postfix is quite surprising,
because a quick survey of sample log files gives the impression that the
number of distinct log lines is quite small; this impression is due to the
uneven distribution exhibited by log lines produced in normal operation
(see \graphref{rule hits graph} for a vivid illustration of this).


Given the diverse nature of the log lines and the ease with which
administrators can cause new lines to be logged (\sectionref{postfix
background}), enabling users to easily extend the parser to deal with new
log lines is a design imperative (\sectionref{parser design}).  Despite the
resulting initial increase in complexity the task is quite tractable,
though it does raise efficiency and optimisation questions (answered in
\sectionref{parser efficiency}).  Providing a tool to ease the generation
of regexes from unparsed log lines should greatly help users add rules
to parse previously unparsed log lines (\sectionref{creating new rules}).


The architecture not only substantially eases the parsing of new log
lines, it makes adding new actions (\sectionref{adding new actions}) a
relatively simple task.  The simplicity of adding a new action frees the
implementor from worrying about how their action might interfere with
existing parsing, so they can focus on correctly implementing their new
action.


The division of the parser into rules, actions, and framework is unusual
because rules are separated so completely from the actions and framework;
although parsers are often divided (the combination of \texttt{lex} and
\texttt{yacc}~\cite{lex-and-yacc-book} being a common example), the parts
are usually quite internally interdependent, and will be combined into a
complete parser by the compilation process; in contrast \parsername{} keeps
the rules and actions separate until the parser runs.  The actions and
framework are not as completely separated, as the actions depend on
services provided by the framework; however the actions and framework are
not tightly integrated: it would be possible, with some work, to completely
separate the two.  XXX I SHOULD DO THAT\@.


% XXX THIS SHOULD PROBABLY BE REMOVED\@.
%The emergent behaviour (\sectionref{Emergent behaviour}) exhibited by the
%rules and actions is also interesting, and is discussed after the flow
%chart (\sectionref{flow-chart}) and explanation of the paths through the
%parser.  This emergent behaviour greatly eases the process of adding new
%actions (\sectionref{adding new actions}), as the actions do not have to
%be inserted into an explicit flow of control; new actions will naturally
%find their place in the paths through the parser.

The flow of control in this parser is quite different from that of most
other parsers.  In most traditional parsers the parser has a current state,
and each state has a fixed set of acceptable next states, with unacceptable
states causing parsing to fail --- e.g.\ when a \textbf{C} parser sees the
keyword \textbf{for} it expects to immediately see a left parenthesis, with
any other input causing parsing to fail (comments will already have been
removed by the preprocessor).  This parser is different: the rule which
matches the next log line for a mail dictates the action that will be
invoked, which is equivalent to the next state in other parsers; thus the
next log line for a mail dictates the next state for that mail.  XXX THINK
ABOUT HOW TO MAKE THE PARSER MORE STATEFUL IF REQUIRED\@; ADD IT TO THE
ARCHITECTURE SECTION\@.

The real difficulties arise once the parser is successfully dealing with
90\% of the log lines, as the irregularities and complications explained in
\sectionref{complications} only become apparent once the vast majority of
log lines have been parsed successfully.  Adding new rules to deal with
numerous, infrequently occurring log lines is a simple, albeit tiresome,
task, (though alleviated by the tool described in \sectionref{adding new
actions}), whereas dealing with mails where information appears to be
missing is much more grueling.  Trawling through the log files, looking for
something out of the ordinary, possibly hundreds or even thousands of lines
away from the last mention of the queueid in question, is extremely time
consuming and error prone.  Sometimes the task is not to spot the line that
is unusual, but to spot that a line that is normally present is missing,
i.e.\ to realise that one line amongst thousands is absent.  In all cases
the evidence must be used to construct a hypothesis to explain the
irregularities, and that hypothesis must then be tested in the parser; if
successful, the parser must be modified to deal with the irregularities,
without adversely affecting the existing functionality.  The complications
described in \sectionref{complications} were solved in the order they are
described in, and that order closely resembles the frequency with which
they occur; the most frequently occurring complications dominate the
warning messages produced, and so naturally they are the first
complications to be dealt with.

A parser's ability to correctly parse its input is extremely important; the
parser's coverage of its test log files is discussed in \sectionref{parsing
coverage}.  Both its success at parsing individual log lines
(\sectionref{log-lines-covered}) and its correctness in reconstructing each
mail's journey through Postfix (\sectionref{mails-covered}) are described
in detail, including the results of manually verifying the correct parsing
of a subset of the test log files.

\newpage


