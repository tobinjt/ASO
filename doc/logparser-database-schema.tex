\section{Database schema}
\label{database schema}

The database is an integral part of the parser presented here: it stores
the rules and the data gleaned by applying those rules to Postfix log
files.  Understanding the database schema is important in understanding the
actions of the parser, and essential to developing further applications
which utilise the data gathered.

\subsection{Introduction}

The database schema can be conceptually divided in two: the rules which are
used to parse log files, and the data saved from the parsing of log files.
Rules have the fields required to parse the log lines, extract data to be
saved, and the action to be executed; they also have several fields which
aid the user in understanding what each rule parses.  The rules are
described in detail in \sectionref{rules} but the fields are covered here
in \sectionref{rule attributes}.

The data saved from parsing the logs is also divided into two tables as
described below: connections and results.  The connections table contains a
row for every mail accepted and every connection where there was a
rejection; the individual fields will be described in
\sectionref{connections table}.  The results table has one or more rows per
row in the connections table; the fields will be covered in detail in
\sectionref{results table}.

An important but easily overlooked benefit of storing the rules in the
database is the connection between rules and results: if more information
is required when examining a result, the rules which produced the database
entries are available for inspection (and each result references the rule
which created it).  There is no ambiguity about which rule resulted in a
particular result, eliminating one potential source of confusion.

\subsection{Rules table}

\label{rule attributes}

Rules are discussed in detail in \sectionref{rules}, but the structure of
the rules table is covered here.  Rules are created by the user, not the
parser, and will not be modified by the parser (except for the hits and
hits\_total fields).  Rules parse the individual log lines, extracting data
to be saved in the connections and results tables, and specifying action to
take for that log line.

Each rule defines the following:

\begin{description}

    \item [id] A unique identifier for each rule which other tables can use
        to refer to a specific rule.

    \item [name] A short name for the rule.

    \item [description] Something must have occurred to cause Postfix to
        log each line (e.g.\ a remote client connecting causes a connection
        line to be logged).  This field describes the action causing the
        log lines this rule matches.

    \item [restriction\_name] The restriction which caused the mail to be
        rejected.  Only applicable to rules which have an action of
        \texttt{REJECTION}, other rules should have an empty string.

    \item [postfix\_action] This is the action Postfix must have taken to
        generate this log line.  This field is mostly ignored, but two
        values (IGNORED and INFO) have special meaning, as described below
        in the list of typical values.\label{postfix_action}

        \begin{description}

            \item [ACCEPTED] Postfix has accepted a mail, and will
                subsequently attempt to deliver it.

            \item [BOUNCED] The mail has bounced, due to a mail loop,
                delivery failure, or five day timeout.

            \item [DELETED] The mail was deleted from the queue by an
                administrator.

            \item [DISCARDED] Postfix discarded the mail it was in the
                process of accepting, because it was either larger than the
                limit set by the administrator, or the client timed out or
                disconnected.

            \item [EXPIRED] The mail has been in the queue for five
                days\footnote{As with much of Postfix's behaviour, this is
                the default value but can be changed by the administrator
                if they choose.} without successful delivery.  A bounce
                mail will be generated and sent to the sender address.

            \item [INFO] Represents an unspecified intermediate action that
                the parser is not interested in per se, but which does log
                useful information, supplementing other log lines.

            \item [IGNORED] An action which is not only uninteresting in
                itself, but which also provides no useful data.

            \item [POSTFIX\_RELOAD] The administrator has instructed
                Postfix to start or stop, and all existing \daemon{smtpd}
                processes will be terminated.  This does not negatively
                impact on the logs or mail queued by Postfix for delivery.

            \item [PROCESSING] \daemon{cleanup} is processing a mail ---
                see~\cite{postfix-cleanup} for details of the processing
                performed by \daemon{cleanup}.

            \item [REJECTED] Postfix rejected a command from the remote
                client, causing at least one recipient to be rejected.

            \item [SENT] Postfix has successfully sent a mail.

        \end{description}

        Uninteresting log lines are parsed so that any lines the parser
        isn't capable of handling become immediately obvious errors.

    \item [program] The program (\daemon{smtpd}, \daemon{qmgr}, etc.) whose
        log lines the rule applies to.  This avoids needlessly trying rules
        which won't match the log line, or worse, might match
        unintentionally.  Rules whose program is \texttt{*} will be tried
        against any log lines which aren't parsed by program specific
        rules.

    \item [regex] The \regex{} to match the log line against.  The \regex{}
        will first have several keywords expanded: this simplifies reading
        and writing rules; avoids needless repetition of complex \regex{}
        components; allows the components to be corrected and/or improved
        in one location; and makes each \regex{} largely self-documenting.

        For efficiency the keywords are expanded and every rule's \regex{}
        is compiled before attempting to parse the log file --- otherwise
        each \regex{} would be recompiled each time it was used, resulting
        in a large, data dependent slowdown.  Rule efficiency concerns are
        discussed in \sectionref{rule efficiency}, with the impact of
        compiling and caching \regexes{} covered in \sectionref{Caching
        each regex}.

    \item [result\_cols, connection\_cols] Specifies how the fields in the
        log line will be extracted.  The format is:
        \newline \tab{} \texttt{smtp\_code = 1; recipient = 2, sender = 4;}
        \newline i.e.\ semi-colon or comma separated assignment statements,
        with the variable name on the left and the matching capture from
        the \regex{} on the right hand side.  The list of acceptable
        variable names is:

        \texttt{connection\_cols: client\_hostname, client\_ip, server\_ip,
        \newline \tab{} server\_hostname} and \texttt{helo.\newline}
        \texttt{result\_cols: sender, recipient, smtp\_code, message\_id,
        \newline \tab{} size,} and \texttt{data}

        Additionally \texttt{child} and \texttt{pid} are used respectively
        by the \texttt{TRACK}, \texttt{BOUNCE} and \texttt{SMTPD\_DIED}
        actions.

    \item [result\_data, connection\_data] Sometimes rules need to supply a
        piece of data which isn't present in the log line: e.g.\ setting
        \texttt{smtp\_code} when mail is accepted.  The format and allowed
        variables are the same as for \texttt{result\_cols} and
        \texttt{connection\_cols}, except that arbitrary
        data\footnote{Commas and semi-colons cannot be escaped and thus
        cannot be used.  This is intended for use with small amounts of
        data rather than large amounts in any one rule, so dealing with
        escape sequences was deemed unnecessary.} is permitted on the right
        hand side of the assignment.

    \item [action] The action that will be invoked when this rule matches a
        log line; a full list of actions and the parameters they are
        invoked with can be found in \sectionref{actions-in-detail}.

    \item [queueid] Specifies the matching capture from the \regex{} which
        gives the queueid, or zero if the log line doesn't contain a
        queueid.  Many log lines won't contain a queueid, e.g.\ rejections
        logged before a mail has been accepted (a queueid won't have been
        allocated), or log lines which aren't tied to one particular mail.

    \item [hits] This counter is maintained for every rule and incremented
        each time the rule successfully matches.  At the start of each run
        the program sorts the rules in descending order of hits, and at the
        end of the run updates every rule's hits.  Assuming that the
        distribution of log lines is reasonably consistent between log
        files, rules matching more commonly occurring log lines will be
        tried before rules matching less commonly occurring log lines,
        lowering the program's execution time.  Rule ordering for
        efficiency is discussed in \sectionref{rule ordering for
        efficiency}.

    \item [hits\_total] The total number of hits for this rule over all
        runs of the parser.

    \item [priority] This is the user-configurable companion to hits: rules
        will be tried in order of priority, overriding hits.  This allows
        more specific rules to take precedence over more general rules
        (described in \sectionref{overlapping rules}).

    \item [cluster\_group] A reference to the \texttt{cluster\_group}
        table.  That table is used by the Decision Tree algorithm described
        in a separate document.

\end{description}


\subsection{Connections table}

\label{connections table}

Every accepted mail and every connection where there was a rejection will
have a single entry in the connections table containing the following
fields:

\begin{description}

    \item [id] This field uniquely identifies the row.

    \item [server\_ip] The \IP{} address (IPv4 or IPv6) of the server: the
        local server when receiving mail, the remote server when sending
        mail.

    \item [server\_hostname] The hostname of the server, it will be
        \texttt{unknown} if the \IP{} address could not be resolved to a
        hostname via \DNS{}\@.

    \item [client\_ip] The client \IP{} address (IPv4 or IPv6): the remote
        server when receiving mail, the local server when sending mail.

    \item [client\_hostname] The hostname of the client, it will be
        \texttt{unknown} if the \IP{} address could not be resolved to a
        hostname via \DNS{}\@.

    \item [helo] The hostname used in the HELO command.  The HELO hostname
        occasionally changes during a connection, presumably because spam
        or virus senders think it's a good idea.  By default Postfix only
        logs the HELO hostname when it rejects an \SMTP{} command, but it
        is quite easy to rectify this:

\label{logging helo}

        \begin{enumerate}

            \item Create \texttt{/etc/postfix/log\_helo.pcre}
                containing:\newline \tab{}\texttt{/./~~~~WARN~Logging~HELO}

            \item Modify \texttt{smtpd\_data\_restrictions} in
                \texttt{/etc/postfix/main.cf} to contain\newline
                \tab{}\texttt{check\_helo\_access~/etc/postfix/log\_helo.pcre}

        \end{enumerate}

        Although \texttt{smtpd\_helo\_restrictions} seems like the natural
        place to log the HELO hostname, there won't be a queueid associated
        with the mail for the first recipient, so that log line cannot be
        associated with the correct mail.  There is guaranteed to be a
        queueid when the DATA command has been reached, and thus it will be
        logged by any restrictions taking effect in
        \texttt{smtpd\_data\_restrictions}.  There is no difficulty in
        specifying a HELO-based restriction in
        \texttt{smtpd\_data\_restrictions}, Postfix will perform the check
        correctly.

        Logging the HELO hostname in this fashion also prevents the
        complication described in \sectionref{Mail deleted before delivery
        is attempted} from occurring when, but only in the case where there
        is a single recipient; in that case the recipient address will be
        logged also, but when there are multiple recipients none are
        logged.  It is also possible to warn for every recipient,
        preventing the complication in \sectionref{Mail deleted before
        delivery is attempted} entirely.

    \item [queueid] The queueid of the mail if the connection represents an
        accepted mail, or \texttt{NOQUEUE} otherwise.

    \item [start] The timestamp of the first log line, in seconds since the
        epoch (explained in the glossary, appendix~\refwithpage{Glossary}).

    \item [end] The timestamp of the last log line, in seconds since the
        epoch.

\end{description}

\subsection{Results table}

\label{results table}

Every log line where the associated rule has a \texttt{postfix\_action}
will have an entry in the results table, e.g.\ rejecting an \SMTP{}
command, delivering a mail, or bouncing a mail (see
\sectionref{postfix_action} for typical values of
\texttt{postfix\_action}).  Each row is associated with a single
connection, though there may be many results per connection.

\begin{description}

    \item [connection\_id] A reference to the row in the connections table
        this result is associated with.

    \item [rule\_id] A reference to the entry in the rules table which
        matched the log line and created this result.

    \item [id] A unique identifier for this result.

    \item [warning] Postfix can be configured to log a warning instead of
        enforcing a restriction that would reject an \SMTP{} command --- a
        facility that is quite useful for testing new restrictions.  This
        field will be 1 if the log line parsed was a warning rather than a
        real rejection, or 0 for a real rejection.  This field should be
        ignored if the result is not a rejection.

    \item [smtp\_code] The \SMTP{} code associated with the log line.  In
        general an \SMTP{} code is only present for a rejection or final
        delivery; results initially missing an \SMTP{} code will duplicate
        the \SMTP{} code of other results in the connection.  Some final
        delivery log lines don't contain an \SMTP{} code: in those cases
        the code is faked based on the success or failure represented by
        the log line.

    \item [sender] The sender's email address.  Because multiple mails may
        be delivered during one connection, there may be different sender
        addresses in the results for one connection; however there should
        not be different sender addresses in the results for one email.

    \item [recipient] The recipient address; there may be multiple
        recipient addresses per mail, but only one per result.

    \item [size] The size of the mail; it will only be present for
        delivered mails.

    \item [message\_id] The message-id of the accepted mail, or
        \texttt{NULL} if no mail was accepted.

    \item [data] A field available for anything not covered by other
        fields, e.g.\ the rejection message from an \RBL{}\@.

    \item [timestamp] The time the log line was logged, in seconds since
        the epoch.

\end{description}

\subsection{Conclusion}

The table containing the rules used by the parser and both tables
containing the data extracted from the Postfix logs were described, with
the purpose of each field discussed in detail.  A clear, comprehensible
schema is essential when using the extracted data; it's more important when
using the data than when storing it, because storing the data is a
write-once operation, whereas utilising the data requires frequent
searching, sorting and manipulation of the data to produce customised
reports and/or statistics.

