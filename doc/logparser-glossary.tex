% vim: set filetype=tex :
% The contents of the glossary.
\newglossaryentry{SQLite}{name={SQLite},description={
    \textit{SQLite is a small C library that implements a self-contained,
    embeddable, zero-configuration SQL database engine.\/} SQLite is an
    \gls{SQL} implementation focusing on correctness, simplicity, and
    speed.  Unlike other \gls{SQL} implementations it does not require a
    separate server process, greatly simplifying deployment of an
    application utilising it.  More details can be found
    at~\url{http://www.sqlite.org/}.
}}

\newglossaryentry{Phishing}{name={Phishing},description={
    Phishing~\cite{Wikipedia-phishing} is an attempt to acquire information
    by masquerading as an entity trusted by the user, e.g.\ a bank.
}}

\newglossaryentry{Backscatter}{name={Backscatter},description={
    When a spam sender or worm sends mail with forged sender addresses,
    innocent sites are flooded with undeliverable mail notifications; this
    is called backscatter mail.
}}

\newglossaryentry{Joe job}{name={Joe job},description={
    A joe~job describes spam mail sent using a faked sender address with
    the intention of sullying the good name of the user of that address.
    Joe~jobs are one cause of backscatter.
}}

\newglossaryentry{Epoch}{name={Epoch},description={
    Most operating systems store the current time and timestamps of files
    etc.\ as seconds elapsed since the epoch, the beginning of time as far
    as the operating system is concerned.  On Unix and Unix-derived systems
    the epoch is 1970/01/01 00:00:00; on other operating systems it may be
    different.
}}


\newglossaryentry{<>}{name={$<>$},sort={<>},description={
    $<>$ is the sender address used for mail that should not be replied to,
    e.g.\ bounce notifications.  In \gls{SMTP} all addresses are enclosed
    in $<>$, so \textit{username\at{}domain\/} becomes
    \textit{$<$username\at{}domain$>$\/}; thus $<>$ is actually an empty
    address, but is always written as $<>$ for clarity.  All mail servers
    must accept mail sent from $<>$, or they are in violation of
    \gls{RFC}~2821~\cite{RFC2821}.
}}

\newglossaryentry{queueid}{name={queueid},description={
    Each mail in Postfix's queue is assigned a queueid to uniquely identify
    it.  Queueids are assigned from a limited pool, so although they are
    guaranteed to be unique for the lifetime of the mail, they may be
    reused later.
}}

%\newglossaryentry{hash}{name={hash},description={
%    A hashing function transforms a string of characters to a number.
%    There are many possible uses for the resulting number: a common usage
%    is to maintain a data structure indexed by strings in an efficient
%    manner.  A full description is beyond the scope of this paper, further
%    information can be found at~\cite{hash-functions}.
%}}

\newglossaryentry{awk}{name={awk},description={
    AWK is a general purpose programming language that is designed for
    processing text-based data, and is available as a standard utility on
    all Unix systems.
}}

\newglossaryentry{syslog}{name={syslog},description={
    Syslog is the standard logging mechanism on Unix systems: the program
    sends log messages to syslog, then syslog filters and stores the
    messages according to the configuration the administrator has chosen.
}}

\newglossaryentry{mail bomb}{name={mail bomb},description={
    A mail bomb occurs when an attacker inflicts a huge volume of mail on
    the victim.  At best a mail bomb is irritating to the victim; at worst
    the deluge of mail can be severe enough to: interrupt service for the
    victim and/or other users; cause mail to be rejected because the victim
    has reached a limit (e.g.\ too many mails, too much disk space
    consumed); the victim may accidentally delete other mail while trying
    to cope with the mail bomb.
}}

\newglossaryentry{mail loop}{name={mail loop},description={
    Sometimes mail sent to one address must be forwarded to a different
    address instead, e.g.\ because a person has changed jobs.  A mail loop
    occurs when the chain of recipient addresses constitute a cyclic
    directed graph; the simplest example is when mail for
    \texttt{alice\at{}example.net} is delivered to
    \texttt{bob\at{}example.com}, and mail for \texttt{bob\at{}example.com}
    is delivered to \texttt{alice\at{}example.net}.
}}
