% vim: set filetype=tex :
% The contents of the glossary.
\glossary{name={SQLite3},description={
    \textit{SQLite is a small C library that implements a self-contained,
    embeddable, zero-configuration SQL database engine.\/} SQLite3 is an
    \SQL{} implementation focusing on correctness, simplicity and speed.
    Unlike other \SQL{} implementations it does not require a separate
    server process, greatly simplifying deployment of an application
    utilising it.  More details can be found at~\cite{sqlite-features} or
    \url{http://www.sqlite.org/}.
}}

\glossary{name={Phishing},description={
    Phishing~\cite{Wikipedia-phishing} is an attempt to acquire information
    by masquerading as an entity trusted by the user, e.g.\ a bank.
}}

\glossary{name={Backscatter},description={
    When a spam sender or worm sends mail with forged sender addresses,
    innocent sites are flooded with undeliverable mail notifications; this
    is called backscatter mail.
}}

\glossary{name={Joe~job},description={
    A joe~job is when spam mail is sent with a faked sender address with
    the intention of sullying the good name of the owner of the address.
    joe~jobs are a cause of backscatter, though by no means the only cause.
}}

\glossary{name={Epoch},description={
    Most operating systems store the current time and timestamps of files
    etc.\ as seconds elapsed since the epoch, the beginning of time as far
    as the operating system is concerned.  On Unix and Unix-derived systems
    the epoch is 1970/01/01 00:00:00; on other operating systems it may be
    different.
}}

\glossary{name={NULL},description={
    NULL is a special term used in \SQL{} databases indicating the absence
    of data for the field.
}}

% Postfix components
\glossary{name={bounce},description={
    The bounce daemon is responsible for sending bounce notifications in
    Postfix versions later than 2.2.  The definitive documentation is
    \url{http://www.postfix.org/bounce.8.html}.
}}

\glossary{name={cleanup},description={
    Cleanup processes all incoming mail after it has been accepted and
    before it is delivered.  It removes duplicate recipient addresses,
    inserts missing headers, and optionally rewrites addresses if
    configured to do so.  The definitive documentation is
    \url{http://www.postfix.org/cleanup.8.html}.
}}

\glossary{name={lmtp},description={
    Delivery of mail over \LMTP{} is performed by the lmtp component.  The
    definitive documentation is \url{http://www.postfix.org/lmtp.8.html}.
}}

\glossary{name={local},description={
    Local is the Postfix component responsible for local delivery of mail
    (i.e.\ delivered on the server Postfix is running on), whether it be to
    a user's mailbox or a program such as a mailing list manager or
    procmail (\url{http://www.procmail.org/}).  It also handles aliases and
    processing of a user's \texttt{.forward} file.  The definitive
    documentation is \url{http://www.postfix.org/local.8.html}.
}}

\glossary{name={pickup},description={
    Pickup is the service that deals with mail submitted locally via
    postdrop and sendmail; it passes all submitted mail to cleanup.  The
    definitive documentation is \url{http://www.postfix.org/pickup.8.html}.
}}

\glossary{name={postdrop},description={
    Postdrop is used when submitting mail locally on the server: it creates
    a new mail in the queue and copies its input into the mail.  Subsequent
    delivery of the mail is the responsibility of other Postfix components.
    The definitive documentation is
    \url{http://www.postfix.org/postdrop.1.html}.
}}

\glossary{name={postsuper},description={
    Maintenance task such as deleting mails from the queue, putting mail on
    hold (no further delivery attempts will be made until it is released
    from hold, also by postsuper), and consistency checking of the mail
    queue.  The definitive documentation is available at
    \url{http://www.postfix.org/postsuper.1.html}.
}}

\glossary{name={qmgr},description={
    Qmgr is the Postfix daemon that manages the mail queue, determining
    which mails will be delivered next.  Qmgr orders the mails based on the
    recipient for local mails and the destination server for remote
    addresses, ensuring that it balances the aims of achieving maximum
    concurrency while avoiding overwhelming destinations or wasting time
    and resources on non-responsive destinations.  The definitive
    documentation is \url{http://www.postfix.org/qmgr.8.html}.
}}

\glossary{name={sendmail},description={
    Postfix provides a command that is compatible with the Sendmail
    (\url{http://www.sendmail.org/}) mail submission program that all Unix
    commands that send mail depend on; Postfix sendmail executes postdrop
    to place a new mail in the queue.  The definitive documentation is
    \url{http://www.postfix.org/sendmail.1.html}.
}}

\glossary{name={smtp},description={
    Delivery of mail over \SMTP{} is performed by the smtp component.  The
    definitive documentation is \url{http://www.postfix.org/smtp.8.html}.
}}

\glossary{name={smtpd},description={
    Smtpd is the Postfix program that accepts mail via \SMTP{}, and
    implements all the anti-spam restrictions Postfix provides.  The
    definitive documentation is \url{http://www.postfix.org/smtpd.8.html}.
}}

\glossary{name={virtual},description={
    Virtual is the Postfix component responsible for delivery of mails to
    virtual domains.  With \daemon{local} delivery the destination is
    determined only by the portion of the email address on the left side of
    the \at{}, whereas with \daemon{virtual} delivery the destination is
    determined by the entire email address, e.g.\ if the server considers
    itself responsible for both \textbf{example.org} and
    \textbf{example.net} domains: \daemon{local} considers
    \textbf{john\at{}example.org} and \textbf{john\at{}example.net} to have
    the same mailbox, whereas \daemon{virtual} considers them to have
    different mailboxes.  Virtual delivery is used when a server hosts
    multiple domains where a username may be present in more than one
    domain but represent different users in each.  The definitive
    documentation is
    \url{http://www.postfix.org/virtual.8.html}.
}}

\glossary{name={Bayesian spam filtering},description={
    Bayesian spam filtering is a method of classifying mail based on the
    frequency that the words in the mail have previously appeared in a spam
    corpus and a ham (non-spam) corpus.  A full description is beyond the
    scope of this document, see~\cite{bayesian-filtering,a-plan-for-spam}
    for a detailed explanation.
}}

\glossary{name={Bayesian poisoning},description={
    Bayesian poisoning is the addition of innocuous or unrelated words to a
    spam mail in the hope of defeating Bayesian spam filtering.  E.g.\ the
    word Viagra would be firmly in the spam corpus for most people, but by
    adding the words \textit{schedule}, \textit{meeting}, \textit{moving
    forward\/} and \textit{best business practices\/} to a mail received by
    a manager, the Bayesian spam filter might tip the balance from bad to
    good, if the non-spam words outweigh the spam words.
}}

\glossary{name={$<>$},sort={<>},description={
    $<>$ is the sender address used for mail that should not be replied
    to, e.g.\ bounce notifications.  In \SMTP{} all addresses are enclosed
    in $<>$, so \textit{username\at{}domain\/} becomes
    \textit{$<$username\at{}domain$>$\/}; thus $<>$ is actually an empty
    address, but is always written as $<>$ for clarity.  All mail servers
    must accept mail sent from $<>$, or they are in violation of
    \RFC{}~2821~\cite{RFC2821}.
}}

\glossary{name={queueid},description={
    Each mail in Postfix's queue is assigned a queueid to uniquely identify
    it.  Queueids are assigned from a limited pool, so although they are
    guaranteed to be unique for the lifetime of the mail, given sufficient
    time they will be reused.
}}

\glossary{name={IPv4},description={
    Internet Protocol~\cite{Wikipedia-ipv4} version 4 is the fourth version
    of the Internet Protocol used to interconnect computers on the
    Internet.  It is the first widely deployed version of IP, and has been
    in use for over 25 years.
}}

\glossary{name={IPv6},description={
    Internet Protocol~\cite{Wikipedia-ipv6} version 6 is the latest version
    of the Internet Protocol used to interconnect computers on the
    Internet.  It is the successor to IPv4, bringing with it a greatly
    expanded address space, allowing many more computers to use the
    Internet simultaneously.  IPv4 and IPv6 will coexist for many years to
    come as existing networks transition from the former to the latter.
}}

\glossary{name={hash},description={
    A hashing function transforms a string of characters to a number.
    There are many possible uses for the resulting number: a common usage
    is to maintain a data structure indexed by strings in an efficient
    manner.  A full description is beyond the scope of this paper, further
    information can be found at~\cite{hash-functions}.
}}

\glossary{name={Solaris},description={
    Solaris is a Unix-derived Operating System produced by Sun Microsystems
    (\url{http://www.sun.com/software/solaris/}).
}}

\glossary{name={awk},description={
    AWK is a general purpose programming language that is designed for
    processing text-based data, and is available as a standard utility on
    all Unix systems.
}}

\glossary{name={syslog},description={
    Syslog is the standard logging mechanism on Unix systems: the program
    sends log messages to syslog, then syslog filters and stores the
    messages according to the configuration the administrator has chosen.
}}

\glossary{name={mail bomb},description={
    A mail bomb occurs when an attacker inflicts a huge volume of mail on
    the victim.  At best a mail bomb is irritating to the victim; at worst
    the deluge of mail can be severe enough to: interrupt service for the
    victim and/or other users; cause mail to be rejected because the victim
    has reached a limit (e.g.\ too many mails, too much disk space
    consumed); the victim may accidentally delete other mail while trying
    to cope with the mail bomb.
}}

\glossary{name={mail loop},description={
    Sometimes mail set to one address must be delivered to a different
    address instead, e.g.\ because a person has changed jobs.  A mail loop
    occurs when the recipient addresses constitute a cyclic directed graph;
    the simplest example is when mail for \texttt{foo\at{}example.net} is
    delivered to \texttt{bar\at{}example.com}, and mail for
    \texttt{bar\at{}example.com} is delivered to
    \texttt{bar\at{}example.com}.
}}
