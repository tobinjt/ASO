\section{Graphs}

\label{graphs}

XXX COMPARE SAVING RESULTS TO RAM DISK VERSUS SAVING TO HARD DISK\@.

XXX DO SOMETHING WITH NORMAL VS SMALLER GRAPH\@.

\showgraph{build/plot-normal-vs-smaller}{NORMAL VS SMALLER}{NORMAL VS
SMALLER}

\renewcommand{\figurename}{Graph}

\subsection{Introduction}

Graphs are an excellent means of displaying data, transforming a
meaningless stream of numbers into an easily comprehensible form, where
anomalies and patterns are immediately obvious.  These graphs are used to
illustrate the topics discussed in \sectionref{rule efficiency}.  The
graphs in the first section cover parser scalability, demonstrating that
performance scales linearly with input size.  The impact of rule ordering
is shown in \sectionref{rule ordering graphs}, and the anomalous dips and
peaks apparent in some graphs are explained.  The third group of graphs
vividly shows the huge impact that caching compiled \regexes{} has on
parser performance.  Miscellaneous graphs are presented in the final group
of graphs; these graphs are referenced from various places in the test.
This section concludes with a breakdown of the rule hits accumulated during
a single test run of the parser.

\subsection{Parser scalability}

\showgraph{build/plot-normal-filesize-numlines}{Execution time vs file
size vs number of lines}{execution time vs file size vs number of lines
graph}

The Y axis in \graphref{execution time vs file size vs number of
lines graph} represents the following:

\begin{enumerate}

    \item The time required, in seconds, to parse the log file.

    \item The size of the log file, in megabytes.

    \item The number of lines in the log file, divided by 10000.

\end{enumerate}

\Graphref{execution time vs file size vs number lines factor}
shows the ratio of execution time vs file size and number of lines (higher
is better, it means more bytes or lines processed per second).  The ratios
are quite tightly banded with the exception of log files 22 and 62--68,
where they are noticeably higher; \graphref{execution time vs file
size vs number lines factor} and \tableref{execution time vs file
size vs number lines factor table} show that the parser's execution time
scales linearly with input size.

\showgraph{build/plot-normal-filesize-numlines-factor}{Ratio of file
size and number of lines to execution time}{execution time vs file size vs
number lines factor}

\showtable{build/stats-normal-filesize-line-count-include}{Ratio of file
size \& number of lines to execution time: statistics}{execution time vs
file size vs number lines factor table}

\clearpage

\subsection{Rule ordering}

\label{rule ordering graphs}

\showgraph{build/plot-normal-shuffle-factor}{Percentage increase of
shuffled over normal}{percentage increase of shuffled over normal}

\showgraph{build/plot-normal-reverse-factor}{Percentage increase of
reversed over normal}{percentage increase of reversed over normal}

\showtable{build/stats-normal-shuffle-reverse-include}{Execution time
increase for different rule orderings}{Execution time increase for
different rule orderings}

\subsubsection{Why are there dips in the graphs?}
\label{Why are there dips in the graphs?}

The dips at log files 22 and 62--68 correspond to peaks in log file size in
\graphref{execution time vs file size vs number of lines graph},
and peaks in \graphref{execution time vs file size vs number lines
factor} (where a peak means that more lines are processed per second, i.e.\
performance is better).  The explanation for this took some time to arrive
at, but it turns out to be reasonably simple.  The large log files in
question were caused by a mail forwarding loop, where the distribution of
log lines is quite different to normal, resulting in different performance
characteristics.

The mail loop was set up by a user modifying his mail forwarding to:
\newline \tab{}\texttt{$\backslash$username, username@domain} \newline This
instructs Postfix to deliver the mail to the local user, and also forward
it to the remote address; this is generally not a problem except that the
remote address in this case is the user's address, creating an infinite
loop.  To prevent this happening Postfix examines the Delivered-To header
in the mail, and if the mail has already been delivered to the current
address it is bounced back to the sender with the error message
\texttt{mail forwarding loop for username@domain}.  Ordinarily this works
well, but unfortunately in this case the user noticed they had not received
any mail in a while and opted to send a test mail to themselves, causing a
loop not caught by Postfix as described below.

\begin{enumerate}

    \item Postfix accepts a mail from username@domain, for username@domain.

    \item Postfix delivers the mail to the local mailbox and
        username@domain, as instructed by the user's forwarding
        instructions. The forwarded mail has a
        \texttt{Delivered-To:~username@domain} header added, and the
        envelope sender address is username@domain.  Log lines are added by
        \daemon{local}, \daemon{qmgr} (twice), \daemon{cleanup} and finally
        \daemon{pickup}.

    \item Postfix accepts the mail for username@domain, but while
        delivering it notices that the \texttt{Delivered-To} header already
        contains the address it is currently delivering to, and therefore
        sends a bounce notification with sender address \textit{$<>$\/} to
        the original sender: username@domain.  Log lines are added by
        \daemon{local}, \daemon{qmgr} (twice) and \daemon{cleanup}.

    \item Postfix accepts the bounce notification and delivers it to both
        the local mailbox and to username@domain, as instructed by the
        user's forwarding instructions.  A
        \texttt{Delivered-To:~username@domain} header is added to the
        forwarded bounce notification, which now has an envelope sender
        address of username@domain.  Log lines are added by \daemon{local},
        \daemon{qmgr} (twice), \daemon{cleanup}, and finally
        \daemon{pickup}.

    \item Postfix accepts the forwarded bounce notification but while
        delivering the mail it notices that the \texttt{Delivered-To}
        header already contains the address currently being delivered to,
        and sends a bounce notification to the sender: username@domain.
        Log lines are added by \daemon{local}, \daemon{qmgr} (twice) and
        \daemon{cleanup}.

    \item Repeat from step two; this will continue indefinitely unless an
        administrator intervenes and deletes the appropriate mails from the
        queue.

\end{enumerate}

The sequence described above occurs extremely rapidly because Postfix does
not have to deliver the mail to an external system, so mails are delivered,
bounced and generated as fast as the disks can keep up, resulting in a huge
volume of log lines.

The vast majority of log lines when a mail loop occurs are from Postfix
components which have a small number of rules associated with them, whereas
in general \daemon{smtpd} adds the majority of log lines, and also has the
highest number of rules.  \daemon{smtpd} log lines are distributed across
rules much more evenly than the log lines of \daemon{qmgr}, \daemon{local},
\daemon{cleanup} or \daemon{pickup}, so the average number of rules
required to parse a \daemon{smtpd} log line is much higher that the average
number required to parse other log lines.

These two characteristics combine to reduce the average number or rules
required to parse a log line when there is a mail loop, as shown by the
peaks in \graphref{execution time vs file size vs number lines
factor}.  When the rule ordering is reversed the majority of log lines
generated by a mail loop will be parsed with very few rules, whereas
without a mail loop the majority of log lines require a large number of
rules; this leads to a noticeable drop in the average time required to
parse a log line, as shown in \graphref{percentage increase of
reversed over normal}.  The number of rules which need to be consulted when
the ordering is shuffled varies between the optimum and nadir, and the
performance varies proportionally.

The difference between log files with a mail loop and log files without can
be seen in \tableref{Execution time increase for different rule
orderings} showing the increases for the different rule orderings and
combinations of log files:



\subsection{Caching regexes}

\label{Caching regexes}

The following graphs show the impact that not caching compiled \regexes{}
has on parser performance: on typical log files the execution time when not
caching compiled \regexes{} is 500--600\% of the execution time when
caching; reversing the perspective shows that cached execution time is
merely 17--20\% of non-cached execution time.  Caching compiled \regexes{}
is probably the single most effective optimisation possible in the parser's
implementation; given that it only required sixteen extra lines of
reasonably simple code,\footnote{The sixteen lines of code breaks down as
follows: three lines to add caching, eight lines of error checking and
reporting, three lines to optionally disable caching for debugging and
performance measurement, and two lines of comments.} the investment in time
was minimal.

\showgraph{build/plot-cached-discarded}{Regex: cached vs
discarded}{normal regex vs discard regex}

\showgraph{build/plot-cached-discarded-factor}{Regex caching: percentage
execution time increase}{normal regex vs discarded regex factor}

Two large dips can be seen in \graphref{normal regex vs discarded
regex factor} at log files 22 and 62--68, corresponding to the spikes in
log file size in graphs~\refwithpage{execution time vs file size vs number
of lines graph} and~\refwithpage{normal regex vs discard regex}.  The
reason for the anomalous log files has already been explained in
\sectionref{Why are there dips in the graphs?}.

The distribution of log lines across rules when there is a mail loop is
much different and the average number of rules consulted per log line is
much lower; this results in far fewer \regex{} compilations per line than
when there is not a mail loop, and a correspondingly decreased execution
time.  The increases in execution time when not caching \regexes{} for
different combinations of log files are summarised in the table below:

\showtable{build/stats-cached-discarded-include-for-graph}
{Regex caching/discarding with different groups of log files}
{Regex caching/discarding with different groups of log files}


\subsection{Rule hits}
\label{rule hits}

The number of hits per rule is quite unevenly spread, resembling a Power
Law distribution~\cite{powerlaw}.

\showgraph{build/plot-hits}{Hits per rule}{rule hits graph}

As \graphref{rule hits graph} is quite difficult to read it has
been separated into three sections: low, middle and high.

\showgraph{build/plot-hits-low}{Hits per rule (low)}{hits per rule low}

It is apparent from the low graph (\graphref{hits per rule low})
that some rules have few or no hits; those with zero hits are rules which
were written to parse log files used during development of the parser but
not utilised in the test runs performed for this document.

\showgraph{build/plot-hits-middle}{Hits per rule (middle)}{hits per rule
middle}

\showgraph{build/plot-hits-high}{Hits per rule (high)}{hits per rule
high}

\clearpage



\subsection{Miscellaneous graphs}

\label{Miscellaneous graphs}

\showgraph{build/plot-mails-received}{Mails received per day}{Mails
received per day}

As expected there are far more mails received on weekdays than at weekends.
Note that this graph and the table below show the number of mails received
by \SMTP{} only; in particular the mail loops noticeable in other graphs
do not contribute to these figures.

\showtable{build/mails-received-include-for-graph}{Number of mails received
per day: statistics}{Number of mails received per day: statistics}

\showgraph{build/plot-action-distribution}{Distribution of rules per
action}{Distribution of rules per action}

\clearpage

\subsection{Summary}

The graphs presented in this section illustrate the topics discussed in
\sectionref{rule efficiency}.  The first collection of graphs are about
parser scalability, showing the linear relationship between execution time
and input size.  \sectionref{rule ordering graphs} demonstrates the effect
of rule ordering on execution time, and the unexpected consequences of
specific inputs.  The necessity of caching compiled \regexes{} is attested
to by the third group of graphs, where the difference between caching and
discarding is staggering.  The penultimate section contains a breakdown of
the rule hits accumulated during a single test run of the parser.
Miscellaneous graphs expected to be useful are collected in the final
section.  
