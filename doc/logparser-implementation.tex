\chapter{Postfix Parser Implementation}

\label{Postfix Parser Implementation}

This chapter documents the implementation of \parsername{}, a parsed based
on the architecture described in \sectionref{parser architecture}.  Any
design may look good when examined in the abstract, but the real test of
the design comes with the first concrete implementation; only then do any
practical difficulties come to light.  Implementing this architecture was
relatively straightforward: the difficulties came not from the architecture
and design, but from anomalies in the log files and Postfix's behaviour, as
described in \sectionref{complications}.  \parsername{} successfully deals
with all the difficulties that were discovered during its development, but
further difficulties may arise during future usage; unfortunately solving
all potential difficulties that may arise is an intractable task, but the
descriptions of the solutions developed thus far should help if someone
needs to develop a solution of their own.

\parsername{} deals with all the eccentricities and oddities of parsing
Postfix log files, presenting the resulting data in a normalised, simple to
use representation.  Unfortunately, dealing with the complications that
arise sometimes requires the parser to discard log lines (e.g.\
\sectionref{aborted delivery attempts}), and, less frequently, to discard a
data structure (e.g.\ \sectionref{timeouts during data phase}).
\parsername{} can parse log files from Postfix version 2.2 through to
version 2.5, and should be capable of parsing log files from later versions
with only minor modifications or an updated ruleset.  The parser's task is
to follow the journey each mail takes through Postfix, combining the data
captured by rules into a coherent whole, saving it in a useful and
consistent form, and performing housekeeping duties.  The intermingling of
log lines from different mails immediately rules out the possibility of
handling each mail in isolation; the parser must handle multiple mails in
parallel, each potentially at a different stage in its journey, without any
interference between mails --- except in the minority of cases where
intra-mail interference is required, e.g.\ mail re-injected for forwarding
(\sectionref{Re-injected mails}).  The best way to deal with intermingling
of log lines is to maintain state information for every unfinished mail,
and to manipulate the appropriate mail correctly for each log line
encountered.

This chapter begins with the assumptions under which \parsername{} was
designed and written, followed by a flowchart showing the most common paths
taken through Postfix and \parsername{}, with a description of the stages
and stage transitions.  The second topic is the \acronym{SQL} database that
provides storage for the parser: any future work that analyses gathered
data will do so using the database, so the database schema acts as an
\acronym{API}.  A diagram of the database schema is provided, plus
documentation for every table and field.

The next three sections document the implementation of the three components
of the architecture.  First is the framework, including its initialisation
phase, the parsing process, and the conveniences it offers to users of
\parsername{}; that section finishes with the features that are important
for this thesis but not strictly necessary for parsing: the performance
data collected by the framework, the optimisations that can be disabled to
show their effect, and the debugging options it provides.  The second
component of the architecture is the actions, starting with a graph of how
often each action is specified by rules, a description of why some actions
are more popular than others, and how this popularity has no influence on
how often an action is invoked.  Every action that is part of \parsername{}
is documented, and the actions section concludes with a description of the
process of adding a new action.  Rules are the final component of the
architecture, and also the most visible to advanced users, e.g.\ systems
administrators, because it is likely that they will need to add their own
rules to recognise their log lines.  An example rule used by \parsername{}
is examined, with every field clearly documented; that is followed by a
description of adding new rules, and how to determine the value of each of
the fields.  \parsername{} provides a utility to create regexes from
unrecognised log lines; the algorithm it uses is documented, including the
differences between it and the original algorithm it is based on.  The
rules section finishes with a discussion of how \parsername{} uses rule
conditions and overlapping rules, plus a description of the regex snippets
provided to ease the process of writing regexes.

On first inspection, Postfix log files look like they will be simple to
parse, but this impression turns out to be incorrect.  The many
complications and difficulties encountered while writing \parsername{}, and
the solutions developed to overcome them, are documented in
\sectionref{complications}.  This chapter finishes with a list of
\parsernames{} limitations, and some possible improvements that could be
implemented.

\section{Assumptions}

\parsername{} makes a small number of (hopefully safe and reasonable)
assumptions:

\begin{itemize}

    \item The log files are whole and complete: nothing has been removed,
        either deliberately or accidentally (e.g.\ log file rotation gone
        awry, file system filling up, logging system unable to cope with
        the volume of log messages).  On a well run mail server it is
        extremely unlikely that any of these problems will arise, though
        the likelihood increases when suffering from a deluge of spam or a
        mail loop.  When parsing individual log files in isolation, it is
        highly likely that some mails will have log lines in previous log
        files, and others will have log lines in subsequent log files; to
        alleviate this problem \parsername{} supports saving and loading
        its state tables, so they can be preserved between log files.

    \item Postfix logs enough information to make it possible to accurately
        reconstruct the actions it has taken.  Heuristics are used in
        several places when parsing; see \sectionref{identifying bounce
        notifications}, \sectionref{aborted delivery attempts}, and
        \sectionref{pickup logging after cleanup} for details.  At least
        one difficulty encountered while writing \parsername{}
        (\sectionref{yet more aborted delivery attempts}) could not be
        solved using the data in the log files, and requires a brute force
        approach.

    \item The Postfix queue has not been tampered with, causing unexplained
        appearance or disappearance of mail.  This may happen if the
        administrator deletes mail from the queue without using
        \daemon{postsuper}, or if the server suffers from filesystem
        corruption.

\end{itemize}

In some ways this task is similar to reverse engineering or replicating a
black box system based solely on its inputs and outputs.  Thus far,
analysis of the log files has been enough to reconstruct Postfix's
behaviour, but for other programs the techniques described in
\cite{black-box-error-reporting} may be useful.  Some advantages come from
treating Postfix as a black box during parser development:

\begin{itemize}

    \item Reading and understanding the source code would require a
        significant investment of time: Postfix 2.5.5 has 17MB of source
        code.  Each subsequent version would require further work to
        investigate the changes; many of those changes, although they
        improve Postfix's internals, would not have any effect on its
        externally observable behaviour.

    \item \parsername{} is developed using real world log files rather than
        the idealised log files someone would naturally envisage when
        reading the source code, which cannot accurately communicate the
        variety of orderings in which log lines are found in log files.

    \item The parser acts as a second source of information about Postfix's
        operation, based on empirical evidence from log files.  A separate
        project could compare the empirical knowledge inherent in
        \parsername{} with Postfix's documentation and source code to see
        how closely the two agree.

\end{itemize}



\section{Parser Flow Chart}

\label{flow chart}

The flow chart in \figureref{flow chart image} shows the most common paths
a connection or mail can take through \parsername{}; decision boxes and the
difficulties described in \sectionref{complications} are excluded for the
sake of clarity.  The flow chart is intended to be a graphical overview of
how a mail progresses through both Postfix and \parsername{}, providing an
overall context into which the detailed descriptions in the remainder of
this chapter will fit, in particular the actions (\sectionref{actions in
implementation}) and complications (\sectionref{complications}).  The
states and state transitions shown in the flow chart will be explained
later in this section.

\showgraph{build/logparser-flow-chart-part-1}{\parsernamelong{} flow
chart}{flow chart image}

Everything starts off with a mail entering the system, whether by local
submission via \daemon{postdrop}, by \acronym{SMTP}, by re-injection
because of forwarding, or generated internally by Postfix.  Local
submission is the simplest of the four: a queueid is assigned immediately
\flowchart{PICKUP}{2}, and the mail moves on to the delivery stage.
Re-injection because of forwarding lacks explicit log lines of its own; it
is explained fully in \sectionref{Re-injected mails}.  Internally generated
mails lack any explicit origin in Postfix 2.2.x and must be detected using
heuristics as described in \sectionref{identifying bounce notifications};
later versions of Postfix do provide log lines for internally generated
mails \flowchart{BOUNCE\_CREATED}{3}.  Bounce notifications are the primary
example of internally generated mails, though other types exist, e.g.\
Postfix may generate mails to the administrator when it encounters
configuration errors.

\acronym{SMTP} is more complicated than the others:

\begin{enumerate}

    \item The remote client connects \flowchart{CONNECT}{1}.

    \item This is followed by rejection of one or more mail delivery attempts
        \flowchart{DELIVERY\_REJECTED}{4}; acceptance of one or more mails
        \flowchart{CLONE}{5}; failure of the remote client's connection
        \flowchart{DELIVERY\_ERROR}{6}; or some random interleaving of two
        or more of the above.

    \item The client disconnects \flowchart{DISCONNECT, TIMEOUT}{6}, either
        normally or with an error.  If Postfix has rejected any
        \acronym{SMTP} commands the data gathered from those rejections
        will be saved to the database; if there were no rejections there
        will not be any data to save.  Any mails accepted will already have
        been cloned so their data is in another data structure, and will be
        delivered in the same way as mails that entered the system by any
        other route.

\end{enumerate}

The obvious counterpart to mail entering the system is mail leaving the
system, whether by deletion, bouncing, expiry, local delivery, or remote
delivery.  All five are handled in the same way:

\begin{enumerate}

    \item The mail will have one or more delivery attempts
        \flowchart{MAIL\_SENT, SAVE\_DATA}{8}.

    \item Sometimes mail is re-injected for forwarding and the child mail
        needs to be tracked with the parent mail \flowchart{TRACK}{9}; the
        handling of re-injected mails is described in \sectionref{tracking
        re-injected mail}.

    \item After one or more delivery attempts the mail will be delivered
        \flowchart{MAIL\_SENT}{8}, bounced \flowchart{MAIL\_BOUNCED}{8},
        expired \flowchart{EXPIRY}{8}, or deleted by the administrator
        \flowchart{DELETE}{8}.

    \item The mail is removed from the Postfix queue.  This is the last log
        line for this particular mail, though it may be indirectly referred
        to if it was re-injected.  The mail is cleaned up and entered in
        the database, then deleted from the state tables
        \flowchart{COMMIT}{11}.

\end{enumerate}

It should be emphasised that the sequence above happens whether the mail is
delivered to a mailbox, piped to a command, delivered to a remote server,
bounced (because of a mail loop, delivery failure, or five day timeout), or
deleted by the administrator, \textit{unless\/} the mail is either parent
or child of re-injection, as explained in \sectionref{tracking re-injected
mail}.

\section{Database}

\label{database}

A database is used to store both the rules and the data gleaned by parsing
Postfix log files.  Understanding the database schema is important in
understanding the actions of the parser, and essential to developing
further applications that utilise the data; \sectionref{database as API}
describes how the database schema functions as an \acronym{API}.

The database schema can be conceptually divided in two: the rules used to
recognise log lines, and the data saved from the parsing of log files.
Each rule has a regex to recognise log lines and capture data from them,
and specifies the action to be invoked when a log line is recognised; they
also have several fields that aid the user in understanding the meaning of
the log lines recognised by each rule.  The rules are described in detail
in \sectionref{rules in implementation}, but the rules table is documented
in \sectionref{rules table} with the rest of the database schema.

The data saved from parsing the log files is divided into two tables:
connections and results.  The connections table contains an entry for every
mail accepted and every connection that rejected a delivery attempt; the
individual fields will be described in \sectionref{connections table}.  The
results table will have at least one entry for each entry in the
connections table; its fields will be covered in detail in
\sectionref{results table}.  A diagram of the database schema is provided
in \figureref{Diagram of the database schema picture}, to complement the
in-depth descriptions of each table.

An important but easily overlooked benefit of storing the rules in the
database is the link between rules and results: if more information is
required when examining a result, the rule that produced the result is
available for inspection because each result references the rule that
created it.  No ambiguity is possible about which rule resulted in a
particular result, eliminating one potential source of confusion.

A clear, comprehensible schema is essential when using the data extracted
from log lines; it is more important when using the data than when storing
it, because storing the data is a once-off operation, whereas utilising
the data requires frequent searching, sorting, and manipulation of the data
to produce customised reports or statistics.

\subsection{Using A Database To Provide An Application Programming Interface}

\label{database as API}

The database populated by \parsername{} provides a simple interface to
Postfix log files.  Although the interface is a database schema rather than
a set of functions in a library, it provides the same benefits as any other
\acronym{API}: a stable interface between the user and the creator of the
data, allowing code on either side of the interface to be changed without
adverse effects, as long as the interface is adhered to.  Programs that use
the database can range from the simple examples in \sectionref{motivation}
to far more complex data mining tools and machine learning algorithms.

Using a database simplifies writing programs that need to interact with the
data in several ways:

\begin{enumerate}

    \item Most programming languages have facilities for database access,
        allowing a developer to write programs that use the gathered data
        in their preferred programming language, rather than being
        restricted to the language the parser is written in.

    \item Databases provide complex querying and sorting functionality for
        the user without requiring large amounts of programming.  All
        databases have one or more programs, of varying complexity and
        sophistication, that can be used for ad hoc queries with minimal
        investment of time.

    \item Databases are easily extensible, e.g.:

        \begin{itemize}

            \item New columns can be added to the tables used by the
                program, using DEFAULT clauses or TRIGGERS to populate
                them.

            \item A VIEW gives a custom arrangement of data with minimal
                effort.

            \item Triggers can be written to perform actions when certain
                events occur.  In pseudo-\acronym{SQL}\@:

\begin{verbatim}
CREATE TRIGGER ON INSERT INTO results
    WHERE sender = "boss@example.com"
        AND rule_id = rules.id
        AND rules.action = "DELIVERY_REJECTED"
    SEND PANIC EMAIL TO "postmaster@example.com";
\end{verbatim}

            \item Other tables can be added to the database, e.g.\ to cache
                historical or computed data, or to incorporate data from
                other sources.

        \end{itemize}

    \item Some databases support granting access on a fine-grained basis,
        e.g.\ allowing the finance department to produce invoices, the
        helpdesk to run limited queries as part of dealing with support
        calls, and the administrators to have full access to the data.


    \item \acronym{SQL} is reasonably standard and many people will already
        be familiar with it; for those unfamiliar with \acronym{SQL}, lots of
        resources are available from which to learn, e.g.\
        \urlLastChecked{http://philip.greenspun.com/sql/}{2009/02/23}.
        Although every vendor implements a different dialect of
        \acronym{SQL}, the basics are the same everywhere.  Depending on
        the database in use there may be tools available that reduce or
        remove the requirement to know \acronym{SQL}.

\end{enumerate}

Storing the results in a database will also increase the efficiency of
using those results, because the log files only need to be parsed once
rather than each time the data is used; indeed the database may be used by
someone with no access to the original log files.

\subsection{Rules Table}

\label{rules table}

\newlength{\belowcaptionskipORIG}
\setlength{\belowcaptionskipORIG}{\belowcaptionskip}
\setlength{\belowcaptionskip}{10pt}
\showgraph{build/database-schema}{Diagram of the database schema}{Diagram
of the database schema picture}
\setlength{\belowcaptionskip}{\belowcaptionskipORIG}

Rules are discussed in detail in \sectionref{rules in implementation}, but
the structure of the rules table is documented here alongside the other
tables in the database.  Rules are created by the user, and will not be
modified by \parsername{}, except when it updates the hits and hits\_total
fields.  Rules recognise the individual log lines, capturing data to be
saved in the connections and results tables, and specifying the action to
invoke for each recognised log line.

Each rule must provide values for most of the fields below; all fields are
required unless otherwise stated in the description.

\begin{boldeqlist}

    \item [id] A unique identifier that other tables can use when referring
        to a specific rule.

    \item [name] A short name for the rule.

    \item [description] This field should describe the event causing the
        log lines this rule recognises, e.g.\ ``Mail has been delivered to
        the LDA (typically procmail)''.

    \item [restriction\_name] The name of the Postfix restriction that
        caused the rejection of the mail delivery attempt.  This field is
        valid only for rules that recognise rejection log lines, i.e.\
        rules that have an action of \action{DELIVERY\_REJECTED}.

    \item [program] The Postfix component (e.g.\ \daemon{smtpd}) whose log
        lines the rule recognises; see \sectionref{rule conditions in
        implementation} for full details of how this attribute is used.

    \item [regex] The regex to recognise log lines with, as documented in
        \sectionref{regex components}.

    \item [connection\_data] Sometimes rules need to provide data that is
        not present in the log line, e.g.\ setting \texttt{client\_ip} when
        a mail is being delivered to another server; any field in the
        connections table can be set in this way.  The format is:
        \newline{} \tab{} \texttt{ client\_hostname = localhost,}
        \newline{} \tab{} \tab{} \texttt{client\_ip = 127.0.0.1} \newline{}
        i.e.\ semi-colon or comma separated assignment statements.  Commas
        and semi-colons cannot be escaped and thus cannot be included in
        data, because this feature is intended for use with small amounts
        of data and dealing with escape sequences was deemed unnecessary.
        This field is optional.

    \item [result\_data] The result table equivalent of
        \texttt{connection\_data}, also optional.

    \item [action] The action to be invoked when this rule recognises a log
        line; a full list of actions and the parameters they are invoked
        with can be found in \sectionref{actions in detail in
        implementation}.

    \item [hits] This counter is maintained for every rule and incremented
        each time the rule successfully recognises a log line.  At the
        start of each run \parsername{} sorts the rules by hits, and at the end of the run it updates
        every rule's hits field in the database.  Assuming that the
        distribution of log lines is reasonably consistent across log
        files, ordering rules by their recognition frequency will reduce
        the parser's execution time.  Rule ordering for efficiency is
        discussed in \sectionref{rule ordering for efficiency}.  This field
        will be set by the parser rather than the rule author.

    \item [hits\_total] The total number of log lines recognised by this
        rule over all runs of the parser; hits is reset to zero each time
        the parser is run, but hits\_total is not.  This field will be set
        by the parser rather than the rule author.

    \item [priority] This is the user-configurable companion to hits: when
        the list of rules is sorted by the parser, priority overrides hits.
        This allows more specific rules to take precedence over more
        general rules, as described in \sectionref{rules in architecture}.
        This field is optional.

    \item [debug] If this field is true, a warning will be issued with
        information about the rule and the log line every time this rule
        recognises a log line.  This field is optional.

\end{boldeqlist}

\subsection{Connections Table}

\label{connections table}

Every accepted mail and every connection that rejected a mail delivery
attempt will have a single entry in the connections table containing all of
the following below.  For an incoming connection, the client is the remote
machine, and the server is the local machine; for outbound mail delivery
attempts, the roles are reversed.

\begin{boldeqlist}

    \item [id] This field uniquely identifies the row.

    \item [server\_ip] The IP address of the server.

    \item [server\_hostname] The hostname of the server, it will be
        \texttt{unknown} if the IP address could not be resolved to a
        hostname via DNS\@.

    \item [client\_ip] The client IP address.

    \item [client\_hostname] The hostname of the client, it will be
        \texttt{unknown} if the IP address could not be resolved to a
        hostname via DNS\@.

    \item [helo] The hostname used in the HELO command.  The HELO hostname
        occasionally changes during a connection, presumably because spam
        or virus senders think it is a good idea.  By default Postfix only
        logs the HELO hostname when it rejects a mail delivery attempt, but
        it is quite simple to rectify this as described in
        \sectionref{logging helo}.

    \item [queueid] The queueid of the mail if the connection represents an
        accepted mail, or \texttt{NOQUEUE} if not.

    \item [start] The timestamp of the first log line.

    \item [end] The timestamp of the last log line.

\end{boldeqlist}

\subsection{Results Table}

\label{results table}

Every recognised log line will have a row in the results table, associated
with a single connection; a single connection will have at least one result
associated with it, but will usually have several, and may have hundreds.

\begin{boldeqlist}

    \item [connection\_id] The id of the row in the connections table this
        result is associated with.

    \item [rule\_id] The id of the rule in the rules table that recognised
        the log line.

    \item [id] A unique identifier for this result.

    \item [warning] Administrators can configure Postfix to log a warning
        instead of enforcing a restriction that would reject a mail
        delivery attempt --- a mechanism that is quite useful for testing
        new restrictions.  This field will be false for a real rejection,
        or true if the log line was a warning.  This field should be
        ignored if the result is not a rejection, i.e.\ the action field of
        the associated rule is not \action{DELIVERY\_REJECTED}.

    \item [smtp\_code] The \acronym{SMTP} code associated with the log
        line.  In general an \acronym{SMTP} code is only present in a log
        line representing a rejection or final delivery; results whose log
        line did not contain an \acronym{SMTP} code will duplicate the
        \acronym{SMTP} code of other results in that connection.  Some
        final delivery log lines do not contain an \acronym{SMTP} code,
        e.g.\ when Postfix delivers to a user's mailbox; in those cases the
        \acronym{SMTP} code is specified by the rule's
        \texttt{result\_data} field, based on the success or failure
        represented by the log line.

    \item [enhanced\_status\_code] The enhanced status code~\cite{RFC3463}
        is similar to the \acronym{SMTP} code, but is intended to be
        interpreted by mail clients so that error messages can be clearly
        conveyed to the user.  Enhanced status code support was added to
        Postfix in version 2.3; log lines from previous versions will not
        contain any enhanced status codes.

    \item [sender] The sender's email address.  This may change during a
        connection if the client uses different sender addresses for
        multiple rejected delivery attempts; however there will not be
        different sender addresses in the results for one accepted mail.

    \item [recipient] The recipient address; there may be multiple
        recipient addresses per mail or connection.

    \item [size] The size of the mail, only available for delivered mails.

    \item [delay] How long the mail was delayed while it was being
        delivered.  This will only be present for delivered mails.

    \item [delays] More detailed information about how long the mail was
        delayed while it was being delivered, again only present for
        delivered mails.

    \item [message\_id] The message-id of the accepted mail, again present
        for delivered mails only.

    \item [data] A field available to store a piece of captured data that
        does not have its own specific field, e.g.\ the rejection message
        from a \acronym{DNSBL}\@.

    \item [timestamp] The log line's timestamp.

\end{boldeqlist}



\section{Framework}

The role of the framework in this architecture was described in detail in
\sectionref{framework in architecture}; this section is concerned with the
implementation of the framework within \parsername{}.  The framework
manages the parsing process, taking care of the drudgery and boring tasks,
provides services to the actions, and implements some efficiency measures.

Each time the parser is run the framework performs some initialisation
tasks, setting up several state tables that will later be used by actions.
Data about the connections and mails being processed is held in
\texttt{connections} and \texttt{queueids} respectively.  The other state
tables are used when working around complications in parsing:
\texttt{timeout\_queueids} is used when dealing with connections that time
out during the DATA phase (\sectionref{timeouts during data phase});
\texttt{bounce\_queueids} is part of the solution to bounce notification
mails being delivered before their creation is logged (\sectionref{Bounce
notification mails delivered before their creation is logged}); and
\texttt{postsuper\_deleted\_queueids} caches information about mails that
were recently deleted by the administrator so that subsequent log lines
processed by the \action{SAVE\_DATA} action can be discarded
(\sectionref{Mail deleted before delivery is attempted}).

The framework verifies each of the rules when it loads the ruleset,
checking:

\begin{itemize}

    \squeezeitems{}

    \item That the specified action is registered with the framework.

    \item That the regex is valid.  The regex will have regex components
        expanded (see \sectionref{regex components}), and will also be
        compiled for efficiency (\sectionref{Caching compiled regexes}).

    \item For overlap between the data captured by the regex and additional
        data specified in either \texttt{connection\_data} or
        \texttt{result\_data}.

    \item That \texttt{connection\_data}, \texttt{result\_data}, and the
        regex captures specify valid fields to save data to.

\end{itemize}

State tables from a previous \parsername{} run, if any, will be loaded now;
the framework supports saving state at any time, without adversely
affecting the parsing process.  The need to track re-injected mail
(\sectionref{Re-injected mails}) complicates the process of saving state,
because the relationships between mails must be maintained; loading state
is also complicated by dealing with aborted delivery attempts
(\sectionref{aborted delivery attempts}), because a separate set of
relationships between connections and mails must be re-created when the
state tables are restored.  The last step in loading the ruleset is to sort
the rules as described in \sectionref{rule ordering for efficiency}.

The framework is now ready to begin parsing.  For each log line it will use
those rules whose \texttt{program} field equals the program in the log line
(previously described in \sectionref{rule conditions in implementation}),
falling back to generic rules if necessary, and finally warning if the log
line is unrecognised.  The repetitive nature of log files gives them high
compression ratios; the framework uses a Perl module named
\texttt{IO::Uncompress::AnyUncompress} to read compressed log files, saving
users the trouble of uncompressing them to a temporary file before parsing
begins.  When used interactively, the framework displays a progress bar to
show how far parsing has progressed through the log file and how long the
remainder of the parsing process is likely to take.  The progress bar is
not as accurate when parsing compressed log files, because a compressed
block uncompresses to a variable number of log lines; variation between the
recognition and processing time of individual log lines also affects its
accuracy, but overall the progress bar is a useful addition.  \parsername{}
is primarily intended for parsing complete log files, but with minor
modifications it could be used to parse a live log file, periodically
checking it for new log lines; this could be very useful for programs that
work best with up to date data, e.g.\ a program for monitoring the health
of the mail system, or graphs showing activity over the last five minutes.

The framework collects data used when evaluating \parsernames{} efficiency
for the evaluation chapter (\sectionref{Evaluation}); some of the
techniques used to improve parsing speed can be turned off or altered to
measure the effect they have.  The data gathered with each optimisation
disabled will be compared to the data gathered with it enabled, to quantify
the benefit each optimisation provides.  Three sets of data are collected:

\begin{enumerate}

    \item How long it takes to parse each log file (\graphref{parsing time
        vs log file size vs number of log lines graph}).

    \item The number of rules used when recognising log lines
        (\graphref{Mean number of rules used per log line}); the framework
        may need to try multiple rules for each log line before it finds
        one that recognises it.

    \item The number of log lines and rules used for each Postfix component
        (\graphref{Mean number of rules used per log line for each Postfix
        component}).

\end{enumerate}

The framework has five ways that it can adapt its behaviour to demonstrate
how effective each optimisation is:

\begin{enumerate}

    \item The rule ordering used can be changed from optimal (the default,
        most efficient) to shuffled (intended to represent an unordered
        ruleset) or reverse (reverse of optimal, least efficient).  The
        results of this are shown in \sectionref{rule ordering for
        efficiency}.

    \item The framework can record which rule recognised each log line, and
        then on a subsequent run consult that information so that it uses
        the correct rule for each log line.  This gives the best possible
        running time, because only one rule is used to recognise each log
        line, as discussed in \sectionref{perfect rule ordering}.

    \item Each regex is compiled and the result cached when the ruleset is
        loaded; this optimisation can be disabled and the regex compiled
        every time it is used when recognising log lines.  The effect of
        this optimisation is shown in \sectionref{Caching compiled
        regexes}.

    \item Normally, when a log line has been recognised the framework
        invokes the action specified by the rule.  Invoking the action can
        be skipped if desired, so the timing information shows how long is
        spent recognising log lines only; this time can be subtracted from
        the time taken by a normal run to show how long is spent processing
        log lines.  This data is analysed in \sectionref{recognising vs
        processing}.

    \item The framework can skip inserting data into the database, because
        that dominates the execution time of the parser; the evaluation
        chapter measures the speed of \parsername{}, not the speed of the
        database or the disks it resides on.  The effect on parsing time of
        storing results in the database is described at the beginning of
        \sectionref{parser efficiency}.

\end{enumerate}

The framework also provides several debugging options, to aid in writing or
correcting rules, or figuring out why the parser is not behaving as
expected.  In increasing order of severity they are:

\begin{enumerate}

    \item Individual rules can set their debug field to true, and debugging
        information will be printed each time they recognise a log line.

    \item Each result can be extended with extra debugging information,
        which is useful when a warning dumps a mail's data for inspection.
        The extra information added is: the log line's timestamp in human
        readable form; the entire log line; the name of the log file and
        the number of the log line within it.  This information is not
        stored in the database.

    \item Every time a log line is recognised, the recognising rule's regex
        and the log line can be printed, so that the user can verify that
        the correct rule recognised each log line.  This is equivalent to
        setting every rule's debug field to true.

    \item Every connection and result added to the database can be dumped
        in a human readable form.  This will result in a huge amount of
        debugging information, so it is only useful for small log file
        snippets, because otherwise the amount of information is
        overwhelming.

\end{enumerate}

A user, typically a mail administrator, would use these options when having
difficulty extending the ruleset to recognise log lines not recognised by
\parsernames{} \numberOFrules{} rules.  Option 1 is useful for debugging a
single rule, to check if it is recognising log lines it should not;
figuring out that a rule is not recognising log lines that it should
recognise is a more difficult task.  Option 2 is useful when \parsername{}
is warning about parsing problems, because extra information about a mail
will be present in the warning message.  Option 3 is less useful because of
the volume of data it produces, though for small log file snippets, e.g.\
1000 log lines, it is possible to manually verify that the correct rule
recognised each log line.  Option 4 is not useful for most users: it is for
debugging problems with the data that is saved to the database, and
sufficient safeguards are in place that there should be plenty of warning
messages explaining what is wrong without using this option.

\section{Actions}

\label{actions in implementation}

Actions are the component of this architecture responsible for processing
all of the inputs recognised by rules; in \parsername{} they reconstruct
the journey each mail takes through Postfix, dealing with all the
complications and difficulties that arise.  \parsername{} has
\numberOFactions{} actions and \numberOFrules{} rules, with an uneven
distribution of rules to actions as shown in \graphref{Distribution of
rules per action}.  Unsurprisingly, the action with the most associated
rules is \action{DELIVERY\_REJECTED}, the action that processes Postfix
rejecting a mail delivery attempt; next is \action{SAVE\_DATA}, the action
that saves useful information without doing any other processing.  The
third most common action is, perhaps surprisingly, \action{UNINTERESTING}:
this action does nothing when invoked, allowing uninteresting log lines to
be parsed without any effects; it does not imply that the input is
ungrammatical or unparsed.  Generally rules specifying the
\action{UNINTERESTING} action recognise log lines that are not associated
with a specific mail, e.g.\ notices about configuration files changing;
these log lines are recognised and processed so that the framework can warn
about unrecognised log lines, informing the user that they need to extend
the ruleset.  Most of the remaining actions have only one or two associated
rules, because they will only ever have one or two log line variants, e.g.\
all log lines showing that a remote client has connected are recognised by
a single rule and processed by the \action{CONNECT} action.

No correlation exists between how often an action is specified by rules and
how often an action is invoked because a rule recognises a log line.
\Graphref{Number of times each action was invoked excluding 22 and 62--68}
shows the number of times each action was invoked when parsing the
\numberOFlogFILES{} log files used to generate the statistics in
\sectionref{parser efficiency}, \textit{excluding\/} log files 22 \&
62--68, because their content is extremely skewed by two mail loops.  The
action most commonly specified by rules, \action{DELIVERY\_REJECTED}, is
the third most commonly invoked action; the most commonly invoked actions,
\action{CONNECT} and \action{DISCONNECT}, are each specified by only one
rule.  The \action{UNINTERESTING} action, the third most commonly specified
action, is halfway down the graph.

As expected, some actions have been invoked almost exactly the same number
of times: almost every \action{CONNECT} will have a \action{DISCONNECT},
with only a 0.00042\% difference between the number of times the two were
invoked; every mail that enters Postfix's queue will be managed by
\daemon{qmgr} (\action{MAIL\_QUEUED}) and processed by \daemon{cleanup}
(\action{CLEANUP\_PROCESSING}), and again the two have only a 0.01648\%
difference between their number of invocations.
\action{CLEANUP\_PROCESSING} is invoked slightly more often than
\action{MAIL\_QUEUED}: occasionally, an accepted mail in the process of
being transferred is interrupted by a timeout, and there is a log line from
\daemon{cleanup} but not from \daemon{qmgr}, as described in
\sectionref{discarding cleanup log lines}.

\showgraph{build/graph-action-distribution}{Number of rules specifying each
action}{Distribution of rules per action}

\showgraph{build/graph-number-of-action-invocations-excluding-22-and-62--68}{Number
of times each action was invoked when parsing \numberOFlogFILES{} log
files, excluding log files 22 \& 62--68}{Number of times each action was
invoked excluding 22 and 62--68}

\subsection{Actions In The \parsernamelong{}}

\label{actions in detail in implementation}

This section documents the actions found in \parsername{}; it may help to
revisit the flow chart in \sectionref{flow chart} to see how a mail passes
from one action to another as its log lines are recognised.  The words
\textit{mail\/} and \textit{connection\/} are used in the actions
descriptions below because they are less unwieldy and more specific than
\textit{state table entry\/}; a connection becomes a mail during the
\action{CLONE} action, which processes Postfix accepting a delivery
attempt, and the data structure moves from the connections state table to
the queueids state table.

The complications and difficulties that arose when parsing real-world log
files are documented in \sectionref{complications}; some action
descriptions refer to specific difficulties they address.  The
complications are documented in a separate section to avoid overwhelming
the action descriptions.

If the log line has enough information to identify the correct connection
or mail, each action will save all the data captured by the recognising
rule's regex; usually, log lines that lack identifying information will be
processed by the \action{UNINTERESTING} action.  Each action is passed the
same arguments:

\begin{boldeqlist}

    \squeezeitems{}

    \item [rule] The recognising rule.

    \item [data] The data captured from the log line by the rule's regex.

    \item [line] The log line, separated into fields:

        \begin{boldeqlist}

            \squeezeitems{}

            \item [timestamp] The time the line was logged at.

            \item [program] The name of the program that generated the log
                line.

            \item [pid] The \acronym{pid} of the process that logged the
                line.

            \item [host] The hostname of the server the line was logged on.

            \item [text] The remainder of the log line, i.e.\ the message
                logged by the program and recognised by the rule.

        \end{boldeqlist}

\end{boldeqlist}

\begin{description}

    \item [BOUNCE\_CREATED] Postfix 2.3 and subsequent versions log the
        creation of bounce messages, and this action processes those log
        lines.  This action creates a new mail if necessary; if the mail
        already exists the unknown origin flag will be removed from it.
        The mail will be marked as a bounce notification.  To deal with
        complication \sectionref{Bounce notification mails delivered before
        their creation is logged}, this action checks a cache of recent
        bounce mails, to avoid incorrectly creating bounce mails when log
        lines are out of order.

    \item [CLEANUP\_PROCESSING] \daemon{cleanup} processes every mail that
        passes through Postfix; details of what it does are available in
        \sectionref{Postfix Daemons}.  This action saves all data captured
        by the rule's regex if the log line has not come after a timeout
        (see \sectionref{discarding cleanup log lines}); it also creates
        the mail if necessary, setting its unknown origin flag (see
        \sectionref{pickup logging after cleanup}).

    \item [CLONE] Multiple mails may be accepted on a single connection, so
        each time a mail is accepted the connection's state table entry
        must be cloned and saved in the state tables under its queueid; if
        the original data structure was used then second and subsequent
        mails would overwrite one another's data.

    \item [COMMIT] Enter the data from the mail into the database.  Entry
        will be postponed if the mail is a child waiting to be tracked
        (\sectionref{Re-injected mails}).  Once entered in the database,
        the mail will be usually be deleted from the state tables, but
        deletion will be postponed if the mail is the parent of mail
        re-injected for forwarding (\sectionref{Re-injected mails}).

    \item [CONNECT] Process a remote client connecting: create a new
        connection, indexed by \daemon{smtpd} \acronym{pid}.  If a
        connection already exists it is treated as a symptom of a bug in
        \parsername{}, and the action will issue a warning containing the
        full contents of the existing connection plus the log line that has
        just been parsed.

    \item [DELETE] Deals with mail deleted using Postfix's administrative
        command \daemon{postsuper}.  This action adds a dummy recipient
        address if required (see \sectionref{Mail deleted before delivery
        is attempted}), then invokes the \action{COMMIT} action to save the
        mail to the database.

    \item [DELIVERY\_ERROR] Process log lines indicating that an error
        occurred and the remote client disconnected; the log lines
        processed by this action will be followed by log lines processed by
        the \action{DISCONNECT} action, so all this action does is save
        data from the log line.

    \item [DELIVERY\_REJECTED] Postfix rejected a mail delivery attempt
        from the remote client.  This is the action most frequently
        specified by rules, because so many different restrictions are used
        to reject delivery attempts.  This action is quite simple: if the
        log line contains a queueid, save the data captured by the rule's
        regex to the mail identified by that queueid; otherwise save it to
        the connection identified by the \acronym{pid} in the log line.

    \item [DISCONNECT] Invoked when the remote client disconnects, it
        enters the connection in the database if it has any useful data,
        performs any required cleanup, and deletes the connection from the
        state tables.  This action deals with aborted delivery attempts
        (\sectionref{aborted delivery attempts}).

    \item [EXPIRY] If Postfix has not managed to deliver a mail after
        trying for five days, it will give up and return the mail to the
        sender.  When this happens the mail will not have a combination of
        Postfix programs that passes the valid combinations check,
        implemented to deal with the complication described in
        \sectionref{out of order log lines}; this action tags the mail as
        having expired, so the \action{COMMIT} action will skip the valid
        combinations check.

    \item [MAIL\_BOUNCED] This action behaves in exactly the same way as
        the \action{SAVE\_DATA} action; it saves all data captured by the
        recognising rule's regex, and does nothing more.  It is a separate
        action to distinguish delivery attempts that bounce from other
        delivery attempts.

    \item [MAIL\_DISCARDED] Sometimes mail is discarded by Postfix, e.g.\
        mail submitted locally that is larger than the limit configured by
        the administrator.  This action is used for those mails; it invokes
        the \action{COMMIT} action, but is a separate action to simplify
        further analysis.

    \item [MAIL\_QUEUED] This action represents Postfix picking a mail from
        the queue to deliver.  This action needs to deal with out of order
        log lines when mail is re-injected for forwarding; see
        \sectionref{Re-injected mails} for details.

    \item [MAIL\_SENT] This action behaves in exactly the same way as the
        \action{SAVE\_DATA} action; it saves all data captured by the
        recognising rule's regex, and does nothing more.  It is a separate
        action to distinguish successful delivery attempts from other
        delivery attempts.

    \item [MAIL\_TOO\_LARGE] When a client tries to send a mail larger than
        the local server accepts, the mail will be discarded by Postfix and
        the client informed of the problem.  The mail may have been
        accepted and partially transferred; if so the parser will have a
        data structure that must be disposed of.  See \sectionref{timeouts
        during data phase} for full details; although that describes
        timeouts, the processing is the same for mails that are too large.

    \item [PICKUP] The \action{PICKUP} action corresponds to the
        \daemon{pickup} service processing a locally submitted mail.  A new
        mail will usually be created, although out of order log lines may
        have caused it to already exist, as documented in
        \sectionref{pickup logging after cleanup}.

    \item [POSTFIX\_RELOAD] When Postfix stops running or reloads its
        configuration, it kills all \daemon{smtpd} processes, requiring all
        of the connections in \parsernames{} state tables to be cleaned up,
        entered in the database, and deleted from the state tables.
        Postfix probably kills all the other components too, but
        \parsername{} is only affected by \daemon{smtpd} processes exiting.

    \item [SAVE\_DATA] Every action that can locate the correct entry in
        the state tables saves any data captured by the recognising rule's
        regex to it.  The \action{SAVE\_DATA} saves data in this way but
        does not do anything else; it is invoked for log lines that contain
        useful data but do not require any further processing.

    \item [SMTPD\_DIED] Sometimes a \daemon{smtpd} dies, is killed by a
        signal, or exits unsuccessfully; the associated connection must be
        cleaned up, entered in the database if it has enough data, and
        deleted from the state tables.  Sometimes the connection will not
        have enough data to satisfy the database schema, so it cannot be
        entered into the database for future analysis; unfortunately this
        means that the small amount of data that has been gathered by
        \parsername{} will be lost.

    \item [SMTPD\_WATCHDOG] \daemon{smtpd} processes have a watchdog timer
        to deal with unusual situations; after five hours the timer will
        expire and the \daemon{smtpd} will exit.  This occurs very
        infrequently, because there are many other timeouts that should
        occur in the intervening hours, e.g.\ timeouts for DNS requests or
        timeouts reading data from the client.  The active connection for
        that \daemon{smtpd} must be cleaned up, entered in the database,
        and deleted from the state tables.

    \item [TIMEOUT] The connection with the remote client timed out, so the
        mail being transferred must be discarded by Postfix.  The mail may
        have been accepted: if so the parser will have a data structure to
        dispose of.  See \sectionref{timeouts during data phase} for full
        details.

    \item [TRACK] Track a mail when it is re-injected for forwarding to
        another mail server; this happens when a local address is aliased
        to a remote address (see \sectionref{tracking re-injected mail} for
        a full explanation).  \action{TRACK} will be called when dealing
        with the parent mail, and will create the child mail if necessary.
        \action{TRACK} checks if the child has already been tracked, either
        with this parent or with another parent, and issues appropriate
        warnings if so.

    \item [UNINTERESTING] This action just returns successfully: it is used
        when a log line needs to be recognised to avoid warning about
        unrecognised log lines, but does not either provide any useful data
        to be saved or require any processing.

\end{description}

\subsection{Adding New Actions}

\label{adding new actions in implementation}

Adding new actions is not as easy as adding new rules, though care has been
taken in the architecture and implementation to make adding new actions as
painless as possible; one of the few limitations is that \parsername{} is
written in Perl, so new actions must also be written in Perl.  The
implementer writes a subroutine that accepts the standard arguments given
to actions, and registers it with the framework by calling the framework's
\texttt{add\_actions()} subroutine.  The new action must be registered
before the rules are loaded, because it is an error for a rule to specify
an unregistered action; this helps catch mistakes made when adding new
rules.  No other work is required from the implementer to integrate the
action into the parser; all of their attention and effort can be focused on
correctly implementing their new action.  The action may need to extend the
list of valid combinations described in \sectionref{out of order log lines}
if the new action creates a different set of acceptable programs, but this
would only be necessary if the new action processes log lines from Postfix
components that \parsername{} does not have rules for, e.g.\
\daemon{virtual} or \daemon{lmtp}.

\section{Rules}

\label{rules in implementation}

The rules are responsible for recognising each log line and specifying the
correct action to be invoked.  The rules will be the most visible component
in any parser implemented using this architecture, and also the component
most likely to be modified by users.  Rules need to be as simple as
possible so that users can easily modify them or add new rules, but each
implementation must balance that simplicity with the need to provide enough
flexibility and power to successfully parse inputs.

The role of the rules in the architecture is covered in detail in
\sectionref{rules in architecture}; this section is concerned with the
practical aspects of how rules are implemented and used in \parsername{}.
The structure of the rules table has already been documented in
\sectionref{rules table}; that description will not be repeated here, but
should be fleshed out by the example rule in \sectionref{example rule in
implementation}.  The process of creating new rules from unparsed log lines
is dealt with in \sectionref{creating new rules in implementation},
followed by the implementation of the utility supplied with \parsername{}
to create regexes from unparsed log lines; the regex components provided by
\parsername{} to ease writing of complex regexes are covered in
\sectionref{regex components}.  The rule conditions used in \parsername{}
are the penultimate topic (\sectionref{rule conditions in implementation}),
and the section concludes with some suggestions for how to detect
overlapping rules.

\subsection{Example Rule}

\label{example rule in implementation}

The example rule in \tableref{Example rule in implementation table}
recognises the log line that results from Postfix rejecting a delivery
attempt because the domain part of the sender address does not have an A,
AAAA, or MX DNS entry, i.e.\ mail could not be delivered to any address in
the sender's domain (for full details
see~\cite{reject-unknown-sender-domain}).  A sample log line that would be
recognised by this rule:

% RFC 3330 says that 192.0.2.0/24 is reserved for example use.

\begin{verbatim}
NOQUEUE: reject: RCPT from smtp.example.com[192.0.2.1]:
  550 5.1.8 <alice@example.com>:
  Sender address rejected: Domain not found;
  from=<alice@example.com> to=<bob@example.net>
  proto=SMTP helo=<smtp.example.com>
\end{verbatim}

% do not reformat this!
\begin{table}[thbp]
    \caption{Example of a rule used by \parsernameshort{}}
    \empty{}\label{Example rule in implementation table}
    \begin{tabular}{ll}
        \tabletopline{}%
        \textbf{Field}      & \textbf{Value}                                    \\
        \tablemiddleline{}%
        name                & Unknown sender domain                             \\
        description         & We do not accept mail from unknown domains        \\
        restriction\_name   & reject\_unknown\_sender\_domain                   \\
        program             & \daemon{smtpd}                                    \\
        regex               & \verb!^__RESTRICTION_START__ <(__SENDER__)>: !    \\
                            & \verb!Sender address rejected: Domain not found;! \\
                            & \verb!from=<\k<sender>> to=<(__RECIPIENT__)> !    \\
                            & \verb!proto=E?SMTP helo=<(__HELO__)>$!            \\
        result\_data        &                                                   \\
        connection\_data    &                                                   \\
        action              & \action{DELIVERY\_REJECTED}                       \\
        hits                & 0                                                 \\
        hits\_total         & 0                                                 \\
        priority            & 0                                                 \\
        debug               & 0                                                 \\
        %cluster\_group      & 400                                               \\
        \tablebottomline{}%
    \end{tabular}
\end{table}

\noindent{}The fields in \tableref{Example rule in implementation table}
are used as follows:

\begin{description}

    \item [name, description, and restriction\_name:] are not used by the
        parser, they serve to document the rule for the user's benefit.

    \item [program and regex:] The program is used to restrict the log
        lines this rule will be used to recognise; see \sectionref{rule
        conditions in implementation} for details.  The regex does the
        actual recognition of the log lines, and data captured by the regex
        (e.g.\ sender, recipient) will be automatically saved to the
        results and connections tables.  The regex components used in the
        regex are described in \sectionref{regex components}.

    \item [action:] The action to be invoked when the rule recognises a log
        line.  See \sectionref{actions in detail in implementation} for
        details of the actions implemented by \parsername{}, and
        \sectionref{actions in architecture} for the role of actions in the
        architecture.

    \item [result\_data and connection\_data:] are used to provide data not
        present in the log line, and are unused in this rule.

    \item [hits, hits\_total, and priority:] hits and priority are used
        when ordering the rules for more efficient parsing (see
        \sectionref{rule ordering for efficiency}).  At the end of each
        parsing run hits is set to the number of log lines recognised by
        the rule.  Hits\_total is the sum of hits over every parsing run,
        but is otherwise unused by the parser.

    \item [debug] enables or disables printing of debugging information
        when this rule recognises a log line.

\end{description}

\subsection{Creating New Rules}

\label{creating new rules in implementation}

The log files produced by Postfix differ from installation to installation,
because administrators have the freedom to choose the subset of available
restrictions that suits their needs, including using different
\acronym{DNSBL} services, policy servers, or custom rejection messages.  To
ease the process of parsing new log lines, the architecture separates rules
from actions: adding new actions requires some effort, but adding new rules
to recognise new log lines is trivial, and occurs much more frequently.

To add a new rule a new row must be added to the rules table in the
database, containing all the required attributes: action, name,
description, program, and regex; all the other attributes are either
optional, set by the parser, or have sensible defaults.  The name and
description fields should be set based on the meaning of the log line, to
help others understand which log lines this rule will recognise; the
program will be obvious from the unrecognised log lines.  The action
depends on the what the log line represents, e.g.\ a delivery rejection, a
mail being delivered, some useful information, or something else; examine
the list of actions in \sectionref{actions in detail in implementation} to
determine the correct one.  The regex needs to be constructed based on the
log line, but see below for a tool to ease the process.

Other attributes may be required to make a rule work correctly:
connection\_data, result\_data, priority, or restriction\_name.  In general
it will only become clear that connection\_data or result\_data are
required when \parsername{} warns about an entry in the connections or
results tables that is missing some required fields, because values for
those fields are not present in any of the log lines for that connection or
mail.  For example, the rule that recognises the \daemon{pickup} component
processing a mail sets client\_hostname to \texttt{localhost} and
client\_ip to \texttt{127.0.0.1}, because the mail originates on the local
server.  If the new rule deliberately uses the architecture's overlapping
rules feature the priority field needs to be set, on this rule and possibly
others; the priority field may be needed on unintentionally overlapping
rules too, but that is more difficult to determine.  Finally, the
restriction\_name field should be set if the rule's action is
\action{DELIVERY\_REJECTED}; the name of the restriction should be clear
from the content of the log line.

A program is provided with \parsername{} to ease the process of creating
new rules from unrecognised log lines, based on the algorithm developed by
Risto Vaarandi for his \acronym{SLCT}~\cite{slct-paper}.  The differences
between the two algorithms will be outlined as part of the general
explanation below.  The core of the \acronym{SLCT} algorithm is quite
simple: programs generally create log lines by substituting variable words
into a fixed pattern, and analysis of the frequency with which each word
occurs can be used to determine whether the word is variable or part of the
fixed pattern.  This classification can be used to group similar log lines
and generate a regex to match each group of log lines.  The algorithm has
five steps:

\begin{description}

    \item [Pre-process the file.]  The modified algorithm begins by
        leveraging the knowledge gained when writing rules and developing
        \parsername{}; it performs many substitutions on the input log
        lines, replacing commonly occurring variable terms (e.g.\ email
        addresses, IP addresses, the standard start of rejection messages)
        with keywords described in \sectionref{regex components}.  The
        purpose of this step is to utilise existing knowledge to create
        more accurate regexes; it replaces many variable words with fixed
        words, improving the subsequent classification of words as fixed or
        variable.  Regex metacharacters in the log line will be escaped, to
        avoid generating invalid or incorrect regexes.  The altered log
        lines are written to a temporary file, which the next stage will
        use instead of the original input file.

        In the original algorithm the purpose of the pre-processing stage
        was to reduce the memory consumption of the program.  In the first
        pass it generates a hash~\cite{hash-functions}\glsadd{hash} (from a
        small range of values) for each word of each log line, incrementing
        a counter for each hash.  The counters will be used in the next
        pass to filter out words: if the word's hash does not have a high
        frequency, the word itself cannot have a high frequency, so it must
        be variable and does not need a counter maintained for it, reducing
        the number of counters required and thus the program's memory
        consumption.

    \item [Calculate word frequencies.]  The position of words within a log
        line is important: a word occurring in two log lines does not
        indicate similarity unless it occupies the same position in both
        log lines.  If a variable term substituted into a log line contains
        spaces, it will appear to the algorithm as more than one word.
        This will alter the position of subsequent words, so a word
        occurring in different positions in two log lines \textit{may\/}
        indicate similarity, but the algorithm does not attempt to deal
        with this possibility.  The modified algorithm maintains a counter
        for each \texttt{(word, word's position within the log line)}
        tuple, incrementing it each time that word occurs in that position.

        The original algorithm hashes each word and checks that result's
        counter from the previous pass: if that counter has a high
        frequency a separate counter will be maintained for this
        \texttt{(word, word's position within the log line)} tuple; if not,
        the tuple will not have a counter.  This reduces the number of
        counters maintained in this step, reducing memory requirements at
        the cost of increased CPU usage.

        As time goes by, the amount of memory typically available to a
        program or algorithm increases, and the need to reduce memory
        requirements correspondingly decreases, so the modified algorithm
        omits the hashing step and maintains counters for all tuples.  Most
        of the infrequently occurring words will have been substituted with
        keywords during the first step, vastly reducing the number of
        tuples to maintain counters for; the original algorithm does not
        have the detailed knowledge leveraged by the modified algorithm,
        because it is a generic tool.

    \item [Classify words based on their frequency.]  The frequency of each
        \texttt{(word, word's position within the log line)} tuple is
        checked: if its frequency is greater than the threshold supplied by
        the user (1\% of all log lines is generally a good starting point),
        it is classified as a fixed word, otherwise it is classified as a
        variable term.  If a variable term appears sufficiently often it
        will be misclassified as a fixed term, but that should be noticed
        by the user when reviewing the new regexes, and will be obvious
        when the new rules do not recognise some log lines they are
        expected to.  Variable terms are replaced by \texttt{.+} to match
        one or more of any character.

    \item [Build regexes.]  The words are reassembled to produce a regex
        matching the log line, and a counter is maintained for each regex.
        Contiguous sequences of \texttt{.+} in the newly assembled regexes
        are collapsed to a single \texttt{.+}; any resulting duplicate
        regexes and their counters are combined.  If a regex's counter is
        lower than the threshold supplied by the user the regex is
        discarded; this second threshold is independent of the threshold
        used to differentiate between fixed and variable words, but once
        again 1\% of log lines is a good starting point.

    \item [Test the new regexes.]  The final step replaces keywords in the
        new regexes in the same way as \parsername{} does, and compiles
        each regex to check they are valid.  A match will be attempted
        between all of the new regexes and each of the unparsed log lines
        the regexes were built from; the user will be warned if a log line
        is not matched by any regex, or if a log line is matched by more
        than one regex.  The number of log lines each regex matches is
        counted, as is the number of log lines matched by all regexes,
        though a log line is counted once only, even if matched by more
        than one regex.

        This step is not performed by the original algorithm.

\end{description}

The new regexes are displayed for the user to add to the ruleset, either as
new rules or merged into the regexes of existing rules; also displayed are
the number of unrecognised log lines each regex was expected to match, and
the number it actually matched, to help the user notice problems in the new
regexes.  Discarding regexes will result in some of the unrecognised log
lines not being matched; when the ruleset has been augmented with the new
regexes, the original log files should be parsed again, and any remaining
unparsed log lines used as input to this utility.

This utility is not expected to create perfect regexes, but it greatly
reduces the effort required to deal with unrecognised log lines.  The
regexes it generates will be self-contained: a parser that relies on using
cascaded parsing would require a modified algorithm, perhaps by replacing
the pre-processing stage with one that applies existing cascading rules to
each log line, and uses the resulting modified log lines for the remainder
of the algorithm.

\subsection{Regex Components}

\label{regex components}

Each rule's regex will have keywords expanded when the ruleset is loaded,
for several reasons:

\begin{itemize}

    \item It simplifies both reading and writing of regexes and helps to
        make each regex largely self-documenting.  For example, the meaning
        of \_\_CLIENT\_HOSTNAME\_\_ is immediately clear, whereas its
        expansion \verb!(?:unknown|(?:[-.\w]+))! needs to be deciphered
        each time it is encountered.

    \item It avoids needless repetition of complex regex components, and
        allows the components to be corrected or improved in one location.
        For example, \_\_SENDER\_\_ is used in 68 rules; if a mistake is
        discovered in it the mistake only needs to be corrected in one
        place.

    \item It enables automatic extraction and saving of captured data.  The
        regex snippets use Perl 5.10's named capture buffers~\cite{perlre}
        to capture data, so the mapping between captures and fields does
        not need to be explicitly specified by the rule.

\end{itemize}

To improve efficiency, the keywords are expanded and every rule's regex is
compiled before attempting to parse the log file, otherwise every regex
would be recompiled each time it was used, resulting in a large, data
dependent slowdown, as described in \sectionref{Caching compiled regexes}.
Most of the keywords are named after the fields in the connections or
results tables they populate: \_\_CLIENT\_HOSTNAME\_\_, \_\_CLIENT\_IP\_\_,
\_\_DELAY\_\_, \_\_DELAYS\_\_, \_\_ENHANCED\_STATUS\_CODE\_\_,
\_\_HELO\_\_, \_\_MESSAGE\_ID\_\_, \newline{} \_\_QUEUEID\_\_,
\_\_RECIPIENT\_\_, \_\_SENDER\_\_, \_\_SERVER\_HOSTNAME\_\_,
\_\_SERVER\_IP\_\_, \_\_SIZE\_\_, and \_\_SMTP\_CODE\_\_.

The other keywords need more explanation:

\begin{eqlist}

    \squeezeitems{}

    \item [\_\_CHILD\_\_]  The queueid of a child mail; see
        \sectionref{Re-injected mails}.

    \item [\_\_COMMAND\_\_]  All \acronym{SMTP} commands.

    \item [\_\_CONN\_USE\_\_]  How many times the connection was reused;
        Postfix tries to reuse connections whenever possible to reduce the
        load on both the sending and receiving servers.

    \item [\_\_DATA\_\_]  This snippet is special: it does not match
        anything by itself, so it must be followed by a pattern written by
        the rule author, but it captures whatever is matched by that
        pattern.  For example, \verb!/(__DATA__connection! \newline{}
        \tab{}\verb!(?:refused|reset by peer))/! matches either
        ``connection refused'' or ``connection reset by peer'' and causes
        it to be saved to the data field of that log line's result.

    \item [\_\_PID\_\_]  The \acronym{pid} of a \daemon{smtpd} process that
        dies or is killed; see the \action{SMTPD\_DIED} action in
        \sectionref{actions in detail in implementation}.

    \item [\_\_RESTRICTION\_START\_\_]  Matches the standard information
        Postfix includes at the start of almost all log lines resulting
        from rejecting a delivery attempt.

    \item [\_\_SHORT\_CMD\_\_]  Postfix sometimes logs \acronym{SMTP}
        commands in a short, single word form; this snippet matches all of
        those, except \texttt{DATA}, which typically has a more specific
        rule.  Priorities could have been used instead of excluding
        \texttt{DATA}.


\end{eqlist}

Some similarity exists between regex components and cascaded parsing: each
regex component resembles a rule that recognises part of a log line and
consumes it, leaving the remainder to be recognised by other components or
cascaded rules.  The major difference between the two is that regex
components are explicitly used by the author of the ruleset, whereas
cascaded parsing would be dynamically applied by the framework.

\subsection{Rule Conditions}

\label{rule conditions in implementation}

Rule conditions as part of the architecture have already been documented in
\sectionref{rule conditions in architecture}, and they can be very complex
and difficult to evaluate.  In contrast, \parsername{} uses quite simple
rule conditions: each rule has a \texttt{program} attribute that specifies
the Postfix component whose log lines it recognises, and each log line
contains the name of the Postfix component that produced it; when trying to
recognise log lines, the framework will only use rules where the two are
equal.  This avoids needlessly trying rules that will not recognise the log
line, or worse, might recognise it unintentionally.  In addition the
framework supports generic rules, whose program attribute is ``*''; these
will be used if none of the program-specific rules recognise the log line.
If none of the rules are successful the framework will warn the user,
informing them that they need to augment their ruleset, and alerting them
that because \parsername{} failed to recognise some of their log lines, the
results stored in the database may be incomplete.

\subsection{Overlapping Rules}

\label{overlapping rules in implementation}

The advantages and difficulties of overlapping rules have already been
addressed in \sectionref{overlapping rules in architecture} and will not be
repeated here.  \parsername{} does not try to detect overlapping rules;
that responsibility is left to the author of the rules.  A mechanism is
provided for ordering overlapping rules: the priority field in each rule;
negative priorities may be useful for catchall rules.

Detecting overlapping rules is difficult, but the following approaches may
be helpful:

\begin{itemize}

    \item Sort rules by program and regex, then visually inspect the list,
        e.g.\ with an \acronym{SQL} query similar to:   \newline{}
        \verb!SELECT program, regex!                    \newline{}
        \verb!    FROM rules!                           \newline{}
        \verb!    ORDER BY program, regex;!             \newline{}
        The rules are sorted first by program, then by regex, because rules
        cannot overlap if their programs are different.  Note that this
        query does not properly deal with generic rules whose program is
        \texttt{*}; those rules will be used on all log lines that have not
        been recognised by program-specific rules.

    \item Compare the results of parsing using different rule orderings, as
        described in \sectionref{rule ordering for efficiency}.  Parse
        several log files using optimal ordering, then dump a textual
        representation of the rules, connections, and results tables.
        Repeat with shuffled and reversed ordering, starting with a fresh
        database.  If the ruleset does not have overlapping rules the
        tables from each run will be identical; differences indicate
        overlapping rules.  The rules that overlap can be determined by
        examining the differences in the tables: each result contains a
        reference to the rule which created it, which will change if that
        rule overlaps with another.  Unfortunately this method cannot prove
        the absence of overlapping rules; it can detect overlapping rules,
        but only if the log files have log lines that are recognised by
        more than one rule.

\end{itemize}

\section{Complications Encountered While Writing \parsernamelong{}}

\label{complications}

It was initially expected that parsing Postfix log files would be a
relatively simple task, requiring a couple of months of work.  The author
had found Postfix log files useful when investigating problems reported by
users, and an examination of the log files gave the impression that they
would be straightforward to parse, process, and understand.  The large
variation in log lines was not apparent, because the majority of log lines
are recognised by a small set of rules, as shown in \figureref{rule hits
graph}.  Most of the myriad complications and difficulties documented in
this section were discovered during \parsernames{} development, but the
first three complications were identified during the planning and design
phase of this project, influencing the architecture's design.

Each of these complications caused \parsername{} to operate incorrectly,
generate warning messages, or leave mails in the state table.  The
complications are listed in the order in which they were overcome during
development of \parsername{}, with the first complication occurring several
orders of magnitude more frequently than the last.  When deciding which
problem to tackle next, the problem causing the highest number of warning
messages or mails incorrectly remaining in the state tables was always
chosen, because that approach yielded the biggest improvement in the
parser, and made the remaining problems more apparent.

\subsection{Queueid Vs Pid}

\label{queueid vs pid}

A delivery attempt lacks a queueid until one recipient has been accepted,
so log lines must first be correlated by \daemon{smtpd} \acronym{pid}, then
transition to being correlated by their queueid.  This is relatively minor,
but does require:

\begin{itemize}

    \item Two versions of several functions: \texttt{by\_pid} and
        \texttt{by\_queueid}.

    \item Two state tables to hold data structures.

    \item Most importantly: every section of code must know which state
        table it should query.

\end{itemize}

\subsection{Connection Reuse}

\label{connection reuse}

Multiple independent mails may be delivered across one connection: this
requires \parsername{} to clone the connection's data as soon as a mail is
accepted, so that subsequent mails will not overwrite each other's data.
This must be done every time a mail is accepted, as it is impossible to
tell in advance which connections will accept multiple mails.  Once a mail
has been accepted its log lines will not be correlated by \acronym{pid} any
more, its queueid will be used instead, as described in \sectionref{queueid
vs pid}.  If the original connection has any useful data (e.g.\ rejections)
it will be saved to the database when the client disconnects.  One unsolved
difficulty is distinguishing between different groups of rejections, e.g.\
when dealing with the following sequence:

\begin{enumerate}

    \item The client attempts to deliver a mail, but it is rejected.

    \item The client issues the RSET command to reset the \acronym{SMTP}
        session.

    \item The client attempts to deliver another mail, likewise rejected.

\end{enumerate}

There should ideally be two separate entries in the database resulting from
the above sequence, but currently there will only be one.



\subsection{Re-injected Mails}

\label{Re-injected mails}

\label{tracking re-injected mail}

Mails sent to local addresses are not always delivered directly to a
mailbox: sometimes they are sent to and accepted for a local address, but
need to be delivered to one or more remote addresses due to aliasing.  When
this occurs, a child mail will be injected into the Postfix queue, but
without the explicit logging that mails injected by \daemon{smtpd} or
\daemon{postdrop} have.  Thus the source of the mail is not immediately
discernible from the log line in which the mail's queueid first appears:
from a strictly chronological reading of the log lines it usually appears
as if the child mail has been created without a source.  Subsequently the
parent mail will log the creation of the child mail, e.g.\ parent mail
\texttt{3FF7C4317} creates child mail \texttt{56F5B43FD}\@:

\texttt{3FF7C4317: to=<username@example.com>, relay=local, \hfill{}
\newline{} \tab{} \tab{} delay=0, status=sent (forwarded as 56F5B43FD)}

Unfortunately, while all log lines from an individual process appear in
chronological order, the order in which log lines from different processes
are interleaved is subject to the vagaries of process scheduling.  In
addition, the first log line belonging to the child mail (the example log
line above belongs to the parent mail) is logged by either \daemon{qmgr} or
\daemon{cleanup}, so the order also depends on how soon they process the
new mail.

Because of the uncertain order that log lines can appear in, \parsername{}
cannot complain when it encounters a log line from either \daemon{qmgr} or
\daemon{cleanup} for a previously unseen mail; instead it must flag the
mail as coming from an unknown origin, and subsequently clear that flag if
and when the origin of the mail becomes clear.  Obviously the parser could
omit checking where mails originate from, but requiring an explicit source
helps to expose bugs in the parser; such checks helped to identify the
complications described in \sectionref{discarding cleanup log lines} and
\sectionref{pickup logging after cleanup}.

Process scheduling can have a still more confusing effect: quite often the
child mail will be created, delivered, and entirely finished with,
\textit{before\/} the parent mail logs its creation!  Thus, mails flagged
as coming from an unknown origin cannot be entered into the database when
their final log line is processed, nor can \parsername{} warn the user;
instead they must be marked as ready for entry and subsequently entered
once their origin has been identified.  The crux of this complication is
that re-injected mails appear in the log files without explicit logging
indicating their source.

Tracking re-injected mail requires \parsername{} to do the following in the
\action{COMMIT} action:

\begin{enumerate}

    \item If a mail is tagged with the unknown origin flag, it is assumed
        to be a child mail whose parent has not yet been identified.  The
        mail is tagged as ready to be entered in the database, but entry is
        postponed until the parent is identified.  The child mail should
        not have any subsequent log lines: only its parent will refer to
        it.

    \item If the mail is a child mail whose parent has been identified, it
        is entered in the database as usual, then removed from its parent's
        list of children.  If this child is the last mail on that list, and
        the parent has already been entered in the database, the parent
        will be removed from the state tables.

    \item If the mail is a parent, it is entered in the database as usual
        because there will be no further log lines for it.  There may be
        child mails waiting to be entered in the database; these are
        entered as normal, and removed from the state tables.  If there are
        incomplete child mails, the parent's removal from the state tables
        will be postponed until the last child has been entered.

\end{enumerate}

\subsection{Identifying Bounce Notifications}

\label{identifying bounce notifications}

Postfix 2.2.x (and presumably previous versions) does not generate a log
line when it generates a bounce notification; suddenly there will be log
entries for a mail that lacks an obvious source.  There are similarities to
the problem of identifying re-injected mails discussed in
\sectionref{tracking re-injected mail}, but unlike the solution described
therein bounce notifications do not eventually have a log line that
identifies their source.  Heuristics must be used to identify bounce
notifications:

\begin{enumerate}

    \item The sender address is \verb!<>!.\glsadd{<>}

    \item Neither \daemon{smtpd} nor \daemon{pickup} have logged any
        messages associated with the mail, indicating it was generated
        internally by Postfix, not accepted via \acronym{SMTP} or submitted
        locally by \daemon{postdrop}.

    \item The message-id has a specific format: \newline{}
        \tab{} \texttt{YYYYMMDDhhmmss.queueid@server\_hostname} \newline{}
        e.g.\ \texttt{20070321125732.D168138A1@smtp.example.com}

    \item The queueid in the message-id must be the same as the queueid of
        the mail: this is what distinguishes a new bounce notification from
        a bounce notification that is being re-injected as a result of
        aliasing.  In the latter case the message-id will be unchanged from
        the original bounce notification, and so even if it happens to be
        in the correct format, e.g.\ if it was generated by Postfix on this
        or another server, it will not correspond with the queueid of the
        mail.

\end{enumerate}

Once a mail has been identified as a bounce notification, the unknown
origin flag is cleared and the mail can be entered in the database.

There is a small chance that a mail will be incorrectly identified as a
bounce notification, as the heuristics used may be too broad.  For this to
occur the following conditions would have to be met:

\begin{enumerate}

    \item The mail must have been generated internally by Postfix.

    \item The sender address must be \verb!<>!.\glsadd{<>}

    \item The message-id must have the correct format and correspond with
        the queueid of the mail.  While a mail sent from elsewhere could
        easily have the correct message-id format, the chance that the
        queueid in the message-id would correspond with the queueid of the
        mail is extremely small.

\end{enumerate}

If a mail is misclassified as a bounce message it will almost certainly
have been generated internally by Postfix; arguably misclassification in
this case is a benefit rather than a drawback, as other mails generated
internally by Postfix will be handled correctly.  Postfix 2.3 (and
hopefully subsequent versions) log the creation of a bounce message.

This check is performed during the \action{COMMIT} action.

\subsection{Aborted Delivery Attempts}

\label{aborted delivery attempts}

Some mail clients behave strangely during the \acronym{SMTP} dialogue: the
client aborts the first delivery attempt after the first recipient is
accepted, then makes a second delivery attempt for the same recipient that
it continues with until delivery is complete.  Microsoft Outlook is one
client that behaves in this fashion; other clients may act in a similar
way.  An example dialogue exhibiting this behaviour is presented below
(lines starting with a three digit number are sent by the server, the other
lines are sent by the client):

\begin{verbatim}
220 smtp.example.com ESMTP
EHLO client.example.com
250-smtp.example.com
250-PIPELINING
250-SIZE 15240000
250-ENHANCEDSTATUSCODES
250-8BITMIME
250 DSN
MAIL FROM: <sender@example.com>
250 2.1.0 Ok
RCPT TO: <recipient@example.net>
250 2.1.5 Ok
RSET
250 2.0.0 Ok
RSET
250 2.0.0 Ok
MAIL FROM: <sender@example.com>
250 2.1.0 Ok
RCPT TO: <recipient@example.net>
250 2.1.5 Ok
DATA
354 End data with <CR><LF>.<CR><LF>
The mail transfer is not shown.
250 2.0.0 Ok: queued as 880223FA69
QUIT
221 2.0.0 Bye
\end{verbatim}

Once again Postfix does not log a message making the client's behaviour
clear, so once again heuristics are required to identify when this
behaviour occurs.  A list of all mails accepted during a connection is
saved in the connection's state table entry, and the accepted mails are
examined when the disconnection action is invoked.  Each mail is checked
for the following:

\begin{itemize}

    \item Was the second result processed by the \action{CLONE} action?
        The first two \daemon{smtpd} log lines will be a connection log
        line and a mail acceptance log line.

    \item Is \daemon{smtpd} the only Postfix component that produced a log
        line for this mail?  Every mail which passes normally through
        Postfix will have a \daemon{cleanup} line, and later a
        \daemon{qmgr} log line; lack of a \daemon{cleanup} line is a sure
        sign the mail did not make it too far.

    \item Can the mail be found in the state tables?  If not it cannot be
        an aborted delivery attempt.

    \item If there are third and subsequent results, were those log lines
        processed by the \action{SAVE\_DATA} action?  If there are any log
        lines after the first two they should be informational only.

\end{itemize}

If all the checks above are successful then the mail is assumed to be an
aborted delivery attempt and is discarded.  There will be no further
entries logged for such mails, so without identifying and discarding them
they accumulate in the state table and will cause clashes if the queueid is
reused.  The mail cannot be entered in the database as the only data
available is the client hostname and \acronym{IP} address, but the database
schema requires many more fields be populated (see \sectionref{connections
table} and \sectionref{results table}).  These heuristics are quite
restrictive, and it appears there is little scope for false positives; if
there are any false positives there will also be warnings when the next log
line for that mail is parsed.  False negatives are less likely to be
detected: there may be queueid clashes (and warnings) if mails remain in
the state tables after they should have been removed, otherwise the only
way to detect false negatives is to examine the state tables after each
parsing run.

This check is performed in the \action{DISCONNECT} action; it requires
support in the \action{CLONE} action where the list of accepted mails is
maintained.


\subsection{Further Aborted Delivery Attempts}

Some mail clients disconnect abruptly if a second or subsequent recipient
is rejected; they may also disconnect after other errors, but such
disconnections are either unimportant or are handled elsewhere in the
parser, e.g.\ \sectionref{timeouts during data phase}.  Sadly, Postfix does
not log a message saying the mail has been discarded, as should be expected
by now.  The checks to identify this happening are:

\begin{itemize}

    \item Is the mail missing its \daemon{cleanup} log line?  Every mail
        which passes normally through Postfix will have a \daemon{cleanup}
        line; lack of one is a sure sign the mail did not make it too far.

    \item Were there three or more \daemon{smtpd} log lines for the mail?
        There should be a connection log line and a mail acceptance log
        line, followed by one or more delivery rejected log lines.

\end{itemize}

If both checks are successful then the mail is assumed to have been
discarded when the client disconnected.  There will be no further log lines
for such mails, so without identifying and entering them in the database
immediately they accumulate in the state table and will cause clashes if
the queueid is reused.

These checks are made during the \action{DISCONNECT} action.

\subsection{Timeouts During DATA Phase}

\label{timeouts during data phase}

The DATA phase of the \acronym{SMTP} conversation is where the headers and
body of the mail are transferred.  Sometimes there is a timeout or the
connection is lost during the DATA phase;\footnote{For the sake of brevity
\textit{timeout\/} will be used throughout this section, but everything
applies equally to lost connections.} when this occurs Postfix will discard
the mail and \parsername{} needs to discard the data associated with that
mail.  It seems more intuitive to save the mail's data to the database, but
if a timeout occurs there is no data available to save; the timeout is
recorded and saved with the connection instead.  To deal properly with
timeouts the \action{TIMEOUT} action does the following:

\begin{enumerate}

    \item Record the timeout and captured data in the connection's results.

    \item If no mails have been accepted yet nothing needs to be done; the
        \action{TIMEOUT} action ends.

    \item If one or more recipients have been accepted, Postfix will have
        allocated a queueid for the incoming mail, and there may be a mail
        in the state tables that needs to be discarded.  The timeout may
        have interrupted transfer of an accepted delivery attempt, or it
        may have occurred after a mail delivery attempt was rejected.  If a
        mail needs to be discarded, the following checks will all pass:

        \begin{itemize}

            \item The timestamp of the log line preceding the timeout log
                line is earlier than the timestamp of the last accepted
                delivery attempt, i.e.\ there have not been any rejections
                since then the delivery attempt was accepted.

            \item The mail must exist in the state tables.

            \item The mail does not have a \daemon{qmgr} log line.

        \end{itemize}

        If all the checks are passed the mail will be discarded from the
        state table, and will not be entered in the database.  If one or
        more checks are not passed it is assumed that the timeout happened
        after a rejected delivery attempt.  This assumption is not
        necessarily correct, because Postfix may have accepted an earlier
        recipient and rejected a later one, in which case the timeout
        applies to the partially accepted mail, which should be discarded;
        this has not been a problem in practice.  Processing timeouts is
        further complicated by the presence of out of order
        \daemon{cleanup} log lines: see \sectionref{discarding cleanup log
        lines} for details.

\end{enumerate}

This complication is dealt with by the \action{TIMEOUT} action, with help
from the \action{CLONE} action.

\subsection{Discarding Cleanup Log Lines}

\label{discarding cleanup log lines}

The author has only observed this complication occurring after a timeout,
though there may be other circumstances that trigger it.  Sometimes the
\daemon{cleanup} log line for a mail being accepted is logged after the
timeout log line, by which time \parsername{} has discarded the mail;
parsing the \daemon{cleanup} log line causes the
\action{CLEANUP\_PROCESSING} action to create a new state table entry, to
help deal with re-injected mails (\sectionref{Re-injected mails}).  This is
incorrect because the log line actually belongs to the mail that has just
been discarded; if the queueid is reused there will be a queueid clash,
otherwise the new mail will just remain in the state tables.

During the \action{TIMEOUT} action, if the \daemon{cleanup} line is still
pending, the action updates the \texttt{timeout\_queueids} state table,
adding the queueid and the timestamp from the log line.  When the next
\daemon{cleanup} line is parsed for that queueid, the list will be checked
and if the log line meets the following requirements it will be deemed part
of the mail where the timeout occurred and discarded.

\begin{itemize}

    \item The queueid must not have been reused yet, i.e.\ it does not have
        a mail in the state tables.

    \item The timestamp of the \daemon{cleanup} log line must be within ten
        minutes of the mail acceptance timestamp.  Timeouts happen after
        five minutes, but some data may have been transferred slowly, and
        empirical evidence shows that ten minutes is not unreasonable;
        hopefully it is a good compromise between false positives (log
        lines incorrectly discarded) and false negatives (new state table
        entries incorrectly created).

\end{itemize}

The \daemon{cleanup} log line must meet the criteria above for it to be
discarded because some, but not all, connections where a timeout occurs
will have an associated \daemon{cleanup} log line; if the
\action{CLEANUP\_PROCESSING} action blindly discarded the next
\daemon{cleanup} log line after a timeout it would sometimes be mistaken.
When the next \daemon{pickup} log line containing that queueid is processed
the queueid will be removed from the cache of timeout queueids, regardless
of whether it meets the criteria above.

This complication is handled by the \action{CLEANUP\_PROCESSING} and
\action{TIMEOUT} actions.

\subsection{Pickup Logging After Cleanup}

\label{pickup logging after cleanup}

When mail is submitted locally, \daemon{pickup} accepts the new mail and
generates a log line.  Occasionally this log line will occur later in the
log file than the \daemon{cleanup} log line, so the \action{PICKUP} action
will find that a state table entry already exists for that queueid.
Normally when this happens a warning is generated by the \action{PICKUP}
action, but if the following conditions are met it is assumed that the log
lines were out of order:

\begin{itemize}

    \item The only program which has logged anything thus far for the mail
        is \daemon{cleanup}.

    \item There is less than a five second difference between the
        timestamps of the \daemon{cleanup} and \daemon{pickup} log lines.

\end{itemize}

As always with heuristics there may be circumstances in which these
heuristics match incorrectly, but none have been identified so far.  This
complication seems to occur during periods of particularly heavy load, so
is most likely caused by process scheduling vagaries.

This complication is dealt with by the \action{PICKUP} action.

\subsection{Smtpd Stops Logging}

\label{smtpd stops logging}

Occasionally a \daemon{smtpd} will just stop logging, without an
immediately obvious reason.  After poring over log files for some time
several reasons have been found for this infrequent occurrence:

\begin{enumerate}

    \item Postfix is stopped or its configuration is reloaded.  When this
        happens all \daemon{smtpd} processes exit, so all entries in the
        connections state table must be cleaned up, entered in the database
        if there is sufficient data, and deleted.

    \item Sometimes a \daemon{smtpd} is killed by a signal (sent by Postfix
        for some reason, by the administrator, or by the OS), so its active
        connection must be cleaned up, entered in the database if there is
        sufficient data, and deleted from the connections state table.

    \item Occasionally a \daemon{smtpd} will exit uncleanly, so the active
        connection must be cleaned up, entered in the database if there is
        sufficient data, and deleted from the connections state table.

    \item Every Postfix process uses a watchdog which kills the process if
        it is not reset for a considerable period of time (five hours by
        default).  This safeguard prevents Postfix processes from running
        indefinitely and consuming resources if a failure causes them to
        enter a stuck state.

\end{enumerate}

The circumstances above account for all occasions where a \daemon{smtpd}
suddenly stops logging.  In addition to removing an active connection the
last accepted mail may need to be discarded, as detailed in
\sectionref{timeouts during data phase}; otherwise the queueid state table
is untouched.

This complication is handled by several actions: \action{POSTFIX\_RELOAD} (1),
\action{SMTPD\_DIED} (2 \& 3), and \action{SMTPD\_WATCHDOG} (4).

\subsection{Out Of Order Log Lines}

\label{out of order log lines}

Occasionally a log file will have out of order log lines which cannot be
dealt with by the techniques described in \sectionref{tracking re-injected
mail}, \sectionref{discarding cleanup log lines}, or \sectionref{pickup
logging after cleanup}.  In the \numberOFlogFILES{} log files used for
testing this problem occurs only five times in 60,721,709 log lines, but
for parser correctness it must be dealt with.  All five occurrences have
the same characteristics: the \daemon{local} log line showing delivery to a
local mailbox comes after the \daemon{qmgr} log line showing removal of the
mail from the queue because delivery is complete.  This causes problems:
the data in the state tables for the mail is not complete, so it cannot be
entered into the database; a new mail is created when the \daemon{local}
log line is processed and remains in the state tables; four warnings are
issued per pair of out of order log lines.

The \action{COMMIT} action examines the list of programs that have produced
log lines for each mail, comparing the list against a table of known-good
program combinations.  If the mail's combination is found in the table the
mail can be entered in the database; if the combination is not found entry
must be postponed and the mail flagged for later entry.  The
\action{MAIL\_DELIVERED} action checks for that flag; if the additional log
lines have caused the mail to reach a valid combination then entry in the
database will proceed, otherwise it must be postponed once more.

The list of valid combinations is explained below.  Every mail will
additionally have log lines from \daemon{cleanup} and \daemon{qmgr}; any
mail may also have a log line from \daemon{bounce}, \daemon{postsuper}, or
both.

% This will put the text on the line following the item name, if the
% enumitem package is loaded.
%\begin{description}[style=nextline]
\begin{description}

    \item [\daemon{local}:] Local delivery of a bounce notification, or
        local delivery of a re-injected mail.

    \item [\daemon{local}, \daemon{pickup}:] Mail submitted locally on the
        server, delivered locally on the server.

    \item [\daemon{local}, \daemon{pickup}, \daemon{smtp}:] Mail submitted
        locally \newline{} on the server, for both local and remote
        delivery.

    \item [\daemon{local}, \daemon{smtp}, \daemon{smtpd}:] Mail accepted
        from a remote client, for both local and remote delivery.

    \item [\daemon{local}, \daemon{smtpd}:] Mail accepted from a remote
        client, for local delivery only.

    \item [\daemon{pickup}, \daemon{smtp}:] Mail submitted locally on the
        server, for remote delivery only.

    \item [\daemon{smtp}:] Remote delivery of either a re-injected mail or
        a bounce notification.

    \item [\daemon{smtp}, \daemon{smtpd}:] Mail accepted from a remote
        client, then remotely delivered (typically relaying mail for
        clients on the local network to addresses outside the local
        network).

    \item [\daemon{smtpd}, \daemon{postsuper}:] Mail accepted from a remote
        client, then deleted by the administrator before any delivery
        attempt was made.  Notice that \daemon{postsuper} is required, not
        optional, for this combination.

\end{description}

This check applies to accepted mails only, not to rejected mails.  This
check is performed during the \action{COMMIT} action, with support from the
\action{MAIL\_DELIVERED} action.

\subsection{Yet More Aborted Delivery Attempts}

\label{yet more aborted delivery attempts}

The aborted delivery attempts described in \sectionref{aborted delivery
attempts} occur frequently, but those described in this section only occur
four times in the \numberOFlogFILES{} log files used for testing.  The
symptoms are the same as in \sectionref{aborted delivery attempts}, except
that there \textit{is\/} a \daemon{cleanup} log line; there is nothing in
the log file to explain why there are no further log lines.  The only way
to detect these mails is to periodically scan all mails in the state
tables, deleting any mails with the following characteristics:

\begin{itemize}

    \item The timestamp of the last log line for the mail must be 12 hours
        or more earlier than the timestamp of the last log line parsed from
        the current log file.

    \item There must be exactly two \daemon{smtpd} and one \daemon{cleanup}
        log lines for the mail, with no additional log lines.

\end{itemize}

12 hours is a somewhat arbitrary time period, but it is far longer than
Postfix would delay delivery of a mail in the queue, unless it was not
running for an extended period of time.  Each time the end of a log file is
reached, the state tables are scanned for mails matching the
characteristics above, and any mails found are deleted.

\subsection{Mail Deleted Before Delivery Is Attempted}

\label{Mail deleted before delivery is attempted}

Postfix logs the recipient address when delivery of a mail is attempted, so
if delivery has yet to be attempted \parsername{} cannot determine the
recipient address or addresses.  This is a problem when mail is arriving
faster than Postfix can attempt delivery, and the administrator deletes
some of the mail before Postfix has had a chance to try to deliver it.  In
this case the recipient address will not have been logged, so a dummy
recipient address needs to be added, as every mail is required by the
database schema (\sectionref{results table}) to have at least one
recipient.  Typically when this complication occurs there are many
instances of it, closely grouped.

This lack of information cannot easily be overcome: it is trivial to log a
warning for every accepted recipient, but Postfix will not yet have
allocated a queueid for the mail when the warning for the first recipient
is logged, so the warning will be associated with the connection rather
than the accepted mail.  A queue file and queueid will be allocated after
Postfix accepts the MAIL FROM command if
\texttt{smtpd\_delay\_open\_until\_valid\_rcpt} is set to ``no'', but that
setting will cause disk IO for every delivery attempt, instead of just for
delivery attempts where recipients are accepted, and consequently a drastic
reduction in the performance of the mail server.

The \action{DELETE} action is responsible for handling this complication.

\subsection{Bounce Notification Mails Are Delivered Before Their Creation
Is Logged}

\label{Bounce notification mails delivered before their creation is logged}

This is yet another complication that only occurs during periods of
extremely high load, when process scheduling and even hard disk access
times cause strange behaviour.  In this complication, bounce notification
mails are created, delivered, and deleted from the queue, \textit{before\/}
the log line from \daemon{bounce} that explains their source.  To deal with
this the \action{COMMIT} action maintains a cache of recently committed
bounce notification mails, which the \action{BOUNCE\_CREATED} action
subsequently checks if the bounce mail is not found in the state tables.
If the queueid exists in the cache, and its start time is less than ten
seconds before the timestamp of the bounce log line, it is assumed that the
bounce notification mail has already been processed and the
\action{BOUNCE\_CREATED} action does not create one.  If the queueid exists
in the cache it is removed, because it has either just been used or the
problem did not occur for that mail.  Whether the \action{BOUNCE\_CREATED}
action creates a mail or finds an existing mail in the state tables, it
flags the mail as having been seen by the \action{BOUNCE\_CREATED} action;
if this flag is present the \action{COMMIT} action will not add the mail to
the cache of recent bounce notification mails.  This is not required to
correctly deal with the complication, but is an optimisation to reduce the
parser's memory usage.  On the occasions the author has observed this
complication occurring there have been a huge number of bounce notification
mails generated: if every bounce notification mail was cached it would
dramatically increase the memory requirements of the parser.  The cache of
bounce notification mails will be pruned whenever the parser's state is
saved, though if the size of the cache ever becomes a problem it could be
pruned periodically to keep it in check.

\subsection{Mails Deleted During Delivery}

\label{Mails deleted during delivery}

The administrator can delete mails using \daemon{postsuper}; occasionally
mails that are in the process of being delivered will be deleted.  This
results in the log lines from the delivery agent (\daemon{local},
\daemon{virtual} or \daemon{smtp}) appearing in the log file
\textit{after\/} the mail has been removed from the state tables and saved
in the database.  The \action{DELETE} action adds deleted mails to a cache,
which is checked by the \action{SAVE\_DATA} action, and the current log
line discarded if the following conditions are met:

\begin{enumerate}

    \item The queueid is not found in the state tables.

    \item The queueid is found in the cache of deleted mails.

    \item The timestamp of the log line is within 5 minutes of the final
        timestamp of the mail.

\end{enumerate}

Sadly this solution involves discarding some data, but the complication
only arises eight times in quick succession in one log file, which is not
in the \numberOFlogFILES{} log files used for testing; if this complication
occurred more frequently it might be desirable to find the mail in the
database and add the log line to it.



\section{Limitations And Possible Improvements}

\label{limitations and improvements in implementation}

Every piece of software suffers from some limitations, and almost always
has room for improvement.  Below are the limitations and possible
improvements that have been identified in \parsername{}.

\subsection{Limitations}

\label{logging helo}

\begin{enumerate}

    \item Each new Postfix release requires writing new rules or modifying
        existing rules to cope with the new or changed log lines.
        Similarly, using a new \acronym{DNSBL}, a new policy server, or new
        administrator-defined rejection messages also requires new rules.

    \item The hostname used in the HELO command is not logged if the
        incoming delivery attempt is successful.  Configuring Postfix to do
        this is relatively simple; create a restriction that is guaranteed
        to warn for every accepted mail, as follows:

        \begin{enumerate}

            \item Create \texttt{/etc/postfix/log\_helo.pcre}
                containing:\newline{}
                \tab{}\texttt{/\^/~~~~WARN~Logging~HELO}

            \item Modify \texttt{smtpd\_data\_restrictions} in
                \texttt{/etc/postfix/main.cf} to contain:\newline{}
                \tab{}\texttt{check\_helo\_access~pcre:/etc/postfix/log\_helo.pcre}

        \end{enumerate}

        Although \texttt{smtpd\_helo\_restrictions} seems like the natural
        place to log the HELO hostname, when it is evaluated for the first
        recipient there will not yet be a queueid allocated for the
        delivery attempt, so the log line cannot be associated with the
        correct mail.  A queueid is guaranteed to have been allocated when
        the DATA command has been reached, and thus the queueid will be
        logged by any restrictions taking effect in
        \texttt{smtpd\_data\_restrictions}, and the log line can be
        associated with the correct mail.  Specifying a HELO-based
        restriction in \texttt{smtpd\_data\_restrictions} does not cause
        any problems; Postfix will perform the check correctly.

        Logging the HELO hostname in this fashion also partially prevents
        the complication described in \sectionref{Mail deleted before
        delivery is attempted} from occurring, but only when the mail has a
        single recipient.  When a mail has a single recipient address it
        will be logged, but when a mail has multiple recipients no
        addresses are logged.

    \item \parsername{} will not detect that it is parsing the same log file
        twice, resulting in the database containing duplicate entries.

    \item \parsername{} does not distinguish between log files produced by
        different servers when parsing; all results will be saved to the
        same database.  This may be viewed as an advantage, because log
        files from different servers can be combined in the same database,
        or it may be viewed as a limitation because it is impossible to
        distinguish between log files from different servers in the same
        database.  If the results of parsing log files from different
        servers must remain separate, \parsername{} can easily be
        instructed to use a different database.

    \item The solution to complication \sectionref{Mails deleted during
        delivery} involves discarding data.

    \item Further complications may arise when parsing log files, and
        \parsername{} will need to be modified to deal with them.

    \item \parsername{} does not limit the size of the database, which will
        grow without bounds unless the user deletes connections and results
        from it.  This is both a limitation and a benefit: the benefit is
        that data will never be unexpectedly deleted, but the limitation is
        that the user must manage the size of the database.

\end{enumerate}

\subsection{Possible Improvements}

\begin{enumerate}

    \item Improve the solution to complication \sectionref{Mails deleted
        during delivery} so that data is not discarded.

\end{enumerate}


\section{Summary}

This chapter has presented \parsername{}, the parser implemented for this
project, beginning with the assumptions made during its development.  A
simplified flowchart shows the most common paths taken through Postfix and
\parsername{}, accompanied by a description of the stages and transitions.
A database provides storage for rules and for data gathered from log files;
any further use of that data is dependent on a clear understanding of the
database schema, so the resemblance between a database schema and an
\acronym{API} is described, followed by a diagram and a detailed
description of the database schema.  The framework is documented next,
including the steps it takes during initialisation, the parsing process,
and the conveniences it offers to users.  The performance data collected by
the framework is described, as are the ways in which various optimisations
can be disabled to demonstrate their effect, and the debugging options the
framework provides.  The implementation of actions in \parsername{} is
documented, including how frequently each action is specified by rules, and
brief descriptions of why some actions are more popular than others.  All
the actions that are part of \parsername{} are described in detail,
followed by the process of adding a new action to the parser.  The final
component of the architecture, the rules, is also the most visible
component, and its implementation is examined in detail with: an example
rule and an explanation of how every field in that rule is used; a
description of how to add new rules and determine the values that should be
used for each field; an explanation of the algorithm used by the utility
that creates new regexes based on unparsed log lines, and how it differs
from the original algorithm it is based on.  How \parsername{} uses rule
conditions and overlapping rules is discussed, accompanied by a description
of the regex snippets the framework provides to simplify the regexes used
in rules.  The many complications and difficulties encountered while
writing \parsername{}, and the solutions developed to overcome them, are
documented in depth; also documented are how some solutions interact, and
which action or actions the solution is implemented in.  This chapter
concludes with a list of the limitations identified in \parsername{},
followed by some possible improvements suggested by the author.  The next
chapter will evaluate this implementation, examining both \parsernames{}
efficiency and the coverage it achieves when parsing log files.
