\chapter{Introduction}

\label{introduction}

The architecture described herein was developed as part of a larger project
to improve anti-spam defences by analysing the performance of the set of
filters currently in use, optimising the order and membership of the set
based on that analysis, and developing supplemental filters where
deficiencies are identified.  Most anti-spam techniques are content-based
(e.g.~\cite{a-plan-for-spam, word-stemming, relaxed-online-svms}) and
require the mail to be accepted before determining if it is spam, but
rejecting mail during the delivery attempt is preferable: senders of
non-spam mail that is mistakenly rejected will receive an immediate
non-delivery notice; resource usage is reduced on the accepting mail server
(allowing more intensive content-based techniques to be used on the mail
that is accepted); users have less spam mail to wade through.  Improving
the performance of anti-spam techniques that are applied when mail is being
transferred via \acronym{SMTP} is the goal of this project, by providing a
platform for reasoning about anti-spam filters.  The approach chosen to
measure performance is to analyse the log files produced by the
\acronym{SMTP} server in use, Postfix
(\urlLastCheckedNoParens{http://www.postfix.org/}{2009/02/28}), rather than
modifying its source code to generate statistics: this approach improves
the chances of other Postfix users testing and using the software developed
for this project.  The need arose for a parser capable of dealing with the
great number and variety of log lines produced by Postfix: the parser must
be designed so that adding support for parsing new inputs is a simple task,
because the log lines to be parsed will change over time.

Most mail server administrators will have performed some basic processing
of the log files produced by their mail server at one time or another,
whether it was to debug a problem, explain to a user why their mail is
being rejected, or check if new anti-spam measures are working.  The more
adventurous will have generated statistics to show how successful each of
their anti-spam measures has been in the last week, and possibly even
generated some graphs to clearly illustrate these statistics to management
or users.\footnote{This was the author's first real foray into processing
Postfix log files.}  Very few will have performed in-depth parsing and
analysis of their log files, where the parsing must correlate the log lines
per-connection or per-queueid rather than processing log lines
independently.  One of the barriers to this kind of processing is the
unstructured nature of Postfix log files, where each log line was added on
an ad hoc basis as a requirement was discovered or new functionality was
added.\footnote{A history of all changes made to Postfix is distributed
with the source code, available from
\urlLastChecked{http://www.postfix.org/}{2009/02/23}.}  Further
complication arises because the set of rejection messages is not fixed: new
messages can be added by the administrator with custom checks; every
\acronym{DNSBL}\footnote{This thesis is supplied with a glossary
(\textsection\ref{Glossary}) and a list of acronyms
(\textsection\ref{Acronyms}).} returns a different explanatory message;
policy servers may log different messages depending on the characteristics
of the connection.  There are many ways in which the log lines may differ
between servers, even within the same organisation: servers may be
configured differently, or running different versions of Postfix.

This thesis documents the difficult process of parsing Postfix log files,
presenting an architecture designed to enable the users of a parser to
easily extend it to parse their particular inputs, without requiring a lot
of work or a high level of understanding of the parsing process and the
parser's internal workings.  This architecture is the basis of
\parsername{}, a program that parses Postfix log files and places the
resulting data into a database for later analysis.  The gathered data can
be used to optimise current anti-spam measures, provide a baseline to test
new anti-spam measures against, or to produce statistics showing how
effective those measures are.  Numerous other uses are possible for such
data: improving server performance by identifying troublesome destinations
and reconfiguring appropriately; identifying regular high volume uses
(e.g.\ customer newsletters) and restricting those uses to off-peak times
or providing a dedicated service for them; detecting virus outbreaks that
propagate via mail; as a base for billing customers on a shared server.
Preserving the raw data enables users to develop a multitude of uses far
beyond those conceived of by the author.

\section{Thesis Layout}

Chapter~\ref{background} provides background information useful in
understanding the motivation behind the project, \acronym{SMTP}, and
Postfix.

Chapter~\ref{state of the art review} reviews both the previously published
research in this area and other Postfix log file parsers, discussing why
they were deemed unsuitable for the task, including why they could not be
improved or expanded upon.

XXX THIS ALL NEEDS TO BE CHECKED AND UPDATED\@.

This algorithm requires a database for storing both the rules used when
parsing and the results gleaned from parsing.  The database schema used is
described in \sectionref{database schema}, explaining in detail the tables
used for storing the data gleaned from the log files and the table that
stores the rules.

Chapter~\ref{rules in architecture} discusses the parsing rules in detail,
explaining the purpose and usage of each field in a rule, referring to an
example rule and sample data it matches successfully against.  The pros and
cons of overlapping rules are considered, including techniques for
detecting unintentional overlaps.  Rule efficiency concerns are discussed,
in particular the optimisations used by the algorithm.  The section
concludes with a description of using the tools provided with the parser to
generate new rules (specifically the regex in each rule) from unparsed log
lines.

Chapter~\ref{Postfix Parser Implementation} contains the core of the paper,
describing a naive parsing algorithm and the complications initially
encountered that shaped the full algorithm.  A flow chart and a discussion
of the emergent behaviour exhibited by the algorithm accompanies a
comprehensive explanation of the different stages of the initial algorithm.
The framework that actions and rules fit into is documented, then the
actions taken during execution of the algorithm are described, followed by
the process of adding a new action.  The section concludes with an in-depth
description of the further complications discovered, and their solutions
that complete the parser.

Chapter~\ref{parsing coverage} analyses the coverage the parser achieves
over a set of \numberOFlogFILES{} log files taken from a mail server
handling mail for over 700 users, averaging 8500 mails per day
(\graphref{Mails received per day}).  Coverage is described both
in terms of the fraction of log lines parsed and the fraction of mails and
connections successfully reconstructed by the parsing algorithm; dealing
with false negatives and a discussion of the difficulties in identifying
false positives is also included.  As part of determining the coverage of
the parser a random sampling of log lines was parsed, and the correctness
of the results manually verified.

Chapter~\ref{limitations and improvements in implementation} lists the
limitations of the algorithm, then suggests some ways of dealing with them,
with the goal of improving parsing and reproduction of the journey a mail
takes through Postfix.

Chapter~\ref{conclusion} contains the conclusion of the thesis, describing
the results of the research, design, and implementation of the parser.

The bibliography (appendix~\ref{bibliography}) contains references to the
resources used in developing the algorithm, writing the program, and
preparing this thesis.  Also listed are some additional resources expected
to be helpful in understanding \acronym{SMTP}, Postfix, anti-spam
techniques, or the thesis.

Appendix~\ref{Glossary} provides a glossary of terms used in the thesis.

Appendix~\ref{Acronyms} provides a list of acronyms used in the thesis.

Appendix~\ref{Postfix Daemons} gives a brief description of Postfix
daemons.

\section{Previously Published Work}

Portions of \textsection\ref{introduction}, \textsection\ref{parser
architecture}, \textsection\ref{Postfix Parser Implementation}, and
\textsection\ref{Results} have previously been published at a
conference~\cite{sgai-2008}, and later reprinted in a
journal~\cite{elsevier-2009}.  Publication of the conference paper and
journal article was supported by Science Foundation Ireland
RFP~05/RF/CMS002.
