\chapter{Introduction}

\label{introduction}

The architecture and implementation described in this thesis were developed
as the foundation of a larger project to improve anti-spam defences, by
analysing the performance of the set of anti-spam techniques currently in
use, optimising the order and membership of the set based on that analysis,
and developing supplemental anti-spam techniques where deficiencies are
identified.  Most anti-spam techniques are content-based
(e.g.~\cite{a-plan-for-spam, relaxed-online-svms, word-stemming}) and
require a mail to be accepted before determining if it is spam, but
rejecting mail during the delivery attempt is preferable: senders of
non-spam mail that is mistakenly rejected will receive an immediate
non-delivery notice; resource usage is reduced on the accepting mail
server, allowing more intensive content-based techniques to be used on the
remaining mail that is accepted; users have less spam mail to wade through.
Improving the performance of anti-spam techniques that are applied when
mail is being transferred via \acronym{SMTP} is the goal of this project,
by providing a platform for reasoning about the performance of anti-spam
techniques.

The approach chosen to measure performance is to analyse the log files
produced by the \acronym{SMTP} server in use, Postfix, rather than
modifying its source code to generate statistics: this approach improves
the chances of other Postfix users testing and using the software developed
for this project.  The need arose for a parser capable of dealing with the
great number and variety of log lines produced by Postfix: the parser must
be designed and implemented so that adding support for parsing new inputs
is a simple task, because the log lines to be parsed will change over time.
This variety in log lines occurs for several reasons:

\begin{itemize}

    \squeezeitems{}

    \item Log lines differ amongst versions of Postfix.

    \item The mail server administrator can define custom rejection
        messages.

    \item Policy servers~\cite{policy-servers} may log different messages
        depending on the characteristics of the connection.

    \item Every \acronym{DNSBL}\footnote{This thesis is supplied with a
        glossary (\textsection\ref{Glossary}) and a list of acronyms
        (\textsection\ref{Acronyms}).} returns a different explanatory
        message.

\end{itemize}

Most mail server administrators will have performed some basic processing
of the log files produced by their mail server at one time or another,
whether it was to debug a problem, explain to a user why their mail is
being rejected, or check if new anti-spam techniques are working.  The more
adventurous will have generated statistics to show how successful each of
their anti-spam measures has been in the last week, and possibly even
generated some graphs to clearly illustrate these statistics to management
or users.\footnote{This was the first real foray the author, a Systems
Administrator for a network of over 2000 computers and over 1800 users,
took into processing Postfix log files.}  Very few will have performed
in-depth parsing and analysis of their log files, where the parsing must
correlate the log lines per-connection or per-queueid rather than
processing log lines independently.  One of the barriers to this kind of
processing is the unstructured nature of Postfix log files, where each log
line was added on an ad hoc basis as a requirement was discovered or new
functionality was added.\footnote{A history of all changes made to Postfix
is distributed with the source code, available from
\urlLastChecked{http://www.postfix.org/}{2009/02/23}.}  Further
complication arises because the set of log lines is not fixed, and log
lines can differ in many ways between servers, even within the same
organisation, where servers may be configured differently or running
different versions of Postfix.

It was hoped to reuse an existing parser rather than writing one from
scratch, but the effort required to adapt and improve an existing parser
was judged to be greater than the effort to write a new one, because the
techniques used by the existing parsers severely limited their potential:

\begin{itemize}

    \item Some parsers ignored the majority of log lines, parsing specific
        log lines accurately, but without any provision for parsing new or
        similar log lines.

    \item Other parsers sloppily parsed the majority of log lines, but were
        incapable of distinguishing between log lines in one same category,
        e.g.\ not distinguishing between different anti-spam techniques
        causing the rejection of delivery attempts.

    \item None of the reviewed parsers extracted enough data for future
        in-depth analysis.

\end{itemize}

The only prior published work on the subject of parsing Postfix log files
that the author is aware of is \textit{Log Mail Analyser: Architecture and
Practical Utilizations\/}~\cite{log-mail-analyser}, which aims to extract
data from log files, correlate it, and present it in a form suitable for a
systems administrator to search using the myriad of standard Unix text
processing utilities already available; it is reviewed alongside the other
parsers in the State of the Art Review in chapter~\ref{state of the art
review}.

Once it was decided that a new parser would be written, an architecture was
required to base the implementation on.  Existing architectures are
tailored towards parsing inputs with a fixed grammar or a tightly
constrained format, whereas Postfix log files lack any form of constraint,
as outlined earlier.  A new architecture was designed and developed for
this parser, with the hope that it will be useful to others.  The resulting
architecture is conceptually simple: provide a few generic functions
(\textit{actions\/}), each capable of dealing with an entire category of
inputs (e.g.\ rejecting a mail delivery attempt), accompanied by a
multitude of precise patterns (\textit{rules\/}), each of which recognises
one input variant and only that variant (e.g.\ rejection by a specific
\acronym{DNSBL}), and specifies which action will process the inputs it
recognises.  This architecture is ideally suited to parsing inputs
where the input is not fully understood or does not conform to a fixed
grammar: the architecture warns about unparsed inputs and other errors, but
continues parsing as best it can, allowing the developer of a new parser to
decide which deficiencies are most important and require immediate
attention, rather than being forced to fix the first error that arises.

This thesis documents the difficult process of parsing Postfix log files,
presenting an architecture designed to enable the users of a parser to
easily extend it to parse their particular inputs, without requiring much
work or a high level of understanding of the parsing process and the
parser's internal workings.  This architecture is the basis of
\parsername{}, a program that parses Postfix log files and places the
resulting data into a database for later analysis.  The gathered data can
be used to optimise current anti-spam defences, to produce statistics
showing how effective each technique in use is, or to provide a baseline to
test new anti-spam measures against.  Numerous other uses are possible for
such data: improving server performance by identifying troublesome
destinations and reconfiguring appropriately; identifying regular high
volume uses (e.g.\ customer newsletters) and restricting those uses to
off-peak times, or providing a dedicated service for them; detecting virus
outbreaks that propagate via mail; as a base for billing customers on a
shared server.  Preserving the raw data enables users to develop a
multitude of uses far beyond those conceived of by the author.

\section{Thesis Layout}

Chapter~\ref{background} provides background information useful in
understanding \acronym{SMTP}, Postfix, and the motivation behind the
project.

Chapter~\ref{state of the art review} reviews the previously published
research in this area and other Postfix log file parsers, discussing why
they were deemed unsuitable for the task, including why they could not be
improved or expanded upon.

Chapter~\ref{parser architecture} describes the parser architecture
developed for this project, beginning with an overview, then describing
each of the components of the architecture in detail.  This chapter
concentrates on the abstract, theoretical, implementation-independent
aspects of the architecture; discussion of the practical aspects is
deferred until chapter~\ref{Postfix Parser Implementation}.

Chapter~\ref{Postfix Parser Implementation} documents \parsername{}, the
parser based on the architecture described in chapter~\ref{parser
architecture}.  The practical difficulties of implementing each of the
components of the architecture are described, accompanied by the many
complications encountered when parsing Postfix log files, and other details
of the implementation.

Chapter~\ref{Evaluation} evaluates \parsernames{} efficiency, discussing
the various optimisations implemented in the framework, and the effect they
have.  It also discusses the coverage achieved by \parsername{} over
\numberOFlogFILES{} log files, with separate sections for the number of log
lines correctly recognised, and the number of connections and mails whose
journey through Postfix was correctly reconstructed.

Chapter~\ref{conclusion} contains the conclusion of the thesis.

The bibliography (appendix~\ref{bibliography}) contains references to the
resources used in designing the architecture and writing \parsername{}.

Appendix~\ref{Glossary} provides a glossary of terms used in the thesis.

Appendix~\ref{Acronyms} contains a list of acronyms used in the thesis;
uncommon acronyms will have an entry in the glossary too.

Appendix~\ref{Postfix Daemons} concludes this thesis with a brief
description of Postfix daemons.

\section{Previously Published Work}

Portions of chapters~\ref{introduction} and~\ref{parser
architecture}--\ref{Evaluation} have previously been published at an
international conference~\cite{sgai-2008}, and later reprinted in a
journal~\cite{elsevier-2009}.  Publication of the conference paper was
supported by Science Foundation Ireland RFP~05/RF/CMS002.
