\section{Introduction}

\label{introduction}

Most mail server administrators will have performed some basic processing
of the log files produced by their mail server at one time or another,
whether it was to debug a problem, explain to a user why their mail is
being rejected, or check whether new anti-spam measures are working.  The
more adventurous will have generated statistics to show how many hits each
of their anti-spam measures has gotten in the last week, and possibly even
generated some graphs to clearly illustrate the point to management or
users.\footnote{This was the author's first real foray into processing
Postfix log files.}  Very few will have performed in-depth parsing and
analysis of their log files, where the parsing must correlate the log lines
per-connection or per-queueid rather than processing log lines
independently.  One of the barriers to this type of processing is the
unstructured nature of Postfix log files, where each log line was added on
an ad hoc basis as a requirement was discovered or new functionality was
added.\footnote{A history of all changes made to Postfix is distributed
with the source code, available from \url{http://www.postfix.org/}} Further
complication arises because the set of rejection messages is not fixed: new
messages can be added by the administrator with custom checks; every
\RBL{}\footnote{This document is supplied with a glossary, see
appendix~\ref{Glossary}.} returns a different explanatory message; policy
servers may log different messages depending on the characteristics of the
connection; there are many ways in which the log lines may differ between
servers, even within the same organisation --- servers may be configured
differently, or running different version of Postfix.  This paper documents
the difficult process of parsing Postfix log files, presenting \PLP{}, a
program which parses log files and places the resulting data into a
database for later use.  The gathered data can then be used to optimise
current anti-spam measures, provide a baseline to test new anti-spam
measures against, or to produce statistics showing how effective those
measures are.  There are numerous other uses for such data: improving
server performance by identifying troublesome destinations and
reconfiguring appropriately; identifying regular high volume uses (e.g.\
customer newsletters) and restricting those uses to off-peak times;
detecting virus outbreaks which propagate via email; as a base for billing
customers on a shared server.  Preserving the raw data enables users to
develop a multitude of uses far beyond those conceived of by the author.

\vspace{1em}\noindent\textbf{Layout of the paper:}

Section~\ref{background} provides background information useful in
understanding the paper, parser and algorithm.  It introduces the idea of
using a database schema as an \API{}, providing an interface to the data
gathered that is language-neutral.  The unusual separation of rules,
actions and algorithm is discussed, giving the reasons that approach was
taken when designing the parser.

This algorithm requires a database for storing both the rules used when
parsing and the results gleaned from parsing.  The database schema used is
described in section~\refwithpage{database schema}, explaining in detail
the tables used for storing the data gleaned from the log files and the
table that stores the rules.

Section~\ref{rules} discusses the parsing rules in detail, explaining the
purpose and usage of each field in a rule, referring to an example rule and
sample data it matches successfully against.  The pros and cons of
overlapping rules are discussed, including techniques for detecting
unintentional overlaps.  Rule efficiency concerns are discussed, in
particular the optimisations used by the algorithm, with reference to the
graphs in appendix~\refwithpage{graphs}.  The final section is a
description of using the tools provided with the parser to generate new
rules (specifically the \regex{} in each rule) from unparsed log lines.

Section~\ref{parsing-algorithm} contains the core of the paper, describing
a naive parsing algorithm and the complications initially encountered which
shaped the full algorithm.  A flow chart and a discussion of the emergent
behaviour exhibited by the algorithm accompanies a comprehensive
explanation of the different stages of the initial algorithm.  The
framework which actions and rules fit into is documented, then the actions
taken during execution of the algorithm are described, followed by the
process of adding a new action.  The section concludes with an in-depth
description of the further complications discovered and their solutions
which complete the parser.

Section~\ref{parsing coverage} analyses the coverage the parser achieves
over a set of \numberOFlogFILES{} log files taken from a mail server
handling mail for over 700 users, averaging 8500 mails per day
(graph~\refwithpage{Mails received per day}).  Coverage is described both
in terms of the fraction of log lines parsed and the fraction of mails and
connections successfully reconstructed by the parsing algorithm, including
dealing with false negatives and a discussion of the difficulties in
identifying false positives.  A random sampling of log lines was parsed,
and the correctness of the results manually verified, as part of
determining the coverage of the parser.

Section~\ref{limitations-improvements} lists the limitations of the
algorithm, then suggests some ways of dealing with them, with the goal of
improving parsing and reproduction of the journey a mail takes through
Postfix.

Section~\ref{conclusion} contains the paper's conclusion, describing the
results of the research, design and implementation of the parser.

Appendix~\ref{other-parsers} discusses other parsers and why they were
deemed unsuitable for the task, including why they could not be improved or
expanded upon.

The bibliography contains references to the resources used in developing
the algorithm, writing the program, and preparing this paper.  Also listed
are some additional resources expected to be helpful in understanding
\SMTP{}, Postfix, anti-spam techniques, or the paper.

Appendix~\ref{graphs} contains graphs illustrating the topics previously
discussed in \sectionref{rule efficiency} (rule efficiency), and explains
the irregularities observed in these graphs.

Appendix~\ref{Glossary} provides a glossary of terms used in the paper.

Appendix~\ref{Acronyms} provides a list of acronyms used in the paper.

