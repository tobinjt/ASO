\section{Limitations and possible improvements}

\label{limitations-improvements}

\subsection{Introduction}

Every piece of software suffers from some limitations and there is almost
always room for improvement.

\subsection{Limitations}

\label{logging helo}

\begin{enumerate}

    \item Each new Postfix release requires new rules to be written to cope
        with the new log lines.  Similarly using a new \DNSBL{}, new policy
        server or new administrator defined rejection messages require new
        rules.

    \item It appears that the hostname used in the HELO command is not
        logged if the mail is accepted.\footnote{Tested with Postfix
        2.2.10, 2.3.11 and 2.4.7; this may possibly have changed in Postfix
        2.5. XXX TEST THIS AGAIN.}  Rectifying this is relatively simple:
        create a restriction which is guaranteed to warn for every accepted
        mail, as follows:

        \begin{enumerate}

            \item Create \texttt{/etc/postfix/log\_helo.pcre}
                containing:\newline \tab{}\texttt{/\^/~~~~WARN~Logging~HELO}

            \item Modify \texttt{smtpd\_data\_restrictions} in
                \texttt{/etc/postfix/main.cf} to contain\newline
                \tab{}\texttt{check\_helo\_access~/etc/postfix/log\_helo.pcre}

        \end{enumerate}

        Although \texttt{smtpd\_helo\_restrictions} seems like the natural
        place to log the HELO hostname, there will not be a queueid
        associated with the mail for the first recipient, so the log line
        cannot be associated with the correct mail.  There is guaranteed
        to be a queueid when the DATA command has been reached, and thus it
        will be logged by any restrictions taking effect in
        \texttt{smtpd\_data\_restrictions}.  There is no difficulty in
        specifying a HELO-based restriction in
        \texttt{smtpd\_data\_restrictions}, Postfix will perform the check
        correctly.

        Logging the HELO hostname in this fashion also prevents the
        complication described in \sectionref{Mail deleted before delivery
        is attempted} from occurring, but only in the case where there is a
        single recipient; in that case the recipient address will be logged
        also, but when there are multiple recipients no addresses are
        logged.  It is also possible to warn for every recipient,
        preventing the complication in \sectionref{Mail deleted before
        delivery is attempted} entirely.

    \item The algorithm does not distinguish between mails where one or
        more mails are rejected and a subsequent mail is accepted; it will
        appear in the database as one mail with lots of rejections followed
        by acceptance (this has already been mentioned in
        \sectionref{connection reuse}).  It does not appear to be possible
        to make this distinction given the data Postfix logs, though it
        might be possible to write a policy server to provide additional
        logging.

    \item The TIMEOUT action uses potentially incorrect heuristics to
        decide whether the timeout applies to an accepted mail or not,
        potentially leaving a mail in the state tables
        (\sectionref{timeouts-during-data-phase}).

    \item The program will not detect parsing the same log file twice,
        resulting in the database containing duplicate entries.

    \item The parser does not distinguish between log files produced by
        different sources when parsing; all results will be saved to the
        same database.  This may be viewed as an advantage, as log files
        from different sources can be combined in the same database, or it
        may be viewed as a limitation as there is no facility to
        distinguish between log files from different sources in the same
        database.  If the results of parsing log files from different
        sources must remain separate, the parser can easily be instructed
        to use a different database to store the results in.

    \item The solution to complication \sectionref{Mails deleted during
        delivery} involves discarding data.

\end{enumerate}

\subsection{Possible improvements}

\begin{itemize}

    \item Investigate and write the policy server referred to in limitation
        3 above.

    \item Improve the solution to complication \sectionref{Mails deleted
        during delivery} so that data is not discarded.

    \item Improve the heuristics used in
        \sectionref{timeouts-during-data-phase}, or develop another
        solution, to avoid incorrectly leaving a mail in the state tables.

\end{itemize}

\subsection{Summary}

This section has covered the limitations of the parser and possible
improvements which may be implemented in the future.


