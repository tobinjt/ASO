\section{Limitations and possible improvements}

\label{limitations-improvements}

\subsection{Introduction}

Every piece of software suffers from some limitations and there is almost
always room for improvement.

\subsection{Limitations}

\begin{enumerate}

    \item Each new Postfix release requires new rules to be written to cope
        with the new log lines.  Similarly using a new \RBL{}, new policy
        server or new administrator defined rejection messages require new
        rules.

    \item It appears that the hostname used in the HELO command is not
        logged if the mail is accepted.\footnote{Tested with Postfix
        2.2.10, 2.3.11 and 2.4.7; this may possibly have changed in Postfix
        2.5.}  Rectifying this has already been described in
        \sectionref{logging helo}.

    \item The algorithm does not distinguish between mails where one or
        more mails are rejected and a subsequent mail is accepted; it will
        appear in the database as one mail with lots of rejections followed
        by acceptance (this has already been mentioned in
        \sectionref{connection reuse}).  It doesn't appear to be possible
        to make this distinction given the data Postfix logs, though it
        might be possible to write a policy server to provide additional
        logging.

    \item The TIMEOUT action uses potentially incorrect heuristics to
        decide whether the timeout applies to an accepted mail or not,
        potentially leaving a mail in the state tables
        (\sectionref{timeouts-during-data-phase}).

    \item The program will not detect parsing the same log file twice,
        resulting in the database containing duplicate entries.

    \item The parser does not distinguish between logs produced by
        different sources when parsing; all results will be saved to the
        same database.  This may be viewed as an advantage, as logs from
        different sources can be combined in the same database, or it may
        be viewed as a limitation as there is no facility to distinguish
        between logs from different sources in the same database.  If the
        results of parsing logs from different sources must remain
        separate, the parser can easily be instructed to use a different
        database to store the results in.

    \item The solution to complication \sectionref{Mails deleted during
        delivery} involves discarding data.

\end{enumerate}

\subsection{Possible improvements}

\begin{itemize}

    \item Investigate and write the policy server referred to in limitation
        3 above.

    \item Improve the solution to complication \sectionref{Mails deleted
        during delivery} so that data is not discarded.

    \item Improve the heuristics used in
        \sectionref{timeouts-during-data-phase}, or develop another
        solution, to avoid incorrectly leaving a mail in the state tables.

\end{itemize}

\subsection{Conclusion}

This section has covered the limitations of the parser and possible
improvements which may be implemented in the future.


