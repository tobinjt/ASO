\section{Other Postfix log parsers reviewed}

\label{other-parsers}

\subsection{Introduction}

It is important to compare and contrast newly developed programs,
algorithms and parsers against those already available, to accurately judge
what, if any, improvements are delivered by the newcomers.  There are not
that many previously developed Postfix log parsers, indeed it was quite
difficult to find ten parsers to review for this project, and the
functionality offered ranges from quite basic to much more mature,
depending on the needs of the creator.  The programs are not reviewed from
an independent viewpoint; the objective is to compare and contrast each
program with \parsername{}.  There are some important differences between
\parsername{} and the parsers reviewed here:

\begin{enumerate}

    \item Only \parsername{} makes it possible to parse new log lines
        without modifying the core parsing algorithm; see \sectionref{why
        separate rules, actions and framework?} for a discussion of why the
        separation of the algorithm into rules, actions and framework is
        beneficial.

    \item \parsername{} does not provide any reporting based on the data it
        gathers.  This is by design: the parser is responsible for parsing
        log files and extracting data --- further processing of that data
        is a separate concern, deferred to another program.  Using an
        \SQL{} database simplifies the process of generating such reports
        (discussed in \sectionref{database as API}); some simple examples
        are given in \sectionref{motivation}.

    \item Most of the parsers silently ignore log lines they cannot handle,
        whereas \parsername{} complains loudly about every single line it
        fails to parse.  The exception is AWStats, which outputs the
        percentage of input lines it was unable to parse, but does not
        output the lines themselves.

    \item A minor difference is that most parsers don't handle compressed
        files; both \parsername{} and Splunk handle them transparently,
        without user intervention; Sawmill and Lire can be configured to
        support compressed files, but Sawmill exhibits a dramatic slow down
        when thus configured.  Although this is a minor difference, support
        for reading compressed log files is quite helpful, as it
        dramatically reduces the disk space required to store old log
        files.

\end{enumerate}

\subsection{Parsers reviewed}

In this section text in italics is a quote extracted from the documentation
of the program being reviewed.


\subsubsection{Pflogsumm}

\textit{pflogsumm is designed to provide an over-view of Postfix activity,
with just enough detail to give the administrator a ``heads up'' for
potential trouble spots.\/}

Pflogsumm produces a report designed for troubleshooting, rather than for
in-depth analysis.  It does not support saving any data, nor does it
extract more data than is required for producing the report.  Both the
parsing and reporting are difficult to extend as it is a specialised tool,
unlike the easily extensible design of \parsername{}.  It does not attempt
to correlate log lines by queueid or \pid{}, nor does it need to deal with
the complications encountered during this project.  Pflogsumm produces a
useful report, and successfully dealt with the \numberOFlogFILES{} log
files files tested with,\footnote{The results it reported were not
verified, but it did not report any errors, and has a very good reputation
amongst Postfix users.} but is not a suitable base for this project.

Pflogsumm's report includes the following:

\begin{itemize}

    \item Total number of mails accepted, delivered and rejected.  Total
        bytes of mail accepted and delivered.

    \item Total number of sender and recipient addresses and domains.

    \item Per-hour averages and per-day summaries of the number of mails
        received, delivered, deferred, bounced and rejected.

    \item For received mail: per-domain totals for mails sent, deferred,
        average delay, maximum delay and bytes delivered.  For received
        mail: per-domain totals for mails received and bytes received.

    \item Number of mails and bytes sent and received for each address.

    \item Summary of why mail delivery was deferred or failed, why mails
        were bounced, why mails were rejected, and warning messages.

\end{itemize}

There are many options available to include or exclude certain sections of
the report.

\url{http://jimsun.linxnet.com/postfix_contrib.html} \newline (Last checked
2008/04/09.)

\subsubsection{Sawmill Universal Log File Analysis and Reporting}

\textit{ Sawmill is a Postfix log analyzer (it also support 686 other log
formats).  It can process log files in Postfix format, and generate dynamic
statistics from them, analyzing and reporting events.  Sawmill can parse
Postfix logs, import them into a SQL database (or its own built-in
database), aggregate them, and generate dynamically filtered reports, all
through a web interface.  Sawmill can perform Postfix analysis on any
platform, including Window, Linux, FreeBSD, OpenBSD, Mac OS, Solaris, other
UNIX, and more.\/}

Sawmill is a general purpose commercial product which parses 687 log file
formats (correct as of 2007/12/04) and produces reports.  Its data
extraction facilities are quite limited, though it does extract three
different sets of data for Postfix (one is beta as of 2007/12/04), but they
do not appear to be interlinked, nor does it save sufficient data for the
purposes of this project.\footnote{The data extracted by Sawmill is
described later in this review, and the data extracted by \PLP{} is
described in \sectionref{connections table} and \sectionref{results
table}.}  No attempt is made to correlate log lines or deal with the
difficulties documented in \sectionref{complications}
\sectionref{additional complications}.\footnote{If any attempt is made
there is no reference to it in the documentation available on the website.}
The source code is available in an obfuscated form only (presumably for a
fee), and the product is quite expensive, as it requires a license per
report which is to be generated; in contrast \parsername{} is free and the
code is freely available.  The web interface allows creation of dynamic
reports based on any field, but due to the associated cost the author has
not experimented with it.  Presumably, as the company charges per report
the user wishes to generate, the data store (if there is one) is
deliberately inaccessible and undocumented to prevent the user bypassing
the program and generating their own reports.

\url{http://www.thesawmill.co.uk/formats/postfix.html} \newline Fields
extracted: from, to, server, UID, relay, status, number of recipients,
origin hostname, origin \IP{} and virus.  The fields \texttt{server},
\texttt{uid} and \texttt{virus} are not explained in their documentation:
\texttt{server} is probably the server the mail is delivered to, and
\texttt{uid} might be the uid of the user submitting mail locally.  Postfix
does not perform any form of virus checking (though it has many options for
cooperating with an external virus scanner), so the \texttt{virus} field is
a mystery.

\url{http://www.thesawmill.co.uk/formats/postfix_ii.html} \newline Fields
extracted: from, to, \RBL{}, client hostname and client \IP{}\@.

\url{http://www.thesawmill.co.uk/formats/beta_postfix.html} \newline Fields
extracted: from, to, client hostname, client \IP{}, relay hostname, relay
\IP{}, status, response code, \RBL{} and message id.

Even if the three data sets were linked together Sawmill would extract less
data than \parsername{}, and it does not appear to extract data about
rejections except when the rejection is caused by an \RBL{} check.

When tested with the \numberOFlogFILES{} test log files it performed
adequately, though the rate it processed log files at did slow down
noticeably as it progressed.  Sawmill supports reading compressed log files
but it exhibits a dramatic slow down when doing so: it took six hours to
parse the first half of the log files; twelve hours to parse the next
third; after twenty four hours parsing the remaining sixth it crashed due
to lack of disk space.  On the second parsing attempt the log files were
uncompressed beforehand and parsing took eight hours.

Sawmill's web interface supports searching on any combination of the fields
it extracts, but the interface is neither as simple to use nor as
informative as the interface provided by Splunk.  The administrative
interface is very easy to use --- it took only five minutes to start
parsing a directory of log files.

(Last checked 2007/12/04.)

\subsubsection{Splunk}

\textit{Splunk is an IT Search engine. It is software that indexes any
format of IT data from any source in real time, including logs,
configurations, scripts, code, messages, traps, alerts, activity reports,
stack traces and metrics from all of your applications, servers and
devices. Splunk lets you search, navigate, alert and report on all your IT
data in real time using an AJAX web interface. You can also share knowledge
and Splunk solutions with other members of the Splunk community via
SplunkBase.\/}

Splunk aims to index all an organisation's log files, providing a
centralised view capable of searching and correlating diverse log sources.
The web interface supports complicated searches, providing statistics and
graphs in real time, a facility not provided by \parsername{} (report
generation has been deferred to a subsequent program).  Saved
searches\footnote{The author was unable to save searches, though that may
have been due to limitations in the free version.} can be run periodically
and the results emailed to a recipient or sent to a shell script, which
presumably can publish the results as required (though possibly without the
graphs and detailed statistics); \parsername{} provides the database and
leaves it to the user to utilise it in any way, whenever they want, with no
restrictions on usage of the data.  The interface is optimised for
interactive rather than automated queries and it does not appear to be
possible to write independent tools to utilise the Splunk database; there
are additional reports available on \url{http://www.splunkbase.com/}, but
the author was unable to find any: every category was empty, even those
which the interface claimed had reports available.  Many types of reports
are bundled with the software, though most are variations of a bar or pie
chart, with the exception of bubble and heatmap graphs.  It is very easy to
drill down through the graphs to extract a portion of the data (e.g.\
select the hour with the largest number of events, then select a particular
host, and finally a specific address), though it is not possible to search
on partial words.

The web interface is quite attractive and easy to use when searching, but
as an administrator it seems unnecessarily difficult to perform simple
tasks.  When testing it took roughly 30 minutes to add a single log file to
be indexed (log files must be indexed before searching), with the downside
that the log file was copied into a spool directory, doubling the disk
space usage.  The most suitable test would be to index all log files in a
particular directory, but after three hours, numerous futile attempts, and
reading all available documentation the author admitted defeat.  Using the
\CLI{} was more successful: invoking the command \newline \tab{}
\texttt{splunk find logs }\textit{log-directory\/}\newline added 40 of the
\numberOFlogFILES{} log files to the queue for indexing.  Repeated attempts
enqueued the same 40 log files, without explanation as to why the others
were excluded.\footnote{The log files appear to have only been indexed
once; presumably Splunk keeps track of the files it has indexed and
discards requests to index files for a second time.  This may or may not be
a useful feature for \parsername{}.} There did not appear to be an option
to order the log files to ensure they would be processed in the order they
were created.  Subsequently the author was successful in adding a single
file at a time using the \CLI{}; a simple loop to add all desired log files
was sufficient to index all files.  Splunk will periodically check all
those files for updates unless they are manually removed from its list;
this may or may not be useful behaviour.  Splunk did not appear to have any
difficulty in parsing the log files, once instructed to do so.  All
searches performed using the indexed data returned reasonable results.

Copious documentation is made available on \url{http://www.splunk.com/},
but poor organisation and sheer abundance makes it extremely hard to find
useful information.  Searching using the website's interface confusingly
tends to return results from old documentation rather than new.  In
general the documentation appears to have been written by someone
intimately acquainted with the software and having difficulty understanding
how a newcomer would approach tasks or the questions they would ask.

Splunk supports reading compressed log files without any configuration by
the user.  The free version of Splunk limits the volume of data indexed per
day to 500MB, though a trial Enterprise licence is available which allows
indexing of up to 5GB of data per day.  The cheapest licensed version costs
\$5000 (plus \$1000 support), and still limits the volume of data indexed
per day to 500MB.

In the specific case of parsing Postfix log files, Splunk extracts some
standard fields: to and from addresses, HELO hostname, date, host the log
files were collected from, and protocol.  It parses the standard syslog
fields at the beginning of the line, and extracts any \texttt{key=<value>}
pairs occurring after the standard syslog prologue; these pairs are the
fields listed above.  Searches can be based on the extracted fields, and
all the text in the line is also available for searching.  \parsername{}
extracts noticeably more data, though it does not make the full text of the
line available, and the full power of \SQL{} is available when searching,
allowing the user to search on arbitrarily complicated conditions.

Splunk is a generic tool, so it lacks any Postfix specific support over and
above extracting the \texttt{key=<value>} fields; most importantly it makes
no attempt to correlate log lines by queueid or \pid{}, nor to handle any
of the myriad complications discussed in this document.  Some additional
Postfix reports are supposedly available at
\url{http://www.splunkbase.com/}, but the author was unable to find them
(or any other reports).

\url{http://www.splunk.com/} \newline (Last checked 2008/04/29.)

\subsubsection{Isoqlog}

\textit{Isoqlog is an MTA log analysis program written in C. It designed to
scan qmail, postfix, sendmail and exim logfile and produce usage statistics
in \HTML{} format for viewing through a browser. It produces Top domains
output according to Sender, Receiver, Total mails and bytes; it keeps your
main domain mail statistics with regard to Days Top Domain, Top Users
values for per day, per month and years.\/}

Isoqlog produces a report listing the number of mails sent by each unique
sender address, and separately the total number of bytes transferred; both
reports are produced for daily, monthly and annual time spans, but only for
the domains listed in its configuration file (making it impossible to
produce reports for every sender domain).  It appears to ignore all log
lines except for those for the current day, though it does maintain a
record of data previously extracted, which the newly extracted data is
merged into (no information is provided on the format of the data store).
The data extracted appears to be limited to the number of mails sent by
each sender, unlike the copious amounts of data extracted by \parsername{}.
It doesn't utilise rejection log lines in any way, so is unsuitable for the
purposes of this project.  Its parsing is completely inextensible, indeed
is almost incomprehensible, relying on \texttt{scanf(3)}, fixed offsets and
low level string manipulation; it is the opposite end of the spectrum to
\parsernames{} parsing.  It doesn't handle any of the complications
discussed in this document, doesn't gather the breadth of data required for
this project, and ignores the majority of log lines produced by Postfix.
It ignores all log lines earlier than today, so testing with the test log
files is pointless.

\url{http://www.enderunix.org/isoqlog/} \newline (Last checked 2007/08/13.)

\subsubsection{AWStats}

\textit{AWStats is a free powerful and featureful tool that generates
advanced web, streaming, ftp or mail server statistics, graphically. This
log analyzer works as a CGI or from command line and shows you all possible
information your log contains, in few graphical web pages. It uses a
partial information file to be able to process large log files, often and
quickly. It can analyze log files from all major server tools like Apache
log files (NCSA combined/XLF/ELF log format or common/CLF log format),
WebStar, IIS (W3C log format) and a lot of other web, proxy, wap, streaming
servers, mail servers and some ftp servers.\/}

AWStats will produce simple graphs for many different services, but
supporting many different services without special purpose code restricts
it to supporting the \LCD{}.  The data it will extract from an \MTA{} log
file is limited in comparison to \parsername{}: \newline \tab{} time2,
email, email\_r, host, host\_r, method, url, code and bytesd.\newline There
does not seem to be an explanation of any of those fields in the
documentation (\parsername{} provides copious documentation).  AWStats has
no special purpose code to deal with the intricacies of Postfix log files,
in fact it operates by coercing Postfix log files into Apache\footnote{The
Apache web server is the most popular HTTP server in use for the past 10
years; more information is available at \url{http://httpd.apache.org/}.}
format log files, for analysis by AWStats' HTTP log file parser.  The
converting parser only deals with a small portion of the log lines
generated by Postfix, silently skipping those it cannot deal with, and does
not distinguish between different types of rejection; there is no easy way
to extend it to handle new log lines.  Although though it does correlate
log lines by queueid, it does not deal with any of the other complications
described in this document.  AWStats supports saving data but the format of
the saved data is not documented (as far as the author could tell).  It
also supports reading compressed log files, but that functionality was not
tested.

When tested with the \numberOFlogFILES{} test log files AWStats' report
says it parsed 9,240,075 (88.70\%) log lines out of 10,416,129, with
1,176,050 (11.29\%) corrupt lines; however there are actually
\numberOFlogLINES{} lines in the \numberOFlogFILES{} log files, so AWStats
only parsed 17.15\% of the input lines, ignoring the remaining 82.85\%.
The graphs it produces give an overview of mails received for the last
calendar month, showing:

\begin{itemize}

    \item The number of mails accepted from each host.

    \item How many mails were received by each recipient.
        
    \item The average number of mails accepted by the server per-day and
        per-hour.

    \item A top ten list of aggregated \SMTP{} error codes, e.g.\ the first
        entry in the list is: \texttt{Requested mail action not taken:
        relaying not allowed, unknown recipient user, \ldots}

\end{itemize}

% Parsed lines in file: 10416129
%  Found 4 dropped records,
%  Found 1176050 corrupted records,
%  Found 0 old records,
%  Found 9240075 new qualified records.

\noindent\url{http://awstats.sourceforge.net/} \newline
\url{http://awstats.sourceforge.net/awstats.mail.html} \newline
\url{http://awstats.sourceforge.net/docs/awstats_faq.html#MAIL}
\newline (Last checked 2007/10/13.)

\subsubsection{Throughput monitor}

This utility tracks the number of events which occurred over a particular
time and warns if the frequency of events passes a certain threshold.  It's
designed to provide real time alerts when dictionary attacks, mail loops or
similar problems occur. It doesn't attempt to extract or save data,
correlate log lines, or any of the more advanced tasks described in this
document, because it is not designed to do so.  The user must construct the
\regexes{} to match significant input lines before it will parse any log
files.  This program was not reviewed in further detail because its aims
are so far removed from the aims of this project.

\url{http://home.uninet.ee/~ragnar/throughput_monitor/} \newline (Last
checked 2008/05/01.)

\subsubsection{Anteater}

\textit{The Anteater project is a Mail Traffic Analyser. Anteater supports
currently the logformat produced by Sendmail and by Postfix. The tool is
written in 100\% C++ and is very easy to customize. Input, output, and the
analysis are modular class objects with a clear interface. There are eight
useful analyse modules, writing the result in plain ASCII or \HTML{}, to
stdout or to files.\/}

Anteater doesn't have any English documentation so it's difficult, nigh
impossible, for this author to accurately comment on what analysis it
performs.  It did not run successfully when tested, and its parsing would
certainly be out of date as Postfix has evolved considerably since this
tool was last updated (November 2003).  As it neither ran successfully nor
has documentation the author can read a detailed review cannot be provided.

The Debian project (\url{http://www.debian.org/}) provides a manual page
with the copy of anteater it distributes, so the author was at least able
to run anteater with the correct arguments; sadly anteater produced zero
for every statistic, presumably because it was unsuccessful in parsing the
log lines.

\url{http://anteater.drzoom.ch/} \newline (Last checked 2007/08/13.)

\subsubsection{Yet Another Advanced Logfile Analyser}

\textit{yaala is a very flexible analyser for all kinds of logfiles. It
uses parsers to extract information from a logfile, an SQL-like query
language to relate the information to each other and an output-module to
format the information appropriately.\/}

YAALA uses a plugin based system to analyse log files and produce \HTML{}
output reports, with all the parsing and report generation handled by
modules.  Using YAALA as a base would be only slightly less work than
starting from scratch, as both input and output modules would need to be
written specially; it may even be more work to implement the parser within
the constraints of YAALA\@.  YAALA supports storing previously gathered
data using Perl's Storable module~\cite{perl-storable}, so with enough
knowledge of the data structure YAALA stores it should be possible to
extract data with another Perl program also using the Storable module;
\parsername{} uses a well documented database which is accessible from the
majority of programming languages.  This information was gleaned from the
source code, as the documentation is sadly lacking.

YAALA provides a Postfix parser which extracts the following fields from
specific log lines:

\noindent\tab{}Aggregations: count (not explained), bytes (sum of bytes
transferred).\newline \tab{}Keyfields [sic]: incoming\_host,
outgoing\_host, date, hour, sender, recipient, defer\_count, delay.

YAALA extracts most of the fields \parsername{} does, but it does not
maintain separate counters for each restriction like \parsername{}; this
rules out the possibility of using the collected data for  optimisation,
testing or understanding of restrictions.  YAALA's Postfix parser does not
deal with the complications explained in this document, though it does
correlate log lines by queueid.

YAALA provides a mini-language based on \SQL{} that is used when generating
reports; sample reports can be seen at~\cite{yaala-samples}.  Example
query: \newline \tab{} \texttt{requests BY file WHERE host =\~{} Google}
\newline The mini-language is quite limited and cannot be used to extract
data for external use, merely to create reports.  Only data selected by the
query will be saved in the data store; other data will be discarded, and
removed from the data store if already present.

Testing YAALA was unsuccessful because all the select clauses tried
produced a similar error message:
\newline\tab{}\texttt{lib/Yaala/Data/Core.pm: Unavailable aggregation
requested:} \newline\tab{}\texttt{``bytes''. Returning 0.} \newline  The
underlying reason for this is that YAALA only parsed 408 log lines of
360632 log lines (0.11\%) in the first log file.

In summary YAALA provides a Postfix parser which only attempts to parse the
most common Postfix log lines (but fails miserably), provides reasonably
flexible report generation from the limited data extracted, but has no
facilities to extract data for use in other tools.

\url{http://yaala.org/} \newline (Last checked 2007/10/09.)

\subsubsection{Lire}

\textit{As any good system administrator knows, there's a lot more to keep
track of in an active network than just webservers. Lire is hands down the
most versatile log analysis software available today. Lire not only keeps
you informed about your HTTP, FTP, and mail traffic, it also reports on
your firewalls, your print servers, and your DNS activity. The ever growing
list of Lire-supported services clearly outstrips any other software, in
large part thanks to the numerous volunteers who have pioneered many new
services and features. Lire is a total solution for your log analysis
needs.\/}

Lire is a general purpose log parser supporting many different types of log
file.  Its Postfix parser extracts the following data from Postfix log
files: \textit{The email servers' reports will show you the number of
deliveries and the volume of email delivered by day, the domains from which
you receive or send the most emails, the relays most used, etc.\/}; notably
rejections are not mentioned or dealt with, and unlike \parsername{} there
is no facility to extend the parser.  It supports multiple output formats
for generated reports (text, \HTML{}, \PDF{} and Excel 95) but the reports
do not appear to be customisable; \parsername{} does not produce any
reports.  Lire supports saving extracted data for later report generation,
but accessing this data from another application is undocumented; given the
source code it should be possible, with sufficient time and effort, to
access the data from an external program.

Like AWStats and Logrep, Lire attempts to correlate log lines by queueid,
but not by \pid{}, so the complete list of recipients for a mail should be
available; however its parser extracts only part of the available data and
makes no attempt to deal with the other complications described in
\sectionref{complications} \sectionref{additional complications}.
\parsername{} uses an \SQL{} database to make accessing the extracted data
as easy as possible.  When testing Lire on the \numberOFlogFILES{} test log
files it performed reasonably well, producing summaries of: 

\begin{itemize}

    \item Delivery status and failed deliveries.

    \item Sender and recipient domains and servers.

    \item Number of deliveries and bytes per day and hour.

    \item Recipients by domain.

    \item Deliveries by relays, by size and by delay.

    \item Delays by server and domain.

    \item Most mails exchanged between a pair of correspondents.

\end{itemize}

The numbers it reports appear reasonable, and the subset verified by the
author were correct.  Lire produces a report with less detail than
Pflogsumm, and is considerable harder to configure.

\url{http://logreport.org/lire.html} \newline (Last checked 2008/04/29.)

\subsubsection{Logrep}

\textit{Logrep is a secure multi-platform framework for the collection,
extraction, and presentation of information from various log files. It
features HTML reports, multi dimensional analysis, overview pages, SSH
communication, and graphs, and supports over 30 popular systems including
Snort, Squid, Postfix, Apache, Sendmail, syslog, ipchains, iptables, NT
event logs, Firewall-1, wtmp, xferlog, Oracle listener and Pix.\/}

Logrep extracts roughly half the fields \parsername{} does:

\begin{itemize}

    \item For mail sent and received: from address, size, and time and
        date.

    \item For mail sent: to addresses, \SMTP{} code, and delay.

    \item For mail received: the hostname of the sender.

\end{itemize}

It also counts the number of log lines parsed and skipped.  Log lines are
correlated based on the queueid (referred to as sessionname [sic] within
Logrep), but not by \pid{}.  The parsing is error prone: empty fields are
saved when the log line doesn't match the \regex{}, though it appears that
they will not overwrite existing data.  Most notably rejections are
completely ignored, making it unsuitable for the purposes of this project.
It doesn't attempt to address any of the complications in
\sectionref{complications} \sectionref{additional complications} except for
correlating by queueid.

Logrep does not come with any documentation, though some scant
documentation is available on its website (\parsername{} provides copious
documentation).  It requires a web browser to interact with it, ensuring
that automated log processing will be difficult, whereas automated
processing is a key part of \parsernames{} design.  Sadly all the author's
attempts to use Logrep failed, as it was unable to access the log files
selected; this appears to be a bug rather than operator error.\footnote{If
it is caused by operator error, the interface needs improvement as the
(minimal) instructions were followed as closely as possible, and multiple
attempts were made.}  As parsing failed it wasn't possible to review the
reports Logrep can generate (available in \HTML{} only), nor to examine the
(undocumented) format in which it can save extracted data for subsequent
reuse.

Logrep extracts far less data from Postfix log files than \parsername{},
completely ignores rejections, is effectively undocumented, doesn't deal
with the more complicated aspects of Postfix log files, and at the time of
writing doesn't work properly.

\url{http://www.itefix.no/phpws/index.php} \newline (Last checked
2007/11/18.)

\subsubsection{Log Mail Analyser}

Please see the previous in-depth discussion of Log Mail Analyser in
\sectionref{prior art}.


\subsection{Conclusion}

\begin{table}[ht]
\begin{tabular}{l|l|l|l|l|l}
    Parser         & Parsed     & Data        & Custom        & Documentation? & Source \\
                   & test logs? & store?      & reports?      &                & code?  \\
    \hline         &            &             &               &                &        \\
    Pflogsumm      & Yes        & No          & Partial\dag   & Good           & Yes    \\
    Sawmill        & Yes        & Yes         & Searches      & Reasonable     & No     \\
    Splunk         & Yes        & Yes         & Searches      & Abundant       & No     \\
                   &            &             & \& reports    & but poor       &        \\
    Isoqlog        & No         & Yes         & No            & No             & Yes    \\
    AWStats        & Partially  & Yes         & Partial\dag   & Good           & Yes    \\
    Throughput     & N/A        & No          & No            & Brief but      & Yes    \\
    \tab{}monitor  &            &             &               & good           &        \\
    Anteater       & No         & No          & No            & No             & Yes    \\
    YAALA          & No         & Yes\ddag    & Searches      & Poor           & Yes    \\
    Lire           & Yes        & Yes         & Yes           & Yes            & Yes    \\
    Logrep         & No         & Yes         & No            & Barely         & Yes    \\
    \LMA{}         & No         & Yes         & No            & No             & Yes    \\ 
    \parsername{}  & Yes        & Yes\dag\dag & No            & Yes            & Yes    \\
\end{tabular}

\dag Sections can be omitted from a report, but extra sections can not be
added.

\ddag YAALA only stores the data required to produce the latest report;
other data will be discarded.

\dag\dag\parsername{} is the only parser with documentation for its data
store.

\caption{Summary of parsers' features}
\label{Summary of parsers' features}
\end{table}

While there are other programs available which perform basic Postfix log
parsing (some to a greater level of detail than others), few attempt to
correlate log lines by queueid (none correlate by \pid{}) to produce an
overall record of the journey each mail traverses through Postfix.  None of
the reviewed parsers collect the breadth of information gathered by
\parsername{}, nor make it as easy to extend the parser to handle new log
lines.  Most other parsers immediately generate a report and discard the
data extracted from the log files; those which don't discard the data
retain it in a format inaccessible to other tools.  Nearly all of the
parsers reviewed can produce a report of greater or lesser detail and
complexity, a facility not offered by \parsername{}; reporting is deferred
to a subsequent application.  The quality of the documentation offered by
the subset of parsers which provide some varies from unusable to good; none
of the parsers provide any documentation on the format of their data stores
(if any).  Less than half of the parsers were capable of parsing the
\numberOFlogFILES{} test log files, and improving or extending the parsers
would have been quite a difficult task for any of the parsers.  Table
\refwithpage{Summary of parsers' features} provides a summary of the
parsers' features.

The overriding difference between \parsername{} and the other parsers
reviewed herein is that none of them aim to achieve the high level of
understanding of Postfix log files achieved by \parsername{}.
