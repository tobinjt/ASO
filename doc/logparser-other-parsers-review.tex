\chapter{State Of The Art Review}

\label{state of the art review}

At the start of this project ten Postfix log file parsers were tested, with
the hope of finding a suitable parser to build upon, rather than starting
from scratch.  It was quite difficult to find ten parsers to review for
this project, and the functionality offered by those parsers ranges from
quite basic to much more mature, depending on the needs of the author of
the parser.

It was hoped to reuse an existing parser rather than writing one from
scratch, but the existing parsers considered were rejected for one or more
reasons.  The effort required to adapt and improve an existing parser was
judged to be greater than the effort to write a new one, because the
techniques used by the existing parsers severely limited their potential:
some ignored the majority of log lines, parsing specific log lines
accurately, but without any provision for parsing new or similar log lines;
others sloppily parsed the majority of log lines, but were incapable of
distinguishing between log lines of the same category, e.g.\ rejecting a
mail delivery attempt.  The first parser reviewed~\cite{log-mail-analyser}
is the only previously published research in this area that the author is
aware of; their research aims to show that providing the data from log
files in a more accessible form is helpful to systems administrators.

The ten parsers have been reviewed and compared with \parsername{}, this
project's finished parser, to show how much effort would have been required
to use those to fulfil the aims and requirements of this project.  It is
important to compare and contrast newly developed algorithms and parsers
against those already available, to accurately judge what improvements, if
any, are delivered by the newcomers.

Some important differences exist between \parsername{} and most or all of
the parsers reviewed here:

\begin{enumerate}

    \item None of the parsers reviewed perform the kind of advanced parsing
        required for this project or deal with the complications described
        in \sectionref{complications}.

    \item Only \parsername{} enables parsing of new log lines without
        extensive and intrusive modifications to the parser; \parsernames{}
        architecture is described in \sectionref{parser architecture}.

    \item The parsers reviewed all produce a report of varying complexity
        and detail, whereas \parsername{} does not; it extracts data and
        leaves generation of reports from the data to other programs.
        Using an \acronym{SQL} database simplifies the process of
        generating such reports (discussed in \sectionref{database as
        API}); some sample queries are given in \sectionref{motivation}.
        The parser developed for this project is designed to enable much
        more detailed log file analysis by providing a stable platform for
        subsequent programs to develop upon.

    \item Most of the reviewed parsers silently ignore log lines they
        cannot handle, whereas \parsername{} complains loudly about every
        single log line it fails to parse.  The exception is AWStats, which
        outputs the percentage of log lines it was unable to parse, but
        does not output the log lines themselves.

    \item A minor difference is that most parsers do not handle compressed
        log files; both \parsername{} and Splunk handle them transparently,
        without user intervention; Sawmill and Lire can be configured to
        support compressed log files, but Sawmill exhibits a dramatic
        increase in parsing time when doing so.  Support for reading
        compressed log files is quite helpful, as it dramatically reduces
        the disk space required to store historical log files.

    \item Most of the reviewed parsers do not distinguish between different
        delivery attempt rejections, so they cannot be used to determine
        the success rate of different anti-spam techniques.  The exception
        is Pflogsumm, which provides a summary of why delivery attempts
        were rejected.

\end{enumerate}

Each of the reviewed parsers was tested with the \numberOFlogFILES{} test
log files described in \sectionref{parser efficiency}; \tableref{Summary of
parsers' features} summarises the results of this review.

The data extracted by \parsername{} is documented in
\sectionref{connections table} and \sectionref{results table}; for
convenience that list is repeated here: client and server \acronym{IP}
address and hostname, HELO hostname, queueid, start time, end time,
\acronym{SMTP} code, enhanced status code, sender, recipient, size, message
ID, delay and delays.

\section{Log Mail Analyser}

\label{log mail analyser}

There only appears to be one prior published paper about parsing Postfix
log files: \textit{Log Mail Analyzer: Architecture and Practical
Utilizations\/}~\cite{log-mail-analyser}.  The aim of \acronym{LMA} is
quite different from \parsername{}: it attempts to present correlated data
from log files in a form suitable for a systems administrator to search
using the myriad of standard Unix text processing utilities already
available.  It produces a \acronym{CSV} file and either a MySQL or Berkeley
DB database.  The decision to support both \acronym{CSV} and Berkeley DB
appears to have been a serious limitation: both formats have limitations
that will be explained later.  Hardly any documentation is provided with
\acronym{LMA}, though some documentation is available
in~\cite{log-mail-analyser}.  Studying the source code is informative,
though this author had difficulty as the authors of \acronym{LMA} wrote in
Italian.

\acronym{CSV} is a very simple format where each record is stored in a
single line, with fields separated by a comma or other punctuation symbol.
Problems with \acronym{CSV} files include the need to escape separators in
the data stored, providing multiple values for a field (e.g.\ multiple
recipients), and adding new fields.  \acronym{CSV} files do not have a
standard mechanism to document the fields or the separator, unlike
\acronym{SQL} databases where every database includes a schema naming the
fields and the type of data they store (e.g.\ integer, text, timestamp).
The \acronym{CSV} record format is not documented, but the output file
contains a comment giving the format:\newline{} \texttt{\# Timestamp|Nome
Client|IP Client|IP Server|From|To|Status|Size} \newline{}\acronym{LMA}
treats \acronym{CSV} lines starting with \texttt{\#} as comments, but not
all \acronym{CSV} parsers will.

Berkeley DB only supports storing simple \textbf{(key, value)} pairs,
unlike \acronym{SQL} databases that store arbitrary tuples.  In
\acronym{LMA}'s main table the key is an integer referred to by secondary
tables, and the value is a \acronym{CSV} line containing all of the data
for that row.  The secondary by-sender, by-recipient, by-date, and
by-\acronym{IP} tables use the sender/recipient/date/\acronym{IP} address
as the key, and the value is a \acronym{CSV} list of integers referring to
the main table.  This effectively re-implements \acronym{SQL} foreign keys,
but without the functionality offered by even the most basic of
\acronym{SQL} databases (e.g.\ joins, ordering, searches).  It also
requires custom code to search on some combination of the above, though the
authors of \acronym{LMA} did provide some queries: IP-STORY, FROM-STORY,
DAILY-EMAIL, and DAILY-REJECT\@.  Berkeley DB appears to be the least
useful of the three output formats: it does not provide the functionality
of a basic \acronym{SQL} database, and unlike \acronym{CSV} files it cannot
be used with standard Unix text processing tools.

The schema used with the MySQL database is undocumented, but at least it is
possible to discover the schema with an existing \acronym{SQL} database,
unlike with Berkeley DB\@; all \acronym{SQL} databases embed the schema
into the database and provide commands for displaying it.  Berkeley DB does
not embed a schema in its files, because there is neither requirement nor
benefit; it only provides \textbf{(key, value)} pairs, so any additional
structuring of the data is imposed by the application, thus the application
must document this structure.  MySQL support was not tested because the
schema required is not documented.

Whether a MySQL database or Berkeley DB table is chosen in addition to the
\acronym{CSV} output, \acronym{LMA} stores the following data: time and
date of the log line, client hostname and \acronym{IP} address, server
\acronym{IP} address, sender and recipient addresses, \acronym{SMTP} code,
and size (for accepted mails only).  Unlike \parsername{} it does not store
the server hostname, HELO hostname, queueid, start and end times,
timestamps for each log line, enhanced status code, delivery delays, or
message id (for accepted mails only).  Handling of multiple recipients,
\acronym{SMTP} codes, or remote servers\footnote{A single mail may be sent
to multiple remote servers if it was addressed to recipients in different
domains, or Postfix needs to try multiple servers for one or more
recipients.} is not explained; experimental observation shows that multiple
records are added when a mail had multiple recipients (sadly the records
are not associated or linked in any way), and presumably the same approach
is taken when there were multiple destination servers.

\acronym{LMA} requires major changes to the parser code to parse new log lines
or to extract additional data.  The code is structured as a long series of
blocks that each handle all log lines matching a single regex, so parsing
new log lines requires modifying an existing regex or carefully inserting a
new block in the correct place; extracting extra data will require
modifying multiple blocks, regexes, or both.

\acronym{LMA} does not deal with any of the complications discussed in
\sectionref{complications}, except for correlating log lines by queueid;
not correlating log lines by pid means it cannot correlate most rejections.
It does not differentiate between different types of rejections, so it is
not suitable for the purposes of this project; the data about which
restriction caused the rejection is discarded, whereas the main goal of
this project is to retain that data to aid optimisation and evaluation of
anti-spam techniques.  \acronym{LMA} fails to parse Postfix log files
generated on Solaris hosts because the fields automatically prepended to
each log line differ from those added on Linux hosts; log files from
Solaris hosts (and possibly other operating systems) thus require
pre-processing before parsing by \acronym{LMA}.  The \numberOFlogFILES{}
pre-processed test log files were parsed without complaint by
\acronym{LMA}, although it produced 32 entries in the output \acronym{CSV}
file for every rejection in the input log file; it also missed some 40\% of
delivered mail.  Once these deficiencies were discovered the author did not
spend any more time checking the results.

\acronym{LMA} does provide some simple reports: IP-STORY, FROM-STORY,
DAILY-EMAIL and DAILY-REJECT\@.  These reports search the Berkeley DB files
for matching records: the first three extract \acronym{CSV} lines for the
specified client \acronym{IP} address, sender address, or date
respectively.  DAILY-REJECT initially failed with an error message from the
Perl interpreter;\footnote{The error messages were:
\newline{}\texttt{Undefined subroutine \&main::LIST called at queryDB.pl
line 372.}\newline{}\texttt{Undefined subroutine \&main::EXTRACT\_FROM\_DB
called at queryDB.pl line 379.}} after correcting the errors in the code it
worked, extracting the \acronym{CSV} lines for the specified day where the
\acronym{SMTP} code signifies a rejection.  All of these reports are
trivially simple to produce from the \acronym{CSV} file using the standard
Unix tool \texttt{awk}\glsadd{awk}; the most complicated, DAILY-REJECT, is
merely:

% perl queryDB.pl -dayreject 2007-01-26 > lma-query

\begin{verbatim}
awk -F\| 'BEGIN { previous = "" };
    $1 ~ /2007-01-26/ && $7 != "250" && $0 != previous
    { print $0; print " "; previous = $0; }' lma_output.txt
\end{verbatim}

Notes about the command above:

\begin{itemize}

    \item It outputs a line containing only a single space after each
        matching record, to accurately replicate the output of
        DAILY-REJECT\@.

    \item DAILY-REJECT considers all \acronym{SMTP} codes except ``250'' to
        be rejections; this includes invalid \acronym{SMTP} codes such as
        \texttt{0} and \texttt{deferred}, so the awk command does too.
        These invalid \acronym{SMTP} codes are most likely present because
        of incorrect parsing by \acronym{LMA}.

    \item \acronym{LMA} produces 32 lines in its \acronym{CSV} file for
        every single line it should have produced; the command above
        suppresses duplicate sequential lines.  DAILY-REJECT produces the
        correct number of output lines, probably because it uses the
        Berkeley DB files as input rather than the \acronym{CSV} file.

\end{itemize}

The output from DAILY-REJECT and the \texttt{awk} command is not exactly
the same; the author did not spend substantial time attempting to explain
these differences.

\begin{enumerate}

    \item The output from DAILY-REJECT is missing some records that are
        present in the \acronym{CSV} file; this may be because it uses the
        Berkeley DB files instead, and there may be differences between the
        contents.

    \item Some records output by DAILY-REJECT are truncated: they are
        missing the last $|$ that separates the fields and the newline
        following it, so the line containing only a single space is
        concatenated with the record.

\end{enumerate}

In summary, \acronym{LMA} appears to be a proof of concept, written to
demonstrate the point of their paper (that having this information in an
accessible fashion is useful to systems administrators), rather than a
program designed to be useful in a production environment.

% Literature review notes:
%
% Hard-coded parsing, requiring code changes to add more.  Attempts to
% correlate log lines, saves data to database for data mining purposes.
% Hard to extend/expand/understand.  Appears to only save: date and hour,
% DNS name and \acronym{IP} address host, mail server \acronym{IP} address,
% sender, receiver and e-mail status (sent, rejected).  Undocumented
% schema.  Design decision to use \acronym{CSV} as an intermediate format
% between the log file and the database seems to have been restrictive.
% Appears to require a queueid but majority of log lines (e.g.\ rejections)
% lack a queueid.  Supports whitelisting \acronym{IP} addresses when
% parsing logs, but whitelisting when generating reports/data mining would
% be preferable.  Supporting Berkeley DB is probably limiting the software;
% an example is the difficulty in searching a pipe-delimited string, so
% they have re-implemented foreign keys with tables keyed by ip address
% etc.\ pointing at the main table - this also will not scale well.  There
% does not appear to be any attempt to deal with the complications I have
% encountered: their parsing is not detailed enough to encounter them.  It
% does not run properly; does not create any output; throws up errors.

\section{Pflogsumm}

\parserblurb{%
    pflogsumm is designed to provide an over-view of Postfix activity, with
    just enough detail to give the administrator a ``heads up'' for
    potential trouble spots.
}
{http://jimsun.linxnet.com/postfix_contrib.html}
{2008/11/23}

Pflogsumm produces a report designed for troubleshooting rather than
in-depth analysis.  It does not support saving any data, and it does not
extract any data that it does not require to produce its report; e.g.\ it
does not extract the HELO hostname, queueid, start and end times,
timestamps for each log line, or message id.  Both the parsing and
reporting are difficult to extend because it is a specialised tool, unlike
the easily extensible design of \parsername{}.  It does not correlate log
lines by queueid or \acronym{pid}, and does not need to deal with the
complications encountered during this project (\sectionref{complications}).
Pflogsumm produces a useful report, and successfully parsed the
\numberOFlogFILES{} log files it was tested with.\footnote{The results it
reported were not verified in detail, but it did not report any errors, and
has an excellent reputation amongst Postfix users.}  Pflogsumm has many
options to include or exclude certain sections of the report, all clearly
documented; by default it includes the following:

\begin{itemize}

    \item Total number of mails accepted, delivered, and rejected.  Total
        size of mails accepted and delivered.  Total number of sender and
        recipient addresses and domains.

    \item Per-hour averages and per-day summaries of the number of mails
        received, delivered, deferred, bounced, and rejected.

    \item For received mail: per-domain totals for mails sent, deferred,
        average delay, maximum delay, and bytes delivered.  For received
        mail: per-domain totals for size and number of mails received.

    \item Number and size of mails sent and received for each address.

    \item Summary of why mail delivery was deferred or failed, why mails
        were bounced, why mails were rejected, and warning messages.

\end{itemize}

Pflogsumm is the only reviewed parser that distinguishes between different
rejections.

\section{Sawmill Universal Log File Analysis And Reporting}

\parserblurb{%
    Sawmill is a Postfix log analyzer (it also support 818 other log
    formats).  It can process log files in Postfix format, and generate
    dynamic statistics from them, analyzing and reporting events.  Sawmill
    can parse Postfix logs, import them into a SQL database (or its own
    built-in database), aggregate them, and generate dynamically filtered
    reports, all through a web interface.  Sawmill can perform Postfix
    analysis on any platform, including Window~[sic], Linux, FreeBSD,
    OpenBSD, Mac OS, Solaris, other UNIX, and more.
}
{http://www.thesawmill.co.uk/formats/postfix.html}
{2008/11/23}

Sawmill is a general purpose commercial product that parses 818 log file
formats (as of 2008/11/23) and produces reports from the extracted data.
Its data extraction facilities (described later) are too limited to save
enough data for the purposes of this project: although it can extract three
different sets of data from Postfix log files, they are not interlinked in
any way.  The documentation does not suggest that Sawmill correlates log
lines by either queueid or pid or deals with the other difficulties
documented in \sectionref{complications}.

Sawmill has three different Postfix log file parsers, extracting three
different sets of data:

\begin{enumerate}

    \item \urlLastChecked[\hfill{}\newline{}]{http://www.thesawmill.co.uk/formats/postfix.html}{2008/11/23}.
        Fields extracted: from, to, server, UID, relay, status, number of
        recipients, origin hostname, origin \acronym{IP} address, and
        virus.  It also counts the number of and total size of all mails
        delivered.  The fields \texttt{server}, \texttt{uid},
        \texttt{relay}, and \texttt{virus} are not explained in the
        documentation: \texttt{server} is probably the hostname or
        \acronym{IP} address of the server the mail is delivered to;
        \texttt{relay} might be the delivery method: \acronym{SMTP}, local
        delivery, or \acronym{LMTP}; \texttt{uid} might be the uid of the
        user submitting mail locally.  Postfix does not do any form of
        virus checking (though it has many options for cooperating with an
        external virus scanner), so the \texttt{virus} field is a mystery.

    \item \urlLastChecked[\hfill{}\newline{}]{http://www.thesawmill.co.uk/formats/postfix_ii.html}{2008/11/23}.
        Fields extracted: from, to, RBL list, client hostname, and client
        \acronym{IP} address.  It also counts the number and total size of
        all mails delivered.

    \item \urlLastChecked[\hfill{}\newline{}]{http://www.thesawmill.co.uk/formats/beta_postfix.html}{2008/11/23}.
        Fields extracted: from, to, client hostname, client \acronym{IP}
        address, relay hostname, relay \acronym{IP} address, status,
        response code, RBL list, and message id.  It also counts the number
        and size of all mails delivered, processed, blocked, expired, and
        bounced.

\end{enumerate}

Even if the three data sets were combined Sawmill would extract less data
than \parsername{}: it omits the HELO hostname, queueid, enhanced status
code, delivery delays, and start and end times.  Sawmill does not extract
any data about rejections except when the rejection is caused by a
\acronym{DNSBL} check (\texttt{RBL list} in the list of fields).

The source code is only available in an encrypted form, to support people
who wish to use Sawmill on operating systems or machine architectures the
company do not provide executables for.  Sawmill is quite expensive,
requiring a \euros{100} + VAT licence per report, with discounts available
when buying multiple licences (correct as of 2008/11/23); in contrast,
\parsername{} is free to use and the code is freely available.  Sawmill is
supplied with thorough and well written documentation; everything the
author searched for was documented, except the MySQL database schema and
some details of the data extracted by the parser.  A commercial version of
MySQL is required because of MySQL licensing restrictions, but Sawmill's
documentation explains why and includes instructions on how to compile
Sawmill so that it can use a non-commercial version of MySQL (this was not
attempted during the review process).

Sawmill's web interface supports searching on any combination of the fields
it extracts, and all searches produced accurate results.  The interface for
searching is neither as simple to use nor as informative as the interface
provided by Splunk (see \sectionref{Splunk review}).  The administrative
interface is much easier to use than Splunk's: it took only five minutes to
start parsing a whole directory of log files.

When tested with the \numberOFlogFILES{} test log files it performed
adequately, though the rate it processed log files at did slow down
noticeably as it progressed.  Sawmill supports reading compressed log files
but it exhibits a dramatic slow down when doing so: it took six hours to
parse the first half of the log files, and twelve hours to parse the next
third; after twenty four hours parsing the remaining sixth it crashed due
to lack of disk space.  On the second parsing attempt the log files were
uncompressed beforehand and parsing took eight hours.

In summary, Sawmill suffers from supporting so many types of log files: it
parses many types of log files, but none of them very well; it is probably
much more useful when parsing log files where each log line is
self-contained (e.g.\ web server log files), rather than log files
containing interlinked log lines.  It is not suitable as a base for this
parser, as the source code made available is encrypted and not intended
for modification; in addition the architecture would probably need to be
overhauled or replaced to deal with correlating log lines.

\section{Splunk}

\label{Splunk review}

\parserblurb{%
    Splunk is IT Search.  Search and navigate IT data from applications,
    servers and network devices in real-time.  Logs, configurations,
    messages, traps and alerts, scripts, code, metrics and more.  If a
    machine can generate it --- Splunk can eat it.  It\empty{}'s easy to
    download and use and it\empty{}'s very powerful.
}
{http://www.splunk.com/}
{2008/11/23}

Splunk aims to index all of an organisation's log files, providing a
centralised view capable of searching and correlating diverse log sources.
The web interface provides search functionality, generating statistics and
graphs in real time, a facility not provided by \parsername{}.  Splunk
allows quite complicated searches, based on the fields extracted by Splunk
(described later) or the full text of the log line, though it is not
possible to search on partial words.  Search results are not available for
use with other tools.  Searches can be saved for reuse; saved searches can
be run periodically and the results mailed to a recipient or sent to an
external program for further processing.  The author was unable to save
searches, though that may have been because of limitations in the free
version.  The data store is not available for use by external programs,
whereas \parsername{} provides the database and leaves it to the user to
utilise it without limit or restriction.  The web interface is optimised
for interactive use rather than automated queries, and it does not appear
to be possible to write independent tools to utilise the Splunk database.
Some additional Postfix reports are supposedly available on
\urlLastChecked{http://www.splunkbase.com/}{2008/11/23}, but the author was
unable to find any Postfix reports, or indeed reports for any other log
file types: every category was empty, even those that the web site claimed
had numerous reports available.  Many types of graphs can be generated,
though most are variations of a bar or pie chart, except the bubble and
heatmap graphs.  It is easy to drill down through the graphs to select a
portion of the data (e.g.\ select the hour with the largest number of
events, then select a particular host, and finally a specific sender
address).  All searches performed using the indexed data returned
reasonable results.

The web interface is quite attractive and simple to use when searching, but
as an administrator it seems unnecessarily difficult to perform simple
tasks.  When testing Splunk it took roughly 30 minutes to figure out how to
add a single log file to be indexed so that it could be searched, with the
downside that the log file was copied into a spool directory before
indexing, doubling the disk space usage.  The next test was to index all
the log files in a particular directory, but after three hours, numerous
futile attempts, and reading all the available documentation, the author
was still unable to index all the log files in a directory using the web
interface.  Using the \acronym{CLI} rather than the web interface was
partially successful: the command ``\texttt{splunk find logs
}\textit{log-directory\/}'' added 40 of the \numberOFlogFILES{} log files
to the queue for indexing.  Further attempts enqueued the same 40 log
files, without explaining why the others were excluded.\footnote{The log
files appear to have been indexed once only; presumably Splunk keeps track
of the log files it has indexed and discards requests to index log files
for a second time.  This may or may not be a useful feature for
\parsername{}.} There did not appear to be an option to ensure the log
files would be processed in the order they were created, though this may be
neither necessary nor beneficial with Splunk.  Subsequently the author was
successful in adding a single file at a time using the \acronym{CLI} and
the command ``\texttt{splunk add tail }\textit{filename\/}''.  A simple
loop using that command was enough to add all the desired log files.
Splunk will periodically check all indexed log files for updates unless
they are manually removed from its list; this may or may not be useful
behaviour.  Splunk did not appear to have any difficulty in indexing the
log files, once they had been successfully added to its queue.
\parsername{} parses the logs it is instructed to parse, in the order
given; periodic parsing of logs is a task an administrator can easily
achieve with \texttt{cron(8)} and \texttt{logrotate(8)}.

Copious documentation is made available on
\urlLastChecked{http://www.splunk.com/}{2008/11/23}, but the organisation
and abundance of material does not support easy identification of useful
information.  Searches confusingly tended to return results from old
documentation rather than new.  In general the documentation appears to
have been written by someone intimately acquainted with the software, who
has difficulty understanding how a newcomer would approach tasks or the
questions they would ask.

Splunk supports reading compressed log files without any configuration by
the user.  The free version of Splunk limits the volume of data indexed per
day to 500MB, though a trial Enterprise licence is available that allows
indexing of up to 5GB of data per day.  In 2007, the cheapest licenced
version cost \$5000 plus \$1000 support, and limited the volume of data
indexed per day to 500MB\@.  Prices were removed from the Splunk website
during 2008; now Splunk's sales team must be asked for a quote.  Typical
log file sizes for a small scale mail server are given in
\sectionref{parser efficiency}.

When parsing Postfix log files Splunk parses the standard
syslog\glsadd{syslog} fields at the beginning of the log line, and extracts
any \texttt{key=value} pairs occurring after the standard syslog prologue:
to and from addresses, HELO hostname, and protocol (\acronym{SMTP} or
\acronym{ESMTP}).  \parsername{} extracts noticeably more data (client and
server \acronym{IP} address and hostname, queueid, start and end times,
timestamps for each log line, \acronym{SMTP} and enhanced status codes,
delivery delays, and message ID), though it does not make the full text of
the line available (this could be trivially added if desired, but would
greatly increase the size of the resulting database).  The full power of
\acronym{SQL} is available when searching the data extracted by
\parsername{}, allowing the user to search on arbitrarily complicated
conditions.

Splunk is a generic tool, so it lacks any Postfix specific support over and
above extracting the \texttt{key=value} fields from a log line; it makes no
attempt to correlate log lines by queueid or \acronym{pid}, or to handle
any of the other myriad complications discussed in
\sectionref{complications}.  Its source code is unavailable, so it could
not be used as a base for this project, even if it fulfilled all other
requirements.

\section{Isoqlog}

\parserblurb{%
    Isoqlog is an MTA log analysis program written in C.  It designed to
    scan qmail, postfix, sendmail and exim logfile and produce usage
    statistics in HTML format for viewing through a browser.  It produces
    Top domains output according to Sender, Receiver, Total mails and
    bytes; it keeps your main domain mail statistics with regard to Days
    Top Domain, Top Users values for per day, per month and years.
}
{http://www.enderunix.org/isoqlog/}
{2009/01/11}

Isoqlog's report misses most of the information gathered by \parsername{}:
the data extracted is limited to the number of mails sent by each sender,
and it only reports on senders from the domains listed in its configuration
file, making it impossible to produce complete reports.  It ignores all log
lines except those with today's date, so it is impossible to analyse
historical log files, and testing with the \numberOFlogFILES{} test log
files was pointless.  It does maintain a record of data previously
extracted and the newly extracted data is merged into it; the format of the
data store is undocumented.  Almost no documentation is provided with
Isoqlog, little more than installation instructions.  It does not utilise
rejection log lines in any way, so is unsuitable for the purposes of this
project.  Its parsing is completely inextensible, indeed is almost
incomprehensible, relying on \texttt{scanf(3)}, unexplained fixed offsets,
and low level string manipulation; it is the opposite end of the spectrum
to \parsernames{} parsing.  It does not handle any of the complications
discussed in \sectionref{complications}, does not gather the breadth of
data required for this project, and ignores most of the log lines produced
by Postfix.

\section{AWStats}

\parserblurb{%
    AWStats is a free powerful and featureful tool that generates advanced
    web, streaming, ftp or mail server statistics, graphically.  This log
    analyzer works as a CGI or from command line and shows you all possible
    information your log contains, in few graphical web pages.  It uses a
    partial information file to be able to process large log files, often
    and quickly.  It can analyze log files from all major server tools like
    Apache log files (NCSA combined/XLF/ELF log format or common/CLF log
    format), WebStar, IIS (W3C log format) and a lot of other web, proxy,
    wap, streaming servers, mail servers and some ftp servers.
}
{http://awstats.sourceforge.net/awstats.mail.html}
{2009/01/11}

AWStats can produce simple graphs from many different services' log files,
but supporting numerous log files formats without special purpose code
limits its functionality.  The data it can extract from an \acronym{MTA}
log file is limited in comparison to \parsername{}: time2, email, email\_r,
host, host\_r, method, url, code, and bytesd.  No explanation for any of
those fields is provided in the documentation at
\urlLastChecked{http://awstats.sourceforge.net/docs/awstats_faq.html\#MAIL}{2008/11/23},
so the author could neither understand what data is extracted, nor
determine what data is missing in comparison to \parsername{} (which fully
documents everything it extracts).  AWStats coerces Postfix log files into
the log file format used by the Apache web server, for analysis by AWStats'
HTTP log file parser.  The converting parser only deals with a small
portion of the log lines generated by Postfix, silently skipping those it
cannot deal with, and does not distinguish between different types of
rejection; extending it to parse all log lines would be at least as much
work as writing a new parser.  It does correlate log lines by queueid (not
by pid), but it does not deal with any of the other complications described
in \sectionref{complications}.  AWStats supports saving data, but the
format of the data store is not documented.  It also supports reading
compressed log files, but that functionality was not tested.

When tested with the \numberOFlogFILES{} test log files AWStats' reported
that it parsed 9,240,075 (88.70\%) of 10,416,129 log lines, skipping
1,176,050 (11.29\%) corrupt log lines; the \numberOFlogFILES{} log files
contain \numberOFlogLINES{} log lines, so AWStats parsed only 15.21\% of
the log lines, declared 1.93\% were corrupt, and ignored the remaining
82.85\%.  The author did not verify the correctness of the parsing of those
log lines AWStats did parse.

The graphs it produces give an overview of mails received for the last
calendar month, showing:

\begin{itemize}

    \item The number of mails accepted from each host.

    \item How many mails were received by each recipient.

    \item The average number of mails accepted by the server per-day and
        per-hour.

    \item A summary of the \acronym{SMTP} codes used when rejecting
        delivery attempts.

\end{itemize}

AWStats was not a suitable base for this project, because it assumes that
all log files can be rewritten to be compatible with web server log files,
and will contain similar data; coercing Postfix log files into web server
log files, without substantial data loss, would require fully parsing the
Postfix log files without using AWStats, i.e.\ would require writing a
separate parser.  It may be possible to use AWStats' graphing capabilities
to generate reports, by generating input for AWStats from the data
extracted by \parsername{}.

\section{Anteater}

\parserblurb{%
    The Anteater project is a Mail Traffic Analyser.  Anteater supports
    currently the logformat produced by Sendmail and by Postfix.  The tool
    is written in 100\% C++ and is very easy to customize.  Input, output,
    and the analysis are modular class objects with a clear interface.
    There are eight useful analyse modules, writing the result in plain
    ASCII or HTML, to stdout or to files.
}
{http://anteater.drzoom.ch/}
{2009/01/11}

Anteater does not have any English documentation except for the quote above
so it is impossible for this author to accurately comment on the analysis
it performs.  It did not run successfully when tested, and its parsing
would certainly be out of date because Postfix has evolved considerably
since this tool was last updated (2003/11/06).  As it neither ran
successfully nor has documentation the author can read, a detailed review
cannot be provided.

The Debian Linux distribution provides a translated manual page with the
copy of anteater it distributes, so the author was at least able to run
anteater with the correct arguments; sadly anteater produced zero for every
statistic, presumably because it was unsuccessful in parsing the log lines.

\section{Yet Another Advanced Logfile Analyser}

\parserblurb{%
    yaala is a very flexible analyser for all kinds of logfiles.  It uses
    parsers to extract information from a logfile, an SQL-like query
    language to relate the information to each other and an output-module
    to format the information appropriately.
}
{http://yaala.org/}
{2009/01/11}

YAALA uses a plugin-based system to analyse log files and produce reports
in HTML format, with all the parsing and report generation handled by
plugins.  Using YAALA as a base for this project would have been as much
work as starting from scratch, as both the input and output modules would
need to be written specially; it might be more work to implement a parser
within the constraints of YAALA\@.  YAALA supports storing previously
gathered data using Perl's Storable module, so other Perl programs can use
Storable to load, examine, and optionally modify the data; \parsername{}
uses a well documented database that is accessible from most common
programming languages.  Information about how YAALA stores data was gleaned
from the source code, as the documentation is sadly lacking.

YAALA provides a Postfix parser that extracts the following of fields from
specific log lines:

\begin{eqlist}

    \item [Aggregations:] count (not explained), bytes (sum of bytes
        transferred).

    \item [Keyfields:] incoming\_host, outgoing\_host, date, hour, sender,
        recipient, defer\_count, delay.  Which date and hour are stored is
        not documented: start time, end time, delivery time, or another
        time?

\end{eqlist}

\noindent{}YAALA's Postfix parser extracts some of the fields \parsername{}
does: for client and server it stores either the \acronym{IP} address or
the hostname, not both; it omits the HELO hostname, queueid, \acronym{SMTP}
and enhanced status codes, size of each accepted mail, start and end times,
timestamps for each log line, and message ID\@.  It extracts some data that
\parsername{} does not: how many times delivery was deferred for each mail;
this information can be determined from the database populated by
\parsername{} if desired.  Unlike \parsername{}, YAALA does not maintain
separate counters for each restriction; this rules out the possibility of
using the collected data for optimisation, testing, or understanding of
restrictions.  YAALA's Postfix parser does not deal with the complications
explained in \sectionref{complications}, except it does correlate log lines
by queueid.

YAALA provides a mini-language based on \acronym{SQL} that is used when
generating reports; sample reports can be seen
at~\urlLastChecked{http://www.yaala.org/samples.html}{2008/11/23}.  Example
query for HTTP proxy servers: \newline{} \tab{} \texttt{requests BY file
WHERE host =\~{} Google} \newline{} The mini-language is quite limited and
cannot be used to extract data for external use, merely to create reports.
Only data selected by the query will be saved in the data store; other data
will be discarded, and removed from the data store if already present.

Testing YAALA was unsuccessful because all the select clauses tried
produced a similar error message:
\newline{}\tab{}\texttt{lib/Yaala/Data/Core.pm: Unavailable aggregation
requested:} \newline{}\tab{}\tab{}\texttt{``bytes''.  Returning 0.}
\newline{}  The underlying reason for this is that YAALA only parsed 408
(0.11\%) of 360632 log lines in the first log file; it was not tested with
the remainder of the \numberOFlogFILES{} log files.

It might be possible to use \parsername{} as a plugin with YAALA, perhaps
with an intermediate plugin interfacing between the two, but YAALA's data
store is insufficient for \parsernames{} needs: \parsername{} uses two
separate tables, whereas YAALA assumes all data will reside in one
structure; YAALA's querying mini-language might not deal successfully with
data in separate structures.  This approach has not been attempted by the
author.

In summary, YAALA provides a Postfix parser that tries to parse the most
common Postfix log lines only, provides reasonably flexible report
generation from the limited data extracted, but has no facilities to
extract data for use in other tools.

\section{Lire}

\sloppy{}%
\parserblurb{%
    As any good system administrator knows, there's a lot more to keep
    track of in an active network than just webservers.  Lire is hands down
    the most versatile log analysis software available today.  Lire not only
    keeps you informed about your HTTP, FTP, and mail traffic, it also
    reports on your firewalls, your print servers, and your DNS activity.
    The ever growing list of Lire-supported services clearly outstrips any
    other software, in large part thanks to the numerous volunteers who
    have pioneered many new services and features.  Lire is a total solution
    for your log analysis needs.
}
{http://logreport.org/lire.html}
{2009/01/11}
\fussy{}

Lire is a general purpose log file parser supporting many different types
of log file.  It takes a similar approach to YAALA, using plugins to parse
different log file types.  The data extracted by its Postfix parser is not
clearly documented:

\begin{quotation}

    \noindent{} The email servers' reports will show you the number of
    deliveries and the volume of email delivered by day, the domains from
    which you receive or send the most emails, the relays most used, etc.

\end{quotation}

\noindent{}Examining the source code reveals that the parser looks for
\texttt{<key>=<value>} pairs in each log line, extracts them, and
correlates the data by queueid.  This approach will extract the following
data: HELO hostname, queueid, \acronym{SMTP} code, sender and recipient
addresses, and size of accepted mails.  It is unclear if the parser will
extract any further data.  Lire misses the following fields extracted by
\parsername{}: client and server \acronym{IP} address and hostname, start
and end times, enhanced status code, delivery delays, timestamps of each
log line, and message ID\@.

Lire supports multiple output formats for generated reports (text, HTML,
PDF, and Excel 95) but the reports do not appear to be customisable;
\parsername{} does not produce any reports.  Lire's report is not as
detailed as Pflogsumm's, and it is considerable harder to configure.  Lire
supports saving extracted data for later report generation, but the format
of this data store is undocumented; given the source code it should be
possible, with enough time and effort, to understand the format.
\parsername{} uses an \acronym{SQL} database to make accessing the
extracted data as easy as possible.  In general, Lire has poor
documentation.

Similar to AWStats and Logrep, Lire attempts to correlate log lines by
queueid, but not by \acronym{pid}, so the complete list of recipients for a
mail should be available; its parser extracts only part of the available
data and makes no attempt to deal with the other complications described in
\sectionref{complications}.  When testing Lire on the \numberOFlogFILES{}
test log files it performed reasonably well: the numbers it reports appear
reasonable, and the subset verified by the author were correct.  Its report
provided summaries of:

\begin{itemize}

    \item Delivery status and failed deliveries.

    \item Sender and recipient domains and servers.

    \item Number of deliveries and bytes per-day and per-hour.

    \item Recipients by domain.

    \item Deliveries by relays, by size, and by delay.

    \item Delays by server and by domain.

    \item The pair of correspondents that exchanged the highest number of
        mails.

\end{itemize}

Lire would not be a suitable base for this project: it does not extract
enough data; does not deal with rejections in any way; does not make the
extracted data easily available to other programs.  Its parser is
in-extensible but could easily be replaced, but that would require writing
a parser from scratch, so would not be any less work.  \parsername{} could
possibly be used to parse Postfix log files for Lire, but the difficulty
may outweigh the benefits.  Lire's data store may not be suitable for
storing the data extracted by \parsername{}, but the lack of documentation
hinders any evaluation.  As with YAALA this approach has not been attempted
by the author.

\section{Logrep}

\parserblurb{%
    Logrep is a secure multi-platform framework for the collection,
    extraction, and presentation of information from various log files.  It
    features HTML reports, multi dimensional analysis, overview pages, SSH
    communication, and graphs, and supports over 30 popular systems
    including Snort, Squid, Postfix, Apache, Sendmail, syslog, ipchains,
    iptables, NT event logs, Firewall-1, wtmp, xferlog, Oracle listener and
    Pix.
}
{http://www.itefix.no/i2/index.php}
{2009/01/11}

Logrep extracts fewer than half the fields \parsername{} does:

\begin{itemize}

    \item For mail sent and received: from address, size, and time and
        date.  Which date and hour are stored is not documented: start
        time, end time, delivery time, or another time?

    \item For mail sent: to addresses, \acronym{SMTP} code, and delay.

    \item For mail received: the hostname of the sender.

\end{itemize}

It also counts the number of log lines parsed and skipped.  It omits client
\acronym{IP} address and hostname, server \acronym{IP} address, HELO
hostname, queueid, timestamps of each log line, enhanced status code, and
message ID\@.  Log lines are correlated based on the queueid (called
sessionname [sic] within Logrep), but not by \acronym{pid}.  The parsing is
error prone: empty fields are saved when the log line does not match the
regex, though it appears that they will not overwrite existing data.  Most
notably rejections are completely ignored, making it unsuitable for the
purposes of this project.  It does not try to address any of the
complications in \sectionref{complications} except for correlating by
queueid.

Logrep does not come with any documentation, though some scant
documentation is available on its website (\parsername{} provides copious
documentation).  It requires a web browser to interact with it, so
automated log file processing will be difficult, whereas enabling automated
processing is a key part of \parsernames{} design.  Sadly all the author's
attempts to use Logrep failed, as it was unable to access the log files
selected; this appears to be a bug rather than operator error.  If it was
caused by operator error, the interface needs improvement as the (minimal)
instructions were followed as closely as possible, and multiple attempts
were made.  Because parsing failed it was not possible to review the
reports Logrep can generate (available in HTML only), or to examine the
(undocumented) format it uses to save extracted data for subsequent reuse.

Logrep extracts far less data from Postfix log files than \parsername{},
completely ignores rejections, is effectively undocumented, does not deal
with the more complicated aspects of Postfix log files, and at the time of
writing does not work properly.

\section{Summary}

This chapter has reviewed ten programs that perform basic Postfix log file
parsing (some to a greater level of detail than others).   None of the
reviewed parsers collect the breadth of information gathered by
\parsername{}, or make it as easy to extend the parser to handle new log
lines.  Some correlate log lines by queueid (none correlate by pid); none
deal with any of the other complications described in
\sectionref{complications}.  All of the reviewed parsers generate a report,
and some provide a greater or lesser degree of customisation.  Most have a
data store, but only \acronym{LMA} provides any documentation on its
format; some deliberately make the data store inaccessible to other tools.
Most but not all of the parsers provide documentation, with the quality
ranging from unusable to very good.  Fewer than half of the parsers were
capable of parsing the \numberOFlogFILES{} test log files; improving or
extending parsing would have been quite a difficult task for any of the
parsers, and one that the author did not have the time to attempt.
\Tableref{Summary of parsers' features} provides a summary of the parsers'
features.  The overriding difference between \parsername{} and the other
parsers reviewed herein is that none of them aim for the high level of
understanding of Postfix log files achieved by \parsername{}.


\begin{table}[htb]
    \caption{Summary of reviewed parsers' features}
    \empty{}\label{Summary of parsers' features}
    \begin{tabular}{llllll}
        \tabletopline{}%
        Parser          & Parsed test   & Data              & Custom            & Documentation  & Source       \\
                        & log files?    & store?            & reports?          & quality?       & code?        \\
        \tablemiddleline{}%
        \acronym{LMA}       & No            & Yes               & No                & Poor           & Yes          \\
        Pflogsumm       & Yes           & No                & Partial \dag{}    & Good           & Yes          \\
        Sawmill         & Yes           & Yes               & Searches          & Very good      & \nialpha{}   \\
        Splunk          & Yes           & Yes               & Searches          & Abundant       & No           \\
                        &               &                   & \& reports        & but poor       &              \\
        Isoqlog         & No            & Yes               & No                & No English     & Yes          \\
                        &               &                   &                   & documentation  &              \\
        AWStats         & Partially     & Yes               & Partial \dag{}    & Good           & Yes          \\
        Anteater        & No            & No                & No                & None           & Yes          \\
        YAALA           & No            & Yes \ddag{}       & Searches          & Poor           & Yes          \\
                        &               &                   & \& reports        &                &              \\
        Lire            & Yes           & Yes               & Yes               & Reasonable     & Yes          \\
        Logrep          & No            & Yes               & No                & None           & Yes          \\
        \parsername{}   & Yes           & Yes \nibeta{}     & No \nichi{}       & \niepsilon{}   & Yes          \\
        \tablebottomline{}%
    \end{tabular}

    \begin{eqlist}

        \item [\dag{}] Sections can be omitted from a report, but extra
            sections cannot be added.

        \item [\ddag{}] YAALA only stores the data required to produce the
            latest report; other data will be discarded.

        \item [\nialpha{}] Sawmill's source code is available in an
            encrypted form, so that customers can compile it on platforms
            that pre-compiled binaries are not available for.

        \item [\nibeta{}] \parsername{} is the only parser with
            documentation for its data store.

        \item [\nichi{}] \parsername{} defers report generation to
            subsequent programs, but all the necessary data and
            documentation to produce reports is provided.

        \item [\niepsilon{}] \parsername{} aims to have thorough and
            complete documentation, but the author cannot provide an
            unbiased review.

    \end{eqlist}

\end{table}

\clearpage{}
