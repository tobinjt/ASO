\section{Parsing Rules}

\label{rules}

This section discusses the rules used in parsing Postfix log files,
starting with rule characteristics, followed by the problems caused by
overlapping rules, and techniques to detect and deal with such overlaps.
An example rule and a log line it would match are provided, plus a
description of how the fields in the rule\footnote{The table containing the
rules has already been described in \sectionref{rule attributes}.} are used
when matching a log line and subsequently performing the requested action.
The section continues with a discussion of rule efficiency concerns, with
reference to the graphs in \sectionref{graphs}, and finishes with an
explanation of the algorithm used to generate new \regexes{} from unparsed
log lines.

Please refer to \sectionref{parser design} for a discussion of why the
rules and actions have been separated in the parser's design.

\subsection{Summary}

This section dealt with the rules used in parsing Postfix log files:

\begin{itemize}

    \item The characteristics of the rules were described.

    \item Detecting overlapping rules and dealing with the problems they
        can cause was covered, including a discussion of why overlapping
        rules can be helpful as well as harmful.

    \item An example log line and the rule matching it illustrated a
        description of how the fields in the rule are used both in the
        matching phase and the action that is subsequently executed.  XXX
        REWRITE ONCE EXAMPLE HAS CHANGED\@; MERGE WITH NEXT ITEM\@.

    \item The structure of the database table containing the rules is dealt
        with in \sectionref{rule attributes}, and is not duplicated in this
        section.

    \item The topic of rule efficiency was discussed, covering the effects
        of caching compiled \regexes{} and optimal ordering of rules, with
        reference to the graphs in \sectionref{graphs}.

    \item This section finished with an explanation of the algorithm used
        to generate a new \regex{} from unparsed log lines.

\end{itemize}

