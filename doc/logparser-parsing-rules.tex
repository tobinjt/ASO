\section{Parsing rules}

\label{rules}

\subsection{Introduction}

This section discusses the rules used in parsing Postfix log files,
starting with rule characteristics, followed by the problems caused by
overlapping rules, and techniques to detect and deal with such rules.  An
example rule and a log line it would match are provided, plus a description
of how the fields in the rule\footnote{The details of the table containing
the rules have already been described in \sectionref{rule attributes}.} are
used when matching a log line and subsequently performing the requested
action.  The section continues with a discussion of rule efficiency
concerns, referring to the graphs in \sectionref{graphs}, and finishes with
an explanation of the algorithm used to generate new \regexes{} from
unparsed lines.

Please refer to \sectionref{parser design} for a discussion of why the
rules and actions have been separated in the parser's design.

\subsection{Rule characteristics}

\label{rule characteristics}

Rule have certain characteristics which may help in understanding the
parser:

\begin{itemize}

    \item Rules are annotated with the name of a Postfix program, and will
        only be used when parsing log lines produced by that
        program.\footnote{There are also generic rules which are used when
        parsing log lines produced by any Postfix program, but only if
        there are also rules specific to that program, and those rules have
        already have been tried and failed on the current log line.}  Any
        given rule will only be used to parse a subset of the log lines,
        and any given log line will only be parsed by a subset of the
        rules.

    \item The first matching rule wins: no further rules are tried against
        that log line, but there is a facility for ordering the rules so
        that more specific rules can be tried first.

    \item Rules are completely self-contained and can be understood in
        isolation, without reference to any other rules.

    \item Rule processing time is a linear function of the number of rules.

\end{itemize}

\label{comparison against context-free grammars}

In context-free grammar terms the parser rules could be described as:

$\text{\textless{}log-line\textgreater{}} \mapsto \text{rule-1} |
\text{rule-2} | \text{rule-3} | \dots | \text{rule-n}$


\subsection{Overlapping rules}

\label{overlapping rules}

The parser does not try to detect overlapping rules;\footnote{It may be
possible to parse each rule's \regex{} and determine if any overlap.  The
author has not attempted to do this: such a project by itself would
probably qualify for a PhD, and may involve solving the Halting
Problem~\cite{Wikipedia-halting-problem} and circumventing the
Church-Turing Thesis~\cite{Wikipedia-church-turing-thesis}.} that
responsibility is left to the author of the rules.  Unintentionally
overlapping rules lead to inconsistent parsing and data extraction because
the order in which rules are tried against each line may change between log
files, and the first matching rule wins.  Overlapping rules are frequently
a requirement, allowing a more specific rule to match some log lines and a
more general rule to match the majority, e.g.\ separating \SMTP{} delivery
to specific sites from \SMTP{} delivery to the rest of the world.  The
algorithm provides a facility for ordering overlapping rules: the priority
field in each rule (defaults to zero).  Rules are sorted by priority,
highest first, and then rules with the same priority are sorted by the
number of successful matches when parsing the previous log file.  Negative
priorities may be useful for catchall rules.

Detecting overlapping rules is difficult, but the following approaches may
be helpful:

\begin{itemize}

    \item Sort by \regex{} and visually inspect the list, e.g.\ with \SQL{}
        similar to: \textbf{select regex from rules order by regex;}

    \item Compare the results of parsing using sorted, shuffled and
        reversed rules.\footnote{See \sectionref{rule efficiency} for more
        details of sorting the rules.}  Parse a number of log files using
        normal sorting, then dump a textual representation of the rules,
        connections and results tables.  Repeat with shuffled and reversed
        rules, starting with a fresh database.  If there are no overlapping
        rules the tables from each run will be identical; differences
        indicate overlapping rules.  Which rules overlap can be determined
        by examining the differences in the tables: each result contains a
        reference to the rule which created it, if the references differ
        between runs the two rules referenced in the differing records
        overlap.  Unfortunately this method cannot prove the absence of
        overlapping rules; it can detect overlapping rules, but only if
        there are log lines in the input files which match more than one
        rule.

\end{itemize}

\subsection{Example rule}

\label{example rule}

This example rule matches the message logged by Postfix when it rejects
mail from a sender address because the appropriate \DNS{} entries are
missing, i.e.\ mail could not be delivered to the sender's address (for
full details see~\cite{reject-unknown-sender-domain}).

The example rule below would match the following log line:

\begin{verbatim}
NOQUEUE: reject: RCPT from example.com[10.1.1.1]: 550
  <foo@example.com>: Sender address rejected: Domain not found;
  from=<foo@example.com> to=<info@example.net>
  proto=SMTP helo=<smtp.example.com>
\end{verbatim}

% Don't reformat this!
\begin{tabular}[]{ll}

\textbf{Field}      & \textbf{Value}                                    \\
name                & Unknown sender domain                             \\
description         & We do not accept mail from unknown domains        \\
restriction\_name   & reject\_unknown\_sender\_domain                   \\
postfix\_action     & REJECTED                                          \\
program             & \daemon{smtpd}                                    \\
regex               & \verb!^__RESTRICTION_START__ <(__SENDER__)>: !    \\
                    & \verb!Sender address rejected: Domain not found;! \\
                    & \verb!from=<\5> to=<(__RECIPIENT__)> !            \\
                    & \verb!proto=E?SMTP helo=<(__HELO__)>$!            \\
result\_cols        & recipient = 6; sender = 5                         \\
connection\_cols    & helo = 7                                          \\
result\_data        &                                                   \\
connection\_data    &                                                   \\
action              & REJECTION                                         \\
queueid             & 1                                                 \\
hits                & 0                                                 \\
hits\_total         & 0                                                 \\
priority            & 0                                                 \\
cluster\_group      & 400                                               \\

\end{tabular}

\vspace{1em}

Additional data will be captured automatically when the \regex{} contains 
\_\_RESTRICTION\_START\_\_, hence the capture numbers in result\_cols and
connection\_cols start at 5.  The various fields are used as follows;

\begin{description}

    \item [name, description, restriction\_name and postfix\_action:] are
        not \newline used by the algorithm, they serve to document the rule
        for the user's benefit.

    \item [program and regex:] If the program in the rule equals the
        program which logged the log line a match using the \regex{} will
        be attempted against the log line; if the match is successful the
        action will be executed, if not the next rule will be tried.  If
        the program-specific rules don't match the log line, the parser
        will fall back to generic rules; if those rules are unsuccessful a
        warning will be issued.

    \item [action:] will be executed if the \regex{} matches successfully
        (see \sectionref{actions-in-detail} for full details).

    \item [result\_cols, connection\_cols, result\_data and
        connection\_data:] are \newline used by the action to extract and
        save data matched by the \regex{}.

    \item [queueid:] The index of the capture in the \regex{} which
        supplies the queueid, or zero if the log line does not contain a
        queueid.  This allows the correct mail can be found by queueid and
        actions performed on it.

    \item [hits, hits\_total and priority:] hits and priority are used in
        ordering the rules (see \sectionref{rule ordering for efficiency});
        hits is set to the number of successful matches at the end of the
        parsing run; hits\_total is the sum of hits over every parsing run,
        but is otherwise unused by the algorithm.

    \item [cluster\_group] The cluster\_group attribute is used by the
        Decision Tree algorithm described in a separate document; the
        parser does not use it in any way.

\end{description}



\subsection{Rule efficiency}

\label{rule efficiency}

Parsing efficiency is an obvious concern when the parser routinely needs to
deal with 75 MB log files containing 300,000 log lines (generated daily on
a mail server handling mail for approximately 700 users --- large scale
mail servers would have much larger log files on a daily basis).  When
generating the data for the graphs included in \sectionref{graphs},
\numberOFlogFILES{} log files (totaling 10.08 GB, \numberOFlogLINEShuman{}
log lines) were each parsed 10 times, the first run discarded, and the
execution time for the remaining 9 runs averaged.  The first run is
discarded for two reasons:

\begin{enumerate}

    \item The execution time will be higher because the log file must be
        read from disk, whereas for subsequent runs the log file will be
        cached in memory by the operating system.

    \item The execution time will also be higher because the rule ordering
        will be sub-optimal compared to subsequent runs.

\end{enumerate}

Saving results to the database was disabled for the test runs, as that
dominates the run time of the program, and the tests are aimed at measuring
the speed of the parser rather than the speed of the database and the disks
the database is stored on.

\subsubsection{Algorithmic complexity}

An important property of a parser is how execution time scales relative to
input size: does it scale linearly, polynomially, or exponentially?
Graph~\refwithpage{execution time vs file size vs number of lines graph}
shows the execution time in seconds, file size in MB and tens of thousands
of log lines per log file.  All three lines run roughly in parallel, giving
a visual impression that the algorithm scales linearly with input size.
This impression is borne out by graph~\refwithpage{execution time vs file
size vs number lines factor} which plots the ratio of file size vs
execution time and ratio of number of log lines vs execution time (higher
is better).\footnote{Table~\refwithpage{execution time vs file size vs
number lines factor table} shows the ratios for different breakdowns of the
log files.}  As the reader can see the ratios are quite tightly banded,
showing that the algorithm scales linearly: the much larger log files
between points 60 and 70 on the X axis in graph~\refwithpage{execution time
vs file size vs number of lines graph} actually cause the ratio to
increase, rather than decrease.  The strange behaviour where larger log
files are parsed more efficiently is explained fully in \sectionref{Why are
there dips in the graphs?}.  

\subsubsection{Rule ordering for efficiency}

\label{rule ordering for efficiency}

Rule ordering was mentioned in \sectionref{rule attributes} and will be
covered in greater detail in this section.  At the time of writing there
are \numberOFrules{} different rules, with the top 10\% matching the vast
majority of the log lines, and the remaining log lines split across the
other 90\% of the rules (as shown in graph~\refwithpage{rule hits graph}).
Assuming that the distribution of log lines is reasonably steady over time,
program efficiency should benefit from trying more frequently matching
rules before those which match less frequently.  To test this hypothesis
three full test runs were performed with different rule orderings:

\begin{description}

    \item [normal]  The most optimal order, according to the hypothesis:
        rules which match most often will be tried first.

    \item [shuffle] Random ordering --- the rules will be shuffled once
        before use and will retain that ordering for the entirety of the
        log file.  Note that the ordering will change every time the parser
        is executed, so 10 different orderings will be generated for each
        log file in the test run.  This is intended to represent an
        unsorted rule set.

    \item [reverse] Hypothetically the worst order: the most frequently
        matching rules will be tried last.

\end{description}

Graphs~\refwithpage{percentage increase of shuffled over normal}
and~\refwithpage{percentage increase of reversed over normal} show the
percentage increase of execution times, with table~\refwithpage{Execution
time increase for different rule orderings} showing the mean increases for
different groupings of log files.  Overall this provides a modest but
worthwhile performance increase of approximately 9\%, for a small
investment in time and programming.

\subsubsection{Caching each regex}

\label{Caching each regex}

Perl compiles the original \regex{} into an internal representation,
optimising the \regex{} to improve the speed of matching, but this
compilation and optimisation takes CPU time; far more CPU time, in fact,
than the actual matching takes.  Perl automatically caches static
\regexes{}, but dynamic \regexes{} need to be explicitly compiled and
cached.  Graph~\refwithpage{normal regex vs discard regex} shows execution
times with and without caching the \regex{}.  Caching the compiled
\regexes{} is obviously far more efficient; graph~\refwithpage{normal regex
vs discarded regex factor} shows the percentage execution time increase
when not caching each \regex{}.

Caching the compiled \regexes{} is quite simple, and is the single most
effective optimisation implemented in the parser.

\subsection{Creating new rules}

\label{creating new rules}

The log files produced by Postfix differ from installation to installation,
because administrators have the freedom to choose the subset of available
restrictions which suits their needs, including using different \DNSBL{}
services, policy servers, or custom rejection messages.  To facilitate easy
parsing of new log lines, the parser's design separates parsing rules from
parsing actions: adding new actions is difficult, but adding new rules to
parse new rejection messages is trivial (and also occurs much more
frequently).  The implementation provides a program to ease the process of
creating new rules from unparsed log lines, based on the algorithm
developed by Risto Vaarandi~\cite{risto-vaarandi} for his
\SLCT{}~\cite{slct-paper}.  The differences between the two algorithms will
be outlined as part of the general explanation below.

The core of the algorithm is quite simple: log lines are generally created
by substituting variable words into a fixed pattern, and analysis of the
frequency with which each word occurs can be used to determine whether the
word is variable or part of the fixed pattern.  This classification can be
used to group similar log lines and generate a \regex{} to match each group
of log lines.

There are 4 steps in the algorithm:

\begin{description}

    \item [Pre-process the file]  The new algorithm leverages the knowledge
        gained while writing rules and performs a large number of
        substitutions on the log input lines, replacing commonly occurring
        variable terms (e.g.\ email addresses, \IP{} addresses, the
        standard start of rejection messages, etc.) with \regex{} keywords
        which the parser will expand when it loads the rule (see the
        \regex{} entry in \sectionref{rule attributes}).  The purpose of
        this phase is to utilise existing knowledge to create more accurate
        \regexes{}.  The new log lines are written to a temporary file,
        which all subsequent stages use instead of the original input file.

        In the original algorithm the purpose of the preprocessing stage
        was to reduce the memory consumption of the program.  In the first
        pass it generated a hash from a small range of values for each word
        of each log line, incrementing a counter for each hash.  The
        counters will be used in the next pass to filter out words: if the
        word's hash does not have a high frequency, the word itself cannot
        have a high frequency, and there is no need to maintain a counter
        for it, reducing the number of counters and thus the program's
        memory consumption.

    \item [Calculate word frequencies]  The position of words within a log
        line is important: a common word does not indicate similarity
        between log lines unless it occupies the same position within both
        log lines.\footnote{If a variable term within a line contains
        spaces, it will appear to the algorithm as two words rather than
        one.  This will alter the position of subsequent words, so a word
        occurring in different positions in two log lines \textit{may\/}
        indicate similarity, but the algorithm does not attempt to deal
        with this possibility.}  The algorithm maintains a counter for each
        \textit{(word, word's position within the log line)\/} tuple,
        incrementing it each time that word occurs in that position.

        The original algorithm only maintains counters for words whose hash
        result from the previous phase has a high frequency; this reduces
        the number of counters maintained by the algorithm, reducing the
        memory requirements of the algorithm.  The modified algorithm omits
        this check because the majority of unique or infrequently occurring
        words will have been substituted with keywords during the first
        phase, vastly reducing the number of tuples to maintain counters
        for.

    \item [Classify words based on their frequency]  The frequency of each
        \textit{(word, word's position within the log line)\/} tuple is
        checked: if its frequency is greater than the threshold supplied by
        the user (1\% of all log lines is generally a good starting point)
        it is classified as a fixed word, otherwise it is classified as a
        variable term.  If a variable term appears sufficiently often it
        will be classified as a fixed term, but that should be noticed by
        the user when reviewing the new \regexes{}.  Variable terms are
        replaced by \texttt{.+}, which means to match zero or more of any
        character.  

    \item [Build regexes]  The words are reassembled to produce a \regex{}
        matching the log line, and a counter is maintained for each
        \regex{}.  Contiguous sequences of \texttt{.+} in the newly
        reassembled \regexes{} are collapsed to a single \texttt{.+}; any
        resulting duplicate \regexes{} are combined, and their counters
        added.  If the frequency of a \regex{} is lower than the threshold
        supplied by the user the \regex{} is discarded.\footnote{This is a
        second threshold, independent of the threshold used to
        differentiate between fixed and variable words, but once again 1\%
        of input log lines is a good starting point; obviously the
        threshold depends on the number and type of input log lines.}  The
        new \regexes{} are printed for the user to add to the database,
        either as new rules or merged into the \regexes{} of existing
        rules; the counter for each \regex{} is also printed, giving the
        user an indication of how many of the input log lines that \regex{}
        should match.  Discarding \regexes{} will result in some of the
        input log lines not being matched; this utility should be run again
        once the unmatched log lines have been isolated by running the
        parser, including the new \regexes{}, with the same input which
        produced the original set of unparsed log lines.

\end{description}

A second utility is also provided which reads a list of new \regexes{} and
the input given to the first utility.  It tries to match each input log
line against each \regex{}, counting the number of log lines which match
each \regex{}, warning the user if an input log line is matched by more
than one \regex{}, and additionally warning if an input log line is not
matched by any \regex{}.  It displays a summary of how many input log lines
each \regex{} matched, comparing it to the expected number of matches; this
provides the user with an easy method of checking if the \regexes{}
produced by the first utility are correctly matching the input log lines
they are based upon.  A future version of this utility will also group
input log lines by \regex{}, so the user can tweak the \regexes{} if
required.

\subsection{Conclusion}

This section dealt with the rules used in parsing Postfix log files:

\begin{itemize}

    \item The characteristics of the rules were described.

    \item Detecting overlapping rules and dealing with the problems they
        can cause was covered, including a discussion of why overlapping
        rules can be helpful as well as harmful.

    \item An example log line and the rule matching it illustrated a
        description of how the fields in the rule are used both in the
        matching phase and the action that is subsequently executed.

    \item The structure of the database table containing the rules is dealt
        with in \sectionref{rule attributes}, and is not duplicated in this
        section.

    \item The topic of rule efficiency was discussed next, covering the
        effects of caching compiled \regexes{} and optimal ordering of
        rules, with reference to the graphs in \sectionref{graphs}.

    \item This section finished with an explanation of the algorithm used
        to generate a new \regex{} from unparsed log lines.

\end{itemize}

