% vim: set filetype=tex :
% Postfix components
\newglossaryentry{bounce}{name={bounce},type={postfix},description={
    The bounce daemon is responsible for generating bounce notifications in
    Postfix version 2.3 and later.
    \daemonDocURL{http://www.postfix.org/bounce.8.html}{2009/02/23}.
}}

\newglossaryentry{cleanup}{name={cleanup},type={postfix},description={
    The cleanup daemon processes all incoming mail after it has been
    accepted and before it is delivered.  It removes duplicate recipient
    addresses, inserts missing headers, and rewrites addresses if
    configured to do so.
    \daemonDocURL{http://www.postfix.org/cleanup.8.html}{2009/02/23}.
}}

\newglossaryentry{lmtp}{name={lmtp},type={postfix},description={
    Delivers mail using the \acronym{LMTP} protocol.
    \daemonDocURL{http://www.postfix.org/lmtp.8.html}{2009/02/23}.
}}

\newglossaryentry{local}{name={local},type={postfix},description={
    The Postfix component responsible for local delivery of mail (i.e.\
    mail delivered on the server Postfix is running on); this includes
    alias expansion, processing of a user's \texttt{.forward} file, and
    delivery of the mail, whether to a user's mailbox or a program such as
    a mailing list manager.
    \daemonDocURL{http://www.postfix.org/local.8.html}{2009/02/23}.
}}

\newglossaryentry{pickup}{name={pickup},type={postfix},description={
    Pickup is the daemon that deals with mail submitted locally via
    \daemon{postdrop}, passing the mail on to \daemon{cleanup} for further
    processing.
    \daemonDocURL{http://www.postfix.org/pickup.8.html}{2009/02/23}.
}}

\newglossaryentry{postdrop}{name={postdrop},type={postfix},description={
    Postdrop is used when submitting mail locally on the server: it copies
    its input into a newly created mail in the queue, for processing by
    \daemon{pickup} and subsequent delivery.
    \daemonDocURL{http://www.postfix.org/postdrop.1.html}{2009/02/23}.
}}

\newglossaryentry{postsuper}{name={postsuper},type={postfix},description={
    Used by the administrator for maintenance tasks such as deleting mails
    from the queue, putting mail on hold and later releasing it, and
    consistency checking of the mail queue.
    \daemonDocURL{http://www.postfix.org/postsuper.1.html}{2009/02/23}.
}}

\newglossaryentry{qmgr}{name={qmgr},type={postfix},description={
    Qmgr is the Postfix daemon that manages the mail queue, determining
    which mails will be delivered next.  Qmgr groups mail based on the
    recipient for local mails and the destination server for remote
    addresses, ensuring that it achieves maximum concurrency without
    overwhelming destinations or wasting resources on non-responsive
    destinations.
    \daemonDocURL{http://www.postfix.org/qmgr.8.html}{2009/02/23}.
}}

\newglossaryentry{sendmail}{name={sendmail},type={postfix},description={
    A Postfix component that is compatible with the Sendmail mail
    submission program which all Unix commands that need to send mail use;
    it executes \daemon{postdrop} to place a new mail in the queue.
    \daemonDocURL{http://www.postfix.org/sendmail.1.html}{2009/02/23}.
}}

\newglossaryentry{smtp}{name={smtp},type={postfix},description={
    Delivers mail using the \acronym{SMTP} protocol.
    \daemonDocURL{http://www.postfix.org/smtp.8.html}{2009/02/23}.
}}

\newglossaryentry{smtpd}{name={smtpd},type={postfix},description={
    The Postfix component that accepts mail via \acronym{SMTP}, and
    implements most of the anti-spam restrictions Postfix provides.
    \daemonDocURL{http://www.postfix.org/smtpd.8.html}{2009/02/23}.
}}

\newglossaryentry{virtual}{name={virtual},type={postfix},description={
    The Postfix component responsible for delivery of mails to virtual
    domains.  When \daemon{local} delivers mail, the destination is
    determined only by the portion of the email address on the left side of
    the \texttt{@}, whereas when \daemon{virtual} delivers mail, the
    destination is determined by the entire email address.  For example, if
    the server is responsible for both the \texttt{example.org} and
    \texttt{example.net} domains: \daemon{local} would deliver mail for
    \texttt{john@example.org} and \texttt{john@example.net} to the same
    mailbox, whereas \daemon{virtual} would deliver mail for those
    addresses to different mailboxes.  Virtual delivery is used where the
    local part of an address may be present in multiple domains, and each
    must be delivered to different users.
    \daemonDocURL{http://www.postfix.org/virtual.8.html}{2009/02/23}.
}}
