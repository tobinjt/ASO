% vim: set filetype=tex :
% Postfix components
\newglossaryentry{bounce}{name={bounce},type={postfix},description={
    The bounce daemon is responsible for sending bounce notifications in
    Postfix versions later than 2.2.  The definitive documentation is
    \url{http://www.postfix.org/bounce.8.html}.
}}

\newglossaryentry{cleanup}{name={cleanup},type={postfix},description={
    Cleanup processes all incoming mail after it has been accepted and
    before it is delivered.  It removes duplicate recipient addresses,
    inserts missing headers, and optionally rewrites addresses if
    configured to do so.  The definitive documentation is
    \url{http://www.postfix.org/cleanup.8.html}.
}}

\newglossaryentry{lmtp}{name={lmtp},type={postfix},description={
    Delivery of mail over \gls{LMTP} is performed by the lmtp component.
    The definitive documentation is
    \url{http://www.postfix.org/lmtp.8.html}.
}}

\newglossaryentry{local}{name={local},type={postfix},description={
    Local is the Postfix component responsible for local delivery of mail
    (i.e.\ delivered on the server Postfix is running on), whether it be to
    a user's mailbox or a program such as a mailing list manager or
    procmail (\url{http://www.procmail.org/}).  It also handles aliases and
    processing of a user's \texttt{.forward} file.  The definitive
    documentation is \url{http://www.postfix.org/local.8.html}.
}}

\newglossaryentry{pickup}{name={pickup},type={postfix},description={
    Pickup is the service that deals with mail submitted locally via
    postdrop and sendmail; it passes all submitted mail to cleanup.  The
    definitive documentation is \url{http://www.postfix.org/pickup.8.html}.
}}

\newglossaryentry{postdrop}{name={postdrop},type={postfix},description={
    Postdrop is used when submitting mail locally on the server: it creates
    a new mail in the queue and copies its input into the mail.  Subsequent
    delivery of the mail is the responsibility of other Postfix components.
    The definitive documentation is
    \url{http://www.postfix.org/postdrop.1.html}.
}}

\newglossaryentry{postsuper}{name={postsuper},type={postfix},description={
    Maintenance tasks such as deleting mails from the queue, putting mail
    on hold and later releasing it (no further delivery attempts will be
    made until it is released), and consistency checking of the mail queue.
    The definitive documentation is available at
    \url{http://www.postfix.org/postsuper.1.html}.
}}

\newglossaryentry{qmgr}{name={qmgr},type={postfix},description={
    Qmgr is the Postfix daemon that manages the mail queue, determining
    which mails will be delivered next.  Qmgr orders the mails based on the
    recipient for local mails and the destination server for remote
    addresses, ensuring that it balances the aims of achieving maximum
    concurrency while avoiding overwhelming destinations or wasting time
    and resources on non-responsive destinations.  The definitive
    documentation is \url{http://www.postfix.org/qmgr.8.html}.
}}

\newglossaryentry{sendmail}{name={sendmail},type={postfix},description={
    Postfix provides a command that is compatible with the Sendmail
    (\url{http://www.sendmail.org/}) mail submission program that all Unix
    commands that send mail depend on; it executes \daemon{postdrop} to
    place a new mail in the queue.  The definitive documentation is
    \url{http://www.postfix.org/sendmail.1.html}.
}}

\newglossaryentry{smtp}{name={smtp},type={postfix},description={
    Delivery of mail over \gls{SMTP} is performed by the smtp component.
    The definitive documentation is
    \url{http://www.postfix.org/smtp.8.html}.
}}

\newglossaryentry{smtpd}{name={smtpd},type={postfix},description={
    Smtpd is the Postfix program that accepts mail via \gls{SMTP}, and
    implements all the anti-spam restrictions Postfix provides.  The
    definitive documentation is \url{http://www.postfix.org/smtpd.8.html}.
}}

\newglossaryentry{virtual}{name={virtual},type={postfix},description={
    Virtual is the Postfix component responsible for delivery of mails to
    virtual domains.  With \daemon{local} delivery the destination is
    determined only by the portion of the email address on the left side of
    the \at{}, whereas with \daemon{virtual} delivery the destination is
    determined by the entire email address, e.g.\ if the server considers
    itself responsible for both \textbf{example.org} and
    \textbf{example.net} domains: \daemon{local} considers
    \textbf{john\at{}example.org} and \textbf{john\at{}example.net} to have
    the same mailbox, whereas \daemon{virtual} considers them to have
    different mailboxes.  Virtual delivery is used when a server hosts
    multiple domains where a username may be present in more than one
    domain but represent different users in each.  The definitive
    documentation is
    \url{http://www.postfix.org/virtual.8.html}.
}}
