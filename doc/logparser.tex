% $Id$
\documentclass[a4paper,12pt,draft]{report}

% Useful stuff for math mode.
\usepackage{amstext}
% Include images
\usepackage[final]{graphicx}
% Provides \FloatBarrier that stops floats going past it.
\usepackage{placeins}
% Relax the parameters Latex uses when placing floats.
\usepackage{flexiblefloats}
% Prevent floats from appearing above the line that includes them
\usepackage{flafter}
% Add the bibliography into the table of contents.
\usepackage[chapter,numbib]{tocbibind}
% Create a label for the last page.  Might be useful for "page 23/79" or
% something.
\usepackage{lastpage}

% College expect one and a half or double spacing for the initial
% submission.
\usepackage{setspace}
\onehalfspacing{}

% Better looking horizontal rules for tables.
\usepackage{booktabs}

% Extra footnote functionality, including references to earlier footnotes.
% Removed for now, because it generates a warning:
% "LaTeX Warning: Command \@makecol  has changed."
%\usepackage[bottom]{footmisc}
% The Euro symbol
\usepackage{eurofont}

% Aligned list environments.
\usepackage{eqlist}
% This voodoo below gives us an eqlist environment with bold items.
\newcommand{\bolditem}[1]{%
    \bfseries#1%
}
\newenvironment{boldeqlist}
    {\begin{eqlist}[\renewcommand{\makelabel}{\bolditem}]}
    {\end{eqlist}}

% Enable customised lists.
\usepackage{enumitem}
% Tell Latex to use scalable fonts
\usepackage{type1cm}
% Enable nice kerning features.
\usepackage[final]{microtype}

% Extra packages recommended by Springer.
%\usepackage{mathptmx}
%\usepackage{helvet}
%\usepackage{courier}
%\usepackage{makeidx}
%\usepackage{multicol}

% Embed SVN Id
\usepackage{svn}

\newcommand{\parserblurb}[3]{%
    \begin{quotation}
        \noindent{}\textit{#1\/}

        \noindent{}\url{#2}\newline
        Last checked #3.
    \end{quotation}
}

% Produce nicer references like "section 4.3 on the next page"
% Must be loaded before hyperref.
\usepackage{varioref}
% Shorten the text used for references with page numbers.
\renewcommand{\reftextfaraway}[1]{%
    [p.~\pageref{#1}]%
}
% Replacement for \ref{}, adds the page number too.
\newcommand{\refwithpage}[1]{%
    \empty{}\vref{#1}%
    %\empty{}\ref{#1} [p.~\pageref{#1}]%
}
% section references, automatically add \textsection
\newcommand{\sectionref}[1]{%
    \textsection{}\vref*{#1}%
    %\textsection{}\refwithpage{#1}%
}
% A reference with a space between the label and the reference.
\newcommand{\refwithlabel}[2]{%
    #1~\vref{#2}%
}
% table references, for consistent formatting.
\newcommand{\tableref}[1]{%
    \refwithlabel{table}{#1}%
}
\newcommand{\Tableref}[1]{%
    \refwithlabel{Table}{#1}%
}
% graph references
% SHOULD THESE BE LABELLED WITH graph OR figure?
\newcommand{\graphref}[1]{%
    \refwithlabel{graph}{#1}%
}
\newcommand{\Graphref}[1]{%
    \refwithlabel{Graph}{#1}%
}
% figure references
\newcommand{\figureref}[1]{%
    \refwithlabel{figure}{#1}%
}
\newcommand{\Figureref}[1]{%
    \refwithlabel{Figure}{#1}%
}


% Acronyms and glossary entries.  Useful options:
% * nonumberlist disables the list of page numbers after each entry.
\usepackage[acronym,toc,numberedsection=nolabel,style=list,nonumberlist]{glossaries}
\renewcommand{\glspostdescription}[0]{}
\renewcommand{\glsautoprefix}[1]{glossary:#1}
\newglossary[plg]{postfix}{pin}{pout}{Postfix Daemons}
\makeglossaries{}
\newcommand{\acronyms}[1]{%
    \acronym[s]{#1}%
}
\newcommand{\acronym}[2][]{%
    \gls{#2}#1%
    % If there's a glossary entry for this acronym add it, otherwise
    % do nothing.
    \ifglsentryexists{#2 glossary}%
        {\glsadd{#2 glossary}}%
        {}%
}
% Define a new glossary style that uses eqlist, so that acronyms line up
% nicely.
\newglossarystyle{eqlist}{%
    \glossarystyle{list}%
    \renewenvironment{theglossary}{\begin{eqlist}}{\end{eqlist}}%
    \renewcommand*{\glossaryentryfield}[5]{%
        \item [\textbf{##2}] ##3%
    }%
    \renewcommand{\glsgroupskip}[0]{}%
}
% Define a new glossary style that removes the space between glossary
% groups.
\newglossarystyle{nospacelist}{%
    \glossarystyle{list}%
%    \renewenvironment{theglossary}{\begin{eqlist}}{\end{eqlist}}%
%    \renewcommand*{\glossaryentryfield}[5]{%
%        \item [\textbf{##2}] ##3%
%    }%
    \renewcommand{\glsgroupskip}[0]{}%
}

% Provides commands to distinguish between pdf and dvi output.
\usepackage{ifpdf}
% When creating a \PDF{} make the table of contents into links to the pages
% (without horrible red borders) and include bookmarks.  The title and
% author do not work - I think either gnuplot or graphviz clobbers it.
% hyperfootnotes need to be disabled to avoid breaking footmisc, but they
% still seem to work, somehow.
\ifpdf{}
    \usepackage[pdftex,hyperfootnotes=false,plainpages=false,pdfpagelabels]{hyperref}
\else{}
    \usepackage[dvips,hyperfootnotes=false,plainpages=false,pdfpagelabels]{hyperref}
    % This is necessary for wrapping URLs in the bibliography when
    % producing a dvi, but causes problems when generating \PDF{} output.
    \usepackage{breakurl}
\fi{}
\hypersetup{
    pdftitle    = {Parsing Postfix Log Files},
    pdfauthor   = {John Tobin},
    final       = true,
    pdfborder   = {0, 0, 0},
}

% Sort numbers where there are multiple citations.  Does not appear to have
% the expected effect (probably clashes with hyperref), though it does
% reduce the space between numbers.
\usepackage{cite}
% Check for unused references, and how the citation key (e.g.\ slct-paper)
% in the margin beside the reference.
%\usepackage[nomsgs]{refcheck}
% Show where references are used; neither work.
%\usepackage{citeref}
%\usepackage{backref}
%\renewcommand{\refname}{Bibliography}

% Reduce the space between items in a list; this is useful when each item
% is a single line, because then the default spacing makes the list look
% very sparse.  This command needs to be the first thing in a list to take
% effect.
\newcommand{\squeezeitems}[0]{%
    \setlength{\itemsep}{0pt}%
    \setlength{\topsep}{0pt}%
    \setlength{\partopsep}{0pt}%
}

% Put each URL in the bibliography on a new line.
\newcommand{\urlprefix}[0]{\newline{}}
% New formatting commands.

% \showgraph{filename}{caption}{label}
\newcommand{\showgraph}[4][thbp]{%
    \begin{figure}[#1]
        \caption{#3}\label{#4}
        \includegraphics{#2}
    \end{figure}
}

%\showtable{filename}{caption}{label}
\newcommand{\showtable}[4][thbp]{%
    \begin{table}[#1]
        \caption{#3}\label{#4}
        \input{#2}
    \end{table}
}

\newcommand{\tabletopline}[0]{%
    \toprule{}%
}

\newcommand{\tablebottomline}[0]{%
    \bottomrule{}%
}

\newcommand{\tablemiddleline}[0]{%
    \midrule{}%
}

% A command to format a Postfix daemon's name
\newcommand{\daemon}[1]{%
    \texttt{postfix/#1}%
}

\newcommand{\flowchart}[2]{%
    (action: \action{#1}\@; flowchart:~#2)%
}

\newcommand{\action}[1]{%
    \texttt{#1}%
}

% Add last checked dates to all URLs
\newcommand{\urlLastChecked}[3][ ]{%
    \url{#2}#1(last checked #3)%
}
\newcommand{\daemonDocURL}[3][]{%
    \hfill{} \newline{} \url{#2}%
    \hfill{} \newline{} #1 Last checked #3%
}

\newcommand{\tab}[0]{%
    \hspace*{1em}%
}

% This is silly, but it keeps chktex happy.
\newcommand{\singlequote}[0]{'}%

% Constant values.
\newcommand{\numberOFlogFILES}[0]{%
    93%
}

\newcommand{\numberOFlogFILESall}[0]{%
    774%
}

\newcommand{\numberOFlogFILESallYEARS}[0]{%
    2 years, 1\textonehalf{} months%
}

\newcommand{\numberOFrules}[0]{%
    184%
}

\newcommand{\numberOFrulesMINUSten}[0]{%
    174%
}

\newcommand{\numberOFrulesMINIMUM}[0]{%
    115%
}

% \numberOFrulesMINIMUM as percentage of \numberOFrules
\newcommand{\numberOFrulesMINIMUMpercentage}[0]{%
    62.500\%%
}

% \numberOFrules as percentage increase of \numberOFrulesMINIMUM
\newcommand{\numberOFrulesMAXIMUMpercentage}[0]{%
    60.000\%%
}

\newcommand{\numberOFlogLINES}[0]{%
    60,721,709%
}

\newcommand{\numberOFlogLINEShuman}[0]{%
    60.722 million%
}

\newcommand{\numberOFactions}[0]{%
    23%
}

\newcommand{\numberOFruleINTERSECTIONS}[0]{%
    16,836%
}

\newcommand{\numberOFconnectionsINlogFILES}[0]{%
    13,850,793%
}

% The name of the program, so I only have to change it in one place.
\newcommand{\parsername}[1][]{\acronym[#1]{PLP}}
\newcommand{\parsernames}[0]{\acronym['s]{PLP}}
\newcommand{\parsernamelong}[0]{Postfix Log Parser}
\newcommand{\parsernameshort}[1][]{PLP#1}

\newcommand{\specialpage}[1]{%
    \phantomsection{}
    \addcontentsline{toc}{chapter}{#1}%
}
\newcommand{\specialpageheading}[1]{%
    \begin{center}
        \textbf{\Large #1}
    \end{center}
}
\newcommand{\specialpageandheading}[1]{%
    \specialpage{#1}
    \specialpageheading{#1}
}

\newcommand{\titleandauthor}[0]{%
\begin{center}

    {\Huge Parsing Postfix Log Files}

    \vfill{}

    {\LARGE John Tobin}

    \vfill{}

\end{center}
}

\begin{document}

\pagestyle{empty}

\specialpage{Title}

\titleandauthor{}

\begin{center}

    A thesis submitted to the University of Dublin, in fulfilment of the
    requirements for the degree of Master of Science in Computer Science.

    \vfill{}

    April 2009


    \vfill{}

    \vfill{}

    \vfill{}

\end{center}

\newpage{}


% Declaration
\specialpageandheading{Declaration}

\vfill{}

\noindent{}I declare that this thesis, and the work described herein, is
entirely my own work, and has not been submitted as an exercise for a
degree at this or any other university.  This thesis may be borrowed or
copied upon request with the permission of the Librarian, Trinity College,
University of Dublin.

\vfill{}

\begin{flushright}
    \underline{\hspace*{15em}} \\~\\
    John Tobin \\
    \today{}
\end{flushright}

\vfill{}

\vfill{}

\vfill{}

\vfill{}

\newpage{}

% Acknowledgements
\specialpageandheading{Acknowledgements}

\bigskip{}

\noindent{}I am indebted to my supervisor, Dr.\ Carl Vogel, for his advice,
assistance, and guidance.

\bigskip{}

\noindent{}I am grateful to my wife, Ariane Tobin, for her patience, good
will, and encouragement; without her, I would not have accomplished this.

\newpage{}

% Abstract
\specialpage{Abstract}

~\empty{}

\vfill{}
\vfill{}
\vfill{}

%\titleandauthor{}

\begin{center}
    \textbf{Abstract}
\end{center}

Parsing Postfix log files is much more difficult than it first appears, but
it \textit{is\/} possible to achieve a high degree of accuracy in
understanding Postfix log files, and subsequently in reconstructing the
actions taken by Postfix when processing mail delivery attempts.  This
thesis describes the creation of a parser for Postfix log files,
documenting the architecture developed for this project and the parser that
implements it, the difficulties encountered and the solutions developed.
The parser stores data gleaned from the log files in an SQL database;
future projects or programs could use the gathered data to optimise current
anti-spam measures, to produce statistics showing how effective those
measures are, or to provide a baseline to test new anti-spam measures
against.  The Postfix log file parser needs to be very precise and strict
when parsing, yet must allow users to easily adapt or extend it to parse
new log lines, without requiring that the user have an in-depth knowledge
and understanding of the parser's internal workings.  The newly developed
architecture is designed to make the process of parsing new inputs as
simple as possible, enabling users to trivially add new rules (to parse
variants of known inputs) and relatively easily add new actions (to process
a previously unknown category of inputs).  The parser implemented for this
project is evaluated on the criteria of efficiency and coverage of Postfix
log files, demonstrating that the conflicting goals of efficiency and
accuracy can be balanced, and that one need not be sacrificed to achieve
the other.

\SVN$Id$

\vfill{}
\vfill{}
\vfill{}

~\empty{}

\newpage{}

% Include the header across the top of each page, after the declaration.
\pagestyle{headings}

% Pull in the acronyms early, so they can be used throughout the text.
% % vim: set textwidth=75 spell :
% Warning: don't use acronyms within definitions, they don't work properly.

\newacronym{RBL}{Real-time Black List}{description={
    Real-time Black Lists are a simple collaborative anti-spam technique
    used to reject or penalise email sent from mail servers reported to
    have sent large volumes of spam.  To use an RBL the mail server makes a
    DNS request incorporating the IP address of the currently connected
    client; if the requested hostname exists the client is on the RBL, and
    the mail server can decide what course of action to take.
}}

\newacronym{API}{Application Programming Interface}{description={
    One of the fundamental concepts when writing programs is the reuse of
    existing code, so that each program does not reinvent the wheel.  When
    a body of code is intended to be reused repeatedly, the user of this
    code needs to be informed of the functionality provided by the code.
    An API defines the interface provided to the user, and acts a contract
    between the user and the provider: if the user adheres to the API the
    provider guarantees it will work, while the provider is free to change
    the implementation as long as the API is preserved.
}}

\newacronym{SMTP}{Simple Mail Transfer Protocol}{description={
    SMTP is the protocol which transfers mail from the sender to the
    recipient across the Internet.  A brief introduction to SMTP is
    provided in section~\refwithpage{SMTP background}.
}}

\newacronym{MTA}{Mail Transfer Agent}{description={
    A Mail Transfer Agent sends and/or receives mail via SMTP\@.
}}

\newacronym{RFC}{Request For Comments}{description={
    The Request For Comments series is a series of proposals defining
    various protocols and file formats, e.g. SMTP\@.  The name is somewhat
    misleading: initially the authors were asking for peer review, but
    these documents are now the de facto standards the Internet runs on.
}}

% vim: set filetype=tex :
% The contents of the glossary.
\newglossaryentry{SQLite}{name={SQLite},description={
    \textit{SQLite is a small C library that implements a self-contained,
    embeddable, zero-configuration SQL database engine.\/} SQLite is an
    \acronym{SQL} implementation focusing on correctness, simplicity, and
    speed.  Unlike other \acronym{SQL} implementations it does not require
    a separate server process, greatly simplifying deployment of an
    application utilising it.  More details can be found
    at~\url{http://www.sqlite.org/}.
}}

\newglossaryentry{Phishing}{name={Phishing},description={
    Phishing~\cite{Wikipedia-phishing} is an attempt to acquire information
    by masquerading as an entity trusted by the user, e.g.\ a bank.
}}

\newglossaryentry{Backscatter}{name={Backscatter},description={
    When a spam sender or worm sends mail with forged sender addresses,
    innocent sites are flooded with undeliverable mail notifications; this
    is called backscatter mail.
}}

\newglossaryentry{Joe job}{name={Joe job},description={
    A joe~job describes spam mail sent using a faked sender address with
    the intention of sullying the good name of the user of that address.
    Joe~jobs are one cause of backscatter.
}}

\newglossaryentry{Epoch}{name={Epoch},description={
    Most operating systems store the current time and timestamps of files
    etc.\ as seconds elapsed since the epoch, the beginning of time as far
    as the operating system is concerned.  On Unix and Unix-derived systems
    the epoch is 1970/01/01 00:00:00; on other operating systems it may be
    different.
}}


\newglossaryentry{<>}{name={$<>$},sort={<>},description={
    $<>$ is the sender address used for mail that should not be replied to,
    e.g.\ bounce notifications.  In \acronym{SMTP} all addresses are
    enclosed in $<>$, so \textit{username@domain\/} becomes
    \textit{$<$username@domain$>$\/}; thus $<>$ is actually an empty
    address, but is always written as $<>$ for clarity.  All mail servers
    must accept mail sent from $<>$, or they are in violation of
    \acronym{RFC}~2821~\cite{RFC2821}.
}}

\newglossaryentry{queueid}{name={queueid},description={
    Each mail in Postfix's queue is assigned a queueid to uniquely identify
    it.  Queueids are assigned from a limited pool, so although they are
    guaranteed to be unique for the lifetime of the mail, they may be
    reused later.
}}

%\newglossaryentry{hash}{name={hash},description={
%    A hashing function transforms a string of characters to a number.
%    There are many possible uses for the resulting number: a common usage
%    is to maintain a data structure indexed by strings in an efficient
%    manner.  A full description is beyond the scope of this paper, further
%    information can be found at~\cite{hash-functions}.
%}}

\newglossaryentry{awk}{name={awk},description={
    AWK is a general purpose programming language that is designed for
    processing text-based data, and is available as a standard utility on
    all Unix systems.
}}

\newglossaryentry{syslog}{name={syslog},description={
    Syslog is the standard logging mechanism on Unix systems: the program
    sends log messages to syslog, then syslog filters and stores the
    messages according to the configuration the administrator has chosen.
}}

\newglossaryentry{mail bomb}{name={mail bomb},description={
    A mail bomb occurs when an attacker inflicts a huge volume of mail on
    the victim.  At best a mail bomb is irritating to the victim; at worst
    the deluge of mail can be severe enough to: interrupt service for the
    victim and/or other users; cause mail to be rejected because the victim
    has reached a limit (e.g.\ too many mails, too much disk space
    consumed); the victim may accidentally delete other mail while trying
    to cope with the mail bomb.
}}

\newglossaryentry{mail loop}{name={mail loop},description={
    Sometimes mail sent to one address must be forwarded to a different
    address instead, e.g.\ because a person has changed jobs.  A mail loop
    occurs when the chain of recipient addresses constitute a cyclic
    directed graph; the simplest example is when mail for
    \texttt{alice@example.net} is delivered to \texttt{bob@example.com},
    and mail for \texttt{bob@example.com} is delivered to
    \texttt{alice@example.net}.
}}

% vim: set filetype=tex :
% Postfix components
\newglossaryentry{bounce}{name={bounce},type={postfix},description={
    The bounce daemon is responsible for sending bounce notifications in
    Postfix versions later than 2.2.  The definitive documentation is
    \url{http://www.postfix.org/bounce.8.html}.
}}

\newglossaryentry{cleanup}{name={cleanup},type={postfix},description={
    Cleanup processes all incoming mail after it has been accepted and
    before it is delivered.  It removes duplicate recipient addresses,
    inserts missing headers, and optionally rewrites addresses if
    configured to do so.  The definitive documentation is
    \url{http://www.postfix.org/cleanup.8.html}.
}}

\newglossaryentry{lmtp}{name={lmtp},type={postfix},description={
    Delivery of mail over \gls{LMTP} is performed by the lmtp component.
    The definitive documentation is
    \url{http://www.postfix.org/lmtp.8.html}.
}}

\newglossaryentry{local}{name={local},type={postfix},description={
    Local is the Postfix component responsible for local delivery of mail
    (i.e.\ delivered on the server Postfix is running on), whether it be to
    a user's mailbox or a program such as a mailing list manager or
    procmail (\url{http://www.procmail.org/}).  It also handles aliases and
    processing of a user's \texttt{.forward} file.  The definitive
    documentation is \url{http://www.postfix.org/local.8.html}.
}}

\newglossaryentry{pickup}{name={pickup},type={postfix},description={
    Pickup is the service that deals with mail submitted locally via
    postdrop and sendmail; it passes all submitted mail to cleanup.  The
    definitive documentation is \url{http://www.postfix.org/pickup.8.html}.
}}

\newglossaryentry{postdrop}{name={postdrop},type={postfix},description={
    Postdrop is used when submitting mail locally on the server: it creates
    a new mail in the queue and copies its input into the mail.  Subsequent
    delivery of the mail is the responsibility of other Postfix components.
    The definitive documentation is
    \url{http://www.postfix.org/postdrop.1.html}.
}}

\newglossaryentry{postsuper}{name={postsuper},type={postfix},description={
    Maintenance tasks such as deleting mails from the queue, putting mail
    on hold and later releasing it (no further delivery attempts will be
    made until it is released), and consistency checking of the mail queue.
    The definitive documentation is available at
    \url{http://www.postfix.org/postsuper.1.html}.
}}

\newglossaryentry{qmgr}{name={qmgr},type={postfix},description={
    Qmgr is the Postfix daemon that manages the mail queue, determining
    which mails will be delivered next.  Qmgr orders the mails based on the
    recipient for local mails and the destination server for remote
    addresses, ensuring that it balances the aims of achieving maximum
    concurrency while avoiding overwhelming destinations or wasting time
    and resources on non-responsive destinations.  The definitive
    documentation is \url{http://www.postfix.org/qmgr.8.html}.
}}

\newglossaryentry{sendmail}{name={sendmail},type={postfix},description={
    Postfix provides a command that is compatible with the Sendmail
    (\url{http://www.sendmail.org/}) mail submission program that all Unix
    commands that send mail depend on; it executes \daemon{postdrop} to
    place a new mail in the queue.  The definitive documentation is
    \url{http://www.postfix.org/sendmail.1.html}.
}}

\newglossaryentry{smtp}{name={smtp},type={postfix},description={
    Delivery of mail over \gls{SMTP} is performed by the smtp component.
    The definitive documentation is
    \url{http://www.postfix.org/smtp.8.html}.
}}

\newglossaryentry{smtpd}{name={smtpd},type={postfix},description={
    Smtpd is the Postfix program that accepts mail via \gls{SMTP}, and
    implements all the anti-spam restrictions Postfix provides.  The
    definitive documentation is \url{http://www.postfix.org/smtpd.8.html}.
}}

\newglossaryentry{virtual}{name={virtual},type={postfix},description={
    Virtual is the Postfix component responsible for delivery of mails to
    virtual domains.  With \daemon{local} delivery the destination is
    determined only by the portion of the email address on the left side of
    the \at{}, whereas with \daemon{virtual} delivery the destination is
    determined by the entire email address, e.g.\ if the server considers
    itself responsible for both \textbf{example.org} and
    \textbf{example.net} domains: \daemon{local} considers
    \textbf{john\at{}example.org} and \textbf{john\at{}example.net} to have
    the same mailbox, whereas \daemon{virtual} considers them to have
    different mailboxes.  Virtual delivery is used when a server hosts
    multiple domains where a username may be present in more than one
    domain but represent different users in each.  The definitive
    documentation is
    \url{http://www.postfix.org/virtual.8.html}.
}}

\glsaddall[types={postfix}]

\setcounter{page}{5}
\tableofcontents
\listoffigures
\listoftables

\newpage

% WSUIPA fonts.
\input{ipamacs}

\chapter{Introduction}

\label{introduction}

Most mail server administrators will have performed some basic processing
of the log files produced by their mail server at one time or another,
whether it was to debug a problem, explain to a user why their mail is
being rejected, or check if new anti-spam measures are working.  The more
adventurous will have generated statistics to show how successful each of
their anti-spam measures has been in the last week, and possibly even
generated some graphs to clearly illustrate these statistics to management
or users.\footnote{This was the author's first real foray into processing
Postfix log files.}  Very few will have performed in-depth parsing and
analysis of their log files, where the parsing must correlate the log lines
per-connection or per-queueid rather than processing log lines
independently.  One of the barriers to this kind of processing is the
unstructured nature of Postfix log files, where each log line was added on
an ad hoc basis as a requirement was discovered or new functionality was
added.\footnote{A history of all changes made to Postfix is distributed
with the source code, available from
\urlLastChecked{http://www.postfix.org/}{2009/02/23}} Further complication
arises because the set of rejection messages is not fixed: new messages can
be added by the administrator with custom checks; every
\acronym{DNSBL}\footnote{This document is supplied with a glossary, see
\textsection\ref{Glossary}.} returns a different explanatory message;
policy servers may log different messages depending on the characteristics
of the connection.  There are many ways in which the log lines may differ
between servers, even within the same organisation: servers may be
configured differently, or running different versions of Postfix.  This
paper documents the difficult process of parsing Postfix log files,
presenting \parsername{}, a program that parses Postfix log files and
places the resulting data into a database for later use.  The gathered data
can then be used to optimise current anti-spam measures, provide a baseline
to test new anti-spam measures against, or to produce statistics showing
how effective those measures are.  Numerous other uses are possible for
such data: improving server performance by identifying troublesome
destinations and reconfiguring appropriately; identifying regular high
volume uses (e.g.\ customer newsletters) and restricting those uses to
off-peak times; detecting virus outbreaks that propagate via mail; as a
base for billing customers on a shared server.  Preserving the raw data
enables users to develop a multitude of uses far beyond those conceived of
by the author.

\vspace{1em}\noindent\textbf{Layout:}

XXX THIS ALL NEEDS TO BE CHECKED AND UPDATED\@.

Section~\ref{background} provides background information useful in
understanding the thesis, parser, and architecture.

Section~\ref{state of the art review} reviews both the previously published
research in this area and other available Postfix log file parsers,
discussing why they were deemed unsuitable for the task, including why they
could not be improved or expanded upon.

This algorithm requires a database for storing both the rules used when
parsing and the results gleaned from parsing.  The database schema used is
described in \sectionref{database schema}, explaining in detail the tables
used for storing the data gleaned from the log files and the table that
stores the rules.

Section~\ref{rules in architecture} discusses the parsing rules in detail,
explaining the purpose and usage of each field in a rule, referring to an
example rule and sample data it matches successfully against.  The pros and
cons of overlapping rules are considered, including techniques for
detecting unintentional overlaps.  Rule efficiency concerns are discussed,
in particular the optimisations used by the algorithm.  The section
concludes with a description of using the tools provided with the parser to
generate new rules (specifically the regex in each rule) from unparsed log
lines.

Section~\ref{Postfix Parser Implementation} contains the core of the paper,
describing a naive parsing algorithm and the complications initially
encountered that shaped the full algorithm.  A flow chart and a discussion
of the emergent behaviour exhibited by the algorithm accompanies a
comprehensive explanation of the different stages of the initial algorithm.
The framework that actions and rules fit into is documented, then the
actions taken during execution of the algorithm are described, followed by
the process of adding a new action.  The section concludes with an in-depth
description of the further complications discovered, and their solutions
that complete the parser.

Section~\ref{parsing coverage} analyses the coverage the parser achieves
over a set of \numberOFlogFILES{} log files taken from a mail server
handling mail for over 700 users, averaging 8500 mails per day
(\graphref{Mails received per day}).  Coverage is described both
in terms of the fraction of log lines parsed and the fraction of mails and
connections successfully reconstructed by the parsing algorithm; dealing
with false negatives and a discussion of the difficulties in identifying
false positives is also included.  As part of determining the coverage of
the parser a random sampling of log lines was parsed, and the correctness
of the results manually verified.

Section~\ref{limitations and improvements in implementation} lists the
limitations of the algorithm, then suggests some ways of dealing with them,
with the goal of improving parsing and reproduction of the journey a mail
takes through Postfix.

Section~\ref{conclusion} contains the conclusion of the thesis, describing
the results of the research, design, and implementation of the parser.

The bibliography contains references to the resources used in developing
the algorithm, writing the program, and preparing this thesis.  Also
listed are some additional resources expected to be helpful in
understanding \acronym{SMTP}, Postfix, anti-spam techniques, or the thesis.

Appendix~\ref{Glossary} provides a glossary of terms used in the thesis.

Appendix~\ref{Acronyms} provides a list of acronyms used in the thesis.

Appendix~\ref{Postfix Daemons} gives a brief description of Postfix
daemons.


% Reset the list of acronyms that have been used, so they will be expanded
% again the first time they are used.  The idea is to have them expanded
% once per chapter; I do not reset them before the appendices, but some may
% not have been used since they were reset, and thus will be expanded
% again.
\glsresetall{}
\section{Background}

\label{background}

This section provides background information helpful in understanding the
remainder of the thesis.  It begins with a discussion of the motivation
underlying the project, followed by some technical information: the use of
a database as an \gls{API}\@; a brief introduction to \gls{SMTP}\@.  It
finishes with a longer introduction to Postfix, concentrating on anti-spam
restrictions and policy servers.

\subsection{Motivation}

\label{motivation}

This thesis and the program it describes are part of a larger project to
optimise a mail server's Postfix-based anti-spam restrictions, generate
statistics and graphs, and provide a platform on which new restrictions on
trial can be evaluated to determine if they are beneficial in the fight
against spam.  The program parses Postfix log files and populates a
database with the data gleaned from those log files, providing a consistent
and simple view of the log files that future tools can utilise.  The
gathered data can be used to optimise current anti-spam measures, to
provide a baseline to test new anti-spam measures against, or to produce
statistics showing how effective those measures are.

Determining which Postfix restrictions reject the highest number of mails
is a short example of the optimisation possible using data from the
database:

\begin{verbatim}
SELECT name, description, restriction_name, hits_total
    FROM rules
    WHERE postfix_action = 'REJECTED'
    ORDER BY hits_total DESC;
\end{verbatim}

If the database supports sub-selects (where the results of one query are
used in another), percentages can be obtained for the top ten restrictions
(sample output shown in \tableref{Sample output from SQL query showing
percentages}):

\input{build/include-sample-sql-query}

\begin{table}[ht]
    \caption{Sample output from SQL query showing percentages}
    \empty{}\label{Sample output from SQL query showing
    percentages}
    \begin{tabular}[]{lrr}
        \tabletopline{}%
        Restriction & Number of hits & Percentage of hits \\
        \tablemiddleline{}%
        \input{build/include-sample-sql-output}
        \tablebottomline{}%
    \end{tabular}
\end{table}


Another example is determining which restrictions are not effective: this
sample query shows which restrictions had fewer than 100 rejections in the
last log file parsed, and the percentage of total rejections each of those
restrictions represents.

\begin{verbatim}
SELECT name, description, restriction_name, hits,
        (hits * 100.0 /
            (SELECT SUM(hits)
                FROM rules
                WHERE postfix_action = 'REJECTED'
            )
        ) || '%' AS percentage
    FROM rules
    WHERE postfix_action = 'REJECTED'
        AND hits < 100
    ORDER BY hits ASC;
\end{verbatim}

The database queries above yield summary statistics about the efficiency of
anti-spam techniques.  These statistics are far less feasible to assess
directly from log files without prior pre-processing into a database in the
fashion proposed, implemented, and tested herein.

\subsection{Database as Application Programming Interface}

\label{database as API}

The database populated by this program provides a simple interface to
Postfix log files.  Although the interface is a database schema rather than
a set of functions provided by shared code, it is in effect quite similar
to any other \gls{API}: it provides a stable interface for users of the
interface, allowing the implementation to be changed without adverse
effects.  As long as the \gls{API} is maintained, the parser can be
improved to handle additional complications; support can be added for
earlier or later releases of Postfix; bugs can be fixed or limitations
removed.

Using a database simplifies writing programs that need to interact with the
data in several ways:

\begin{enumerate}

    \item Facilities are provided to access databases from most programming
        languages, allowing a developer to access the data gathered using
        their preferred programming language, rather than being restricted
        to the language the parser is written in.  It is often possible to
        write an interface layer allowing code written in one language to
        be used in another language, but this greatly increases the effort
        required to use the parser.

    \item Databases provide complex querying and sorting functionality to
        the user without requiring large amounts of programming.  All
        databases provide a program, of varying complexity and
        sophistication, that can be used for ad hoc queries with minimal
        investment of time.

    \item Databases are easily extensible, e.g.:

        \begin{itemize}

            \item New columns can be added to the tables used by the
                program, with sufficient DEFAULT clauses or a clever
                TRIGGER or two.\footnote{Please refer to an \gls{SQL} guide
                for explanations of these terms, e.g.\
                \url{http://philip.greenspun.com/sql/}.}

            \item A VIEW gives a custom arrangement of data with very
                little effort.

            \item If the database supports it, access can be granted on a
                fine-grained basis, e.g.\ allowing the finance department
                to produce invoices, the helpdesk to run limited queries as
                part of dealing with support calls, and the administrators
                to have full access to the data.

            \item Triggers can be written to perform actions when certain
                events occur.  In pseudo-\gls{SQL}\@:

\begin{verbatim}
CREATE TRIGGER ON INSERT INTO results
    WHERE sender = 'boss@example.com'
        AND postfix_action = 'REJECTED'
    SEND PANIC EMAIL TO 'postmaster@example.com';
\end{verbatim}

            \item Other tables can be added to the database, e.g.\ to cache
                historical, summary or computed data.

        \end{itemize}


    \item \gls{SQL} is reasonably standard and many people will already be
        familiar with it; for those unfamiliar with it there are lots of
        readily available resources from which to learn (a good
        introduction to \gls{SQL} can be found at
        \url{http://philip.greenspun.com/sql/}, others are
        \url{http://www.w3schools.com/sql/default.asp},
        \url{http://sqlcourse.com/}).  Although every vendor implements a
        different dialect of \gls{SQL}, the basics are the same everywhere
        (analogous to the overall similarities and minor differences
        between Irish English, British English, American English, and
        Australian English).  Depending on the database in use there may be
        tools available that reduce or remove the requirement to know
        \gls{SQL}; several are available for \gls{SQLite}
        (\url{http://www.sqlite.org/}), the default database used by the
        parser implementation.

\end{enumerate}

Storing the results in a database will also increase the efficiency of
using those results, because the log files need only be parsed once rather
than each time the data is used; indeed the results may be used by someone
with no access to the original log files.



\subsection{Simple Mail Transfer Protocol}

\label{SMTP background}

The \acrlong{SMTP}, originally defined in \gls{RFC}~821~\cite{RFC821} and
updated in \gls{RFC}~2821~\cite{RFC2821}, is used for transferring mail
between the sending and receiving \gls{MTA}\@.  It is a simple, human
readable, plain text protocol, making it quite easy to test and debug
problems with it.  Despite the simplicity of the protocol many virus and
spam sending programs fail to implement it properly, so requiring strict
adherence to the protocol specification is beneficial in protecting against
spam and viruses.\footnote{\label{footnote:rfc760}Originally all mail
servers adhered to the principle of \textit{Be liberal in what you accept,
and conservative in what you send\/} from \gls{RFC}~760~\cite{rfc760}, but
unfortunately that principle was written in a friendlier time.  Given the
deluge of spam that mail servers are subjected to daily, a more appropriate
maxim could be: \textit{Require strict adherence to \gls{RFC}~2821;
implement the strongest restrictions you can; relax the restrictions and
adherence only when legitimate mail is impeded.\/}  it is not as friendly,
nor as catchy, but it more accurately reflects the current situation.} A
typical \gls{SMTP} conversation resembles the following (the lines starting
with a three digit number are sent by the server, all other lines are sent
by the client):

\begin{verbatim}
220 smtp.example.com ESMTP
HELO client.example.com
250 smtp.example.com
MAIL FROM: <alice@example.com>
250 2.1.0 Ok
RCPT TO: <bob@example.com>
250 2.1.5 Ok
DATA
354 End data with <CR><LF>.<CR><LF>
Message headers and body sent here.
.
250 2.0.0 Ok: queued as D7AFA38BA
QUIT
221 2.0.0 Bye
\end{verbatim}

An example deviation from the protocol:

\begin{verbatim}
220 smtp.example.com ESMTP
HELO client.example.com
250 smtp.example.com
MAIL FROM: Alice N. Other alice@example.com
501 5.1.7 Bad sender address syntax
RCPT TO: Bob in Sales/Marketing bob@example.com
503 5.5.1 Error: need MAIL command
DATA
503 5.5.1 Error: need RCPT command
Message headers and body sent here.
.
502 5.5.2 Error: command not recognized
QUIT
221 2.0.0 Bye
\end{verbatim}

This example client is so poorly written that not only does it present the
sender and recipient addresses improperly, it ignores the error messages
returned by the server and carries on regardless.  Many spam and virus
sending programs are this deficient --- unfortunately others (particularly
newer programs) were written by competent programmers, or utilise
competently written programs (e.g.\ Postfix or Sendmail on Unix hosts,
Microsoft Outlook on Windows hosts).  Traditionally a mail server would
have done its best to deal with deficient clients, with the intention of
accepting as much mail destined for its users as
possible,\footref{footnote:rfc760} e.g.\ by ignoring the absence of a HELO
command, or by accepting sender or recipient addresses that were not
enclosed in \texttt{<>}.  

A detailed description of \gls{SMTP} is beyond the scope of this thesis:
introductory guides can be found at~\cite{smtp-intro-01,smtp-intro-02},
and the definitive references are~\cite{RFC821,RFC2821}.

\subsection{Postfix}

\label{postfix background}

Postfix is a \acrlong{MTA}~(\gls{MTA}) with the following design aims (in
order of importance): security, flexibility of configuration, scalability,
and high performance.  It features extensive optional anti-spam
restrictions, allowing an administrator to deploy those restrictions which
they judge suitable for their site's needs, rather than a fixed set chosen
by Postfix's author.  These restrictions can be selectively applied,
combined, and bypassed on a per-client, per-recipient, or per-sender basis,
allowing varying levels of stricture and permissiveness.  Postfix leverages
simple lookup tables to support arbitrarily complicated user-defined
sequences of restrictions and exceptions; policy servers
(\sectionref{policy servers}) provide a simple way to write new
restrictions.  Administrators can also supply their own rejection messages
to make it clear to senders exactly why their mail was rejected.
Unfortunately this flexibility has a cost: complexity in the log files
generated.  Although it is easy to use standard Unix text processing
utilities to determine the fate of an individual email, following the
journey an email takes through Postfix can be quite difficult.  For most
mails the journey is simple and brief, but the remaining minority can have
quite complex journeys (see \sectionref{complications} for details).

Postfix's design follows the Unix philosophy of \textit{``Write programs
that do one thing and do it well''\/}~\cite{unix-philosophy}, and is
separated into various component programs to perform the tasks required of
an \gls{MTA}\@: receive mail, send mail, local delivery of mail, etc.\ ---
full details can be found in~\cite{postfix-overview}.  Each log line
contains the name of the Postfix component that produced it, and this
information is used when determining which rules should be used to parse
each log line (see \sectionref{rule characteristics} for details).
Postfix's design is strongly influenced by security concerns: those
components that interact with other hosts are not
privileged,\footnote{Privilege means the power to perform actions that are
limited to the administrator, and not available to ordinary users.} so bugs
in those components will not give an attacker extra privileges; those
components that are privileged do not interact with other hosts, making it
much more difficult for an attacker to exploit any bugs that may exist in
those components.

\subsubsection{Mixing and matching Postfix restrictions}

\label{Mixing and matching Postfix restrictions}

Postfix restrictions are documented fully in~\cite{smtpd_access_readme,
smtpd_per_user_control, policy-servers}; the following is a brief overview
only.

Postfix uses one restriction list (containing zero or more restrictions)
for each stage of the \gls{SMTP} conversation: client connection, HELO
command, MAIL FROM command, RCPT TO commands, DATA command, and end of
data.  The appropriate restriction list is evaluated for each stage
(evaluation will be explained shortly), though by default the restriction
lists for client connection, HELO, and MAIL FROM commands will not be
evaluated until the first RCPT TO command is received, because some clients
do not deal properly with rejections before the first RCPT TO command; a
benefit of this delay is that Postfix has more information available when
logging rejections.

Each restriction is evaluated to produce a result of \textit{reject},
\textit{permit}, \textit{dunno\/} or the name of another restriction to be
evaluated.\footnote{Other results are possible as described
in~\cite{smtpd_access_readme,smtpd_per_user_control,policy-servers}.} The
meaning of \textit{permit\/} and \textit{reject\/} is fairly obvious;
\textit{dunno\/} means to stop evaluating the current restriction and
continue processing the remainder of the restriction list, allowing
exceptions to more general rules.  When the result is the name of another
restriction Postfix will evaluate that restriction, allowing restrictions
to be chosen based on the client \gls{IP} address, client hostname, HELO
hostname, sender address, or recipient address.\footnote{E.g.\ the
administrator may require that all clients on the local network have valid
DNS entries, to prevent people sending mail from unknown machines.}  The
administrator can define new restrictions as a list of existing
restrictions, allowing arbitrarily long and complex sequences of lookups,
restrictions and exceptions.  Postfix tries to protect the administrator in
as far as is reasonable, e.g.\ the restriction
\texttt{check\_helo\_mx\_access} cannot cause a mail to be accepted,
because the parameter it checks (the hostname given in the HELO command) is
under the control of the remote client.  Despite this, it is possible for
the administrator to make catastrophic mistakes, e.g.\ rejecting all mail
--- the administrator must be cognisant of the effects their configuration
changes will have.  This is similar to one of UNIX's design philosophies:
\textit{``UNIX was not designed to stop its users from doing stupid things,
as that would also stop them from doing clever
things''\/}~\cite{unix-philosophy}.

Postfix uses simple lookup tables to make decisions when evaluating some
restrictions, e.g.\newline{}
\tab{}\texttt{check\_client\_access~cidr:/etc/postfix/client\_access}

\begin{eqlist}

    \item [check\_client\_access] The name of the restriction to evaluate.

    \item [cidr] The type of the lookup table.

    \item [/etc/postfix/client\_access] The file containing the lookup
        table.

\end{eqlist}

The restriction \texttt{check\_client\_access} checks if the \gls{IP}
address of the connected client is found in the specified table and returns
the associated action if found; the method of searching the file is
dependant on the type of the file (\texttt{cidr} in the example) --- see
\cite{postfix-lookup-tables} for more details.  Other restrictions
determine their result by consulting external sources, e.g.\newline{}
\tab{}\texttt{reject\_rbl\_client dnsbl.example.com}\newline{} checks the
\gls{DNSBL} \texttt{dnsbl.example.com} and rejects the command if the
client's \gls{IP} address is listed.

For further information about Postfix restrictions
see~\cite{smtpd_access_readme,smtpd_per_user_control,policy-servers}.

\subsubsection{Policy servers}

\label{policy servers}

A policy server~\cite{policy-servers} is an external program consulted by
Postfix to determine the fate of an \gls{SMTP} command.  The policy server
is given state information by Postfix (sample state information is shown in
\tableref{Example attributes sent to policy servers}) and returns a verdict
from the set described in \sectionref{Mixing and matching Postfix
restrictions}.  The policy server can perform more complex checks than
those provided by Postfix: a trivial example is allowing addresses
associated with the payroll system to send mail on the third Tuesday after
pay day only, to help prevent problems from phishing mails using faked
sender addresses.\footnote{A phishing mail might claim that the payroll
system had a disastrous disk failure, and until the server is replaced all
salary payments will have to be processed manually, so please reply to this
mail with your name, address, and bank account details.}

\begin{table}[ht]

    \caption{Example attributes sent to policy servers}
    \empty{}\label{Example attributes sent to policy servers}

    \centering{}

    \begin{tabular}[]{ll}

        request                 & smtpd\_access\_policy     \\
        protocol\_state         & RCPT                      \\
        protocol\_name          & SMTP                      \\
        helo\_name              & some.domain.tld           \\
        queue\_id               & 8045F2AB23                \\
        sender                  & foo@bar.tld               \\
        recipient               & bar@foo.tld               \\
        recipient\_count        & 0                         \\
        client\_address         & 1.2.3.4                   \\
        client\_name            & another.domain.tld        \\
        reverse\_client\_name   & another.domain.tld        \\
        instance                & 123.456.7                 \\

    \end{tabular}

\end{table}

Some widely deployed policy servers:

\begin{itemize}

    \item Sender Policy Framework (SPF)~\cite{openspf}.  SPF\label{spf
        introduction} records specify which mail servers are allowed to
        send mail claiming to be from a particular domain.  The intention
        is to reduce spam from faked sender addresses,
        backscatter~\cite{postfix-backscatter}, and
        joe~jobs~\cite{Wikipedia-joe-job}\glsadd{Joe job}.  There has been
        considerable resistance to the proposal because it breaks or vastly
        complicates some features of \gls{SMTP}, e.g.\ forwarding mail from
        one company or university to another when a user moves.

    \item Greylisting~\cite{greylisting} is a technique that temporarily
        rejects a delivery attempt when the triple of \newline{}
        \tab{}\texttt{(sender address, recipient address, remote \gls{IP}
        address)}\newline{} has not been seen before; on second and
        subsequent delivery attempts from that triple the mail will be
        accepted.  This blocks spam because maintaining a list of failed
        addresses and retrying after a temporary failure is uneconomical
        for a spam sender, but a legitimate mail server must retry
        deliveries that temporarily failed.  Sadly spam senders are using
        increasingly complex and well written programs to distribute spam,
        frequently using an ISP provided \gls{SMTP} server from a
        compromised machine on the ISP's network.  Greylisting will slowly
        become less useful, but it does block a large percentage of spam
        mail at the moment; the most effective restrictions from the
        \numberOFlogFILES{} log files used in testing the parser are shown
        in \tableref{Summary of rejections}.  Greylisting is obviously
        worth using, at least at the moment, particularly when you consider
        Greylisting's position as the final restriction that a mail must
        overcome:\footnote{Greylisting is the final restriction a mail must
        overcome in the configuration used on the mail server the log files
        were obtained from; an administrator is free to use Greylisting
        wherever in the restriction list they feel is most appropriate for
        their mail system.} Greylisting only takes effect for mails that
        have passed every other restriction.  Some problems may be
        encountered when using greylisting: some servers fail to retry;
        legitimate mail may be delayed, particularly when coming from a
        pool of servers.

        \begin{table}[ht]
            \caption{Summary of rejections}\label{Summary of rejections}
            \input{build/include-restriction-summary-table}
        \end{table}

    \item Scoring systems such as Policyd-weight~\cite{policyd-weight}
        perform tests on features of the delivery attempt (e.g.\ \gls{IP}
        address, sender address), incrementing or decrementing a score
        based on the results; if the eventual score is higher than a
        threshold the mail is rejected.  The administrator must manually
        whitelist clients if they are to bypass a Postfix restriction;
        using a threshold that requires a delivery attempt to hit several
        restrictions frees the administrator from whitelisting clients that
        fall foul of one restriction only.

    \item Rate limiting on a per-sender, per-client or per-recipient basis
        as performed by Policyd~\cite{policyd}.

\end{itemize}



\subsection{Summary}

This section has provided background information on several topics,
starting with the motivation behind the project, continuing with an
explanation of the use of a database as an \gls{API} and an introduction to
\gls{SMTP}.  The focus then switched to Postfix, the \gls{MTA} in use, and
its restrictions and policy servers.

\glsresetall{}
\chapter{State Of The Art Review}

\label{state of the art review}

At the start of this project ten Postfix log file parsers were tested, with
the hope of finding a suitable parser to build upon, rather than starting
from scratch.  It was quite difficult to find ten parsers to review for
this project, and the functionality offered by those parsers ranges from
quite basic to much more mature, depending on the needs of the author of
the parser.

It was hoped to reuse an existing parser rather than writing one from
scratch, but the existing parsers considered were rejected for one or more
reasons.  The effort required to adapt and improve an existing parser was
judged to be greater than the effort to write a new one, because the
techniques used by the existing parsers severely limited their potential:
some ignored the majority of log lines, parsing specific log lines
accurately, but without any provision for parsing new or similar log lines;
others sloppily parsed the majority of log lines, but were incapable of
distinguishing between log lines of the same category, e.g.\ rejecting a
mail delivery attempt.  The first parser reviewed~\cite{log-mail-analyser}
is the only previously published research in this area that the author is
aware of; their research aims to show that providing the data from log
files in a more accessible form is helpful to systems administrators.

The ten parsers have been reviewed and compared with \parsername{}, this
project's finished parser, to show how much effort would have been required
to use those to fulfil the aims and requirements of this project.  It is
important to compare and contrast newly developed algorithms and parsers
against those already available, to accurately judge what improvements, if
any, are delivered by the newcomers.

Some important differences exist between \parsername{} and most or all of
the parsers reviewed here:

\begin{enumerate}

    \item None of the parsers reviewed perform the kind of advanced parsing
        required for this project or deal with the complications described
        in \sectionref{complications}.

    \item Only \parsername{} enables parsing of new log lines without
        extensive and intrusive modifications to the parser; \parsernames{}
        architecture is described in \sectionref{parser architecture}.

    \item The parsers reviewed all produce a report of varying complexity
        and detail, whereas \parsername{} does not; it extracts data and
        leaves generation of reports from the data to other programs.
        Using an \acronym{SQL} database simplifies the process of
        generating such reports (discussed in \sectionref{database as
        API}); some sample queries are given in \sectionref{motivation}.
        The parser developed for this project is designed to enable much
        more detailed log file analysis by providing a stable platform for
        subsequent programs to develop upon.

    \item Most of the reviewed parsers silently ignore log lines they
        cannot handle, whereas \parsername{} complains loudly about every
        single log line it fails to parse.  The exception is AWStats, which
        outputs the percentage of log lines it was unable to parse, but
        does not output the log lines themselves.

    \item A minor difference is that most parsers do not handle compressed
        log files; both \parsername{} and Splunk handle them transparently,
        without user intervention; Sawmill and Lire can be configured to
        support compressed log files, but Sawmill exhibits a dramatic
        increase in parsing time when doing so.  Support for reading
        compressed log files is quite helpful, as it dramatically reduces
        the disk space required to store historical log files.

    \item Most of the reviewed parsers do not distinguish between different
        delivery attempt rejections, so they cannot be used to determine
        the success rate of different anti-spam techniques.  The exception
        is Pflogsumm, which provides a summary of why delivery attempts
        were rejected.

\end{enumerate}

Each of the reviewed parsers was tested with the \numberOFlogFILES{} test
log files described in \sectionref{parser efficiency}; \tableref{Summary of
parsers' features} summarises the results of this review.

The data extracted by \parsername{} is documented in
\sectionref{connections table} and \sectionref{results table}; for
convenience that list is repeated here: client and server \acronym{IP}
address and hostname, HELO hostname, queueid, start time, end time,
\acronym{SMTP} code, enhanced status code, sender, recipient, size, message
ID, delay and delays.

\section{Log Mail Analyser}

\label{log mail analyser}

There only appears to be one prior published paper about parsing Postfix
log files: \textit{Log Mail Analyzer: Architecture and Practical
Utilizations\/}~\cite{log-mail-analyser}.  The aim of \acronym{LMA} is
quite different from \parsername{}: it attempts to present correlated data
from log files in a form suitable for a systems administrator to search
using the myriad of standard Unix text processing utilities already
available.  It produces a \acronym{CSV} file and either a MySQL or Berkeley
DB database.  The decision to support both \acronym{CSV} and Berkeley DB
appears to have been a serious limitation: both formats have limitations
that will be explained later.  Hardly any documentation is provided with
\acronym{LMA}, though some documentation is available
in~\cite{log-mail-analyser}.  Studying the source code is informative,
though this author had difficulty as the authors of \acronym{LMA} wrote in
Italian.

\acronym{CSV} is a very simple format where each record is stored in a
single line, with fields separated by a comma or other punctuation symbol.
Problems with \acronym{CSV} files include the need to escape separators in
the data stored, providing multiple values for a field (e.g.\ multiple
recipients), and adding new fields.  \acronym{CSV} files do not have a
standard mechanism to document the fields or the separator, unlike
\acronym{SQL} databases where every database includes a schema naming the
fields and the type of data they store (e.g.\ integer, text, timestamp).
The \acronym{CSV} record format is not documented, but the output file
contains a comment giving the format:\newline{} \texttt{\# Timestamp|Nome
Client|IP Client|IP Server|From|To|Status|Size} \newline{}\acronym{LMA}
treats \acronym{CSV} lines starting with \texttt{\#} as comments, but not
all \acronym{CSV} parsers will.

Berkeley DB only supports storing simple \textbf{(key, value)} pairs,
unlike \acronym{SQL} databases that store arbitrary tuples.  In
\acronym{LMA}'s main table the key is an integer referred to by secondary
tables, and the value is a \acronym{CSV} line containing all of the data
for that row.  The secondary by-sender, by-recipient, by-date, and
by-\acronym{IP} tables use the sender/recipient/date/\acronym{IP} address
as the key, and the value is a \acronym{CSV} list of integers referring to
the main table.  This effectively re-implements \acronym{SQL} foreign keys,
but without the functionality offered by even the most basic of
\acronym{SQL} databases (e.g.\ joins, ordering, searches).  It also
requires custom code to search on some combination of the above, though the
authors of \acronym{LMA} did provide some queries: IP-STORY, FROM-STORY,
DAILY-EMAIL, and DAILY-REJECT\@.  Berkeley DB appears to be the least
useful of the three output formats: it does not provide the functionality
of a basic \acronym{SQL} database, and unlike \acronym{CSV} files it cannot
be used with standard Unix text processing tools.

The schema used with the MySQL database is undocumented, but at least it is
possible to discover the schema with an existing \acronym{SQL} database,
unlike with Berkeley DB\@; all \acronym{SQL} databases embed the schema
into the database and provide commands for displaying it.  Berkeley DB does
not embed a schema in its files, because there is neither requirement nor
benefit; it only provides \textbf{(key, value)} pairs, so any additional
structuring of the data is imposed by the application, thus the application
must document this structure.  MySQL support was not tested because the
schema required is not documented.

Whether a MySQL database or Berkeley DB table is chosen in addition to the
\acronym{CSV} output, \acronym{LMA} stores the following data: time and
date of the log line, client hostname and \acronym{IP} address, server
\acronym{IP} address, sender and recipient addresses, \acronym{SMTP} code,
and size (for accepted mails only).  Unlike \parsername{} it does not store
the server hostname, HELO hostname, queueid, start and end times,
timestamps for each log line, enhanced status code, delivery delays, or
message id (for accepted mails only).  Handling of multiple recipients,
\acronym{SMTP} codes, or remote servers\footnote{A single mail may be sent
to multiple remote servers if it was addressed to recipients in different
domains, or Postfix needs to try multiple servers for one or more
recipients.} is not explained; experimental observation shows that multiple
records are added when a mail had multiple recipients (sadly the records
are not associated or linked in any way), and presumably the same approach
is taken when there were multiple destination servers.

\acronym{LMA} requires major changes to the parser code to parse new log lines
or to extract additional data.  The code is structured as a long series of
blocks that each handle all log lines matching a single regex, so parsing
new log lines requires modifying an existing regex or carefully inserting a
new block in the correct place; extracting extra data will require
modifying multiple blocks, regexes, or both.

\acronym{LMA} does not deal with any of the complications discussed in
\sectionref{complications}, except for correlating log lines by queueid;
not correlating log lines by pid means it cannot correlate most rejections.
It does not differentiate between different types of rejections, so it is
not suitable for the purposes of this project; the data about which
restriction caused the rejection is discarded, whereas the main goal of
this project is to retain that data to aid optimisation and evaluation of
anti-spam techniques.  \acronym{LMA} fails to parse Postfix log files
generated on Solaris hosts because the fields automatically prepended to
each log line differ from those added on Linux hosts; log files from
Solaris hosts (and possibly other operating systems) thus require
pre-processing before parsing by \acronym{LMA}.  The \numberOFlogFILES{}
pre-processed test log files were parsed without complaint by
\acronym{LMA}, although it produced 32 entries in the output \acronym{CSV}
file for every rejection in the input log file; it also missed some 40\% of
delivered mail.  Once these deficiencies were discovered the author did not
spend any more time checking the results.

\acronym{LMA} does provide some simple reports: IP-STORY, FROM-STORY,
DAILY-EMAIL and DAILY-REJECT\@.  These reports search the Berkeley DB files
for matching records: the first three extract \acronym{CSV} lines for the
specified client \acronym{IP} address, sender address, or date
respectively.  DAILY-REJECT initially failed with an error message from the
Perl interpreter;\footnote{The error messages were:
\newline{}\texttt{Undefined subroutine \&main::LIST called at queryDB.pl
line 372.}\newline{}\texttt{Undefined subroutine \&main::EXTRACT\_FROM\_DB
called at queryDB.pl line 379.}} after correcting the errors in the code it
worked, extracting the \acronym{CSV} lines for the specified day where the
\acronym{SMTP} code signifies a rejection.  All of these reports are
trivially simple to produce from the \acronym{CSV} file using the standard
Unix tool \texttt{awk}\glsadd{awk}; the most complicated, DAILY-REJECT, is
merely:

% perl queryDB.pl -dayreject 2007-01-26 > lma-query

\begin{verbatim}
awk -F\| 'BEGIN { previous = "" };
    $1 ~ /2007-01-26/ && $7 != "250" && $0 != previous
    { print $0; print " "; previous = $0; }' lma_output.txt
\end{verbatim}

Notes about the command above:

\begin{itemize}

    \item It outputs a line containing only a single space after each
        matching record, to accurately replicate the output of
        DAILY-REJECT\@.

    \item DAILY-REJECT considers all \acronym{SMTP} codes except ``250'' to
        be rejections; this includes invalid \acronym{SMTP} codes such as
        \texttt{0} and \texttt{deferred}, so the awk command does too.
        These invalid \acronym{SMTP} codes are most likely present because
        of incorrect parsing by \acronym{LMA}.

    \item \acronym{LMA} produces 32 lines in its \acronym{CSV} file for
        every single line it should have produced; the command above
        suppresses duplicate sequential lines.  DAILY-REJECT produces the
        correct number of output lines, probably because it uses the
        Berkeley DB files as input rather than the \acronym{CSV} file.

\end{itemize}

The output from DAILY-REJECT and the \texttt{awk} command is not exactly
the same; the author did not spend substantial time attempting to explain
these differences.

\begin{enumerate}

    \item The output from DAILY-REJECT is missing some records that are
        present in the \acronym{CSV} file; this may be because it uses the
        Berkeley DB files instead, and there may be differences between the
        contents.

    \item Some records output by DAILY-REJECT are truncated: they are
        missing the last $|$ that separates the fields and the newline
        following it, so the line containing only a single space is
        concatenated with the record.

\end{enumerate}

In summary, \acronym{LMA} appears to be a proof of concept, written to
demonstrate the point of their paper (that having this information in an
accessible fashion is useful to systems administrators), rather than a
program designed to be useful in a production environment.

% Literature review notes:
%
% Hard-coded parsing, requiring code changes to add more.  Attempts to
% correlate log lines, saves data to database for data mining purposes.
% Hard to extend/expand/understand.  Appears to only save: date and hour,
% DNS name and \acronym{IP} address host, mail server \acronym{IP} address,
% sender, receiver and e-mail status (sent, rejected).  Undocumented
% schema.  Design decision to use \acronym{CSV} as an intermediate format
% between the log file and the database seems to have been restrictive.
% Appears to require a queueid but majority of log lines (e.g.\ rejections)
% lack a queueid.  Supports whitelisting \acronym{IP} addresses when
% parsing logs, but whitelisting when generating reports/data mining would
% be preferable.  Supporting Berkeley DB is probably limiting the software;
% an example is the difficulty in searching a pipe-delimited string, so
% they have re-implemented foreign keys with tables keyed by ip address
% etc.\ pointing at the main table - this also will not scale well.  There
% does not appear to be any attempt to deal with the complications I have
% encountered: their parsing is not detailed enough to encounter them.  It
% does not run properly; does not create any output; throws up errors.

\section{Pflogsumm}

\parserblurb{%
    pflogsumm is designed to provide an over-view of Postfix activity, with
    just enough detail to give the administrator a ``heads up'' for
    potential trouble spots.
}
{http://jimsun.linxnet.com/postfix_contrib.html}
{2008/11/23}

Pflogsumm produces a report designed for troubleshooting rather than
in-depth analysis.  It does not support saving any data, and it does not
extract any data that it does not require to produce its report; e.g.\ it
does not extract the HELO hostname, queueid, start and end times,
timestamps for each log line, or message id.  Both the parsing and
reporting are difficult to extend because it is a specialised tool, unlike
the easily extensible design of \parsername{}.  It does not correlate log
lines by queueid or \acronym{pid}, and does not need to deal with the
complications encountered during this project (\sectionref{complications}).
Pflogsumm produces a useful report, and successfully parsed the
\numberOFlogFILES{} log files it was tested with.\footnote{The results it
reported were not verified in detail, but it did not report any errors, and
has an excellent reputation amongst Postfix users.}  Pflogsumm has many
options to include or exclude certain sections of the report, all clearly
documented; by default it includes the following:

\begin{itemize}

    \item Total number of mails accepted, delivered, and rejected.  Total
        size of mails accepted and delivered.  Total number of sender and
        recipient addresses and domains.

    \item Per-hour averages and per-day summaries of the number of mails
        received, delivered, deferred, bounced, and rejected.

    \item For received mail: per-domain totals for mails sent, deferred,
        average delay, maximum delay, and bytes delivered.  For received
        mail: per-domain totals for size and number of mails received.

    \item Number and size of mails sent and received for each address.

    \item Summary of why mail delivery was deferred or failed, why mails
        were bounced, why mails were rejected, and warning messages.

\end{itemize}

Pflogsumm is the only reviewed parser that distinguishes between different
rejections.

\section{Sawmill Universal Log File Analysis And Reporting}

\parserblurb{%
    Sawmill is a Postfix log analyzer (it also support 818 other log
    formats).  It can process log files in Postfix format, and generate
    dynamic statistics from them, analyzing and reporting events.  Sawmill
    can parse Postfix logs, import them into a SQL database (or its own
    built-in database), aggregate them, and generate dynamically filtered
    reports, all through a web interface.  Sawmill can perform Postfix
    analysis on any platform, including Window~[sic], Linux, FreeBSD,
    OpenBSD, Mac OS, Solaris, other UNIX, and more.
}
{http://www.thesawmill.co.uk/formats/postfix.html}
{2008/11/23}

Sawmill is a general purpose commercial product that parses 818 log file
formats (as of 2008/11/23) and produces reports from the extracted data.
Its data extraction facilities (described later) are too limited to save
enough data for the purposes of this project: although it can extract three
different sets of data from Postfix log files, they are not interlinked in
any way.  The documentation does not suggest that Sawmill correlates log
lines by either queueid or pid or deals with the other difficulties
documented in \sectionref{complications}.

Sawmill has three different Postfix log file parsers, extracting three
different sets of data:

\begin{enumerate}

    \item \urlLastChecked[\hfill{}\newline{}]{http://www.thesawmill.co.uk/formats/postfix.html}{2008/11/23}.
        Fields extracted: from, to, server, UID, relay, status, number of
        recipients, origin hostname, origin \acronym{IP} address, and
        virus.  It also counts the number of and total size of all mails
        delivered.  The fields \texttt{server}, \texttt{uid},
        \texttt{relay}, and \texttt{virus} are not explained in the
        documentation: \texttt{server} is probably the hostname or
        \acronym{IP} address of the server the mail is delivered to;
        \texttt{relay} might be the delivery method: \acronym{SMTP}, local
        delivery, or \acronym{LMTP}; \texttt{uid} might be the uid of the
        user submitting mail locally.  Postfix does not do any form of
        virus checking (though it has many options for cooperating with an
        external virus scanner), so the \texttt{virus} field is a mystery.

    \item \urlLastChecked[\hfill{}\newline{}]{http://www.thesawmill.co.uk/formats/postfix_ii.html}{2008/11/23}.
        Fields extracted: from, to, RBL list, client hostname, and client
        \acronym{IP} address.  It also counts the number and total size of
        all mails delivered.

    \item \urlLastChecked[\hfill{}\newline{}]{http://www.thesawmill.co.uk/formats/beta_postfix.html}{2008/11/23}.
        Fields extracted: from, to, client hostname, client \acronym{IP}
        address, relay hostname, relay \acronym{IP} address, status,
        response code, RBL list, and message id.  It also counts the number
        and size of all mails delivered, processed, blocked, expired, and
        bounced.

\end{enumerate}

Even if the three data sets were combined Sawmill would extract less data
than \parsername{}: it omits the HELO hostname, queueid, enhanced status
code, delivery delays, and start and end times.  Sawmill does not extract
any data about rejections except when the rejection is caused by a
\acronym{DNSBL} check (\texttt{RBL list} in the list of fields).

The source code is only available in an encrypted form, to support people
who wish to use Sawmill on operating systems or machine architectures the
company do not provide executables for.  Sawmill is quite expensive,
requiring a \euros{100} + VAT licence per report, with discounts available
when buying multiple licences (correct as of 2008/11/23); in contrast,
\parsername{} is free to use and the code is freely available.  Sawmill is
supplied with thorough and well written documentation; everything the
author searched for was documented, except the MySQL database schema and
some details of the data extracted by the parser.  A commercial version of
MySQL is required because of MySQL licensing restrictions, but Sawmill's
documentation explains why and includes instructions on how to compile
Sawmill so that it can use a non-commercial version of MySQL (this was not
attempted during the review process).

Sawmill's web interface supports searching on any combination of the fields
it extracts, and all searches produced accurate results.  The interface for
searching is neither as simple to use nor as informative as the interface
provided by Splunk (see \sectionref{Splunk review}).  The administrative
interface is much easier to use than Splunk's: it took only five minutes to
start parsing a whole directory of log files.

When tested with the \numberOFlogFILES{} test log files it performed
adequately, though the rate it processed log files at did slow down
noticeably as it progressed.  Sawmill supports reading compressed log files
but it exhibits a dramatic slow down when doing so: it took six hours to
parse the first half of the log files, and twelve hours to parse the next
third; after twenty four hours parsing the remaining sixth it crashed due
to lack of disk space.  On the second parsing attempt the log files were
uncompressed beforehand and parsing took eight hours.

In summary, Sawmill suffers from supporting so many types of log files: it
parses many types of log files, but none of them very well; it is probably
much more useful when parsing log files where each log line is
self-contained (e.g.\ web server log files), rather than log files
containing interlinked log lines.  It is not suitable as a base for this
parser, as the source code made available is encrypted and not intended
for modification; in addition the architecture would probably need to be
overhauled or replaced to deal with correlating log lines.

\section{Splunk}

\label{Splunk review}

\parserblurb{%
    Splunk is IT Search.  Search and navigate IT data from applications,
    servers and network devices in real-time.  Logs, configurations,
    messages, traps and alerts, scripts, code, metrics and more.  If a
    machine can generate it --- Splunk can eat it.  It\empty{}'s easy to
    download and use and it\empty{}'s very powerful.
}
{http://www.splunk.com/}
{2008/11/23}

Splunk aims to index all of an organisation's log files, providing a
centralised view capable of searching and correlating diverse log sources.
The web interface provides search functionality, generating statistics and
graphs in real time, a facility not provided by \parsername{}.  Splunk
allows quite complicated searches, based on the fields extracted by Splunk
(described later) or the full text of the log line, though it is not
possible to search on partial words.  Search results are not available for
use with other tools.  Searches can be saved for reuse; saved searches can
be run periodically and the results mailed to a recipient or sent to an
external program for further processing.  The author was unable to save
searches, though that may have been because of limitations in the free
version.  The data store is not available for use by external programs,
whereas \parsername{} provides the database and leaves it to the user to
utilise it without limit or restriction.  The web interface is optimised
for interactive use rather than automated queries, and it does not appear
to be possible to write independent tools to utilise the Splunk database.
Some additional Postfix reports are supposedly available on
\urlLastChecked{http://www.splunkbase.com/}{2008/11/23}, but the author was
unable to find any Postfix reports, or indeed reports for any other log
file types: every category was empty, even those that the web site claimed
had numerous reports available.  Many types of graphs can be generated,
though most are variations of a bar or pie chart, except the bubble and
heatmap graphs.  It is easy to drill down through the graphs to select a
portion of the data (e.g.\ select the hour with the largest number of
events, then select a particular host, and finally a specific sender
address).  All searches performed using the indexed data returned
reasonable results.

The web interface is quite attractive and simple to use when searching, but
as an administrator it seems unnecessarily difficult to perform simple
tasks.  When testing Splunk it took roughly 30 minutes to figure out how to
add a single log file to be indexed so that it could be searched, with the
downside that the log file was copied into a spool directory before
indexing, doubling the disk space usage.  The next test was to index all
the log files in a particular directory, but after three hours, numerous
futile attempts, and reading all the available documentation, the author
was still unable to index all the log files in a directory using the web
interface.  Using the \acronym{CLI} rather than the web interface was
partially successful: the command ``\texttt{splunk find logs
}\textit{log-directory\/}'' added 40 of the \numberOFlogFILES{} log files
to the queue for indexing.  Further attempts enqueued the same 40 log
files, without explaining why the others were excluded.\footnote{The log
files appear to have been indexed once only; presumably Splunk keeps track
of the log files it has indexed and discards requests to index log files
for a second time.  This may or may not be a useful feature for
\parsername{}.} There did not appear to be an option to ensure the log
files would be processed in the order they were created, though this may be
neither necessary nor beneficial with Splunk.  Subsequently the author was
successful in adding a single file at a time using the \acronym{CLI} and
the command ``\texttt{splunk add tail }\textit{filename\/}''.  A simple
loop using that command was enough to add all the desired log files.
Splunk will periodically check all indexed log files for updates unless
they are manually removed from its list; this may or may not be useful
behaviour.  Splunk did not appear to have any difficulty in indexing the
log files, once they had been successfully added to its queue.
\parsername{} parses the logs it is instructed to parse, in the order
given; periodic parsing of logs is a task an administrator can easily
achieve with \texttt{cron(8)} and \texttt{logrotate(8)}.

Copious documentation is made available on
\urlLastChecked{http://www.splunk.com/}{2008/11/23}, but the organisation
and abundance of material does not support easy identification of useful
information.  Searches confusingly tended to return results from old
documentation rather than new.  In general the documentation appears to
have been written by someone intimately acquainted with the software, who
has difficulty understanding how a newcomer would approach tasks or the
questions they would ask.

Splunk supports reading compressed log files without any configuration by
the user.  The free version of Splunk limits the volume of data indexed per
day to 500MB, though a trial Enterprise licence is available that allows
indexing of up to 5GB of data per day.  In 2007, the cheapest licenced
version cost \$5000 plus \$1000 support, and limited the volume of data
indexed per day to 500MB\@.  Prices were removed from the Splunk website
during 2008; now Splunk's sales team must be asked for a quote.  Typical
log file sizes for a small scale mail server are given in
\sectionref{parser efficiency}.

When parsing Postfix log files Splunk parses the standard
syslog\glsadd{syslog} fields at the beginning of the log line, and extracts
any \texttt{key=value} pairs occurring after the standard syslog prologue:
to and from addresses, HELO hostname, and protocol (\acronym{SMTP} or
\acronym{ESMTP}).  \parsername{} extracts noticeably more data (client and
server \acronym{IP} address and hostname, queueid, start and end times,
timestamps for each log line, \acronym{SMTP} and enhanced status codes,
delivery delays, and message ID), though it does not make the full text of
the line available (this could be trivially added if desired, but would
greatly increase the size of the resulting database).  The full power of
\acronym{SQL} is available when searching the data extracted by
\parsername{}, allowing the user to search on arbitrarily complicated
conditions.

Splunk is a generic tool, so it lacks any Postfix specific support over and
above extracting the \texttt{key=value} fields from a log line; it makes no
attempt to correlate log lines by queueid or \acronym{pid}, or to handle
any of the other myriad complications discussed in
\sectionref{complications}.  Its source code is unavailable, so it could
not be used as a base for this project, even if it fulfilled all other
requirements.

\section{Isoqlog}

\parserblurb{%
    Isoqlog is an MTA log analysis program written in C.  It designed to
    scan qmail, postfix, sendmail and exim logfile and produce usage
    statistics in HTML format for viewing through a browser.  It produces
    Top domains output according to Sender, Receiver, Total mails and
    bytes; it keeps your main domain mail statistics with regard to Days
    Top Domain, Top Users values for per day, per month and years.
}
{http://www.enderunix.org/isoqlog/}
{2009/01/11}

Isoqlog's report misses most of the information gathered by \parsername{}:
the data extracted is limited to the number of mails sent by each sender,
and it only reports on senders from the domains listed in its configuration
file, making it impossible to produce complete reports.  It ignores all log
lines except those with today's date, so it is impossible to analyse
historical log files, and testing with the \numberOFlogFILES{} test log
files was pointless.  It does maintain a record of data previously
extracted and the newly extracted data is merged into it; the format of the
data store is undocumented.  Almost no documentation is provided with
Isoqlog, little more than installation instructions.  It does not utilise
rejection log lines in any way, so is unsuitable for the purposes of this
project.  Its parsing is completely inextensible, indeed is almost
incomprehensible, relying on \texttt{scanf(3)}, unexplained fixed offsets,
and low level string manipulation; it is the opposite end of the spectrum
to \parsernames{} parsing.  It does not handle any of the complications
discussed in \sectionref{complications}, does not gather the breadth of
data required for this project, and ignores most of the log lines produced
by Postfix.

\section{AWStats}

\parserblurb{%
    AWStats is a free powerful and featureful tool that generates advanced
    web, streaming, ftp or mail server statistics, graphically.  This log
    analyzer works as a CGI or from command line and shows you all possible
    information your log contains, in few graphical web pages.  It uses a
    partial information file to be able to process large log files, often
    and quickly.  It can analyze log files from all major server tools like
    Apache log files (NCSA combined/XLF/ELF log format or common/CLF log
    format), WebStar, IIS (W3C log format) and a lot of other web, proxy,
    wap, streaming servers, mail servers and some ftp servers.
}
{http://awstats.sourceforge.net/awstats.mail.html}
{2009/01/11}

AWStats can produce simple graphs from many different services' log files,
but supporting numerous log files formats without special purpose code
limits its functionality.  The data it can extract from an \acronym{MTA}
log file is limited in comparison to \parsername{}: time2, email, email\_r,
host, host\_r, method, url, code, and bytesd.  No explanation for any of
those fields is provided in the documentation at
\urlLastChecked{http://awstats.sourceforge.net/docs/awstats_faq.html\#MAIL}{2008/11/23},
so the author could neither understand what data is extracted, nor
determine what data is missing in comparison to \parsername{} (which fully
documents everything it extracts).  AWStats coerces Postfix log files into
the log file format used by the Apache web server, for analysis by AWStats'
HTTP log file parser.  The converting parser only deals with a small
portion of the log lines generated by Postfix, silently skipping those it
cannot deal with, and does not distinguish between different types of
rejection; extending it to parse all log lines would be at least as much
work as writing a new parser.  It does correlate log lines by queueid (not
by pid), but it does not deal with any of the other complications described
in \sectionref{complications}.  AWStats supports saving data, but the
format of the data store is not documented.  It also supports reading
compressed log files, but that functionality was not tested.

When tested with the \numberOFlogFILES{} test log files AWStats' reported
that it parsed 9,240,075 (88.70\%) of 10,416,129 log lines, skipping
1,176,050 (11.29\%) corrupt log lines; the \numberOFlogFILES{} log files
contain \numberOFlogLINES{} log lines, so AWStats parsed only 15.21\% of
the log lines, declared 1.93\% were corrupt, and ignored the remaining
82.85\%.  The author did not verify the correctness of the parsing of those
log lines AWStats did parse.

The graphs it produces give an overview of mails received for the last
calendar month, showing:

\begin{itemize}

    \item The number of mails accepted from each host.

    \item How many mails were received by each recipient.

    \item The average number of mails accepted by the server per-day and
        per-hour.

    \item A summary of the \acronym{SMTP} codes used when rejecting
        delivery attempts.

\end{itemize}

AWStats was not a suitable base for this project, because it assumes that
all log files can be rewritten to be compatible with web server log files,
and will contain similar data; coercing Postfix log files into web server
log files, without substantial data loss, would require fully parsing the
Postfix log files without using AWStats, i.e.\ would require writing a
separate parser.  It may be possible to use AWStats' graphing capabilities
to generate reports, by generating input for AWStats from the data
extracted by \parsername{}.

\section{Anteater}

\parserblurb{%
    The Anteater project is a Mail Traffic Analyser.  Anteater supports
    currently the logformat produced by Sendmail and by Postfix.  The tool
    is written in 100\% C++ and is very easy to customize.  Input, output,
    and the analysis are modular class objects with a clear interface.
    There are eight useful analyse modules, writing the result in plain
    ASCII or HTML, to stdout or to files.
}
{http://anteater.drzoom.ch/}
{2009/01/11}

Anteater does not have any English documentation except for the quote above
so it is impossible for this author to accurately comment on the analysis
it performs.  It did not run successfully when tested, and its parsing
would certainly be out of date because Postfix has evolved considerably
since this tool was last updated (2003/11/06).  As it neither ran
successfully nor has documentation the author can read, a detailed review
cannot be provided.

The Debian Linux distribution provides a translated manual page with the
copy of anteater it distributes, so the author was at least able to run
anteater with the correct arguments; sadly anteater produced zero for every
statistic, presumably because it was unsuccessful in parsing the log lines.

\section{Yet Another Advanced Logfile Analyser}

\parserblurb{%
    yaala is a very flexible analyser for all kinds of logfiles.  It uses
    parsers to extract information from a logfile, an SQL-like query
    language to relate the information to each other and an output-module
    to format the information appropriately.
}
{http://yaala.org/}
{2009/01/11}

YAALA uses a plugin-based system to analyse log files and produce reports
in HTML format, with all the parsing and report generation handled by
plugins.  Using YAALA as a base for this project would have been as much
work as starting from scratch, as both the input and output modules would
need to be written specially; it might be more work to implement a parser
within the constraints of YAALA\@.  YAALA supports storing previously
gathered data using Perl's Storable module, so other Perl programs can use
Storable to load, examine, and optionally modify the data; \parsername{}
uses a well documented database that is accessible from most common
programming languages.  Information about how YAALA stores data was gleaned
from the source code, as the documentation is sadly lacking.

YAALA provides a Postfix parser that extracts the following of fields from
specific log lines:

\begin{eqlist}

    \item [Aggregations:] count (not explained), bytes (sum of bytes
        transferred).

    \item [Keyfields:] incoming\_host, outgoing\_host, date, hour, sender,
        recipient, defer\_count, delay.  Which date and hour are stored is
        not documented: start time, end time, delivery time, or another
        time?

\end{eqlist}

\noindent{}YAALA's Postfix parser extracts some of the fields \parsername{}
does: for client and server it stores either the \acronym{IP} address or
the hostname, not both; it omits the HELO hostname, queueid, \acronym{SMTP}
and enhanced status codes, size of each accepted mail, start and end times,
timestamps for each log line, and message ID\@.  It extracts some data that
\parsername{} does not: how many times delivery was deferred for each mail;
this information can be determined from the database populated by
\parsername{} if desired.  Unlike \parsername{}, YAALA does not maintain
separate counters for each restriction; this rules out the possibility of
using the collected data for optimisation, testing, or understanding of
restrictions.  YAALA's Postfix parser does not deal with the complications
explained in \sectionref{complications}, except it does correlate log lines
by queueid.

YAALA provides a mini-language based on \acronym{SQL} that is used when
generating reports; sample reports can be seen
at~\urlLastChecked{http://www.yaala.org/samples.html}{2008/11/23}.  Example
query for HTTP proxy servers: \newline{} \tab{} \texttt{requests BY file
WHERE host =\~{} Google} \newline{} The mini-language is quite limited and
cannot be used to extract data for external use, merely to create reports.
Only data selected by the query will be saved in the data store; other data
will be discarded, and removed from the data store if already present.

Testing YAALA was unsuccessful because all the select clauses tried
produced a similar error message:
\newline{}\tab{}\texttt{lib/Yaala/Data/Core.pm: Unavailable aggregation
requested:} \newline{}\tab{}\tab{}\texttt{``bytes''.  Returning 0.}
\newline{}  The underlying reason for this is that YAALA only parsed 408
(0.11\%) of 360632 log lines in the first log file; it was not tested with
the remainder of the \numberOFlogFILES{} log files.

It might be possible to use \parsername{} as a plugin with YAALA, perhaps
with an intermediate plugin interfacing between the two, but YAALA's data
store is insufficient for \parsernames{} needs: \parsername{} uses two
separate tables, whereas YAALA assumes all data will reside in one
structure; YAALA's querying mini-language might not deal successfully with
data in separate structures.  This approach has not been attempted by the
author.

In summary, YAALA provides a Postfix parser that tries to parse the most
common Postfix log lines only, provides reasonably flexible report
generation from the limited data extracted, but has no facilities to
extract data for use in other tools.

\section{Lire}

\sloppy{}%
\parserblurb{%
    As any good system administrator knows, there's a lot more to keep
    track of in an active network than just webservers.  Lire is hands down
    the most versatile log analysis software available today.  Lire not only
    keeps you informed about your HTTP, FTP, and mail traffic, it also
    reports on your firewalls, your print servers, and your DNS activity.
    The ever growing list of Lire-supported services clearly outstrips any
    other software, in large part thanks to the numerous volunteers who
    have pioneered many new services and features.  Lire is a total solution
    for your log analysis needs.
}
{http://logreport.org/lire.html}
{2009/01/11}
\fussy{}

Lire is a general purpose log file parser supporting many different types
of log file.  It takes a similar approach to YAALA, using plugins to parse
different log file types.  The data extracted by its Postfix parser is not
clearly documented:

\begin{quotation}

    \noindent{} The email servers' reports will show you the number of
    deliveries and the volume of email delivered by day, the domains from
    which you receive or send the most emails, the relays most used, etc.

\end{quotation}

\noindent{}Examining the source code reveals that the parser looks for
\texttt{<key>=<value>} pairs in each log line, extracts them, and
correlates the data by queueid.  This approach will extract the following
data: HELO hostname, queueid, \acronym{SMTP} code, sender and recipient
addresses, and size of accepted mails.  It is unclear if the parser will
extract any further data.  Lire misses the following fields extracted by
\parsername{}: client and server \acronym{IP} address and hostname, start
and end times, enhanced status code, delivery delays, timestamps of each
log line, and message ID\@.

Lire supports multiple output formats for generated reports (text, HTML,
PDF, and Excel 95) but the reports do not appear to be customisable;
\parsername{} does not produce any reports.  Lire's report is not as
detailed as Pflogsumm's, and it is considerable harder to configure.  Lire
supports saving extracted data for later report generation, but the format
of this data store is undocumented; given the source code it should be
possible, with enough time and effort, to understand the format.
\parsername{} uses an \acronym{SQL} database to make accessing the
extracted data as easy as possible.  In general, Lire has poor
documentation.

Similar to AWStats and Logrep, Lire attempts to correlate log lines by
queueid, but not by \acronym{pid}, so the complete list of recipients for a
mail should be available; its parser extracts only part of the available
data and makes no attempt to deal with the other complications described in
\sectionref{complications}.  When testing Lire on the \numberOFlogFILES{}
test log files it performed reasonably well: the numbers it reports appear
reasonable, and the subset verified by the author were correct.  Its report
provided summaries of:

\begin{itemize}

    \item Delivery status and failed deliveries.

    \item Sender and recipient domains and servers.

    \item Number of deliveries and bytes per-day and per-hour.

    \item Recipients by domain.

    \item Deliveries by relays, by size, and by delay.

    \item Delays by server and by domain.

    \item The pair of correspondents that exchanged the highest number of
        mails.

\end{itemize}

Lire would not be a suitable base for this project: it does not extract
enough data; does not deal with rejections in any way; does not make the
extracted data easily available to other programs.  Its parser is
in-extensible but could easily be replaced, but that would require writing
a parser from scratch, so would not be any less work.  \parsername{} could
possibly be used to parse Postfix log files for Lire, but the difficulty
may outweigh the benefits.  Lire's data store may not be suitable for
storing the data extracted by \parsername{}, but the lack of documentation
hinders any evaluation.  As with YAALA this approach has not been attempted
by the author.

\section{Logrep}

\parserblurb{%
    Logrep is a secure multi-platform framework for the collection,
    extraction, and presentation of information from various log files.  It
    features HTML reports, multi dimensional analysis, overview pages, SSH
    communication, and graphs, and supports over 30 popular systems
    including Snort, Squid, Postfix, Apache, Sendmail, syslog, ipchains,
    iptables, NT event logs, Firewall-1, wtmp, xferlog, Oracle listener and
    Pix.
}
{http://www.itefix.no/i2/index.php}
{2009/01/11}

Logrep extracts fewer than half the fields \parsername{} does:

\begin{itemize}

    \item For mail sent and received: from address, size, and time and
        date.  Which date and hour are stored is not documented: start
        time, end time, delivery time, or another time?

    \item For mail sent: to addresses, \acronym{SMTP} code, and delay.

    \item For mail received: the hostname of the sender.

\end{itemize}

It also counts the number of log lines parsed and skipped.  It omits client
\acronym{IP} address and hostname, server \acronym{IP} address, HELO
hostname, queueid, timestamps of each log line, enhanced status code, and
message ID\@.  Log lines are correlated based on the queueid (called
sessionname [sic] within Logrep), but not by \acronym{pid}.  The parsing is
error prone: empty fields are saved when the log line does not match the
regex, though it appears that they will not overwrite existing data.  Most
notably rejections are completely ignored, making it unsuitable for the
purposes of this project.  It does not try to address any of the
complications in \sectionref{complications} except for correlating by
queueid.

Logrep does not come with any documentation, though some scant
documentation is available on its website (\parsername{} provides copious
documentation).  It requires a web browser to interact with it, so
automated log file processing will be difficult, whereas enabling automated
processing is a key part of \parsernames{} design.  Sadly all the author's
attempts to use Logrep failed, as it was unable to access the log files
selected; this appears to be a bug rather than operator error.  If it was
caused by operator error, the interface needs improvement as the (minimal)
instructions were followed as closely as possible, and multiple attempts
were made.  Because parsing failed it was not possible to review the
reports Logrep can generate (available in HTML only), or to examine the
(undocumented) format it uses to save extracted data for subsequent reuse.

Logrep extracts far less data from Postfix log files than \parsername{},
completely ignores rejections, is effectively undocumented, does not deal
with the more complicated aspects of Postfix log files, and at the time of
writing does not work properly.

\section{Summary}

This chapter has reviewed ten programs that perform basic Postfix log file
parsing (some to a greater level of detail than others).   None of the
reviewed parsers collect the breadth of information gathered by
\parsername{}, or make it as easy to extend the parser to handle new log
lines.  Some correlate log lines by queueid (none correlate by pid); none
deal with any of the other complications described in
\sectionref{complications}.  All of the reviewed parsers generate a report,
and some provide a greater or lesser degree of customisation.  Most have a
data store, but only \acronym{LMA} provides any documentation on its
format; some deliberately make the data store inaccessible to other tools.
Most but not all of the parsers provide documentation, with the quality
ranging from unusable to very good.  Fewer than half of the parsers were
capable of parsing the \numberOFlogFILES{} test log files; improving or
extending parsing would have been quite a difficult task for any of the
parsers, and one that the author did not have the time to attempt.
\Tableref{Summary of parsers' features} provides a summary of the parsers'
features.  The overriding difference between \parsername{} and the other
parsers reviewed herein is that none of them aim for the high level of
understanding of Postfix log files achieved by \parsername{}.


\begin{table}[thbp]
    \caption{Summary of reviewed parsers' features}
    \empty{}\label{Summary of parsers' features}
    \begin{tabular}{llllll}
        \tabletopline{}%
        Parser          & Parsed test   & Data              & Custom            & Documentation  & Source       \\
                        & log files?    & store?            & reports?          & quality?       & code?        \\
        \tablemiddleline{}%
        \acronym{LMA}       & No            & Yes               & No                & Poor           & Yes          \\
        Pflogsumm       & Yes           & No                & Partial \dag{}    & Good           & Yes          \\
        Sawmill         & Yes           & Yes               & Searches          & Very good      & \nialpha{}   \\
        Splunk          & Yes           & Yes               & Searches          & Abundant       & No           \\
                        &               &                   & \& reports        & but poor       &              \\
        Isoqlog         & No            & Yes               & No                & No English     & Yes          \\
                        &               &                   &                   & documentation  &              \\
        AWStats         & Partially     & Yes               & Partial \dag{}    & Good           & Yes          \\
        Anteater        & No            & No                & No                & None           & Yes          \\
        YAALA           & No            & Yes \ddag{}       & Searches          & Poor           & Yes          \\
                        &               &                   & \& reports        &                &              \\
        Lire            & Yes           & Yes               & Yes               & Reasonable     & Yes          \\
        Logrep          & No            & Yes               & No                & None           & Yes          \\
        \parsername{}   & Yes           & Yes \nibeta{}     & No \nichi{}       & \niepsilon{}   & Yes          \\
        \tablebottomline{}%
    \end{tabular}

    \begin{eqlist}

        \item [\dag{}] Sections can be omitted from a report, but extra
            sections cannot be added.

        \item [\ddag{}] YAALA only stores the data required to produce the
            latest report; other data will be discarded.

        \item [\nialpha{}] Sawmill's source code is available in an
            encrypted form, so that customers can compile it on platforms
            that pre-compiled binaries are not available for.

        \item [\nibeta{}] \parsername{} is the only parser with
            documentation for its data store.

        \item [\nichi{}] \parsername{} defers report generation to
            subsequent programs, but all the necessary data and
            documentation to produce reports is provided.

        \item [\niepsilon{}] \parsername{} aims to have thorough and
            complete documentation, but the author cannot provide an
            unbiased review.

    \end{eqlist}

\end{table}

\clearpage{}

\glsresetall{}
\chapter{Parser Architecture}

\label{parser architecture}

XXX REWRITE\@: MERGE CONTENT FROM ELSEWHERE IN THE THESIS AND PAPER\@.

XXX WHEN FINISHED, CHECK ALL REFERENCES TO ENSURE THEY POINT AT THE CORRECT
CHAPTER\@.

To avoid cluttering the explanation of the parser architecture with the
details involved in parsing Postfix log files and the description of the
parser implemented for this project, the two have been separated.  This
chapter presents the architecture designed and developed for this project,
beginning with the overall architecture and design, followed by the three
components of the architecture: Framework, Actions, and Rules, and
finishes by describing the characteristics of this architecture.

This chapter centers on the theoretical, implementation-independent aspects
of the architecture; the practical difficulties of writing a parser for
Postfix log files are covered in detail in \sectionref{Postfix Parser
Implementation}.

\section{Parser Architecture and Design}

\label{parser design}

It should be clear from the earlier Postfix background (\sectionref{postfix
background}) that log files produced by Postfix vary widely from host to
host, depending on the set of restrictions chosen by the administrator.
With this in mind, one of the parser's design aims was to make adding new
rules as easy as possible, to enable administrators to properly parse their
own log files.  To enable this the architecture is divided into three
parts: framework, actions and rules.  Each will be discussed separately,
but first an overview:

\begin{eqlist}

    \item [Framework]  The framework is the structure that actions and
        rules plug into.  It provides the parsing loop, shared data
        storage, loading and validation of rules, storage of results, and
        other support functions.

    \item [Actions] Each action performs the work required to deal with a
        single category of inputs, e.g.\ processing data from rejections.
        Actions are invoked to deal with a log line once it has been
        identified by the rules: actions modify data structures, handle
        complications, and cause data to be saved to the database.

    \item [Rules]  The rules are responsible for classifying inputs; they
        specify the action to invoke and the regex that matches the inputs
        and extracts data.  Rules provide an easily extensible method of
        associating log lines with actions.

\end{eqlist}

For each input the framework tries each rule in turn until it finds a rule
that matches the input, then invokes the action specified by that rule.

\label{why separate rules, actions, and framework?}

Decoupling the parsing rules from their associated actions allows new rules
to be written and tested without requiring modifications to the parser
source code, significantly lowering the barrier to entry for casual users
who need to parse new inputs, e.g.\ part-time systems administrators
attempting to combat and reduce spam; it also allows companies to develop
user-extensible parsers without divulging their source code.  Decoupling
the framework, actions, and rules simplifies all three and creates a clear
separation of functionality: the framework provides services to the
actions, but does not need to perform any tasks specific to the input being
parsed; actions benefit from having services provided by the framework,
freeing them to concentrate on the task of accurately and correctly
processing the information provided by rules; rules handle the low level
details of classifying inputs and extracting data from those inputs.

Separating the rules from the actions and framework makes it possible to
parse new log lines without modifying the core parsing algorithm.  Adding a
new rule with the action to invoke and a regex to match the log lines is
trivial in comparison to understanding a program's entire parser,
identifying the correct location to change, and making the appropriate
changes.  Bear in mind that changes to the parser must be made without
adversely affecting existing parsing, particularly as there may be edge
cases that are not immediately obvious ---
\sectionref{yet-more-aborted-delivery-attempts} describes a complication
that occurs only four times in \numberOFlogFILES{} log files.  Requiring
changes to the parser's source code also complicates upgrades, as the
changes must be preserved during the upgrade, and may clash with changes
made by the developer.  This architecture allows the user to add new rules
without changing the parser, unless the new log lines require functionality
not already provided by the existing actions.  If the new log lines do
require new functionality, new actions can be added to the parser without
modifying existing actions (\sectionref{adding new actions in
implementation} describes how to safely add new actions); only in the rare
event that the new actions require support from other sections of the code
will more extensive changes be required.

There is some similarity between the parser's design and William Wood's
\acronym{ATN}~\cite{atns,nlpip}, used in Computational Linguistics for creating
grammars to parse or generate sentences.  The resemblance between the two
(shown in \tableref{Similarities between ATN and this architecture}) is
accidental, but it is obvious that the two different approaches share a
similar division of responsibilities, despite having different semantics.

% Do Not Reformat!

\begin{table}[ht]
    \caption{Similarities between ATN and this architecture}
    \empty{}\label{Similarities between ATN and this architecture}
    \begin{tabular}[]{lll}
        \tabletopline{}%
        \acronym{ATN}   & Architecture  & Similarity                  \\
        \tablemiddleline{}%
        Networks        & Overall       & Determines the sequence 
                                          of transitions              \\
                        & Algorithm     & or actions that 
                                          constitutes a valid input.  \\
        Transitions     & Actions       & Assembles data and
                                          imposes conditions          \\
                        &               & the input must meet to be
                                          accepted as                 \\
                        &               & valid.                      \\
        Abbreviations   & Rules         & Responsible for 
                                          classifying input.          \\
        \tablebottomline{}%
    \end{tabular}
\end{table}

\section{Framework}

XXX EXTEND FRAMEWORK SECTION\@: EXPAND ON ALL OF THE POINTS IN THE LIST IN
THE NEXT PARAGRAPH\@.

\label{framework in architecture}

The framework takes care of miscellaneous support functions and low level
details of parsing, freeing the programmers writing actions to concentrate
on writing productive code.  It links actions and rules, allowing either to
be improved independently of the other.  It provides shared storage to pass
data between actions, loads and saves state, loads and validates rules,
manages parsing, invokes actions, tracks how often each rule matches to
optimise rule ordering (\sectionref{rule ordering for efficiency}), stores
results of parsing, and miscellaneous other tasks.
Most parsers will require the same basic functionality from the framework,
plus some specialised support functions.  The framework is the core of the
architecture and is deliberately quite simple: the rules deal with the
variation in inputs, and the actions deal with the intricacies and
complications encountered when parsing.

The function that finds the rule matching the input and invokes the
requested action can be expressed in pseudo-code (indentation denoting flow
of control) as:

% DO NOT REFORMAT!

\begin{verbatim}
for each input:
    for each rule defined by the user: 
        if this rule matches the input:
            perform the action specified by the rule
            skip the remaining rules
            process the next input
    warn the user that the input was not parsed
\end{verbatim}

\section{Actions}

\label{actions in architecture}

XXX TO BE WRITTEN --- CHECK THE IMPLEMENTATION SECTION\@.

Each action is a separate procedure written to deal with a particular
category of input, e.g.\ rejections.  The actions are parser-specific: each
parser author will need to write the required actions from scratch unless
extending an existing parser.  It is anticipated that parsers based on this
architecture will have a high ratio of rules to actions, with the aim of
having simpler rules and clearer distinctions between the inputs parsed by
different rules.  

The ability to add special purpose actions to deal with difficulties and
new requirements that are discovered during parser development is one of
the strengths of this architecture.  Instead of writing a single monolithic
function that must be modified to support new behaviour, with all the
attendant risks of adversely affecting the existing parser, when a new
requirement arises an independent action can be written to satisfy it.
Sometimes the new action will require the cooperation of other actions,
e.g.\ to set or check a flag.  There is a possibility of introducing
failure when modifying existing actions in this way, but the modifications
will be smaller and occur less frequently than with a monolithic
architecture, thus failures will be less likely to occur and will be easier
to test for and diagnose.  The architecture can be implemented in an object
oriented style, allowing sub-classes to extend or override actions in
addition to adding new actions; because each action is an independent
procedure, the sub-class need only modify the action it is overriding,
rather than reproducing large chunks of functionality.

During development of the Postfix log parser it became apparent that in
addition to the obvious variety in log lines there were many complications
to be overcome.  Some were the result of deficiencies in Postfix's logging
(some of which were rectified by later versions of Postfix); others were
due to the vagaries of process scheduling, client behaviour, and
administrative actions.  All were successfully accommodated in the Postfix
log parser: adding new actions was enough to overcome several of the
complications; others required modifications to a single existing action to
work around the difficulties; the remainder were resolved by adapting
existing actions to cooperate and exchange extra data (via the framework),
changing their behaviour as appropriate based on that extra data.

Actions may return a modified input line that will be parsed as if read
from the input stream, allowing for a simplified version of cascaded
parsing~\cite{cascaded-parsing}.  This powerful facility allows several
rules and actions to parse a single input, potentially simplifying both
rules and actions.


\section{Rules}

\label{rules in architecture}

XXX TO BE WRITTEN\@.  TAKE CONTENT FROM THE PAPER AND POSSIBLY FROM THE
IMPLEMENTATION SECTION\@.

Rules categorise inputs, specifying both the regex to match against each
input and the action to invoke when the match is successful.  Parsing new
inputs is generally achieved by creating a new rule that pairs an existing
action with a new regex.  Decoupling the rules from the actions and
framework enables other rule management approaches to be used, e.g.\
instead of manually adding new rules, machine learning techniques could be
used to automatically generate new rules.  If this approach was taken the
choice of machine learning technique would be constrained by the size of
typical data sets (see \sectionref{Results} XXX CHECK THIS REFERENCE WHEN
THE RESULTS SECTION IS FINISHED).  Techniques requiring the full data set
when training would be impractical; Instance Based
Learning~\cite{instance-based-learning} techniques that automatically
determine which inputs from the training set are valuable and which inputs
can be discarded might reduce the data required to a manageable size.  A
parser might also dynamically create new rules in response to certain
inputs, e.g.\ diagnostic messages indicating the program which produced the
input being parsed had read a new configuration file.  These avenues of
research and development has not been pursued by the author, but could
easily be undertaken independently.

The architecture does not try to detect overlapping rules: that
responsibility is left to the author of the rules.  Unintentionally
overlapping rules lead to inconsistent parsing and data extraction because
the first matching rule wins, and the order in which rules are tried
against each input might change between parser invocations.  Overlapping
rules are frequently a requirement, allowing a more specific rule to match
some inputs and a more general rule to match the remainder, e.g.\
separating \acronym{SMTP} delivery to specific sites from \acronym{SMTP}
delivery to the rest of the world.  Allowing overlapping rules simplifies
both the general rule and the more specific rule; additionally rules from
different sources can be combined with a minimum of prior cooperation or
modification required.  Overlapping rules should have a priority attribute
to specify their relative ordering; negative priorities may be useful for
catchall rules.

Decoupling the rules from the actions allows external tools to be written
to detect overlapping rules.  Traditional regexes are equivalent in
computational power to \acronym{FA} and can be converted to \acronym{FA},
so regex overlap can be detected by finding a non-empty intersection of two
\acronym{FA}\@.  The standard equation for \acronym{FA} intersection (given
for example in~\cite{intersection-of-NFA-using-Z}) is: $FA1 \cap{} FA2 =
\overline{(\overline{FA1} \cup{} \overline{FA2})}$, which has considerable
computation complexity.  Perl 5.10 regexes are more powerful than
traditional regexes: it is possible to match correctly balanced brackets
nested to an arbitrary depth, e.g.\
\verb!/^[^<>]*(<(?:(?>[^<>]+)|(?1))*>)[^<>]*$/! matches
\verb!z<123<pq<>rs>j<r>ml>s!.  Perl 5.10 regexes can maintain an arbitrary
state stack and are thus equivalent in computational power to \acronym{PDA}
or \acronym{CFL}, so detecting overlap may require calculating the
intersection of two \acronym{PDA} or \acronym{CFL}s.  The intersection of
two \acronym{CFL}s is not closed, i.e.\ the resulting language cannot
always be parsed by a \acronym{CFL}, so intersection may be intractable in
some cases e.g.:
$a^{*}b^{n}c^{n}~\cap~a^{n}b^{n}c^{*}~\rightarrow~a^{n}b^{n}c^{n}$.
Detecting overlap amongst $n$ regexes requires calculating $n(n-1)/2$
intersections, resulting in $O(n^2x)$ complexity, where $O(x)$ is the
complexity of calculating intersection.  This is certainly not a task to be
performed every time the parser is used: detecting overlap amongst the
Postfix log parser's \numberOFrules{} rules would require calculating
\numberOFruleINTERSECTIONS{} intersections.

The framework requires each rule to have \texttt{action} and \texttt{regex}
attributes; each implementation is free to add any additional attributes it
requires.

It is possible to define pathological regexes which fall into two main
categories: regexes that match every input, and regexes that consume
excessive amounts of CPU time during matching.  Defining a regex to match
all inputs is trivial: \verb!/^/! matches the start of every input.
Usually excessive CPU time is consumed when a regex with a lot of
alteration and variable quantifiers fails to match, but successful
matching is generally quite fast (see~\cite{mastering-regular-expressions}
for in-depth discussion).

XXX RULES CAN HAVE CONDITIONS ATTACHED

\label{comparison against context-free grammars}

In context-free grammar terms the parser rules could be described as:

$\text{\textless{}log-line\textgreater{}} \mapsto \text{rule-1} |
\text{rule-2} | \text{rule-3} | \dots | \text{rule-n}$


Rule have certain characteristics that may help in understanding the
parser architecture:

XXX IMPROVE THIS SECTION\@; SHOULD IT BE A SUBSECTION\@?

\begin{itemize}

    \item Rules are annotated with the name of a Postfix program, and will
        only be used when parsing log lines produced by that
        program.

    \item The first matching rule wins: no further rules are tried against
        that log line, but there is a mechanism for prioritising the rules
        so that more specific rules can be tried first.

    \item Rules are completely self-contained and can be understood in
        isolation, without reference to any other rules.

    \item Rule processing time is a linear function of the number of rules.
        XXX IMPROVE THIS\@.

\end{itemize}

\section{Architecture Characteristics}

\label{Architecture characteristics}

\begin{description}

    \item [Matching rules against inputs is simple:]  The first matching
        rule determines the action that will be invoked: there is no
        backtracking to try alternate rules, no attempt is made to pick a
        \textit{best\/} rule.

    \item [Line oriented:]  The architecture is line oriented at present:
        there is no facility for rules to consume more input or push unused
        input back onto the input stream.  This was not a deliberate design
        decision, rather a consequence of the line oriented nature of
        Postfix log files; more flexible approaches could be pursued.

    \item [Context-free rules:]  Rules can not take into account past or
        future inputs.  In context-free grammar terms the parser rules
        could be described as:
        \newline{}$\text{\textless{}input\textgreater{}} \mapsto
        \text{rule-1} | \text{rule-2} | \text{rule-3} | \dots |
        \text{rule-n}$.

    \item [Context-aware actions:] Actions can consult the results (or lack
        of results) of previous actions during execution, providing some
        context sensitivity.  
        
    \item [Cascaded parsing:] Actions can return a modified input to be
        parsed as if read from the input stream, allowing for a simplified
        version of cascaded parsing~\cite{cascaded-parsing}.

    \item [Transduction:]  The architecture can be thought of as
        implementing transduction: it takes data in one form (log files)
        and transforms it to another form (a database); other formats may
        be more suitable for other implementations.

    \item [Similarity to \acronym{NLP}:] \hfill{} \newline{}
        Unlike traditional parsers such as those used when compiling a
        programming language, this architecture does not require a fixed
        grammar specification that inputs must adhere to.  The architecture
        is capable of dealing with interleaved inputs, out of order inputs,
        and ambiguous inputs where heuristics must be applied --- all have
        arisen and been successfully accommodated in the Postfix log
        parser.

\end{description}


\section{Conclusion}

XXX TO BE WRITTEN\@.

\glsresetall{}
\chapter{Postfix Parser Implementation}

\label{Postfix Parser Implementation}

XXX WRITE AN INTRODUCTION\@.  MOST OF THIS IS CRAP, BUT I MIGHT BE ABLE TO
RESCUE SOME OF IT\@.

The parser deals with all the complications of parsing, the eccentricities
and oddities of Postfix log files, presenting the resulting data in a
normalised, simple to use representation.  The parser's task is to follow
the journey each mail takes through Postfix, combining the data captured by
rules into a coherent whole, saving it in a useful and consistent form, and
performing housekeeping duties.

Please refer to \sectionref{parser design} for a discussion of why the
rules, actions, and framework have been separated in the parser's design.
In this section algorithm can be taken to mean the combination of framework
and actions.

The intermingling of log entries from different mails immediately rules out
the possibility of handling each mail in isolation; the parser must be able
to handle multiple mails in parallel, each potentially at a different stage
in its journey, without any interference between mails --- except in the
minority of cases where intra-mail interference is required.  The best way
to implement this is to maintain state information for every unfinished
mail and manipulate the appropriate mail correctly for each log line
encountered.

There is a similarity between this design and the event-driven programming
paradigm commonly used in GUI programs, where one part of the program
responds to events (mouse clicks in a GUI program, log lines being matched
in the parser) and invokes the correct action.


\section{Assumptions}

\parsername{} makes a small number of (hopefully safe and reasonable)
assumptions:

\begin{itemize}

    \item The log files are whole and complete: nothing has been removed,
        either deliberately or accidentally (e.g.\ log file rotation gone
        awry, file system filling up, logging system unable to cope with
        the volume of log messages).  On a well run system it is extremely
        unlikely that any of these problems will arise, though it is of
        course possible, particularly when suffering from a deluge of spam
        or a mail loop.

    \item Postfix logs enough information to make it possible to accurately
        reconstruct the actions it has taken.  There are several heuristics
        used when parsing; see \sectionref{identifying bounce
        notifications}, \sectionref{aborted delivery attempts}, and
        \sectionref{pickup logging after cleanup} for details.

    \item The Postfix queue has not been tampered with, causing unexplained
        appearance or disappearance of mail.  This may happen if the
        administrator deletes mail from the queue without using
        \daemon{postsuper}, or if there is filesystem corruption.

\end{itemize}

In some ways this task is similar to reverse engineering or replicating a
black box program based solely on its inputs and outputs.  Thus far
analysis of the log files has been sufficient to reconstruct Postfix's
behaviour, but for more difficult programs the techniques described in
\cite{black-box-error-reporting} may be useful.  There are some advantages
to treating Postfix as a black box during parser development:

\begin{itemize}

    \item Reading and understanding the source code would require a
        significant investment of time: Postfix 2.5.5 has 17MB of source
        code.

    \item The parser is developed using real world log files rather than
        the idealised log files someone would naturally envisage reading
        the source code.  The source code cannot accurately communicate the
        variety of orderings in which log lines are written to the log
        file, as process scheduling independently interferes with logging
        and other processing.

    \item The parser acts as a second source of information about Postfix's
        operation, using information gathered from empirical evidence.  A
        separate project could compare the empirical knowledge inherent in
        the parsing algorithm with Postfix's documentation and source code
        to see how closely the two agree.

\end{itemize}



\section{Parser Flow Chart}

\label{flow chart}

A picture is said to be worth a thousand words: the flow chart in
\figureref{flow chart image} shows the most common paths a connection or
mail can take through \parsername{}; the parsing complications described in
\sectionref{complications} are excluded for the sake of clarity.  The flow
chart is intended to be a graphical overview of how a mail progresses
through both Postfix and \parsername{}, providing an overall context into
which the detailed descriptions in the remainder of this chapter will fit,
in particular the actions (\sectionref{actions in implementation}) and
complications (\sectionref{complications}).

\showgraph{build/logparser-flow-chart-part-1}{Parser flow chart}{flow chart
image}

Everything starts off with a mail entering the system, whether by local
submission via \daemon{postdrop}, by \acronym{SMTP}, by re-injection due to
forwarding, or internally generated by Postfix.  Local submission is the
simplest case: a queueid is assigned immediately and the sender address is
logged \flowchart{PICKUP}{2}.  Re-injection due to forwarding sadly lacks
explicit log lines of its own; it is explained fully in
\sectionref{Re-injected mails}.  Internally generated mails lack any
explicit origin in Postfix 2.2.x and must be detected using heuristics as
described in \sectionref{identifying bounce notifications}; later versions
of Postfix do provide log lines for internally generated mails.  Bounce
notifications are the primary example of internally generated mails, though
there are other types, e.g.\ Postfix may generate mails to the
administrator when it encounters configuration errors, but such mails are
presumably rare.

\acronym{SMTP} is more complicated than the others:

\begin{enumerate}

    \item First there is a connection from the remote client
        \flowchart{CONNECT}{1}.

    \item This is followed by rejection of sender address, recipient
        addresses, client \acronym{IP} address or hostname, HELO hostname,
        etc.\ \flowchart{DELIVERY\_REJECTED}{4}; acceptance of one or more
        mails \flowchart{CLONE}{5}; or some interleaving of both.

    \item The client disconnects \flowchart{DISCONNECT}{6}.  If Postfix has
        rejected any \acronym{SMTP} commands the data will be saved to the
        database; if not there will not be any data to save (any mails
        accepted will already have been cloned so their data is in another
        data structure).

    \item If one or more mails were accepted there will be more log lines
        for those mails.

\end{enumerate}

The obvious counterpart to mail entering the system is mail leaving the
system, whether by deletion, bouncing, local delivery, or remote delivery.
All four are handled in exactly the same way:

\begin{enumerate}

    \item Postfix will log the sender and recipient addresses separately
        \flowchart{SAVE\_DATA}{9}.

    \item Sometimes mail is re-injected and the child mail needs to be
        tracked by the parent mail \flowchart{TRACK}{10}; the handling of
        re-injected mails is described in \sectionref{tracking re-injected
        mail}.

    \item Eventually the mail will be delivered, bounced, or deleted by the
        administrator \flowchart{COMMIT}{12}.  This is the last log line
        for this particular mail (though it may be indirectly referred to
        if it was re-injected).  If it is neither parent nor child of
        re-injection the data is cleaned up and entered in the database,
        then deleted from the state tables.

\end{enumerate}

It should be emphasised that the mail delivery process above happens
whether the mail is delivered to a mailbox, piped to a command, delivered
to a remote server, bounced (due to a mail loop, delivery failure, or five
day timeout), or deleted by the administrator, \textit{unless\/} the mail
is either parent or child of re-injection, as explained in
\sectionref{tracking re-injected mail}.

\section{Database}

\label{database}

The database is an integral part of the parser presented here: it stores
the rules and the data gleaned by using those rules to parse Postfix log
files.  Understanding the database schema is important in understanding the
actions of the parser, and essential to developing further applications
which utilise the data gathered; \sectionref{database as API} describes
how the database schema functions as an \acronym{API}.

The database schema can be conceptually divided in two: the rules used to
recognise log lines, and the data saved from the parsing of log files.
Each rule has a regex to recognise log lines and capture data from them,
and specifies the action to be invoked when a log line is recognised; they
also have several fields that aid the user in understanding the meaning of
the log lines recognised by each rule.  The rules are described in detail
in \sectionref{rules in implementation}, but the rules table is documented
in \sectionref{rules table} with the rest of the database schema.

The data saved from parsing the log files is also divided into two tables:
connections and results.  The connections table contains an entry for every
mail accepted and every connection where there was a rejection; the
individual fields will be described in \sectionref{connections table}.
There will be at least one entry in the results table for each entry in the
connections table; the result table's fields will be covered in detail in
\sectionref{results table}.  A diagram of the database schema is provided
in \figureref{Diagram of the Database Schema picture}, alongside the
in-depth descriptions of each table.

An important but easily overlooked benefit of storing the rules in the
database is the link between rules and results: if more information is
required when examining a result, the rule that produced the result is
available for inspection because each result references the rule that
created it.  There is no ambiguity about which rule resulted in a
particular result, eliminating one potential source of confusion.

A clear, comprehensible schema is essential when using the data extracted
from log lines; it is more important when using the data than when storing
it, because storing the data is a write-once operation, whereas utilising
the data requires frequent searching, sorting, and manipulation of the data
to produce customised reports or statistics.

\subsection{Using A Database To Provide An Application Programming Interface}

\label{database as API}

The database populated by \parsername{} provides a simple interface to
Postfix log files.  Although the interface is a database schema rather than
a set of functions in a library, it has the same benefits as any other
\acronym{API}: it provides a stable interface, allowing code on either side
of the interface to be changed without adverse effects, as long as the
interface is adhered to.  Programs that use the database can range from the
simple examples in \sectionref{motivation} to far more complex data mining
tools.

Using a database simplifies writing programs that need to interact with the
data in several ways:

\begin{enumerate}

    \item Most programming languages have facilities for database access,
        allowing a developer to write programs that use the gathered data
        in their preferred programming language, rather than being
        restricted to the language the parser is written in.

    \item Databases provide complex querying and sorting functionality for
        the user without requiring large amounts of programming.  All
        databases provide a program, of varying complexity and
        sophistication, that can be used for ad hoc queries with minimal
        investment of time.

    \item Databases are easily extensible, e.g.:

        \begin{itemize}

            \item New columns can be added to the tables used by the
                program, using DEFAULT clauses or TRIGGERS to populate
                them.

            \item A VIEW gives a custom arrangement of data with minimal
                effort.

            \item Some databases support granting access on a fine-grained
                basis, e.g.\ allowing the finance department to produce
                invoices, the helpdesk to run limited queries as part of
                dealing with support calls, and the administrators to have
                full access to the data.

            \item Triggers can be written to perform actions when certain
                events occur.  In pseudo-\acronym{SQL}\@:

\begin{verbatim}
CREATE TRIGGER ON INSERT INTO results
    WHERE sender = 'boss@example.com'
        AND action = 'DELIVERY_REJECTED'
    SEND PANIC EMAIL TO 'postmaster@example.com';
\end{verbatim}

            \item Other tables can be added to the database, to cache
                historical, summary, or computed data.

        \end{itemize}


    \item \acronym{SQL} is reasonably standard and many people will already
        be familiar with it; for those unfamiliar with it there are lots of
        readily available resources from which to learn, e.g.\
        \urlLastChecked{http://philip.greenspun.com/sql/}{2009/02/23}.
        Although every vendor implements a different dialect of
        \acronym{SQL}, the basics are the same everywhere, analogous to the
        overall similarities and minor differences amongst Irish English,
        British English, American English, and Australian English.
        Depending on the database in use there may be tools available that
        reduce or remove the requirement to know \acronym{SQL}; several are
        available for \gls{SQLite}, the default database used by
        \parsername{}.

\end{enumerate}

Storing the results in a database will also increase the efficiency of
using those results, because the log files need only be parsed once rather
than each time the data is used; indeed the results may be used by someone
with no access to the original log files.

\subsection{Rules Table}

\label{rules table}

\newlength{\belowcaptionskipORIG}
\setlength{\belowcaptionskipORIG}{\belowcaptionskip}
\setlength{\belowcaptionskip}{10pt}
\showgraph{database-schema}{Diagram of the Database Schema}{Diagram
of the Database Schema picture}
\setlength{\belowcaptionskip}{\belowcaptionskipORIG}

Rules are discussed in detail in \sectionref{rules in implementation}, but
the structure of the rules table is documented here along with the other
tables in the database.  Rules are created by the user, not the parser, and
will not be modified by the parser (except for the hits and hits\_total
fields).  Rules classify the individual log lines, capturing data to be
saved in the connections and results tables, and specifying the action to
take for that log line.

The following attributes are defined for each rule:

\begin{boldeqlist}

    \item [id] A unique identifier for each rule that other tables can use
        when referring to a specific rule.

    \item [name] A short name for the rule.

    \item [description] Something must have occurred to cause Postfix to
        log each line (e.g.\ a remote client connecting causes a connection
        line to be logged).  This field describes the event causing the log
        lines this rule matches.

    \item [restriction\_name] The restriction that caused the mail to be
        rejected.  Only applicable to rules that recognise rejection log
        lines, i.e.\ rules that have an action of
        \texttt{DELIVERY\_REJECTED}, other rules should have an empty
        string.

    \item [program] The Postfix component (e.g.\ \daemon{smtpd}) whose log
        lines the rule applies to; see \sectionref{rule conditions in
        implementation} for full details of how this attribute is used.

    \item [regex] The regex to recognise log lines with, documented in
        \sectionref{regex components}.

    \item [connection\_data] Sometimes rules need to save data that is not
        present in the log line: e.g.\ setting \texttt{client\_ip} when a
        mail is being delivered to another server.  The format is:
        \newline{} \tab{} \texttt{ client\_hostname = localhost,}
        \newline{} \tab{} \tab{} \texttt{client\_ip = 127.0.0.1} \newline{}
        i.e.\ semi-colon or comma separated assignment statements, with the
        variable name on the left and the value on the right hand side.
        Any field in the connections table can be set in this way.  Commas
        and semi-colons cannot be escaped and thus cannot be included in
        data.  This feature is intended for use with small amounts of data,
        so dealing with escape sequences was deemed unnecessary.

    \item [result\_data] The result table equivalent of
        \texttt{connection\_data}.

    \item [action] The action that will be invoked when this rule
        recognises a log line; a full list of actions and the parameters
        they are invoked with can be found in \sectionref{actions in detail
        in implementation}.

    \item [hits] This counter is maintained for every rule and incremented
        each time the rule successfully recognises a log line.  At the
        start of each run the parser sorts the rules by how frequently each
        rule recognised log lines, and at the end of the run it updates
        every rule's hits field in the database.  Assuming that the
        distribution of log lines is reasonably consistent across log
        files, ordering rules by their recognition frequency will reduce
        the parser's execution time.  Rule ordering for efficiency is
        discussed in \sectionref{rule ordering for efficiency}.

    \item [hits\_total] The total number of hits for this rule over all
        runs of the parser.

    \item [priority] This is the user-configurable companion to hits: when
        the list of rules is sorted, priority overrides hits.  This allows
        more specific rules to take precedence over more general rules
        (described in \sectionref{rules in architecture}).

\end{boldeqlist}

\subsection{Connections Table}

\label{connections table}

Every accepted mail and every connection where there was a rejection will
have a single entry in the connections table containing the following
fields:

\begin{boldeqlist}

    \item [id] This field uniquely identifies the row.

    \item [server\_ip] The \acronym{IP} address (IPv4 or IPv6) of the
        server: the local address when receiving mail, the remote address
        when sending mail.

    \item [server\_hostname] The hostname of the server, it will be
        \texttt{unknown} if the \acronym{IP} address could not be resolved
        to a hostname via DNS\@.

    \item [client\_ip] The client \acronym{IP} address (IPv4 or IPv6): the
        remote address when receiving mail, the local address when sending
        mail.

    \item [client\_hostname] The hostname of the client, it will be
        \texttt{unknown} if the \acronym{IP} address could not be resolved
        to a hostname via DNS\@.

    \item [helo] The hostname used in the HELO command.  The HELO hostname
        occasionally changes during a connection, presumably because spam
        or virus senders think it is a good idea.  By default Postfix only
        logs the HELO hostname when it rejects an \acronym{SMTP} command,
        but it is quite simple to rectify this, as described in
        \sectionref{logging helo}.

    \item [queueid] The queueid of the mail if the connection represents an
        accepted mail, or \texttt{NOQUEUE} otherwise.

    \item [start] The timestamp of the first log line, in seconds since the
        epoch.\glsadd{Epoch}

    \item [end] The timestamp of the last log line, in seconds since the
        epoch.

\end{boldeqlist}

\subsection{Results Table}

\label{results table}

Every recognised log line will have a row in the results table, associated
with a single connection; a single connection will have at least one result
associated with it, but will usually have several more, and may have
hundreds.

\begin{boldeqlist}

    \item [connection\_id] The id of the row in the connections table this
        result is associated with.

    \item [rule\_id] The id of the rule in the rules table that recognised
        the log line and created this result.

    \item [id] A unique identifier for this result.

    \item [warning] Postfix can be configured to log a warning instead of
        enforcing a restriction that would reject an \acronym{SMTP} command
        --- a mechanism that is quite useful for testing new restrictions.
        This field will be set to 1 if the log line parsed was a warning rather
        than a real rejection, or to 0 for a real rejection.  This field
        should be ignored if the result is not a rejection, i.e.\ the
        action field of the associated rule is not
        \texttt{DELIVERY\_REJECTED}.

    \item [smtp\_code] The \acronym{SMTP} code associated with the log
        line.  In general an \acronym{SMTP} code is only present in a log
        line for a rejection or final delivery; results whose log line did
        not contain an \acronym{SMTP} code will duplicate the
        \acronym{SMTP} code of other results in that connection.  Some
        final delivery log lines do not contain an \acronym{SMTP} code: in
        those cases the \acronym{SMTP} code is specified by the rule's
        \texttt{result\_data} field, based on the success or failure
        represented by the log line.

    \item [enhanced\_status\_code] The enhanced status code~\cite{RFC3463}
        is similar to the \acronym{SMTP} code, but is intended to be parsed
        by mail clients so that error messages can be clearly conveyed to
        the user.  Enhanced status code support was added to Postfix in
        version 2.3; logs generated by previous versions will not contain
        any enhanced status codes.

    \item [sender] The sender's email address.  Because multiple mails may
        be delivered during one connection, there may be different sender
        addresses in the results for one connection; however there should
        not be different sender addresses in the results for one mail.
        Mails sent over the same connection can be distinguished by their
        queueids.

    \item [recipient] The recipient address; there may be multiple
        recipient addresses per mail or connection.

    \item [size] The size of the mail; it will only be present for
        delivered mails.

    \item [delay] How long the mail was delayed while it was being
        delivered.  This will only be present for delivered mails.

    \item [delays] More detailed information about how long the mail was
        delayed while it was being delivered, again only present for
        delivered mails.

    \item [message\_id] The message-id of the accepted mail, or
        \texttt{NULL} if no mail was accepted.

    \item [data] A field available for anything not covered by other
        fields, e.g.\ the rejection message from a \acronym{DNSBL}\@.

    \item [timestamp] The time at which the log line was logged, in seconds
        since the epoch.\glsadd{Epoch}

\end{boldeqlist}



\section{Framework}

The role of the framework in this architecture was described in detail in
\sectionref{framework in architecture}; this section is concerned with the
implementation of the framework within \parsername{}.  The framework
manages the parsing process, taking care of the drudgery and boring tasks,
provides services to the actions, and implements some efficiency measures.

Each time the parser is run the framework performs some initialisation
tasks, setting up data structures that will later be used by actions,
including several state tables.  Data about the connections and mails being
processed is held in \texttt{connections} and \texttt{queueids}
respectively.  There are also state tables used when working around
complications in parsing: \texttt{timeout\_queueids} is used when dealing
with connections that time out during the DATA phase (\sectionref{timeouts
during data phase}); \texttt{bounce\_queueids} is part of the solution to
bounce notification mails being delivered before their creation is logged
(\sectionref{Bounce notification mails delivered before their creation is
logged}); and \texttt{postsuper\_deleted\_queueids} caches information
about mails recently deleted by the administrator so that subsequent log
lines processed by the SAVE\_DATA action can be discarded (\sectionref{Mail
deleted before delivery is attempted}).

When the framework loads the ruleset, it verifies each of the rules,
checking:

\begin{itemize}

    \squeezeitems{}

    \item That the specified action is registered with the framework.

    \item Ensuring the regex is valid and compiling it for efficiency
        (\sectionref{Caching compiled regexes}).

    \item For overlap between the data captured by the regex and additional
        data specified in either \texttt{connection\_data} or
        \texttt{result\_data}.

    \item That \texttt{connection\_data}, \texttt{result\_data}, and the
        regex captures specify valid fields to save data to.

\end{itemize}

If there are state tables from a previous \parsername{} run to be loaded,
they will be loaded now; obviously the framework supports saving its state
tables too.  The need to track re-injected mail (\sectionref{Re-injected
mails}) complicates the process of saving state, as the relationships
between mails must be maintained; loading state is complicated by dealing
with aborted delivery attempts (\sectionref{aborted delivery attempts}),
because a different set of relationships between connections and mails must
be re-created when the state tables are restored.

The framework will now move on to parsing each log file.  For each log line
it will use those rules whose \texttt{program} field matches the program in
the log line (see \sectionref{rule conditions in implementation}), falling
back to generic rules if necessary, and warning if the log line is
unrecognised.  The rules are sorted for increased efficiency; see
\sectionref{rule ordering for efficiency} for details.  The repetitive
nature of log files gives them high compression ratios; the framework
supports reading compressed files, simplifying the process of parsing them
because they do not have to be uncompressed to a temporary file before
parsing begins.  The framework also displays a progress bar to show how far
parsing has progressed through the log file, and to give the user an
indication of how long the remainder of the parsing process is likely to
take.  The progress bar is not as reliable when parsing compressed files
because a compressed block may uncompress to a variable number of log
lines; there is also variation between the recognition and processing time
of individual log lines, but overall the progress bar is a useful addition.

The framework collects data used when evaluating \parsernames{} efficiency
for the results chapter (\sectionref{Results}); in addition some of the
techniques used to improve parsing speed can be turned off or altered to
show the effect they have.  Three sets of data are collected:

\begin{enumerate}

    \item How long it takes to parse each log file.

    \item The number of rules tried when recognising log lines; the
        framework may need to try multiple rules for a given log line
        before it finds one that recognises it.

    \item How many log lines did the framework attempt to recognise, how
        many were skipped because there were no rules whose condition
        matched that log line's program, how many were successfully
        recognised, and how many were not recognised.

\end{enumerate}

There are five ways the framework can adapt its behaviour to demonstrate
how effective the optimisations it uses are, and how efficient
\parsername{} is:

\begin{enumerate}

    \item The rule ordering used can be changed from optimal (the default,
        most efficient) to shuffled (intended to represent an unordered
        ruleset) or reverse (reverse of optimal, least efficient).  The
        results of this are shown in \sectionref{rule ordering for
        efficiency}.

    \item The framework can record which rule recognised each log line, and
        then on a subsequent run use that information so that it uses the
        correct rule for each log line.  This gives the best possible
        running time, as only one rule is used for each log line.  A
        variation of this is to use the correct rule last, to get the worst
        possible running time.  \sectionref{perfect rule ordering}
        discusses this in detail.

    \item Each regex is compiled and the result cached when the rules are
        loaded; this can be disabled and the regex compiled every time it
        is used when recognising log lines, to demonstrate the effect this
        optimisation has on parser efficiency, as shown in
        \sectionref{Caching compiled regexes}.

    \item Normally once a log line has been recognised, the framework
        invokes the action specified by the rule.  Invoking the action can
        be skipped if desired, so the timing information shows how long is
        spent recognising log lines; this time can be subtracted from the
        time taken by a normal run to show how long is spent processing log
        lines.  This data is analysed in \sectionref{recognising vs
        processing}.

    \item The framework can skip inserting data into the database, because
        that dominates the execution time of the parser, and the evaluation
        in \sectionref{Results} is measuring the speed of \parsername{},
        not the speed of the database or the disks it resides on.

\end{enumerate}

The framework also provides several debugging options, to aid in writing or
correcting rules, or figuring out why the parser is not behaving as
expected.  In increasing order of severity they are:

\begin{enumerate}

    \item The regex from the recognising rule and the log line it
        recognised can be printed, so that the user can verify that the
        correct rule recognised each log line.  A possible variant of this
        would be to mark specific rules so only their regexes and the log
        lines they recognise would be printed.

    \item Each result can be extended with extra debugging information, so
        that when a mail's data is dumped for inspection more information
        is available.  The information currently added is: the log line's
        timestamp in human readable form; the entire log line; the name of
        the log file and the number of the log line within it.

    \item Every connection and result added to the database can be dumped
        in a human readable form.  This will result in a huge amount of
        debugging information, so it is only useful for small log file
        snippets, because otherwise the amount of information is
        overwhelming.

\end{enumerate}

\section{Actions}

\label{actions in implementation}

Actions are the component of this architecture responsible for processing
all of the inputs recognised by rules; in \parsername{} they reconstruct
the journey each mail takes through Postfix, dealing with all the
complications and difficulties that arise.  \parsername{} has
\numberOFactions{} actions and \numberOFrules{} rules, with an uneven
distribution of rules to actions as shown in \graphref{Distribution of
rules per action}.  Unsurprisingly, the action with the most associated
rules is \texttt{DELIVERY\_REJECTED}, the action that processes Postfix
rejecting a mail delivery attempt.  The next most common action is, perhaps
surprisingly, \texttt{UNINTERESTING}: this action does nothing when
invoked, allowing uninteresting log lines to be parsed without any effects
(it does not imply that the input is ungrammatical or unparsed).  Generally
rules specifying the \texttt{UNINTERESTING} action recognise log lines that
are not associated with a specific mail, e.g.\ notices about configuration
files changing; these log lines are recognised and processed so that the
framework can warn about unrecognised log lines, informing the user that
they need to add to or extend their ruleset.  Most of the remaining actions
have only one or two associated rules: some are specialised actions
required to address a deficiency in the log files, or a complication that
arises during parsing; other actions will only ever have one log line
variant, e.g.\ all log lines showing that a remote client has connected are
recognised by a single rule and handled by the \texttt{CONNECT} action.
\parsernames{} actions are described in detail in \sectionref{actions in
detail in implementation}, and \sectionref{adding new actions in
implementation} covers the process of adding a new action.

\showgraph{build/graph-action-distribution}{Distribution of rules per
action}{Distribution of rules per action}

\subsection{Actions in the \parsernamelong{}}

\label{actions in detail in implementation}

This section explains the actions found in \parsername{}; it may help to
revisit the flow chart in \sectionref{flow chart} to see how a mail passes
from one action to another as its log lines are recognised.  The words
\textit{mail\/} and \textit{connection\/} are used in the actions
descriptions below because they are less unwieldy and more specific than
\textit{state table entry\/}; a connection becomes a mail during the
\texttt{CLONE} action, which processes Postfix accepting a delivery
attempt, and the data structure moves from one state table to another
within the parser.

The complications and difficulties that arose when parsing log files
generated on a production mail server are documented in
\sectionref{complications}; some action descriptions refer to specific
difficulties they address.  The complications are documented in a separate
section to avoid overwhelming the action descriptions.

If there is enough information in the log line to identify the correct
connection or mail, each action will save all the data captured by the
recognising rule's regex.  Each action is passed the same arguments:

\begin{boldeqlist}

    \squeezeitems{}

    \item [rule] The recognising rule.

    \item [data] The data captured from the log line by the rule's regex.

    \item [line] The log line, separated into fields:

        \begin{boldeqlist}

            \squeezeitems{}

            \item [timestamp] The time the line was logged at.

            \item [program] The name of the program that logged the line.

            \item [pid] The \acronym{pid} of the program that logged the
                line.

            \item [host] The hostname of the server the line was logged on.

            \item [text] The remainder of the log line, i.e.\ the message
                logged by the program and recognised by the rule.

        \end{boldeqlist}

\end{boldeqlist}

\begin{description}

    \item [BOUNCE\_CREATED] Postfix 2.3 and subsequent versions log the
        creation of bounce messages.  This action creates a new mail if
        necessary; if the mail already exists the unknown origin flag will
        be removed.  The action also marks the mail as a bounce
        notification.  To deal with complication \sectionref{Bounce
        notification mails delivered before their creation is logged} this
        action checks a cache of recent bounce mails, to avoid creating
        bogus bounce mails when log lines are out of order.

    \item [CLEANUP\_PROCESSING] \daemon{cleanup} processes every mail that
        passes through Postfix; details of what it does are available in
        \sectionref{Postfix Daemons}.  This action saves all data captured
        by the rule's regex if the log line has not come after a timeout
        (see \sectionref{discarding cleanup log lines}); it also creates
        the mail if necessary, setting its unknown origin flag (see
        \sectionref{pickup logging after cleanup}).

    \item [CLONE] Multiple mails may be accepted on a single connection, so
        each time a mail is accepted the connection's state table entry
        must be cloned and saved in the state tables under its queueid; if
        the original data structure was used then second and subsequent
        mails would overwrite one another's data.

    \item [COMMIT] Enter the data from the mail into the database.  Entry
        will be postponed if the mail is a child waiting to be tracked
        (\sectionref{Re-injected mails}).  Once entered in the database,
        the mail will be usually be deleted from the state tables, but
        deletion will be postponed if the mail is the parent of re-injected
        mail (\sectionref{Re-injected mails}).

    \item [CONNECT] Handle a remote client connecting: create a new
        connection, indexed by \daemon{smtpd} \acronym{pid}.  If a
        connection already exists it is treated as a symptom of a bug in
        \parsername{}, and the action will issue a warning containing the
        full contents of the existing connection plus the log line that has
        just been parsed.

    \item [DELETE] Deals with mail deleted using Postfix's administrative
        command \daemon{postsuper}.  This action adds a dummy recipient
        address if required, then invokes the COMMIT action to save the
        mail to the database.  The complication this action deals with is
        described fully in \sectionref{Mail deleted before delivery is
        attempted}.

    \item [DELIVERY\_REJECTED] Deals with Postfix rejecting an
        \acronym{SMTP} command from the remote client.  This is the action
        most frequently specified by rules, because so many different
        restrictions are used to reject delivery attempts.  This action is
        quite simple: if the log line contains a queueid, save the data
        captured by the rule's regex to the mail identified by that
        queueid; otherwise save it to the connection identified by the
        \acronym{pid} in the log line.  No further processing of the log
        line is required.

    \item [DISCONNECT] Invoked when the remote client disconnects, it
        enters the connection in the database, performs any required
        cleanup, and deletes the connection from the state tables.  This
        action deals with aborted delivery attempts (\sectionref{aborted
        delivery attempts}).

    \item [EXPIRY] If Postfix has not managed to deliver a mail after
        trying for five days it will give up and return the mail to the
        sender.  When this happens the mail will not have a combination of
        Postfix programs which passes the valid combinations check as
        described in \sectionref{out of order log lines}; this action, with
        cooperation from the COMMIT and SAVE\_DATA actions, deals with that
        complication.

    \item [MAIL\_BOUNCED] This action behaves exactly like the
        \texttt{SAVE\_DATA} action; it saves all data captured by the
        recognising rule's regex, and does nothing more.  It is a separate
        action to help distinguish delivery attempts that bounce from other
        delivery attempts.

    \item [MAIL\_DISCARDED] Sometimes mail is discarded, e.g.\ mail
        submitted locally that is larger than the limit configured by the
        administrator.  This action is used for those mails; it invokes the
        \texttt{COMMIT} action, but is a separate action to simplify
        further analysis.

    \item [MAIL\_QUEUED] This action represents Postfix picking a mail from
        the queue to deliver.  This action needs to deal with out of order
        log lines when mail is re-injected for forwarding; see
        \sectionref{Re-injected mails} for details.

    \item [MAIL\_SENT] This action behaves exactly like the
        \texttt{SAVE\_DATA} action; it saves all data captured by the
        recognising rule's regex, and does nothing more.  It is a separate
        action to help distinguish successful delivery attempts from other
        delivery attempts.

    \item [MAIL\_TOO\_LARGE] When a client tries to send a mail larger than
        the local server accepts it will be discarded and the client
        informed of the problem.  The mail may have been accepted and
        partially transferred, in which case there's a data structure to
        dispose of, otherwise there will not be.  See \sectionref{timeouts
        during data phase} for the gory details; although that describes
        timeouts the processing is the same for mails that are too large.

    \item [PICKUP] The PICKUP action corresponds to the \daemon{pickup}
        service dealing with a locally submitted mail.  A new mail will be
        created, although out of order log entries may have caused the
        state table entry to already exist, as documented in
        \sectionref{pickup logging after cleanup}.

    \item [POSTFIX\_RELOAD] When Postfix stops or reloads its configuration
        it kills all \daemon{smtpd} processes, requiring any active
        connections to be cleaned up, entered in the database, and deleted
        from the state tables.  Postfix probably kills all the other
        components too, but \parsername{} is only affected by \daemon{smtpd}
        processes exiting.

    \item [SAVE\_DATA] Every action that can locate the correct entry in
        the state tables saves any data captured by the recognising rule's
        regex to it.  The \texttt{SAVE\_DATA} saves data in this way but
        does not do anything else; it is invoked for log lines that contain
        useful data but do not require any further processing.

    \item [SMTPD\_DIED] Sometimes a \daemon{smtpd} dies or exits
        unsuccessfully; the active connection for that \daemon{smtpd} must
        be cleaned up, entered in the database if it has sufficient data,
        and deleted from the state tables.  In some cases the connection
        will not have enough data to satisfy the database schema, so it
        cannot be entered into the database; unfortunately this means that
        some data is lost.

    \item [SMTPD\_KILLED] Sometimes a \daemon{smtpd} is killed by a signal;
        this is processed in the same way as a \daemon{smtpd} dying, see
        the \texttt{SMTPD\_DIED} action for details.

    \item [SMTPD\_WATCHDOG] \daemon{smtpd} processes have a watchdog timer
        to deal with unusual situations; after five hours the timer will
        expire and the \daemon{smtpd} will exit.  This occurs very
        infrequently, as there are many other timeouts that should occur in
        the intervening hours, e.g.\ timeouts for DNS requests or timeouts
        reading data from the client.  The active connection for that
        \daemon{smtpd} must be cleaned up, entered in the database, and
        deleted from the state tables.

    \item [TIMEOUT] The connection with the remote client timed out so the
        mail currently being transferred must be discarded.  The mail may
        have been accepted, in which case there's a data structure to
        dispose of, or it may not in which case there is not.  See
        \sectionref{timeouts during data phase} for full details.

    \item [TRACK] Track a mail when it is re-injected for forwarding to
        another mail server; this happens when a local address is aliased
        to a remote address (see \sectionref{tracking re-injected mail} for
        a full explanation).  \texttt{TRACK} will be called when dealing
        with the parent mail, and will create the child mail if necessary.
        \texttt{TRACK} checks if the child has already been tracked, either
        by this parent or by another parent, and issues appropriate
        warnings in either case.

    \item [UNINTERESTING] This action just returns successfully: it is used
        when a log line needs to be parsed for completeness, but does not
        either provide any useful data to be saved or require anything to
        be done.

\end{description}

\subsection{Adding New Actions}

\label{adding new actions in implementation}

Adding new actions is not as easy as adding new rules, though care has been
taken in the architecture and implementation to make adding new actions as
painless as possible.  The implementer writes a subroutine that accepts the
standard arguments given to actions, and registers it with the framework.
The new action must be registered before the rules are loaded, because it
is an error for a rule to specify an unregistered action; this helps catch
mistakes made when adding new rules.  No other work is required from the
implementer to integrate the action into the parser; all of their attention
and effort can be focused on correctly implementing their new action.  The
action may need to extend the list of valid combinations described in
\sectionref{out of order log lines} if the new action creates a different
set of acceptable programs, but this would only be necessary if the new
action processes log lines from Postfix components \parsername{} currently
does not have rules for.

\section{Rules}

\label{rules in implementation}

The rules are responsible for classifying each log line and specifying the
correct action to be invoked.  For any parser implemented using this
architecture, the rules are the most visible component and most likely to
be modified by users.  Rules need to be as simple as possible so that users
can easily modify or add to them, but the implementation must balance that
simplicity with the need to provide enough flexibility and power to
successfully parse the inputs.

The role of the rules in the architecture is covered in detail in
\sectionref{rules in architecture}; this section is concerned with the
practical aspects of how rules are implemented and used in \parsername{}.
The structure of the rules table has already been documented in
\sectionref{rules table}; that abstract description will not be repeated
here, but should be fleshed out by the concrete example rule in
\sectionref{example rule in implementation}.  The process of creating new
rules from unparsed log lines is dealt with in \sectionref{creating new
rules in implementation}; the implementation of the utility supplied with
\parsername{} to create regexes from unparsed log lines is also described
in that section.  The rule conditions used in \parsername{} are explained
next, and the final portion of this section provides some
suggestions for how to detect overlapping rules.

\subsection{Example Rule}

\label{example rule in implementation}

The example rule in \tableref{Example rule in implementation table}
recognises the log line resulting from Postfix rejecting a delivery attempt
because the domain in the sender address does not have an A, AAAA,
or MX DNS entry, i.e.\ mail could not be delivered to the sender's address
(for full details see~\cite{reject-unknown-sender-domain}).

Example log line recognised by this rule:

% RFC 3330 says that 192.0.2.0/24 is reserved for example use.

\begin{verbatim}
NOQUEUE: reject: RCPT from smtp.example.com[192.0.2.1]:
  550 5.1.8 <foo@example.com>:
  Sender address rejected: Domain not found;
  from=<foo@example.com> to=<info@example.net>
  proto=SMTP helo=<smtp.example.com>
\end{verbatim}

% do not reformat this!
\begin{table}[ht]

    \caption{Example rule}
    \empty{}\label{Example rule in implementation table}
    \begin{tabular}{ll}

    \textbf{Field}      & \textbf{Value}                                    \\
    name                & Unknown sender domain                             \\
    description         & We do not accept mail from unknown domains        \\
    restriction\_name   & reject\_unknown\_sender\_domain                   \\
    program             & \daemon{smtpd}                                    \\
    regex               & \verb!^__RESTRICTION_START__ <(__SENDER__)>: !    \\
                        & \verb!Sender address rejected: Domain not found;! \\
                        & \verb!from=<\k<sender>> to=<(__RECIPIENT__)> !    \\
                        & \verb!proto=E?SMTP helo=<(__HELO__)>$!            \\
    result\_data        &                                                   \\
    connection\_data    &                                                   \\
    action              & DELIVERY\_REJECTED                                \\
    hits                & 0                                                 \\
    hits\_total         & 0                                                 \\
    priority            & 0                                                 \\
    %cluster\_group      & 400                                               \\

    \end{tabular}

\end{table}

The fields in \tableref{Example rule in implementation table} are used as
follows:

\begin{description}

    \item [name, description, and restriction\_name:] are not used by the
        parser, they serve to document the rule for the user's benefit.

    \item [program and regex:] The program is used to restrict the log
        lines this rule will be used to recognise; see \sectionref{rule
        conditions in implementation} for details.  The regex does the
        actual recognition of the log lines, and data captured by the regex
        (sender, recipient, etc.) will be automatically saved to the
        results and connections tables.

    \item [action:] The action to be invoked when the rule recognises a log
        line.  See \sectionref{actions in detail in implementation} for
        details of the actions implemented by \parsername{}, and
        \sectionref{actions in architecture} for the role of actions in the
        parser architecture.

    \item [result\_data and connection\_data:] are used to provide data not
        present in the log line, and are unused in this rule.

    \item [hits, hits\_total, and priority:] hits and priority are used
        when ordering the rules for more efficient parsing (see
        \sectionref{rule ordering for efficiency}).  At the end of each
        parsing run hits is set to the number of log lines recognised by
        the rule.  Hits\_total is the sum of hits over every parsing run,
        but is otherwise unused by the parser.

\end{description}

\subsection{Creating New Rules}

\label{creating new rules in implementation}

The log files produced by Postfix differ from installation to installation,
because administrators have the freedom to choose the subset of available
restrictions which suits their needs, including using different
\acronym{DNSBL} services, policy servers, or custom rejection messages.  To
facilitate parsing new log lines, the architecture separates rules from
actions: adding new actions can be difficult, but adding new rules to
recognise new log lines is trivial, and occurs much more frequently.

To add a new rule a new row must be added to the rules table in the
database, containing all the required attributes: action, name,
description, program, and regex.  The name and description fields should be
set based on the meaning of the log line, to help others understand which
log lines this rule will recognise; the program will be obvious from the
unrecognised log lines.  The action depends on the what the log line
represents, e.g.\ a delivery rejection, a mail being delivered, some useful
information, or something else; examine the list of actions in
\sectionref{actions in detail in implementation} to determine the correct
one.  The regex will be constructed based on the log line, but see below
for a tool to ease the process.

There are other attributes that may be required in a rule:
connection\_data, result\_data, priority, or restriction\_name.  In general
it will only become clear that connection\_data or result\_data are
required when \parsername{} warns about an entry in the connections or
results tables that is missing some required fields, because values for the
missing fields are not present in any of the log lines.  E.g.\ the rule
recognising the \daemon{pickup} component processing a mail sets the
client\_hostname to \texttt{localhost} and the client\_ip to
\texttt{127.0.0.1}, because the mail originates on the local server.  If
the new rule deliberately uses the architecture's overlapping rules feature
it will be obvious that the priority field needs to be set, on this rule and
possibly others; the priority field may be needed on unintentionally
overlapping rules too, but that is more difficult to determine.  Finally
the restriction\_name field should be set when the rule's action is
\texttt{DELIVERY\_REJECTED}; hopefully the restriction in question will be
clear from the content of the log line.

A program is provided with \parsername{} to ease the process of creating
new rules from unrecognised log lines, based on the algorithm developed by
Risto Vaarandi for his \acronym{SLCT}~\cite{slct-paper}.  The differences
between the two algorithms will be outlined as part of the general
explanation below.  The core of the \acronym{SLCT} algorithm is quite
simple: log lines are generally created by substituting variable words into
a fixed pattern, and analysis of the frequency with which each word occurs
can be used to determine whether the word is variable or part of the fixed
pattern.  This classification can be used to group similar log lines and
generate a regex to match each group of log lines.  There are 4 steps in
the algorithm:

\begin{description}

    \item [Pre-process the file.]  The new algorithm leverages the
        knowledge gained when writing rules and developing \parsername{};
        it performs a large number of substitutions on the input log lines,
        replacing commonly occurring variable terms (e.g.\ email addresses,
        \acronym{IP} addresses, the standard start of rejection messages)
        with keywords that \parsername{} will expand when it loads the new
        rule.  The purpose of this step is to utilise existing knowledge to
        create more accurate regexes; it replaces a large number of
        variable words with fixed words, improving the subsequent
        classification of words as fixed or variable.  The altered log
        lines are written to a temporary file, which subsequent stages will
        use instead of the original input file.

        In the original algorithm the purpose of the pre-processing stage
        was to reduce the memory consumption of the program.  In the first
        pass it generates a hash~\cite{hash-functions}\glsadd{hash} (from a
        small range of values) for each word of each log line, incrementing
        a counter for each hash.  The counters will be used in the next
        pass to filter out words: if the word's hash does not have a high
        frequency, the word itself cannot have a high frequency, so there
        is no need to maintain a counter for it, reducing the number of
        counters required and thus the program's memory consumption.

    \item [Calculate word frequencies.]  The position of words within a log
        line is important: a word occurring in two log lines does not
        indicate similarity unless it occupies the same position in both
        log lines.  If a variable term within a line contains
        spaces, it will appear to the algorithm as more than one word.
        This will alter the position of subsequent words, so a word
        occurring in different positions in two log lines \textit{may\/}
        indicate similarity, but the algorithm does not attempt to deal
        with this possibility.  The algorithm maintains a counter for each
        \texttt{(word, word's position within the log line)} tuple,
        incrementing it each time that word occurs in that position.

        The original algorithm hashes each word and checks that result's
        counter from the previous pass; if that counter has a high
        frequency a separate counter will be maintained for this word in
        this position.  This reduces the number of counters maintained by
        the algorithm in this step, reducing the memory requirements of the
        algorithm at a cost of increased CPU usage.

        As time goes on the amount of memory typically available increases
        and the need to reduce memory requirements decreases, so the
        modified algorithm omits the hashing step and maintains counters
        for all tokens.  In addition most of the infrequently occurring
        words will have been substituted with keywords during the first
        step, vastly reducing the number of tuples to maintain counters
        for.

    \item [Classify words based on their frequency.]  The frequency of
        every tuple \texttt{(word, word's position within the log line)} is
        checked: if its frequency is greater than the threshold supplied by
        the user (1\% of all log lines is generally a good starting point),
        it is classified as a fixed word, otherwise it is classified as a
        variable term.  If a variable term appears sufficiently often it
        will be misclassified as a fixed term, but that should be noticed
        by the user when reviewing the new regexes.  Variable terms are
        replaced by \texttt{.+} to match one or more of any character.

    \item [Build regexes.]  The words are reassembled to produce a regex
        matching the log line, and a counter is maintained for each regex.
        Contiguous sequences of \texttt{.+} in the newly reassembled
        regexes are collapsed to a single \texttt{.+}; any resulting
        duplicate regexes are combined, and their counters added together.
        If a regex's counter is lower than the threshold supplied by the
        user the regex is discarded; this second threshold is independent
        of the threshold used to differentiate between fixed and variable
        words, but once again 1\% of log lines is a good starting point.
        The new regexes are displayed for the user to add to the database,
        either as new rules or merged into the regexes of existing rules;
        the counter for each regex is also displayed, giving the user an
        indication of how many of the log lines that regex should match.
        Discarding regexes will result in some of the log lines not being
        matched; when the rules have been augmented with the new regexes,
        the original log files should be parsed again, and any remaining
        unparsed log lines used as input to this utility.

\end{description}

XXX MERGE logs2regexes WITH check-regexes.

A second utility is also provided that reads a list of new regexes and
the input given to the first utility.  It tries to match each input log
line against each regex, counting the number of log lines that match
each regex, warning the user if an input log line is matched by more
than one regex, and additionally warning if an input log line is not
matched by any regex.  It displays a summary of how many input log lines
each regex matched, comparing it to the expected number of matches; this
provides the user with an easy method of checking if the regexes
produced by the first utility are correctly matching the input log lines
they are based upon.

These utilities are not intended to create perfect regexes, but they
greatly reduce the effort required to parse new or different log lines.

\subsection{Rule Conditions}

\label{rule conditions in implementation}

Rule conditions as part of the architecture have already been documented in
\sectionref{rule conditions in architecture}, and they can be very complex
and difficult to evaluate.  In contrast, \parsername{} uses very simple
rule conditions: each rule has a \texttt{program} attribute that specifies
the Postfix component whose log lines it recognises, and each log line
contains the name of the Postfix component that produced it; the framework
will only use rules where the two are equal when trying to recognise log
lines.  This avoids needlessly trying rules that will not match the log
line, or worse, might match unintentionally.  In addition the framework
supports generic rules, whose program attribute is ``*''; these will be
used if none of the program-specific rules recognise the log line.  If none
of the rules are successful the framework will warn the user.

\subsection{Regex Components}

\label{regex components}

Each rule's regex will have keywords expanded when the rules are loaded,
for several reasons:

\begin{itemize}

    \item It simplifies both reading and writing of regexes and makes each
        regex largely self-documenting.  E.g.\ the meaning of
        \_\_CLIENT\_HOSTNAME\_\_ is immediately clear, whereas
        \verb!(?:unknown|(?:[-.\w]+))! needs to be deciphered.

    \item It avoids needless repetition of complex regex components, and
        allows the components to be corrected or improved in one
        location.  E.g.\ \_\_SENDER\_\_ is used in 68 rules; if a mistake
        is discovered it can be corrected in one place only.

    \item It enables automatic extraction and saving of captured data.  The
        regex snippets use Perl 5.10's named capture buffers~\cite{perlre},
        so the mapping between captures and fields does not need to be
        explicitly specified by the rule.

\end{itemize}

For efficiency the keywords are expanded and every rule's regex is compiled
before attempting to parse the log file, otherwise every regex would be
recompiled each time it was used, resulting in a large, data dependent
slowdown, as shown in \sectionref{Caching compiled regexes}.  Most of the
keywords are named after the fields in the connections or results tables
they populate: \_\_CLIENT\_HOSTNAME\_\_, \_\_CLIENT\_IP\_\_, \_\_DELAY\_\_,
\_\_DELAYS\_\_, \_\_ENHANCED\_STATUS\_CODE\_\_, \_\_HELO\_\_,
\_\_MESSAGE\_ID\_\_, \_\_QUEUEID\_\_, \_\_RECIPIENT\_\_, \_\_SENDER\_\_,
\_\_SERVER\_HOSTNAME\_\_, \_\_SERVER\_IP\_\_, \_\_SIZE\_\_, and
\_\_SMTP\_CODE\_\_.

Some of the keywords need more explanation:

\begin{eqlist}

    \squeezeitems{}

    \item [\_\_CHILD\_\_]  The queueid of a child mail; see
        \sectionref{Re-injected mails}.

    \item [\_\_COMMAND\_\_]  All \acronym{SMTP} commands, except
        \texttt{DATA}, which typically has a more specific rule.
        Priorities could have been used instead of excluding \texttt{DATA}.

    \item [\_\_CONN\_USE\_\_]  How many times the connection was reused;
        Postfix tries to reuse connections whenever possible to reduce the
        load on both the sending and receiving servers.

    \item [\_\_DATA\_\_]  This snippet is special: it matches a zero-length
        string, so it needs to be followed by a pattern that matches
        appropriately, but it captures whatever is matched so that it can
        be saved.  E.g.\ \verb!/(__DATA__connection (?:refused|reset!
        \newline{} \tab{}\verb!by peer))/! matches either ``connection
        refused'' or ``connection reset by peer'' and causes it to be saved
        to the data field of that log line's result.

    \item [\_\_PID\_\_]  The \acronym{pid} of a \daemon{smtpd} process that
        dies or is killed; see the SMTPD\_DIED action in
        \sectionref{actions in detail in implementation}.

    \item [\_\_RESTRICTION\_START\_\_]  Matches the start of a delivery
        rejected log line: XXX SHOULD I INCLUDE THIS\@? \newline{}
        \verb!/(__QUEUEID__): reject(?:_warning)?: ! \newline{}
        \verb!(?:RCPT|DATA) from ! \newline{}
        \verb!(?>(__CLIENT_HOSTNAME__)\\[)! \newline{}
        \verb!(?>(__CLIENT_IP__)\\]): (__SMTP_CODE__)! \newline{}
        \verb!(?: __ENHANCED_STATUS_CODE__)?/!

    \item [\_\_SHORT\_CMD\_\_]  Postfix sometimes logs \acronym{SMTP}
        commands in a short, single word form.

\end{eqlist}

\subsection{Overlapping Rules}

\label{overlapping rules in implementation}

\parsername{} does not try to detect overlapping rules; that responsibility
is left to the author of the rules.  The advantages and difficulties of
overlapping rules have already been addressed in \sectionref{overlapping
rules in architecture} and will not be repeated here.  \parsername{}
provides a mechanism for ordering overlapping rules: the priority field in
each rule (defaulting to zero); negative priorities may be useful for
catchall rules.

Detecting overlapping rules is difficult, but the following approaches may
be helpful:

\begin{itemize}

    \item Sort by program and regex, then visually inspect the list, e.g.\
        with \acronym{SQL} similar to:      \newline{}
        \verb!SELECT program, regex!        \newline{}
        \verb!    FROM rules!               \newline{}
        \verb!    ORDER BY program, regex;! \newline{}
        The rules are sorted first by program, then by regex, because rules
        with overlapping regexes will not overlap if their programs are
        different.  Note that this does not properly deal with rules whose
        program is \texttt{*}; those rules will be used on all log lines
        that have not been recognised by program-specific rules.

    \item Compare the results of parsing using different rule orderings
        (described in \sectionref{rule ordering for efficiency}).  Parse
        several log files using optimal ordering, then dump a textual
        representation of the rules, connections, and results tables.
        Repeat with shuffled and reversed ordering, starting with a fresh
        database.  If there are no overlapping rules the tables from each
        run will be identical; differences indicate overlapping rules.  The
        rules that overlap can be determined by examining the differences
        in the tables: each result contains a reference to the rule which
        created it, which will change if that rule overlaps with another.
        Unfortunately this method cannot prove the absence of overlapping
        rules; it can detect overlapping rules, but only if there are log
        lines in the log files that are recognised by more than one rule.

\end{itemize}

\section{Complications Encountered}

\label{complications}

XXX IMPROVE THE INTRO\@: ``needs more of a lead-in''.

The complications described in this section are listed in the order in
which they were encountered during development of the parser.  Each of
these complications caused the parser to operate incorrectly, generating
either warning messages or leaving mails in the state table.  The frequency
of occurrence is much higher at the start of the list, with the first
complication occurring several orders of magnitude more frequently than the
last.  When deciding which problem to address next, the most common was
always chosen, as resolving the most common problem would yield the biggest
improvement in the parser, prune the greatest number of mails from the
state tables and error messages, and make the remaining problems more
apparent.  The first three complications were encountered early in the
parser's implementation and guided its design and development.

\subsection{Queueid Vs Pid}

The mail lacks a queueid until it has been accepted, so log lines must
first be correlated by the \daemon{smtpd} \acronym{pid}, then transition to
being correlated by the queueid.  This is relatively minor, but does
require:

\begin{itemize}

    \item Two versions of several functions: \texttt{by\_pid} and
        \texttt{by\_queueid}.

    \item Two state tables to hold the data structure for each connection
        and mail.

    \item Most importantly: every section of code must know whether it
        needs to lookup the data structures by \acronym{pid} or queueid.

\end{itemize}

\subsection{Connection Reuse}

\label{connection reuse}

Multiple independent mails may be delivered across one connection: this
requires the algorithm to clone the current data as soon as a mail is
accepted, so that subsequent mails will not trample over each other's data.
This must be done every time a mail is accepted, as it is impossible to
tell in advance which connections will accept multiple mails.  Once the
mail has been accepted its log lines will not be correlated by
\acronym{pid} any more, its queueid will be used instead (except when
timeouts occur during the data phase \sectionref{timeouts during data
phase}).  If the original connection has any useful data (e.g.\ rejections)
it will be saved to the database when the client disconnects.  One unsolved
difficulty is distinguishing between different groups of rejections, e.g.\
when dealing with the following sequence:

\begin{enumerate}

    \item The client attempts to deliver a mail, but it is rejected.

    \item The client issues the RSET command to reset the \acronym{SMTP}
        session.

    \item The client attempts to deliver another mail, likewise rejected.

\end{enumerate}

There should ideally be two separate entries in the database resulting from
the above sequence, but currently there will only be one.



\subsection{Re-injected Mails}

\label{Re-injected mails}

\label{tracking re-injected mail}

The most difficult complication initially encountered is that locally
addressed mails are not always delivered directly to a mailbox: sometimes
they are addressed to and accepted for a local address but need to be
delivered to one or more remote addresses due to aliases.  When this
occurs a child mail will be injected into the Postfix queue, but without
the explicit logging that \daemon{smtpd} or \daemon{postdrop} injected
mails have.  Thus the source of the mail is not immediately discernible
from the log line in which the mail first appears; from a strictly
chronological reading of the log files it usually appears as if the child
mail has appeared from thin air.  Subsequently the parent mail will log the
creation of the child mail, e.g.\ parent mail 3FF7C4317 creates child mail
56F5B43FD\@:

\texttt{3FF7C4317: to=<username@example.com>, relay=local, \newline{}
\tab{} delay=0, status=sent (forwarded as 56F5B43FD)}

Unfortunately, while all log lines from an individual process appear in
chronological order, the order in which log lines from different processes
are interleaved is subject to the vagaries of process scheduling.  In
addition, the first log line belonging to the child mail (the log line
cited above belongs to the parent mail) is logged by \daemon{qmgr}, so the
order also depends on how soon \daemon{qmgr} processes the new mail.

Because of the uncertain order the parser cannot complain when it
encounters a log line from \daemon{qmgr} for a previously unseen mail;
instead it must flag the mail as coming from an unknown origin, and
subsequently clear the flag if and when the origin of the mail becomes
clear.  Obviously the parser could omit checking where mails originate
from, but requiring an explicit source helps to expose bugs in the parser;
such checks helped to identify the complications described in
\sectionref{discarding cleanup log lines} and \sectionref{pickup logging
after cleanup}.

Process scheduling can have a still more confusing effect: quite often the
child mail will be created, delivered, and entirely finished with
\textbf{before} the parent logs the creation log line!  Thus, mails flagged
as coming from an unknown origin cannot be entered into the database when
their final log line is parsed; instead they must be marked as ready for
entry and subsequently entered once the parent mail has been identified.

XXX MERGE THE REMAINDER OF THIS SECTION INTO THE PRIOR MATERIAL\@.

The crux of this complication is that re-injected mails appear in the log
files without explicit logging indicating their source.  There are two
implicit indications:

\begin{enumerate}

    \item The indicator which more commonly introduces re-injection is when
        \daemon{qmgr} selects a mail with a previously unseen queueid for
        delivery, in which case a new data structure will be created for
        that mail.  The mail will be flagged as having unknown origins;
        this flag will subsequently be cleared once the origin has been
        established.  This may also be an indicator that the mail is a
        bounce notification, see \sectionref{identifying bounce
        notifications} for details.

    \item Local delivery re-injects the mail and logs a relayed delivery
        rather than delivering directly to a mailbox or program as it
        usually would.\footnote{Relayed delivery is performed by the
        \acronym{SMTP} client; local delivery means local to the server,
        i.e.\ an address the server is final destination for.}  In this
        case the mail may already have been created (described above) and
        the unknown origin flag will be cleared; if not a new data
        structure will be created.  In both cases the re-injected mail is
        marked as a child of the original mail.  The log line in question
        is:

        \texttt{3FF7C4317: to=<username@example.com>, relay=local,
        \newline{} \tab{} delay=0, status=sent (forwarded as 56F5B43FD)}

        This always occurs for re-injected mail but typically occurs after
        the first indicator.  This log line is required to tie the parent
        and child mails together and so is central to the process of
        tracking re-injected mails.

\end{enumerate}

The algorithm for tracking and saving re-injected mail to the database can
finally be described:

\begin{itemize}

    \item If the mail is of unknown origin it is assumed to be a child mail
        whose parent has not yet been identified.  Mark the mail as ready
        for entry in the database and wait for the parent to deal with it.
        The mail should not have any subsequent log lines; only its parent
        should refer to it.

    \item If the mail is a child mail then it has already been tracked: as
        with all other mail, the data is cleaned up, the child is entered
        in the database, and then deleted from the state tables.  The child
        mail will be removed from the parent mail's list of children; if
        this is the last child and the parent has already been entered in
        the database, the parent will also be deleted from the state
        tables.

    \item The last alternative is that the mail is a parent mail.
        Regardless of the state of its children its data is cleaned up and
        entered in the database.  The parent may have children that are
        waiting to be entered in the database; the data for each of those
        children is cleaned up and entered in the database, then deleted
        from the state tables.  The parent may also have outstanding
        children which are not yet delivered, in which case the parent must
        be retained in the state tables until those children are finished
        with.  As soon as the last child is deleted from the state tables
        the parent will also be deleted from the state tables.

\end{itemize}

A parent mail can have multiple children, which may be delivered before or
after the parent mail.


\subsection{Identifying Bounce Notifications}

\label{identifying bounce notifications}

Postfix 2.2.x (and presumably previous versions) does not generate a log
line when it generates a bounce notification; suddenly there will be log
entries for a mail that lacks an obvious source.  There are similarities to
the problem of identifying re-injected mails discussed in
\sectionref{tracking re-injected mail}, but unlike the solution described
therein bounce notifications do not eventually have a log line that
identifies their source.  Heuristics must be used to identify bounce
notifications, and those heuristics are:

\begin{enumerate}

    \item The sender address is \verb!<>!.\glsadd{<>}

    \item Neither \daemon{smtpd} nor \daemon{pickup} have logged any
        messages associated with the mail, indicating it was generated
        internally by Postfix, not accepted via \acronym{SMTP} or submitted
        locally by \daemon{postdrop}.

    \item The message-id has a specific format: \newline{}
        \tab{} \texttt{YYYYMMDDhhmmss.queueid@server.hostname} \newline{}
        e.g.\ \texttt{20070321125732.D168138A1@smtp.example.com}

    \item The queueid in the message-id must be the same as the queueid of
        the mail: this is what distinguishes bounce notifications generated
        locally from bounce notifications which are being re-injected as a
        result of aliasing.  In the latter case the message-id will be
        unchanged from the original bounce notification, and so even if it
        happens to be in the correct format (e.g.\ if it was generated by
        Postfix on this or another server) it will not match the queueid of
        the mail.

\end{enumerate}

Once a mail has been identified as a bounce notification, the unknown
origin flag is cleared and the mail can be entered in the database.

There is a small chance that a mail will be incorrectly identified as a
bounce notification, as the heuristics used may be too broad.  For this to
occur the following conditions would have to be met:

\begin{enumerate}

    \item The mail must have been generated internally by Postfix.

    \item The sender address must be \verb!<>!.\glsadd{<>}

    \item The message-id must have the correct format and match the queueid
        of the mail.  While a mail sent from elsewhere could easily have
        the correct message-id format, the chance that the queueid in the
        message-id would match the queueid of the mail is extremely small.

\end{enumerate}

If a mail is misclassified as a bounce message it will almost certainly
have been generated internally by Postfix; arguably misclassification in
this case is a benefit rather than a drawback, as other mails generated
internally by Postfix will be handled correctly.  Postfix 2.3 (and
hopefully subsequent versions) log the creation of a bounce message.

This check is performed during the COMMIT action.

\subsection{Aborted Delivery Attempts}

\label{aborted delivery attempts}

Some mail clients behave unexpectedly during the \acronym{SMTP} dialogue:
the client aborts the first delivery attempt after the first recipient is
accepted, then makes a second delivery attempt for the same recipient which
it continues with until delivery is complete.  Microsoft Outlook is one
client that behaves in this fashion; other clients may act in a similar
way.  An example dialogue exhibiting this behaviour is presented below
(lines starting with a three digit number are sent by the server, the other
lines are sent by the client):

\begin{verbatim}
220 smtp.example.com ESMTP
EHLO client.example.com
250-smtp.example.com
250-PIPELINING
250-SIZE 15240000
250-ENHANCEDSTATUSCODES
250-8BITMIME
250 DSN
MAIL FROM: <sender@example.com>
250 2.1.0 Ok
RCPT TO: <recipient@example.net>
250 2.1.5 Ok
RSET
250 2.0.0 Ok
RSET
250 2.0.0 Ok
MAIL FROM: <sender@example.com>
250 2.1.0 Ok
RCPT TO: <recipient@example.net>
250 2.1.5 Ok
DATA
354 End data with <CR><LF>.<CR><LF>
The mail transfer is not shown.
250 2.0.0 Ok: queued as 880223FA69
QUIT
221 2.0.0 Bye
\end{verbatim}

Once again Postfix does not log a message making the client's behaviour
clear, so once again heuristics are required to identify when this
behaviour occurs.  In this case a list of all mails accepted during a
connection is saved in the connection state, and the accepted mails are
examined when the disconnection action is executed.  Each mail is checked
for the following:

\begin{itemize}

    \item Was the second result processed by the \texttt{CLONE} action?
        The first two \daemon{smtpd} log lines will be a connection log
        line and a mail acceptance log line; Postfix logs acceptance as
        soon as the first recipient has been accepted.

    \item Is \daemon{smtpd} the only Postfix component that produced a log
        line for this mail?  Every mail which passes normally through
        Postfix will have a \daemon{cleanup} line, and later a
        \daemon{qmgr} log line; lack of a \daemon{cleanup} line is a sure
        sign the mail did not make it too far.

    \item Does the queueid exist in the state tables?  If not it cannot be
        an aborted delivery attempt.

    \item If there are third and subsequent results, were those log lines
        processed by the \texttt{SAVE\_DATA} action?  If there are any log
        lines after the first two they should be informational only.

\end{itemize}

If all the checks above are successful then the mail is assumed to be an
aborted delivery attempt and is discarded.  There will be no further
entries logged for such mails, so without identifying and discarding them
they accumulate in the state table and will cause clashes if the queueid is
reused.  The mail cannot be entered in the database as the only data
available is the client hostname and \acronym{IP} address, but the database
schema requires many more fields be populated (see \sectionref{connections
table} and \sectionref{results table}).  This heuristic is quite
restrictive, and it appears there is little scope for false positives; if
there are any false positives there will be warnings when the next log line
for that mail is parsed.  False negatives are less likely to be detected:
there may be queueid clashes (and warnings) if mails remain in the state
tables after they should have been removed, otherwise the only way to
detect false negatives is to examine the state tables after each parsing
run.

This check is performed in the DISCONNECT action; it requires support in
the CLONE action where a list of cloned connections is maintained.


\subsection{Further Aborted Delivery Attempts}

Some mail clients disconnect abruptly if a second or subsequent recipient
is rejected; they may also disconnect after other errors, but such
disconnections are either unimportant or are handled elsewhere in the
parser (\sectionref{timeouts during data phase}).  Sadly, Postfix does not
log a message saying the mail has been discarded, as should be expected by
now.  The checks to identify this happening are:

\begin{itemize}

    \item Is the mail missing its \daemon{cleanup} log line?  Every mail
        which passes through Postfix will have a \daemon{cleanup} line;
        lack of a \daemon{cleanup} line is a sure sign the mail did not
        make it too far.

    \item Were there three or more \daemon{smtpd} log lines for the mail?
        There should be a connection log line and a mail accepted log line,
        followed by one or more rejection log lines.

    \item Is the last \daemon{smtpd} log line a rejection line?

\end{itemize}

If all checks are successful then the mail is assumed to have been
discarded when the client disconnected.  There will be no further entries
logged for such mails, so without identifying and entering them in the
database immediately they accumulate in the state table and will cause
clashes if the queueid is reused.

These checks are made during the DISCONNECT action.

\subsection{Timeouts During DATA Phase}

\label{timeouts during data phase}

The DATA phase of the \acronym{SMTP} conversation is where the headers and
body of the mail are transferred.  Sometimes there is a timeout or the
connection is lost\footnote{For brevity's sake \textit{timeout\/} will be
used throughout this section, but everything applies equally to lost
connections.} during the DATA phase; when this occurs Postfix will discard
the mail and the parser needs to discard the data associated with that
mail.  It seems more intuitive to save the mail's data to the database, but
if a timeout occurs there is no data available to save; the timeout is
recorded with the connection data instead, which is saved.

To deal properly with timeouts the parsing algorithm needs to do the
following in the TIMEOUT action:

\begin{enumerate}

    \item Record the timeout and associated data in the connection's
        results.

    \item If no mails have been accepted yet nothing needs to be done; the
        TIMEOUT action ends.

    \item If one or more recipients have been accepted Postfix will have
        allocated a queueid for the incoming mail, and there will be a mail
        in the state tables that needs to be dealt with.

\end{enumerate}

XXX MAKE THIS PARAGRAPH CLEARER\@.  A timeout may thus apply either to an
accepted mail or a rejected mail.  To distinguish between the two cases the
algorithm compares the timestamp of the last accepted mail against the
timestamp of the last line logged by \daemon{smtpd} for that connection
(the TIMEOUT action is dependant on the CLONE action keeping a list of all
mails accepted on each connection).  If the mail acceptance timestamp is
later then the timeout applies to the just-accepted mail, which will be
discarded.  If the \daemon{smtpd} timestamp is later there was a rejection
between the accepted mail and the timeout: the action assumes that the
timeout applies to a rejected delivery attempt and finishes.  This
assumption is not necessarily correct, because Postfix may have accepted an
earlier recipient and rejected a later one, in which case the timeout
applies to the accepted mail, which should be discarded.  This has not been
a problem in practice, though it may be in future.  This complication is
further complicated by the presence of out of order \daemon{cleanup} log
lines: see \sectionref{discarding cleanup log lines} for details.

This complication is dealt with in the TIMEOUT action.

\subsection{Discarding Cleanup Log Lines}

\label{discarding cleanup log lines}

The author has only observed this complication occurring after a timeout,
though there may be other circumstances that trigger it.  Sometimes the
\daemon{cleanup} line for a mail being accepted is logged after the timeout
line; parsing this line causes the MAIL\_QUEUED action to create a new
state table entry for the queueid in the log line.  This is incorrect
because the line actually belongs to the mail that has just been discarded;
the next log line for that queueid will be seen when the queueid is reused
for a different mail, causing a queueid clash and the appropriate warning.

When the \daemon{cleanup} line is still pending during the TIMEOUT action,
the action updates a global list of queueids, adding the queueid and the
timestamp from the log line.  When the next \daemon{cleanup} line is parsed
for that queueid the list will be checked (during the MAIL\_QUEUED action),
and the log line will be deemed part of the mail where the timeout occurred
and discarded if it meets the following requirements:

\begin{itemize}

    \item The queueid must not have been reused yet, i.e.\ it does not have
        an entry in the state tables.

    \item The timestamp of the \daemon{cleanup} log line must be within ten
        minutes of the mail acceptance timestamp.  Timeouts happen after
        five minutes, but some data may have been transferred slowly
        (perhaps because either the client or server is suffering from
        network congestion or rate limiting), and empirical evidence shows
        that ten minutes is not unreasonable; hopefully it is a good
        compromise between false positives (log lines incorrectly
        discarded) and false negatives (new state table entries incorrectly
        created).

\end{itemize}

The next \daemon{cleanup} line must meet the criteria above for it to be
discarded because some, but not all connections where a timeout occurs will
have an associated \daemon{cleanup} line logged; if the algorithm blindly
discarded the next \daemon{cleanup} line after a timeout it would sometimes
be mistaken.  When the next \daemon{pickup} line containing that queueid is
parsed the queueid will be removed from the cache of timeout queueids,
regardless of whether it meets the criteria above.

This complication is handled by the TIMEOUT and MAIL\_QUEUED actions.

\subsection{Pickup Logging After Cleanup}

\label{pickup logging after cleanup}

When mail is submitted locally, \daemon{pickup} accepts the new mail and
generates a log line showing the source.  Occasionally this log line will
occur later in the log file than the \daemon{cleanup} log line, so the
PICKUP action will find that a state table entry exists for that queueid.
Normally if the queueid given in the PICKUP line exists in the state tables
a warning is generated by the \daemon{pickup} action, but if the following
conditions are met it is assumed that the log lines were out of order:

\begin{itemize}

    \item The only program which has logged anything thus far for the mail
        is \daemon{cleanup}.

    \item There is less than a five second difference between the
        timestamps of the \daemon{cleanup} and \daemon{pickup} log lines.

\end{itemize}

As always with heuristics there may be circumstances in which these
heuristics match incorrectly, but none have been identified so far.  This
complication seems to occur during periods of particularly heavy load, so
is most likely caused by process scheduling vagaries.

This complication is dealt with during the PICKUP action.

\subsection{Smtpd Stops Logging}

\label{smtpd stops logging}

Occasionally a \daemon{smtpd} will just stop logging, without an
immediately obvious reason.  After poring over log files for some time
several reasons have been found for this infrequent occurrence:

\begin{enumerate}

    \item Postfix is stopped or its configuration is reloaded.  When this
        happens all \daemon{smtpd} processes exit, so all entries in the
        connections state table must be cleaned up, entered in the database
        if there is sufficient data, and deleted.

    \item Sometimes a \daemon{smtpd} is killed by a signal (sent by Postfix
        for some reason, by the administrator, or by the OS), so its active
        connection must be cleaned up, entered in the database if there is
        sufficient data, and deleted from the connections state table.

    \item Occasionally a \daemon{smtpd} will exit uncleanly, so the active
        connection must be cleaned up, entered in the database if there is
        sufficient data, and deleted from the connections state table.

    \item Every Postfix process uses a watchdog which kills the process if
        it is not reset for a considerable period of time (five hours by
        default).  This safeguard prevents Postfix processes from running
        indefinitely and consuming resources if a failure causes them to
        enter a stuck state.

\end{enumerate}

The circumstances above account for all occasions identified thus far where
a \daemon{smtpd} suddenly stops logging.  In addition to removing an active
connection the last accepted mail may need to be discarded, as detailed in
\sectionref{timeouts during data phase}; otherwise the queueid state table
is untouched.

These occurrences are handled by the three actions POSTFIX\_RELOAD,
SMTPD\_DIED, and SMTPD\_WATCHDOG\@.

\subsection{Out of Order Log Lines}

\label{out of order log lines}

Occasionally a log file will have out of order log lines which cannot be
dealt with by the techniques described in \sectionref{tracking re-injected
mail}, \sectionref{discarding cleanup log lines} or \sectionref{pickup
logging after cleanup}.  In the \numberOFlogFILES{} log files used for
testing this occurs only five times in 60,721,709 log lines, but for parser
correctness it must be dealt with.  The five occurrences have the same
characteristics: the \daemon{local} log line showing delivery to a local
mailbox occurs after the \daemon{qmgr} log line showing removal of the mail
from the queue because delivery is completed.  This causes problems: the
data in the state tables for the mail is not complete, so entering it into
the database fails; a new mail is created when the \daemon{local} line is
parsed and remains in the state tables; four warnings are issued per pair
of out of order log lines.

The COMMIT action examines the list of programs that have produced a log
lines for each mail, comparing the list against a table of known-good
program combinations.  If the mail's combination is found in the table the
mail can be entered in the database; if the combination is not found entry
must be postponed and the mail flagged for later entry.  The SAVE\_DATA
action checks for that flag; if the additional log lines have caused the
mail to reach a valid combination then entry in the database will proceed,
otherwise it must be postponed once more.

The list of valid combinations is explained below.  Every mail will
additionally have log line from \daemon{cleanup} and \daemon{qmgr}; any
mail may also have log line from \daemon{bounce}, \daemon{postsuper}, or
both.

% This will put the text on the line following the item name, if the
% enumitem package is loaded.
%\begin{description}[style=nextline]
\begin{description}

    \item [\daemon{local}:] Local delivery of a bounce notification, or
        local delivery of a re-injected mail.

    \item [\daemon{local}, \daemon{pickup}:] Mail submitted locally on the
        server, delivered locally on the server.

    \item [\daemon{local}, \daemon{pickup}, \daemon{smtp}:] Mail submitted
        locally \newline{} on the server, for both local and remote
        delivery.

    \item [\daemon{local}, \daemon{smtp}, \daemon{smtpd}:] Mail accepted
        from a remote client, for both local and remote delivery.

    \item [\daemon{local}, \daemon{smtpd}:] Mail accepted from a remote
        client, for local delivery only.

    \item [\daemon{pickup}, \daemon{smtp}:] Mail submitted locally on the
        server, for remote delivery only.

    \item [\daemon{smtp}:] Remote delivery of either a re-injected mail or
        a bounce notification.

    \item [\daemon{smtp}, \daemon{smtpd}:] Mail accepted from a remote
        client, then remotely delivered (typically relaying mail for
        clients on the local network to addresses outside the local
        network).

    \item [\daemon{smtpd}, \daemon{postsuper}:] Mail accepted from a remote
        client, then deleted by the administrator before any delivery
        attempt was made (the unwanted mail is typically due to a mail loop
        or joe~job\glsadd{Joe-job}).  Notice that \daemon{postsuper} is
        required, not optional, for this combination.

\end{description}

This check applies to accepted mails only, not to rejected mails.  This
check is performed during the COMMIT action.

\subsection{Yet More Aborted Delivery Attempts}

\label{yet more aborted delivery attempts}

The aborted delivery attempts described in \sectionref{aborted delivery
attempts} occur frequently, but the aborted delivery attempts described in
this section only occur four times in the \numberOFlogFILES{} log files
used for testing.  The symptoms are the same as in \sectionref{aborted
delivery attempts}, except that there \textit{is\/} a \daemon{cleanup} log
line; there is nothing in the log file to explain why there are no further
log lines.  The only way to detect these mails is to periodically scan all
mails in the state tables, deleting any mails displaying the following
characteristics:

\begin{itemize}

    \item The timestamp of the last log line for the mail must be 12 hours
        or more earlier than the last log line parsed from the current log
        file.

    \item There must be exactly two \daemon{smtpd} and one \daemon{cleanup}
        log lines for the mail, with no additional log lines.

\end{itemize}

12 hours is a somewhat arbitrary time period, but it is far longer than
Postfix would delay delivery of a mail in the queue (unless Postfix is not
running for an extended period of time).  The state tables are scanned for
mails matching the characteristics above each time the end of a log file is
reached, and matching mails are deleted.

\subsection{Mail Deleted Before Delivery is Attempted}

\label{Mail deleted before delivery is attempted}

Postfix logs the recipient address when delivery of a mail is attempted, so
if delivery has yet to be attempted the parser cannot determine the
recipient address or addresses.  This is a problem when mail is arriving
faster than Postfix can attempt delivery, and the administrator deletes
some of the mail (because it is the result of a mail loop\glsadd{mail
loop}, mail bomb\glsadd{mail bomb}, or joe~job\glsadd{Joe-job}) before
Postfix has had a chance to try to deliver it.  In this case the recipient
address will not have been logged, so a dummy recipient address needs to be
added, as every mail is required by the database schema
(\sectionref{results table}) to have at least one recipient.  Typically
when this complication occurs there are many instances of it, closely
grouped.

This lack of information cannot easily be overcome: it is trivial to log a
warning for every recipient accepted, but when the warning for the first
recipient is logged Postfix will not yet have allocated a queue file and
queueid for the mail, so the warning will be associated with the connection
rather than the accepted mail.  A queue file and queueid will be allocated
after Postfix accepts the MAIL FROM command if
\texttt{smtpd\_delay\_open\_until\_valid\_rcpt} is set to ``no'', but that
setting will cause disk IO for almost every delivery attempt, instead of
just for delivery attempts where recipients are accepted, and consequently
a drastic reduction in the performance of the mail server.

The DELETE action is responsible for handling this complication.

\subsection{Bounce Notification Mails Are Delivered Before Their Creation
Is Logged}

\label{Bounce notification mails delivered before their creation is logged}

This is yet another complication that only occurs during periods of
extremely high load, when process scheduling and even hard disk access
times cause strange behaviour.  In this complication bounce notification
mails are created, delivered, and deleted from the queue, \textit{before\/}
the log line from \daemon{bounce} that explains their source is logged.  To
deal with this the COMMIT action maintains a cache of recently committed
bounce notification mails, which the BOUNCE action subsequently checks if
the bounce mail is not already in the state tables.  If the queueid exists
in the cache, and its start time is less than ten seconds before the
timestamp of the bounce log line, it is assumed that the bounce
notification mail has already been processed and the BOUNCE action does not
create one.  If the queueid exists it is removed from the cache, because it
has either just been used or it is too old to use in future.  Whether the
BOUNCE action creates a mail or finds an existing mail in the state tables,
it flags the mail as having been seen by the BOUNCE action; if this flag is
present the COMMIT action will not add the mail to the cache of recent
bounce notification mails.  This is not required to correctly deal
with the complication, but is an optimisation to reduce the parser's memory
usage; on the occasions the author has observed this complication occurring
there have been a huge number of bounce notification mails generated --- if
every bounce notification mail was cached it would dramatically increase
the memory requirements of the parser.  The cache of bounce notification
mails will be pruned whenever the parser's state is saved, though if the
size of the cache ever becomes a problem it could be pruned periodically to
keep the size in check.

\subsection{Mails Deleted During Delivery}

\label{Mails deleted during delivery}

The administrator can delete mails using \daemon{postsuper}; occasionally
mails that are in the process of being delivered will be deleted.  This
results in the log lines from the delivery agent (\daemon{local},
\daemon{virtual} or \daemon{smtp}) appearing in the log file
\textit{after\/} the mail has been removed from the state tables and saved
in the database.  The DELETE action adds deleted mails to a cache, which is
checked by the SAVE\_DATA action, and the current log line discarded if the
following conditions are met:

\begin{enumerate}

    \item The queueid is not found in the state tables.

    \item The queueid is found in the cache of deleted mails.

    \item The timestamp of the log line is within 5 minutes of the final
        timestamp of the mail.

\end{enumerate}

Sadly this solution involves discarding some data, but the complication
only arises eight times in quick succession in one log file, which is not
in the \numberOFlogFILES{} log files used for testing; if this complication
occurred more frequently it might be desirable to find the mail in the
database and add the log line to it.

\section{Limitations and Possible Improvements}

\label{limitations and improvements in implementation}

Every piece of software suffers from some limitations, and there is almost
always room for improvement.  Below are the limitations and possible
improvements that have been identified in \parsername{}.

\subsection{Limitations}

\label{logging helo}

\begin{enumerate}

    \item Each new Postfix release requires writing new rules or modifying
        existing rules to cope with the new or changed log lines.
        Similarly using a new \acronym{DNSBL}, a new policy server, or new
        administrator-defined rejection messages requires new rules.

    \item The hostname used in the HELO command is not logged if the
        incoming delivery attempt is successful.  Configuring Postfix to do
        this is relatively simple: create a restriction that is guaranteed
        to warn for every accepted mail, as follows:

        \begin{enumerate}

            \item Create \texttt{/etc/postfix/log\_helo.pcre}
                containing:\newline{}
                \tab{}\texttt{/\^/~~~~WARN~Logging~HELO}

            \item Modify \texttt{smtpd\_data\_restrictions} in
                \texttt{/etc/postfix/main.cf} to contain:\newline{}
                \tab{}\texttt{check\_helo\_access~pcre:/etc/postfix/log\_helo.pcre}

        \end{enumerate}

        Although \texttt{smtpd\_helo\_restrictions} seems like the natural
        place to log the HELO hostname, there will not yet be a queueid
        associated with the mail when \texttt{smtpd\_helo\_restrictions} is
        evaluated for the first recipient, so the log line cannot be
        associated with the correct mail.  There is guaranteed to be a
        queueid when the DATA command has been reached, and thus the
        queueid will be logged by any restrictions taking effect in
        \texttt{smtpd\_data\_restrictions}, and the log line can be
        associated with the correct mail.  There is no difficulty in
        specifying a HELO-based restriction in
        \texttt{smtpd\_data\_restrictions}, Postfix will perform the check
        correctly.

        Logging the HELO hostname in this fashion also partially prevents
        the complication described in \sectionref{Mail deleted before
        delivery is attempted} from occurring, but only when there is a
        single recipient; in that case the recipient address will be logged
        also, but when there are multiple recipients no addresses are
        logged.

    \item \parsername{} does not create separate mails where one or more
        delivery attempts are rejected and subsequently a mail is accepted;
        it will appear in the database as one mail with lots of rejections
        followed by acceptance (this has already been mentioned in
        \sectionref{connection reuse}).  It does not appear to be possible
        to make this distinction given the data Postfix logs, though it
        might be possible to write a policy server to provide additional
        logging.

    \item \parsername{} will not detect that it is parsing the same log file
        twice, resulting in the database containing duplicate entries.

    \item \parsername{} does not distinguish between log files produced by
        different sources when parsing; all results will be saved to the
        same database.  This may be viewed as an advantage, because log
        files from different sources can be combined in the same database,
        or it may be viewed as a limitation because there is no facility to
        distinguish between log files from different sources in the same
        database.  If the results of parsing log files from different
        sources must remain separate, the parser can easily be instructed
        to use a different database.

    \item The solution to complication \sectionref{Mails deleted during
        delivery} involves discarding data.

\end{enumerate}

\subsection{Possible Improvements}

\begin{enumerate}

    \item Investigate and write the policy server referred to in limitation
        3 above.

    \item Improve the solution to complication \sectionref{Mails deleted
        during delivery} so that data is not discarded.

\end{enumerate}


\section{Summary}

XXX TO BE WRITTEN\@.  START WITH THE OLD CONTENT BELOW\@.

This section presented the core of the parser, starting with a very high
level view and the initial complications that arose.  A flow chart showing
the paths a mail may take through the nascent simplified algorithm was
provided, followed by an explanation of those paths, and a discussion of
the parser's emergent behaviour --- the data from the log files creates the
paths in the flow chart, they are not specified anywhere in the parser.
The framework which holds the parser together was covered next, after which
came a description of the current actions provided by the parser, and the
algorithm for analysing unparsed log lines to create regexes for new rules.
Detecting, diagnosing, and defeating complications forms the largest single
portion of this section, mirroring the development of the parser.  The
complications are described in the order they were overcome, with
subsequent problems affecting fewer mails (often by an order or magnitude),
though the time required to solve problems increased with each successive
problem.



\glsresetall{}
\chapter{Evaluation}

\label{Evaluation}

\renewcommand{\figurename}{Graph}

% Reduce the gap between columns in tables.
\addtolength{\tabcolsep}{-2pt}

This chapter evaluates \parsername{} on two criteria: efficiency and
coverage.  \parsername{} is intended to be used in a production
environment, and must be capable of parsing log files generated by a high
volume mail server in a reasonable time period.  Accurate but slow parsing
is preferable to sloppy but quick parsing, so the goal of better parser
efficiency must be balanced against the requirement for precise and correct
processing of log files.

The performance evaluation begins by describing the characteristics of the
mail server that produced the log files used to test and evaluate the
parser's performance, and also the computer that the tests were run on.
How the parser scales as the size of log files increases is described next,
with an explanation of why \parsername{} has better performance with the
larger log files in the group.  The effect of using different rule
orderings is explored, and the use of optimal rule ordering is compared to
use of an oracle that allows the parser to use only one rule when
recognising each log line.  When \parsername{} is used by other mail
administrators they will need to extend the ruleset to parse their own log
lines, so the next section addresses the question of how parser performance
is affected as the number of rules in the ruleset rises.  The simple
optimisation of caching the results of compiling regexes, and the huge
effect it has on parser efficiency, is the penultimate topic to be covered.
The performance evaluation concludes by examining where the parsing time is
spent: recognition of log lines or their subsequent processing?

The second criterion the parser is evaluated on is its coverage of Postfix
log files.  This topic consists of two sections: what proportion of log
lines are correctly recognised by the ruleset, and what proportion of mail
delivery attempts are correctly understood and reconstructed by the
actions?  The former is a requirement before the latter can be achieved,
and the latter is important because the data provided by \parsername{} must
be both complete and correct for it to be of use to others.

\section{Parser Efficiency}

\label{parser efficiency}

Parsing efficiency is an obvious concern when the parser routinely needs to
parse large log files.  The server that generated the log files used in
testing this parser accepts approximately 10,000 mails for 700 users each
weekday; \graphref{Mails received per day} and \tableref{Number of mails
received per day: statistics} show that, as expected, far more mails are
received on weekdays than at weekends.  Note that these figures count mails
received by \acronym{SMTP} only, and the mail loops noticeable in later
graphs are not included.  Median log file size is 50MB, containing 285,000
log lines; large scale mail servers would have much larger log files.  The
mail server in question is a production mail server handling mail for a
university department; the benefit of using this server is that its log
files exhibit the idiosyncrasies and peculiarities a mail server in the
wild must deal with, but the downside is that significantly altering the
configuration to accommodate this project is not an option.

\showgraph{build/graph-mails-received}{Number of mails received via SMTP
per day}{Mails received per day}

\showtable{build/include-mails-received-table}{Number of mails received via
SMTP per day}{Number of mails received per day: statistics}

\begin{table}[thbp]
    \caption{Details of the computer used to generate statistics}
    \empty{}\label{Details of the computer used to generate statistics}
    \centering{}
    \begin{tabular}[]{ll}
        \tabletopline{}%
        Component  & Component in use                                   \\
        \tablemiddleline{}%
        CPU         & One dual core 2.40GHz Intel\textregistered{}
                        Core\texttrademark{}~2 CPU,                     \\
                    & with 32KB L1 cache and 4MB L2 cache.              \\
        RAM         & 2GB 667 MHz DDR RAM\@.                            \\
        Hard disk   & One Seagate Barracuda 7200 RPM 250GB SATA disk.   \\
        \tablebottomline{}%
    \end{tabular}
\end{table}

When generating the timing data used in this section, \numberOFlogFILES{}
log files (totaling 10.08 GB, \numberOFlogLINEShuman{} log lines) were each
parsed 10 times, and the mean parsing time used.  The computer used for
test runs was a Dell Optiplex 745, shown in \tableref{Details of the
computer used to generate statistics}; it was dedicated to the task of
gathering statistics from test runs, and did not run any other programs
while test runs were in progress.  Saving results to the database was
disabled for the test runs, because that dominates the run time of the
parser, and the tests are aimed at measuring the speed of \parsername{}
rather than the speed of the database and the disks the database is stored
on.  Parsing all \numberOFlogFILES{} log files in one run without saving
results to the database took \input{build/include-full-run-duration}, with
mean throughput of \input{build/include-full-run-throughput}.  In contrast,
when saving results to the database, parsing all \numberOFlogFILES{} log
files took \input{build/include-insert-results-duration}, with mean
throughput of \input{build/include-insert-results-throughput} --- parsing
makes up only
\input{build/include-skip-inserting-as-percentage-of-inserting} of the
execution time when saving results to the database.

\subsection{Architecture Scalability: Input Size}

XXX THIS SECTION DOES NOT SHOW ANY VARIETY IN INPUT SIZE, SO DOES NOT SHOW
SCALABILITY\@.  MAYBE I SHOULD RENAME IT TO ``Characteristics Of The Input
Log Files'' OR SOMETHING\@?  THE GRAPHS SHOW THAT LOGS HAVE CONSISTENT
CONTENT EXCEPT WHEN THERE IS A MAIL LOOP\@.

An important property of a parser is how parsing time scales relative to
input size: does it scale linearly, polynomially, or exponentially?
\Graphref{parsing time vs file size vs number of log lines graph} shows the
parsing time in seconds, file size in MB, and number of log lines in tens
of thousands, for each of the \numberOFlogFILES{} log files.  The three
lines run roughly in parallel, giving the impression that the algorithm
scales linearly with input size.  This impression is borne out by
\graphref{parsing time vs file size vs number of log lines factor}, which
plots both the ratio of file size vs parsing time, and the ratio of number
of log lines vs parsing time (higher is better in both cases);
\tableref{parsing time vs file size vs number of log lines factor table}
shows the same ratios for different groups of log files.  The ratios are
quite tightly banded, showing that the algorithm scales linearly; the
ratios increase (i.e.\ improve) for log files 22 \& 62--68, despite their
larger than usual size.  Both groups of log files are much larger than
usual because of a mail loop caused by a user who set up mail forwarding
incorrectly, resulting in a very different distribution of log lines:
normally most log lines are generated by mail delivery attempts from other
hosts, but when the mail loops occurred most of the log lines resulted from
failed delivery of mail generated on the server itself.  The Postfix
components that generated most of the log lines during the mail loop have
fewer associated rules than the Postfix components whose log lines normally
make up the bulk of each log file, so the mean number of rules tried per
log line reduces and so does the mean parsing time per log line.
\Graphref{Number of rules per Postfix component} shows the number of rules
per Postfix component, \graphref{Mean number of rules tried per log line}
shows the drop in the mean number of rules tried per log line for log files
containing a mail loop, and \graphref{Mean number of rules tried per log
line for each Postfix component} shows the mean number of rules tried per
log line for each Postfix component.

\showtable{build/include-file-size-and-number-of-log-lines-vs-parsing-time}{Ratio
of file size and number of log lines to parsing time}{parsing time vs file
size vs number of log lines factor table}

\showgraph{build/graph-input-size-vs-parsing-time}{Parsing time, file size,
and number of log lines for \numberOFlogFILES{} log files}{parsing time vs
file size vs number of log lines graph}

\showgraph{build/graph-input-size-vs-parsing-time-ratio}{Ratio of file size
and number of log lines to parsing time (higher is better)}{parsing time vs
file size vs number of log lines factor}

\showgraph{build/graph-average-number-of-rules-tried-per-log-line}{Mean
number of rules tried per log line}{Mean number of rules tried per log
line}

\showgraph{build/graph-average-number-of-rules-tried-per-program}{Mean
number of rules tried per log line for each Postfix component}{Mean number
of rules tried per log line for each Postfix component}

\begin{table}[thbp]
    \caption{Number of rules per Postfix component}
    \empty{}\label{Number of rules per Postfix component}
    \centering{}
    \begin{tabular}{lr}
        \tabletopline{}%
        Postfix component & Number of rules \\
        \tablemiddleline{}%
        \input{build/include-number-of-rules-per-program}
        \tablebottomline{}%
    \end{tabular}
\end{table}

\FloatBarrier{}

\subsection{Rule Ordering For Efficiency}

\label{rule ordering for efficiency}

\parsername{} has \numberOFrules{} rules: the top 10\% recognise
\input{build/include-top-ten-hits}\% of the log lines in the
\numberOFlogFILES{} log files, with the remaining log lines split across
the other 90\% of the rules, as shown in \graphref{rule hits graph}.
Assuming that the distribution of log lines is reasonably consistent over
time, \parsernames{} efficiency should benefit from trying rules that
recognise log lines more frequently before those rules that recognise log
lines less frequently.  To test this hypothesis, three full test runs were
performed with different rule orderings:

\begin{eqlist}

    \item [Optimal]  The most optimal order, according to the hypothesis:
        rules that recognise log lines most often will be tried first.

    \item [Shuffle] This ordering is intended to represent a randomly
        ordered rule set.  The rules will be shuffled once before use and
        will retain that ordering until the parser exits.  Note that the
        ordering will change every time the parser is executed, so 10
        different rule orderings will be generated for each log file in the
        test run.

    \item [Reverse] Hypothetically the worst order: rules that recognise
        log lines most frequently will be tried last.

\end{eqlist}

\Graphref{Parsing time relative to shuffled ordering graph} shows the
parsing times of optimal and reverse orderings relative to shuffled
ordering; the mean relative parsing times for different groupings of log
files are given in \tableref{Parsing time relative to shuffled ordering
table}.  This optimisation provides a mean reduction in parsing time of
\input{build/include-optimal-ordering-parsing-time-reduction-other-logs}\%
with normal log files,
\input{build/include-optimal-ordering-parsing-time-reduction-logs-22-62-68}\%
when a mail loop occurs and the distribution of log lines is unusual.
\Tableref{Parsing time relative to shuffled ordering table} shows that
differences in rule ordering have less effect on parsing time when parsing
log files 22 \& 62--68, because of the different distribution of log lines
in those log files.  A careful examination of \graphref{Parsing time
relative to shuffled ordering graph} shows that, for the first log file,
optimal and reverse orderings perform identically: this is because the hits
field of each rule is zero for the first log file, so optimal and reverse
orderings produce identical rule orderings.  For the first log file,
shuffled ordering is the most efficient of the three, but that is
accidental and cannot be relied upon.

\showgraph{build/graph-hits}{Log lines recognised per rule}{rule hits graph}

\showgraph{build/graph-optimal-and-reverse-vs-shuffle}{Parsing time
relative to shuffled ordering}{Parsing time relative to shuffled ordering
graph}

\showtable{build/include-optimal-and-reverse-vs-shuffle}{Parsing time
relative to shuffled ordering}{Parsing time relative to shuffled ordering
table}

\FloatBarrier{}

\subsection{Comparing Optimal Ordering Against A Perfect Oracle}

\label{perfect rule ordering}

Optimal rule ordering, as described in \sectionref{rule ordering for
efficiency}, is the best rule ordering it is possible to achieve without
having an oracle that magically divines which rule should be used to
recognise each log line.  Such an oracle would give perfect performance,
because only one rule would need to be used to recognise each log line.
\parsername{} can save a list showing which rule recognised each log line,
and use that list to simulate an oracle and improve parsing speed the
\textit{second\/} time a log file is parsed.  This does not provide a
practical benefit, but it does provide a means to evaluate the performance
of optimal rule ordering in comparison to an oracle.

\Graphref{Parsing time of oracle and optimal ordering relative to shuffled
ordering graph} shows the parsing times of the oracle and the optimal
ordering, relative to shuffled ordering; \tableref{Parsing time of oracle
and optimal ordering relative to shuffled ordering table} shows mean and
standard deviation.  As expected, the oracle is more efficient than optimal
ordering, but not by much.  \Graphref{Percentage increase in parsing time
when using optimal ordering instead of oracle graph} shows the percentage
increase in parsing time when using optimal ordering instead of the oracle,
with mean and standard deviation in \tableref{Percentage increase in
parsing time when using optimal ordering instead of oracle table}.

Once again, the difference between the oracle and optimal ordering is at
its lowest when parsing log files resulting from a mail loop (log files 22
\& 62--68), because the mean number of rules tried per log line is lower
(see \graphref{Mean number of rules tried per log line}).  When parsing the
first log file, the performance of optimal ordering relative to the oracle
is much worse than for the remainder of the log files, because for the
first log file the hits field of every rule is zero, so optimal ordering
does not provide any benefit for that log file; the oracle, in contrast, is
flawless for every log file.

Optimal ordering proves to be quite efficient: \tableref{Percentage
increase in parsing time when using optimal ordering instead of oracle
table} shows that optimal ordering is less than
\input{build/include-perfect-best-vs-optimal-mean} slower than parsing
using a magical oracle that divines the correct rule to use for each log
line.


\showgraph{build/graph-perfect-best-and-optimal-vs-shuffled}{Parsing time
of oracle and optimal ordering relative to shuffled ordering}{Parsing time
of oracle and optimal ordering relative to shuffled ordering graph}

\showtable{build/include-perfect-best-and-optimal-vs-shuffle}{Parsing time
of oracle and optimal ordering relative to shuffled ordering}{Parsing time
of oracle and optimal ordering relative to shuffled ordering table}

\showgraph{build/graph-perfect-best-vs-optimal}{Percentage increase in
parsing time when using optimal ordering instead of oracle}{Percentage
increase in parsing time when using optimal ordering instead of oracle
graph}

\showtable{build/include-perfect-best-vs-optimal-stddev}{Percentage
increase in parsing time when using optimal ordering instead of
oracle}{Percentage increase in parsing time when using optimal ordering
instead of oracle table}

\FloatBarrier{}

\subsection{Scalability As The Number Of Rules Rises}

\label{scalability as the number of rules rises}

How any architecture scales as the number of rules increases is important,
but it is particularly important in this architecture because it is
expected that the typical parser will have a large ruleset.  The full
\parsername{} ruleset has \numberOFrules{} rules, whereas the minimum
number of rules required to parse the \numberOFlogFILES{} log files is
\numberOFrulesMINIMUM{}, \numberOFrulesMINIMUMpercentage{} of the full
ruleset.  The full ruleset is larger because \parsername{} is tested with
\numberOFlogFILESall{} log files; testing with more log files increases the
chance of finding bugs in the parser or new complications to be overcome.
The \numberOFlogFILES{} log files were each parsed 10 times using the
minimum ruleset, and the parsing times compared to those generated using
the full ruleset: the percentage parsing time increase when using the full
ruleset instead of the minimal ruleset for optimal, shuffled, and reverse
orderings is shown in \graphref{Percentage parsing time increase when using
the maximum ruleset instead of the minimum ruleset}, with mean and standard
deviation in \tableref{Percentage parsing time increase when using the
maximum ruleset instead of the minimum ruleset table}.

Clearly the increased number of rules causes a noticeable performance
decrease with reverse ordering, and a lesser decrease with shuffled
ordering, whereas optimal ordering shows scant change.  Log files 22 \&
62--68 show much smaller increases in parsing time than other log files do,
because most of the log lines in those log files are produced by Postfix
components with few rules, so removing unnecessary rules has little effect
on the total number of rules used; \tableref{Number of rules per Postfix
component in the maximum and minimum rulesets} shows the number of rules
per Postfix component for each ruleset.  Once again, for the first log
file, optimal and reverse orderings have identical performance, because the
hits field of every rule is zero.

The optimal ordering has a mean increase of just
\input{build/include-full-ruleset-vs-minimum-ruleset-mean} in parsing time
for a \numberOFrulesMAXIMUMpercentage{} increase in the number of rules.
These results show that both the architecture and \parsername{} scale
extremely well as the number of rules increases, and that optimally sorting
the rules is an important optimisation contributing to this scalability.

\showgraph{build/graph-full-ruleset-vs-minimum-ruleset}{Percentage parsing
time increase when using the maximum ruleset instead of the minimum
ruleset}{Percentage parsing time increase when using the maximum ruleset
instead of the minimum ruleset}

\showtable{build/include-full-ruleset-vs-minimum-ruleset}{Percentage
parsing time increase when using the maximum ruleset instead of the minimum
ruleset}{Percentage parsing time increase when using the maximum ruleset
instead of the minimum ruleset table}

\begin{table}[thbp]
    \caption{Number of rules per Postfix component in the maximum and
    minimum rulesets}
    \empty{}\label{Number of rules per Postfix component in the maximum and
    minimum rulesets}
    \centering{}
    \begin{tabular}{lrr}
        \tabletopline{}%
        Postfix component & Maximum ruleset & Minimum ruleset \\
        \tablemiddleline{}%
        \input{build/include-number-of-rules-per-program-minimum-ruleset}
        \tablebottomline{}%
    \end{tabular}
\end{table}

\FloatBarrier{}

\subsection{Caching Compiled Regexes}

\label{Caching compiled regexes}

Before the Perl interpreter attempts to match a regex against a piece of
text, the regex is compiled into an internal representation and optimised
to improve the speed of matching.  This compilation and optimisation takes
CPU time: in many cases it takes far more CPU time than the actual
matching.  If the interpreter is certain that a regex will not change while
the program is running, it will automatically cache the results of
compiling and optimising the regex for later use.  The results of compiling
a dynamically generated regex can be cached and used in preference to the
original regex, but it is the responsibility of the programmer to do this;
\parsername{} does this with every rule's regex when the rules are loaded
from the database.

\Graphref{Increase in parsing time when not caching compiled regexes graph}
shows the effect that not caching compiled regexes has on parser
performance, with mean and standard deviation in \tableref{Increase in
parsing time when not caching compiled regexes table}.  For typical log
files, the mean increase in parsing time when not caching compiled regexes
is \input{build/include-cached-regexes-vs-discarded-regexes-mean}.  Caching
compiled regexes is probably the single most effective optimisation
possible in \parsername{}, and was quite simple to implement.  As with
previous optimisations, log files 22 \& 62--68 do not suffer such a large
increase in parsing time when not caching compiled regexes; this is
because, on average, fewer regexes are compiled per log line for those log
files.  The increase in parsing time when parsing the first log file is
much greater than for the other log files; again, this is because every
rule's hits field is zero, so optimal ordering is less efficient than
usual, and the mean number of rules tried for each log line will be higher
than usual.

\showgraph{build/graph-cached-regexes-vs-discarded-regexes}{Increase in
parsing time when not caching compiled regexes}{Increase in parsing time
when not caching compiled regexes graph}

\showtable{build/include-cached-regexes-vs-discarded-regexes}{Increase in
parsing time when not caching compiled regexes}{Increase in parsing time
when not caching compiled regexes table}

\FloatBarrier{}

\subsection{Where Is The Time Spent: Recognising Or Processing Log Lines?}

\label{recognising vs processing}

The optimisations described in this chapter have optimised the process of
recognising log lines, but have not optimised actions at all.  Optimisation
efforts have concentrated on recognition of log lines for two reasons:

\begin{enumerate}

    \item With the optimisations described in this chapter enabled,
        recognising log lines still dominates the execution time of the
        parser.  \Graphref{Percentage of parsing time spent recognising log
        lines graph} shows the percentage of parsing time spent recognising
        log lines for each of the \numberOFlogFILES{} log files, with mean
        and standard deviation shown in \tableref{Percentage of parsing
        time spent recognising log lines table}; for normal log files,
        \input{build/include-percentage-time-spent-recognising-log-lines-mean}
        of parsing time is spent recognising log lines.  Optimal ordering
        reduces parsing time by
        \input{build/include-optimal-ordering-parsing-time-reduction-other-logs}\%
        relative to shuffled ordering; processing recognised log lines
        occupies
        \input{build/include-percentage-time-spent-processing-log-lines-mean}
        of parsing time, less if using shuffled ordering, so optimising
        actions could not possibly provide as big a performance increase as
        optimally ordering rules.  Similarly, caching compiled regexes
        provides an
        \input{build/include-cached-regexes-vs-discarded-regexes-mean-reduction}
        reduction in parsing time; not invoking actions at all reduces the
        most optimised parsing time by less than half that.  If processing
        of recognised log lines was optimised to 1\% of its original
        parsing time, it would be slightly better than optimal ordering,
        but it would be vastly harder to implement; optimising actions
        would not provide enough reduction in parsing time to justify the
        amount of effort required.

    \item The process of recognising log lines is not parser-specific
        (excluding evaluation of rule conditions), so the optimisations
        described in this chapter are applicable to all parsers based on
        this architecture.  Actions are parser-specific, so it is unlikely
        that any optimisations made to actions would be portable to other
        parsers.

\end{enumerate}

Individual actions or the framework could and have been optimised, but
plenty of existing literature is available on the topic of optimising
programs, so the subject will not be dealt with here.

\showgraph{build/graph-percentage-time-spent-recognising-log-lines}{Percentage
of parsing time spent recognising log lines}{Percentage of parsing time
spent recognising log lines graph}

\showtable{build/include-percentage-time-spent-recognising-log-lines-table}{Percentage
of parsing time spent recognising log lines}{Percentage of parsing time
spent recognising log lines table}

\FloatBarrier{}

\section{Coverage}

\label{parsing coverage}

The discussion of \parsernames{} coverage of Postfix log files is separated
into two parts: log lines correctly recognised, and mail delivery attempts
correctly understood --- the former is a requirement for the latter to be
achieved.  Correctly understanding and reconstructing every mail delivery
attempt, whether it was successful or not, is important so that the
information in the database is correct and complete.  Improving the
proportion of log lines correctly recognised is the less difficult of the
two, because usually it just requires new rules to be written or existing
rules to be changed.  Improving the proportion of correctly understood and
reconstructed mail delivery attempts is more difficult and intrusive,
because it requires adding or changing actions, and it can be much harder
to realise that a deficiency exists and needs to be addressed.

\subsection{Log Lines Correctly Recognised}

\label{log-lines-covered}

Parsing a log line is a three stage process:

\begin{enumerate}

    \squeezeitems{}

    \item Skip the log line if the ruleset does not contain any rules for
        the Postfix component that produced the log line.

    \item Try each rule until a recognising rule is found; if the log line
        is not recognised, issue a warning and move on to the next log
        line.

    \item Invoke the action specified by the recognising rule.

\end{enumerate}

Correctly recognising all log lines requires that each Postfix component
whose log lines are of interest must have at least one rule, or its log
lines will be silently skipped; in the extreme case of an empty ruleset the
parser would skip every log line.  \parsername{} skips those log lines
because there may be any number of log lines from other programs
intermingled in the log file, and some Postfix components that do not
produce any log lines of interest.  There must be a rule to recognise each
log line variant produced by each Postfix component; if a log line is not
recognised the parser will issue a warning, to inform the user that they
need to extend their ruleset.  \parsername{} does not parse log lines from
non-Postfix programs, e.g.\ Amavisd-new or SpamAssassin; it could easily be
extended to do so, if a method could be developed to correctly associate
such log lines with existing state table entries.  Each rule's regex should
be as specific and precise as possible, to ensure accurate parsing: a rule
with a regex that matches zero or more of any character will recognise
every log line, but not in a meaningful way.  Most log lines contain fixed
strings, so this is not a problem in practice.

Full coverage of log lines can be achieved without undue effort, yet it is
hard to maintain.  Maintaining full coverage is hard because log lines
change over time, e.g.\ administrators add restrictions with custom
messages, \acronym{DNSBL} messages change, or major releases of Postfix
change log lines (usually adding more information).  Warnings are issued
for any log lines that are not recognised; no warnings are issued for
unrecognised log lines while parsing the \numberOFlogFILES{} log files, so
it can be safely concluded that zero false negatives arise.  False
positives are harder to quantify, short of examining each of the 60,721,709
log lines and ensuring that the correct rule recognised it.  However, a
random sample of 6039 log lines was parsed, and the results manually
verified by inspection to ensure that the correct rule recognised each log
line.  The sample was generated by running the command: \texttt{perl -n -e
\singlequote{}print if (rand 1 < 0.0001)\singlequote{} LOG\_FILES} to
randomly extract roughly one log line in every 10,000 (it actually
extracted 0.00994\% instead of 0.01\%).  Each log line was examined and the
correct rule identified from the \numberOFrules{} rules in the database;
the correct rule was then compared to the rule that recognised the log line
when parsing.  The sample results contained zero false positives, and this
check has been automated to ensure continued accuracy.  Based on this, the
author is confident that zero false positives occur when parsing the
\numberOFlogFILES{} log files.  On initial appearances, exercising only 36
rules from a total of \numberOFrules{} when parsing 6039 log lines seems
low, but after examining \graphref{rule hits graph} it becomes apparent
that parsing using such a low number of rules is to be expected.  The
reader should also bear in mind that even when parsing all
\numberOFlogFILES{} log files, only \numberOFrulesMINIMUM{} are used,
because some of the rules are for recognising log lines that only appear in
other log files.

\subsection{Mail Delivery Attempts Correctly Understood And Reconstructed}

\label{mails-covered}

The proportion of mail delivery attempts that are correctly understood and
reconstructed is much more difficult to determine accurately than the
proportion of log lines that are correctly recognised.  The parser can dump
its state tables in a human readable form; examining those tables with
reference to the log files is the best way to detect mails that were not
handled properly (many of the complications discussed in
\sectionref{complications} were detected in this way).  \parsername{}
issues warnings when it detects any errors or discrepancies, alerting the
user to the problem, e.g.\ when a queueid is reused but the previous mail
remains in the state tables, when a queueid or \acronym{pid} is not found
in the state tables, or when a mail does not include sufficient data to
satisfy the database schema.  The parser should produce few or no warnings
during parsing, and when finished parsing the state tables should only
contain entries for mails that have log lines in subsequent log files.
There will often be warnings about a missing queueid or \acronym{pid} in
the first few thousand log lines, because the earlier log lines for those
connections or mails are in a previous log file; loading saved state tables
from the previous log file will solve this problem.

The data used in this chapter is generated by parsing \numberOFlogFILES{}
log files.  5 are warnings produced, but because \parsername{} errs on the
side of producing more warnings rather than fewer, those 5 warnings
represent 3 instances of 1 problem: 3 connections that started before the
first log file, so their initial log lines are missing, leading to warnings
when their remaining log lines are parsed.  None of the warnings are false
positives.

The state tables will contain entries for mails not yet delivered when the
parser finishes execution.  Ideally, that is all they will contain, though
they may also contain mails whose initial log lines are not contained in
the log files.  Any other entries in the state tables are evidence of
either a failure in parsing, or an aberration in the log files.  After
parsing the \numberOFlogFILES{} log files, the state tables contain 18
entries, breaking down into:

\begin{itemize}

    \squeezeitems{}

    \item 1 connection that started only seconds before the log files
        ended and had not yet been fully transferred from client to server.

    \item 1 mail that had been accepted only seconds before the log files
        ended and had not yet been delivered.

    \item 9 mails whose initial log lines were not present in the log
        files.  These mails did not produce warnings because they resemble
        child mails waiting to be tracked with a parent; see
        \sectionref{Re-injected mails} for details.

    \item 7 mails that had yet to be delivered because of repeated
        failures.

\end{itemize}

None of the mails in the state tables should not be present, thus it can be
concluded that zero false negatives occur when parsing the
\numberOFlogFILES{} log files.  Once again, determining the false positive
rate is much harder, as manually checking the results of parsing 13,850,793
connections and mails accepted, rejected, bounced, or delivered is
infeasible.  Considerable evidence exists that the false positive rate is
extremely low:

\begin{itemize}

    \item The parser is quite verbose when complaining about known
        problems, e.g.\ if a mail is missing required data, and no such
        warnings are produced during the test runs.

    \item Queueids and \glspl{pid} naturally identify log lines belonging
        to one mail or connection respectively; it is extremely unlikely
        that a log line would be associated with the wrong connection.

    \item When dealing with the complications described in
        \sectionref{complications}, the solutions are as specific and
        restrictive as possible, with the goal of minimising the number of
        false positives.  In addition, the solution to the complication
        described in \sectionref{out of order log lines} imposes conditions
        that each reconstructed mail must comply with to be acceptable.

    \item Every effort has been made to make \parsername{} as precise,
        demanding, and particular as possible.

\end{itemize}

%\verb!perl -Mstrict -Mwarnings -MList::Util=shuffle -e! \newline{}
%\verb!     '@ARGV = [shuffle(@ARGV)]->[0];!             \newline{}
%\verb!      while (<>) {!                               \newline{}
%\verb!          if (rand 1 < 0.0001) {!                 \newline{}
%\verb!              foreach my $i (1 .!.  6000) {!        \newline{}
%\verb!                  print scalar <>;!               \newline{}
%\verb!              }!                                  \newline{}
%\verb!              exit;!                              \newline{}
%\verb!          }!                                      \newline{}
%\verb!      }' LOG_FILES!

Verifying by inspection that the parser correctly deals with all 60,721,709
log lines in the \numberOFlogFILES{} log files is infeasible, but verifying
the parsing of a sample from those log lines is a tractable, if extremely
time consuming, task.  A sample of log lines was obtained by randomly
selecting a log file, and then randomly selecting a block of 6000
contiguous log lines from it (0.00988\% of the total number of log lines).
It is important that the log lines are contiguous, so that all log lines
are present for as many of the mail delivery attempts contained in the
block as possible.  This log segment was parsed with all debugging options
enabled, resulting in 167,448 lines of output.\footnote{A mean of 27.908
lines of output per log line; each connection has 30 debugging lines, plus
21 debugging lines per result.  Connections which have been cloned will
have the cloned connection in their debugging output, plus another 33
debugging lines.  Those numbers are approximate, and may vary $\pm{}$ 2.
An approximate linear relationship between the number of log lines and
debugging lines is: $33(connections) + 30(accepted~~mails) + 21(results)$.}
All 167,448 lines were examined in conjunction with the log file segment
and a dump of the resulting database, verifying that for each of the log
lines \parsername{} used the correct rule and invoked the correct action,
which in turn produced the correct result, and the correct data was
inserted in the database.  The log file segment produced 4 warnings, 10
mails remaining in the state tables, 1625 mail delivery attempts correctly
entered in the database, zero false positives, and zero false negatives.

Given the evidence detailed above, the author is confident that the false
positive rate when reconstructing a mail delivery attempt from the
\numberOFlogFILES{} log files is exceedingly low, if not zero.

\section{Summary}

This chapter evaluated \parsername{} on two criteria: efficiency, and
coverage of Postfix log files.  The former began by describing the mail
server the log files were taken from, the computer used to generate the
statistics in this chapter, and how the parser scales as the size of log
files increases, including why performance is better on the larger log
files.  The framework optimises the order in which rules are used when
trying to recognise each log line, and the effect that optimisation has on
parsing time is explored; this is followed by the effect that adding more
rules to the ruleset has on parser performance.  The simplest and most
effective of the optimisations, caching compiled regexes, is described
next, and the efficiency evaluation concludes with an examination of where
parsing time is spent: recognising log lines or processing them?

Coverage of Postfix log files is divided into two topics in this chapter:
log lines correctly recognised, and mail delivery attempts correctly
understood and reconstructed.  The former is initially more important,
because the parser must correctly recognise every log line if it is to be
complete, but subsequently the latter takes precedence because correctly
reconstructing the journey a mail delivery attempt takes through Postfix is
the aim of the parser.  Increasing the proportion of log lines correctly
recognised is relatively simple and non-intrusive: adding new rules or
modifying existing rules is very easy because of the separation of rules,
actions, and framework.  Improving the understanding and reconstruction of
mail delivery attempts is harder, because Postfix's behaviour must be
analysed and figured out, and the new behaviour integrated into the actions
without breaking the existing parsing.  Detecting a deficiency in the
parser is also significantly harder, because the parser will warn about
unrecognised log lines, whereas discovering a flaw in the parser requires
careful study of any warnings produced and the entries remaining in the
state table.  Rectifying a flaw in the parser requires a deep understanding
of both the parser and Postfix's log files, investigative work to determine
the cause of the deficiency, further examination of the log files to aid in
developing a solution, and finally implementation, integration, and testing
of the solution.

This chapter shows that it is possible to balance the conflicting goals of
efficient and accurate parsing, and that one does not have to be sacrificed
to achieve the other.

\glsresetall{}
\chapter{Conclusion}

\label{conclusion}

Parsing Postfix log files appears at first sight to be an uncomplicated
task, especially if one has previous experience in parsing log files, but
it turns out to be a much more taxing project than initially expected.  The
variety and breadth of log lines produced by Postfix is quite surprising,
because a quick survey of sample log files gives the impression that the
number of distinct log line variants is quite small; this mistaken
impression comes from the uneven distribution exhibited by log lines
produced in normal operation, vividly illustrated in \graphref{rule hits
graph}.  Given the diverse nature of Postfix log lines, and the ease with
which administrators can cause new log lines to be logged
(\sectionref{postfix background}), enabling users to easily extend the
parser to deal with new log lines is a design imperative
(\sectionref{parser design}).  Providing a tool to ease the generation of
regexes from unrecognised log lines (\sectionref{creating new rules in
implementation}) should greatly help users who need to extend their ruleset
to recognise previously unrecognised log lines.


This architecture's greatest strength is the ease with which parsers based
on it can be adapted to deal with new requirements and inputs.  Parsing a
variation of an existing input is a trivial task: simply modify an existing
rule or add a new rule and the task is complete.  Parsing a new category of
input is achieved by writing a new action and a rule for each input
variant; quite often the new action will not need to interact with existing
actions, but when interaction is required the framework provides the
necessary facilities.  The architecture imposes very little red tape when
writing new actions, allowing the implementer to focus their time and
energy on correctly implementing their new action (\sectionref{actions in
architecture}).  The separation of the architecture into rules, actions,
and framework (\sectionref{parser design}) is unusual, partly because the
three are separated so completely.  Although parsers are often divided into
separate source code files (the combination of lex \&
yacc~\cite{lex-and-yacc-book} being a common example), the parts are
usually quite internally interdependent, and will be combined into a
complete parser by the compilation process; in contrast \parsername{} keeps
the rules and actions separate until the parser runs.  This separation
enables the optimisations discussed in \sectionref{parser efficiency}, and
it also allows different approaches to ruleset management, e.g.\ using
machine learning techniques to seamlessly create or alter rules to
recognise new inputs (\sectionref{adding new rules in architecture}).  The
decoupling of rules from actions allows different sets of rules to be used
with one set of actions, e.g.\ a parser might have actions to process
versions one and two of a file format; by choosing the appropriate ruleset
the parser will parse version one, or version two, or both versions.  A
general purpose framework can be written, so that writing a parser just
requires writing actions and rules.  The architecture makes it possible to
apply commonly used programming techniques (such as object orientation,
inheritance, composition, delegation, roles, modularisation, or closures)
when designing and implementing a parser, simplifying the process of
working within a team or developing and testing additional functionality.
This architecture is ideally suited to parsing inputs that are not fully
understood or do not follow a fixed grammar: the architecture warns about
unrecognised inputs and errors encountered by actions, but continues
parsing as best it can, allowing the developer of a new parser to decide
which deficiencies are most important and require attention first, rather
than being forced to fix the first error that arises.

The flow of control in this architecture is quite different from other
architectures, e.g.\ those used for compiling a programming language.
Typically, those parsers have a current state: each state has a fixed set
of acceptable next states, processing is determined by the state transition
that takes place, and unacceptable state transitions cause parsing to fail.
This architecture is different: the rule that recognises the input dictates
the action that will be invoked.  Rule conditions (\sectionref{rule
conditions in architecture}) enable stateful parsing, where the list of
rules used to recognise an input is constrained by the parser's current
state, but the recognising rule still dictates the action that is invoked
and, whether directly or indirectly, the next state.

When writing \parsername{}, the real difficulties arose once the parser was
successfully recognising almost all of the log lines, because most of the
irregularities and complications explained in \sectionref{complications}
started to become apparent then.  Adding new rules to deal with numerous
infrequently occurring log line variants was a simple if tiresome task,
whereas dealing with mails that were missing information or where Postfix's
actions were not being correctly reconstructed was much more grueling.
Trawling through log files was extremely time consuming and quite error
prone, searching for something out of the ordinary that might help diagnose
the problem, and eventually finding it --- sometimes hundreds or even
thousands of log lines away from the last occurrence of the queueid for the
mail in question.  Sometimes the task was not to identify the unusual log
line, but to spot that a log line normally present was missing, i.e.\ to
realise that one log line amongst thousands was absent.  In all cases the
evidence was used to construct a hypothesis to explain the irregularities,
and that hypothesis was tested in \parsername{}; if successful, the parser
was modified to deal with the irregularities, without adversely affecting
existing parsing.  The complications documented in
\sectionref{complications} are presented in the order they were solved in,
and that order closely resembles the frequency in which they occur; the
most frequently occurring complications dominate the warning messages
produced, and so naturally they were the first complications to be dealt
with.

\parsername{} is not merely a proof of concept: it is intended to be used
for parsing real-world log files from production mail servers, and the
resulting data used to improve anti-spam defences.  This means that
efficiency is important: parsing must complete in an reasonable period of
time, so that the results can be used in a timely manner.  \parsernames{}
efficiency is evaluated in \sectionref{parser efficiency}, where
optimisations and the effect they have are explored.

A parser's ability to correctly parse its inputs is extremely important;
\parsernames{} coverage of \numberOFlogFILES{} log files, each containing
one day's log lines, is discussed in \sectionref{parsing coverage}.  Both
its success at recognising individual log lines  and its correctness in
reconstructing each mail's journey through Postfix are described in detail,
including the results of manually verifying that a randomly selected
portion of a log file was correctly parsed.  Experience implementing
\parsername{} shows that full input coverage is relatively easy to achieve
with this architecture, and that with enough time and effort a full
understanding of the input is possible.  Postfix log files would require
substantial time and effort to correctly parse regardless of the
architecture used; this architecture enables an iterative approach to be
used~\cite{stepwise-refinement}, as is practiced in many other software
engineering disciplines.

The data gathered by \parsername{} provides the foundation for the future
of this project: using machine-learning algorithms to analyse the data and
optimise the set of anti-spam defences in use, followed by identifying
patterns in the data that could be used to write new anti-spam techniques
to recognise and reject spam rather than accepting it.  The database
provides the data in a normalised form that is far easier to use as input
to new or existing implementations of machine-learning algorithms than
trying to adapt each algorithm to extract data directly from log files.
New policy servers, written to implement new anti-spam measures, can be
tested or trained by using the collected data to simulate mail delivery
attempts; this would allow simple, fast, reproducible testing, without the
risk of adversely affecting a production mail server.  Development of
\parsername{} is finished, i.e.\ it correctly parses Postfix log files, and
in future it will only require maintenance; however, one avenue of future
development under consideration is to extend it to parse non-Postfix log
lines, e.g.\ SpamAssassin or Amavisd-new log lines.  \parsername{} can
easily be extended to do this, but it requires a method of associating the
non-Postfix log lines with the existing data structures and state tables,
so that all of the data for a mail delivery attempt can be stored together.

\parsername{} provides a basis for systems administrators to monitor the
effectiveness of their anti-spam measures and adapt their defences to
combat new techniques used by those sending spam.  \parsername{} is a fully
usable application, built to address a genuine need, rather than a proof of
concept whose sole purpose is to illustrate a new idea; it deals with the
oddities and difficulties that occur in the real world, rather than a
clean, idealised scenario developed to showcase the best features of a new
approach.



\appendix

\bibliographystyle{logparser-bibliography-style}
\bibliography{logparser-bibliography}

% Add some glossary entries that should be present, but lack an appropriate
% place in the text to mark them.
\glsadd{queueid}
\glsresetall{}
\renewcommand{\glossarypostamble}{\label{Glossary}}
\printglossary[style=nospacelist]{}
\renewcommand{\glossarypostamble}{\label{Acronyms}}
\printglossary[type=\acronymtype,style=eqlist]{}
\glsresetall{}
\renewcommand{\glossarypostamble}{\label{Postfix Daemons}}
\printglossary[type=postfix,style=nospacelist]{}

\end{document}
