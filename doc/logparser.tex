% $Id$
\documentclass[a4paper,12pt,draft]{article}

% Include images
\usepackage[final]{graphicx}
% Change how nested enumerate environments are labelled.
\renewcommand{\labelenumii}{\roman{enumii}:}
\renewcommand{\refname}{Bibliography}
% Add the bibliography into the table of contents.
\usepackage[section,numbib,nottoc]{tocbibind}
% When creating a PDF make the table of contents into links to the pages
% (without horrible red borders) and include bookmarks.
\usepackage{hyperref}
\hypersetup{%
    pdftitle    = {Parsing Postfix log files},
    pdfauthor   = {John Tobin},
    final,
    pdfborder   = {0 0 0},
    breaklinks,
}
    
\begin{document}

\title{Parsing Postfix log files}
\author{John Tobin \\ School of Computer Science and Statistics \\ 
Trinity College \\ Dublin 2 \\ Ireland \\ tobinjt@cs.tcd.ie}
\date{}
\maketitle

\begin{abstract}

    Parsing Postfix logs is much more difficult than it first appears but
    it is possible to achieve a high degree of accuracy in understanding
    the logs and reconstructing the actions taken by Postfix to generate
    the logs.  This paper describes the process required, documenting the
    parsing algorithm and rules, explaining the difficulties encountered,
    with reference to an implementation which stores data gathered from the
    logs in an SQL database.  The gathered data can then be used to
    optimise current anti-spam measures, provide a baseline to test new
    anti-spam measures against, or to produce statistics showing how
    effective those measures are.

\end{abstract}

XXX WHAT IS THE CONCLUSION???  FIRST THING IS TO COMPARE TO OTHER PARSERS
REVIEWED

XXX LOOK AT citeseer.com AND scholar.google.com FOR RELATED PAPERS

XXX LOTS MORE REFERENCES

\newpage
\tableofcontents

\newpage
\section{Introduction}

\label{introduction}

Most mail server administrators will have performed some basic processing
of Postfix logs at one time or another, whether it was to debug a problem,
explain to a user why their mail is being rejected, or check whether their
new anti-spam measures are working.  The more adventurous will have
generated statistics to show how many hits each of their anti-spam measures
has gotten in the last week, and possibly even generated some graphs to
clearly illustrate the point to management or complaining
users.\footnote{This was the author's first real foray into processing
Postfix logs.}  Very few will have performed in-depth parsing and analysis
of their logs, where the parsing must correlate the log lines
per-connection or per-queueid rather than processing lines independently.
One of the barriers to this type of processing is the unstructured nature
of Postfix logs, where each logging line was added on an ad hoc basis as a
requirement was discovered or new functionality was added.  Further
complication arises because the list of rejection messages is not fixed:
new messages can be added by the administrator with custom checks; every
Real-time Black List returns a different explanatory message; policy
servers may have different messages depending on the characteristics of the
connection; there are many ways in which the log lines may differ between
servers.  This paper documents the difficult process of parsing Postfix
logs, presenting a program which parses logs and places the resulting data
into a database.  The gathered data can then be used to optimise current
anti-spam measures, provide a baseline to test new anti-spam measures
against, or to produce statistics showing how effective those measures are.
There are numerous other uses for such data: improving server performance
by identifying troublesome destinations and reconfiguring appropriately or
dedicating transports to those destinations; identifying regular high
volume uses (e.g.\ customer newsletters) and restricting those uses to
off-peak times; detecting virus outbreaks which propagate via email; as a
base for billing customers on a shared server.

\vspace{1em}\noindent\textbf{Layout of the paper:}

Section~\ref{background} provides background information useful in
understanding the paper, parser and algorithm.

Section~\ref{rules} discusses the parsing rules in detail, explaining their
structure, giving an example rule and sample data it matches successfully
against, followed by a discussion of rule efficiency concerns.

Section~\ref{parsing-algorithm} contains the meat of the paper, describing
a naive parsing algorithm and the complications encountered which shaped
the full algorithm, followed by a comprehensive explanation of the
different stages of the algorithm, the actions taken during the execution
of the algorithm, and further complications whose solution completes the
algorithm.

Section~\ref{limitations-improvements} lists the limitations of the program
and/or the algorithm, then suggests some ways of dealing with the
limitations, with the goal of improving parsing and reproducing the journey
a mail takes through the internals of Postfix.

Section~\ref{conclusion} describes the results of the project to date and
contains the paper's conclusion.

Section~\ref{other-parsers} discusses other parsers and why they were
deemed unsuitable for the task.

Section~\ref{bibliography} is the bibliography, containing the resources
used in developing the algorithm, writing the program and preparing this
paper.  Also listed are some additional resources expected to be helpful in
understanding SMTP, Postfix, anti-spam techniques, or the paper.

Section~\ref{graphs} contains graphs illustrating the topics discussed in
section~\ref{rule efficiency}.

\section{Background}

\label{background}

\subsection{Motivation}

This paper and the program it describes are part of a larger project to
optimise a server's Postfix restrictions, generate statistics and graphs,
and provide a platform on which new restrictions can be trialled and
evaluated to see if they are worth using in the fight against spam.  The
program parses Postfix logs and populates a database with the data gleaned
from those logs, providing a consistent and simple view of the logs which
future tools can utilise.  The gathered data can then be used to optimise
current anti-spam measures, provide a baseline to test new anti-spam
measures against, or to produce statistics showing how effective those
measures are.\footnote{Section~\ref{introduction} lists some more example
uses.}

A short example of the optimisation possible using data from the database
is which Postfix restrictions reject the highest number of mails:

\begin{verbatim}
SELECT name, description, restriction_name, rule_order
    FROM rules
    WHERE postfix_action = 'REJECTED'
    ORDER BY rule_order DESC;
\end{verbatim}

If the database supports sub-selects percentages can be
obtained:

\begin{verbatim}
SELECT name, description, restriction_name, rule_order,
        (rule_order * 100.0 /
            (SELECT SUM(rule_order)
                FROM rules
                WHERE postfix_action = 'REJECTED'
            )
        ) || '%' AS percentage
    FROM rules
    WHERE postfix_action = 'REJECTED'
    ORDER BY rule_order DESC;
\end{verbatim}

SQL note: $||$ is the concatenation operator in SQLite3; if the database
containing the extracted data does not support this syntax, then simply
remove `` $||$ '$\%$'\hspace{1ex}'' from the query --- the results will be
the same, just slightly less visually pleasing.

Another example is determining which restrictions are not effective: this
example shows which rules had less than 100 hits on the last log file
parsed.

\begin{verbatim}
SELECT name, description, restriction_name, rule_order,
        (rule_order * 100.0 /
            (SELECT SUM(rule_order)
                FROM rules
                WHERE postfix_action = 'REJECTED'
            )
        ) || '%' AS percentage
    FROM rules
    WHERE postfix_action = 'REJECTED'
        AND rule_order < 100
    ORDER BY rule_order ASC;
\end{verbatim}

\subsection{Database as Application Programming Interface}

The database populated by this program provides a simple interface to
Postfix logs.  Although the interface is a database schema, it is in effect
quite similar to the API provided by a library: it insulates both user and
provider of the API from changes in the implementation of the other.  The
algorithm implemented by the program can be improved; support can be added
for earlier or later releases of Postfix; bugs can be fixed or limitations
removed from the program.  Statistics and/or graphs can be generated from
the database; new restrictions can be tested and the results inspected;
trends in the fight against spam can emerge from historical data saved in
the database.  Using a database simplifies writing programs which need to
interact with the data in several ways:

\begin{enumerate}

    \item Libraries exist for the majority of programming languages
        allowing access to databases, whereas if this parser was written
        as a library a new interface layer would need to be written for
        every programming language using the API\@.

    \item Databases provide complex querying and sorting functionality to
        the user without requiring large amounts of programming.  All
        databases provide a program, of variable complexity and
        sophistication, which can be used for ad hoc queries with minimal
        investment of time.

    \item Databases are easily extensible, e.g.:
        
        \begin{itemize}

            \item Other tables can be added to the database.

            \item New columns can be added to the tables used by the
                program, with sufficient DEFAULT clauses or a clever
                TRIGGER or two.

            \item A VIEW gives a custom arrangement of data with very
                little effort.

            \item If the database supports it, access can be granted on a
                fine-grained basis so that the finance department can
                produce invoices, the helpdesk can run limited queries as
                part of dealing with support calls, and the administrators
                have full access to the data.

            \item Triggers can be written to perform actions when certain
                events occur.  In pseudo-SQL\@:

\begin{verbatim}
CREATE TRIGGER ON INSERT INTO results
    WHERE sender = 'boss@example.com'
        AND postfix_action = 'REJECTED'
    SEND PANIC EMAIL TO 'postmaster@example.com';
\end{verbatim}

        \end{itemize}


    \item SQL is reasonably standard and many people will already be
        familiar with it; for those unfamiliar with it there are already
        lots of readily available resources from which to learn.  Although
        every vendor implements a different dialect of SQL, the basics are
        the same everywhere.

\end{enumerate}




\subsection{Database schema}
\label{database schema}

The database schema can be conceptually divided in two: the rules which are
used to parse log files, and the data saved from the parsing of log files.
Rules have the fields required to parse the log lines, extract data to be
saved, and cause the correct action to be executed; they also have several
fields which aid the user in understanding what each rule parses.  The
rules are described in detail in section~\ref{rules} and the fields in
particular in section~\ref{rule attributes}, and that discussion won't be
repeated here.  The data saved from parsing the logs is divided into two
tables: connections and results.

\subsubsection{Connections table}

Every connection and/or mail will have a single entry in the connections
table containing the following fields:

\begin{description}

    \item [id] This field uniquely identifies the row.

    \item [server\_ip] The IP address (IPv4 or IPv6) of the server: the
        local server when receiving mail, the remote server when sending
        mail.

    \item [server\_hostname] The hostname of the server, it will be
        \textit{unknown} if the IP address could not be resolved to a
        hostname via DNS.

    \item [client\_ip] The client IP address (IPv4 or IPv6): the remote
        server when receiving mail, the local server when sending mail.

    \item [client\_hostname] The hostname of the client, it will be
        \textit{unknown} if the IP address could not be resolved to a
        hostname via DNS.

    \item [helo] The hostname used in the HELO command.  The HELO
        occasionally changes during a connection, presumably because spam
        or virus senders think it's a good idea.  By default Postfix only
        logs this when it rejects an SMTP command, but it is trivial to log
        this for every connection:

        \texttt{smtpd\_helo\_restrictions = \newline
        \hspace*{2em} check\_helo\_access regexp:/etc/postfix/log\_helo.regex}

        Where \texttt{/etc/postfix/log\_helo.regex} contains:

        \texttt{/./ WARN Logging HELO for diagnostic purposes}

    \item [queueid] The queueid of the mail, or \textit{NOQUEUE} if a mail
        wasn't accepted.

    \item [start] The timestamp of the first log line, in seconds since the
        epoch.

    \item [end] The timestamp of the last log line, in seconds since the
        epoch.

\end{description}

\subsubsection{Results table}

Every log line corresponding to Postfix performing an action other than
INFO will have an entry in the results table, e.g. rejecting an SMTP
command, delivering a mail, or bouncing a mail.  Any log line where the
Postfix action is INFO is not interesting in and of itself, but provides
additional information which will be saved with other results.

\begin{description}

    \item [connection\_id] A reference to the entry in the connections
        table this results entry belongs to.

    \item [rule\_id] A reference to the entry in the rules table which
        matched the log line and created this result.

    \item [warning] Postfix can log a warning instead of rejecting an SMTP
        command - a facility that is quite useful for testing new
        restrictions.  This field will be 1 if the log line parsed was a
        warning rather than a real rejection, or 0 for a real rejection or
        any other result.

    \item [smtp\_code] The SMTP code extracted from the log line.  In
        general an SMTP code is only present for a rejection or final
        delivery; results missing an SMTP code will duplicate the SMTP code
        of other results.  Some final delivery log lines don't contain an
        SMTP code: in those cases the code is faked based on the success or
        failure represented by the log line.

    \item [sender] The sender's email address.  This can change during one
        single connection, when the connection is reused to send multiple
        mails.

    \item [recipient] The recipient address.

    \item [message\_id] The message-id of the accepted mail, or NULL if no
        mail was accepted.

    \item [data] A field available for anything not covered by other
        fields, e.g. the rejection message from an RBL.

    \item [timestamp] The time the line was logged, in seconds since the
        epoch.

\end{description}

\subsection{SMTP background}
\label{smtp background}

XXX WRITE THIS


\subsection{Postfix background}

Postfix is a highly configurable, high performance, secure and scalable
Mail Transport Agent.  It features extensive optional anti-spam
restrictions, allowing an administrator to deploy those restrictions which
they judge suitable for their site's needs, rather than a fixed set chosen
by Postfix's author.  These restrictions can be selectively applied,
combined and bypassed on a per-client, per-recipient or per-sender basis,
allowing varying levels of severity and/or permissiveness.  Postfix
leverages simple lookup tables to support arbitrarily complicated
user-defined sequences of restrictions and exceptions, with policy
servers\footnote{Policy servers will be explained shortly.} as the ultimate
in flexibility.  Administrators can also supply their own rejection
messages to make it clear to senders why exactly their mail was rejected.
Unfortunately this flexibility has a cost: complexity in the logs
generated.  While it is easy to use \texttt{grep(1)} to determine the fate
of an individual email, following the journey an email takes through
Postfix can be quite difficult.  The logs tend to follow a 90\%-10\%
pattern: 90\% of the time the journey is simple, but the other 10\% of the
time requires 90\% of the code.\footnote{These numbers don't have a solid
scientific basis, they're based on gut feeling from writing and debugging
the software.}

\subsubsection{Mixing and matching Postfix restrictions}

XXX TO BE WRITTEN

\subsubsection{Policy servers}

A policy server~\cite{policy-servers} is an external program that accepts
state information from Postfix for each SMTP command not rejected by an
earlier restriction; the policy server can utilise that state information
to implement whatever logic is required.  E.g.\ some users can be
restricted to sending mail on the third Tuesday after pay day only --- this
example may actually be useful in a payroll system to prevent problems from
spam or (worse) phishing mails with faked sender addresses.  More commonly
encountered scenarios are:

\begin{itemize}

    \item Checking Sender Policy Framework records~\cite{openspf,
        wikipedia-spf}.  SPF records specify which mail servers are allowed
        to send mail claiming to be from a particular domain.  The
        intention is to reduce spam from faked sender addresses,
        backscatter~\cite{postfix-backscatter} and
        joe-jobs~\cite{wikipedia-joe-job}; however there has been a lot of
        resistance to the proposal because it breaks or vastly complicates
        some features of SMTP\@.

    \item Greylisting~\cite{greylisting}, is a technique that temporarily
        rejects mail when the triple of (sender, recipient, remote IP
        address) is unknown; on the second and subsequent delivery attempts
        from that triple the mail will be accepted.  The assumption is that
        retrying after a temporary failure is uneconomical for a spammer,
        but that a legitimate mail server must retry.  Sadly spammers are
        using increasingly complex and well written programs to distribute
        spam, frequently using an ISP provided SMTP server from a
        compromised machine on the ISP's network.  Greylisting is slowly
        becoming less useful, but it does block a large percentage of spam
        mail at the moment.

    \item Using a scoring system such as
        Policyd-weight~\cite{policyd-weight} where tests accumulate points
        against the sending system --- if the eventual score is too high
        the mail is rejected.
        
    \item Rate limiting or throttling on a per-sender, per-client or
        per-recipient basis as performed by Policyd~\cite{policyd}.

\end{itemize}

Example attributes taken from~\cite{policy-servers}:

\begin{tabular}[]{ll}

    request                 & smtpd\_access\_policy     \\
    protocol\_state         & RCPT                      \\
    protocol\_name          & SMTP                      \\
    helo\_name              & some.domain.tld           \\
    queue\_id               & 8045F2AB23                \\
    sender                  & foo@bar.tld               \\
    recipient               & bar@foo.tld               \\
    recipient\_count        & 0                         \\
    client\_address         & 1.2.3.4                   \\
    client\_name            & another.domain.tld        \\
    reverse\_client\_name   & another.domain.tld        \\
    instance                & 123.456.7                 \\

\end{tabular}



\subsection{Parser background}

Before getting into detail about the parser a brief overview is in order.
The parser is split into two parts: the parsing algorithm and the rules
which are applied to the lines.  Rules can be thought of as the Lex part of
a Lex and YACC style parser: rules identify the line and return some data,
which the algorithm (the YACC part) receives and uses when performing the
requested action.  Rules are solely concerned with identifying a line and
extracting data from it, whereas the algorithm's task is to follow the
journey each mail takes through Postfix, piecing the data together into a
coherent whole, saving it in a useful and consistent form, and performing
housekeeping duties.  Readers familiar with context-free grammars may find
Section~\ref{comparison against context-free grammars} helpful.

\subsection{Assumptions}

The algorithm described and the program implementing it make a small number
of (hopefully safe and reasonable) assumptions:

\begin{itemize}

    \item The logs are whole and complete: nothing has been removed, either
        deliberately or accidentally (e.g.\ log rotation gone awry, file
        system filling up, logging system unable to cope with the volume of
        logs).

    \item Postfix logs sufficient information to make it possible to
        accurately reconstruct the actions it has taken.

    \item The Postfix queue has not been tampered with, causing unexplained
        appearance or disappearance of mail.

\end{itemize}

In some ways this task is similar to reverse engineering or replicating a
black box program based solely on its inputs and outputs.  Although the
source code is available, reading and understanding it would require a
significant investment of time:

\begin{tabular}[]{llll}

    Postfix 2.2.11  & Postfix 2.3.8   & Postfix 2.4.0 &                   \\
    71548           & 82224           & 83965         & lines of code     \\
    60962           & 67146           & 68675         & lines of comments \\
    16117           & 17647           & 18069         & lines are blank   \\
    148627          & 167017          & 170709        & lines in total    \\

\end{tabular}


\subsection{Other parsers}

10 other parsers have been reviewed in Appendix~\ref{other-parsers} as part
of the background research for this project.  None of the reviewed parsers
perform the type of advanced parsing and log correlation described here;
all are intended to perform a specific parsing and reporting task, rather
than be a generic parser, extracting data and leaving generation of reports
from the data to other programs.  Some parsers save the data extracted to a
database but the majority discard all data once they have finished running,
making historical analysis impossible.  The other parsers reviewed all
produce a report of greater or lesser complexity and detail, whereas the
program described here doesn't attempt to produce a report at all; that
responsibility is deferred to a separate program, to be developed later.
The parsing algorithm and program described here are designed to enable
much more detailed log analysis by providing a stable platform for
subsequent programs to develop upon.



\subsection{Conventions used in the paper}

The words \textit{connection\/} and \textit{mail\/} are often used
interchangeably in this paper; in general the word used was chosen based on
the context it appears in.

\section{Adaptable parsing: user defined rules}

\label{rules}

The complexity and variation in Postfix's logs requires similar flexibility
in the parser; decoupling the parsing rules from the associated actions
allows new rules to be written and tested without requiring modifications
to the algorithm source code (significantly lowering the barrier to entry
for new or casual users), and greatly simplifies both algorithm and rules.
It also creates a clear separation of functionality: rules handle low level
details of identifying log lines and extracting data from a line, whereas
the algorithm handles the higher level details of following the path a mail
takes through Postfix, assembling the required data, etc.

Rule have certain characteristics which may help in understanding the
parser:

\begin{itemize}

    \item The first matching rule wins: no further rules are tried against
        that line, but there is a facility for specifying the order of
        rules so that more specific rules can be tried first.

    \item Rules are completely self-contained and can be understood in
        isolation, without reference to any other rules.

    \item There are no sub-rules, so rules have linear computational
        complexity.

\end{itemize}


\subsection{Comparison against context-free grammars}

\label{comparison against context-free grammars}

XXX DESCRIBE IT IN LEFT\@: RIGHT PRODUCTION TERMS, IF POSSIBLE\@.

\subsection{Rule attributes}

\label{rule attributes}

Each rule defines the following:

\begin{description}

    \item [name] A short name for the rule.

    \item [description] Something must have occurred to cause Postfix to
        log each line (e.g.\ a remote client connecting causes a connection
        line to be logged).  The description field is generally used to
        describe the action causing the log lines this rule matches.

    \item [restriction\_name] The restriction which caused the mail to be
        rejected.  Only applicable to rules which have a result of
        \texttt{rejected}, other rules will have an empty string.

    \item [postfix\_action] This is the action Postfix must have taken to
        generate this line, with two exceptions:

        \begin{itemize}

            \item [INFO] Represents an unspecified intermediate action that
                the parser is not interested in per se, but which does log
                useful information, supplementing other log lines.

            \item [IGNORED] An action which is not only uninteresting in
                itself, but which also provides no useful data.

        \end{itemize}

        Uninteresting lines are parsed so that any lines the parser isn't
        capable of handling become immediately obvious errors.

    \item [program] The program (postfix/smtp, postfix/smtpd, postfix/qmgr,
        etc.) whose log lines the rule applies to.  This avoids needlessly
        trying rules which won't match the line, or worse, might match
        unintentionally.

    \item [regex] The regular expression\footnote{See
        http://en.wikipedia.org/wiki/Regular\_expression for more
        information about regular expressions in general.} to match the log
        line against.  The regex will first have several keywords expanded:
        this simplifies reading and writing rules, avoids needless
        repetition of complex regex components, allows the components
        to be corrected and/or improved in one location, and makes each
        regex largely self-documenting.
        
        The following keywords are expanded (full explanations can be found
        in the source code):

        \_\_SENDER\_\_, \_\_RECIPIENT\_\_, \_\_MESSAGE\_ID\_\_,
        \_\_HELO\_\_, \newline \_\_EMAIL\_\_, \_\_HOSTNAME\_\_, \_\_IP\_\_,
        \_\_IPv4\_\_, \_\_IPv6\_\_, \newline \_\_SMTP\_CODE\_\_,
        \_\_RESTRICTION\_START\_\_, \_\_QUEUEID\_\_, \newline
        \_\_COMMAND\_\_, \_\_SHORT\_CMD\_\_, \_\_DELAYS\_\_, \_\_DELAY\_\_,
        \_\_DSN\_\_ and \_\_CONN\_USE\_\_.

        Additional fields are captured by \_\_RESTRICTION\_START\_\_, so
        rules using it will start the fields in result\_cols,
        connection\_cols, etc.\ at 5.

        For efficiency the keywords are expanded and every rule's regex is
        compiled before attempting to parse the log file --- otherwise each
        regex would be recompiled each time it was used, resulting in a
        large, data dependent slowdown.  Rule efficiency concerns are
        discussed in section~\ref{rule efficiency}.
        
    \item [result\_cols, connection\_cols] Specifies how the fields in the
        log line will be extracted.  The format is: \newline
        \texttt{smtp\_code = 1; recipient = 2, sender = 4;} \newline
        i.e.\ semi-colon or comma separated assignment statements, with the
        variable name on the left and the matching field from the regex on
        the right hand side.  The list of acceptable variable names is:

        \texttt{connection\_cols: client\_hostname, client\_ip, server\_ip,
        \newline server\_hostname and helo.\newline} \texttt{result\_cols:
        sender, recipient, smtp\_code, data \newline and child.}

    \item [result\_data, connection\_data] Sometimes rules need to supply a
        piece of data which isn't present in the log line: e.g.\ setting
        \texttt{smtp\_code} when mail is accepted.  The format and allowed
        variables are the same as for \texttt{result\_cols} and
        \texttt{connection\_cols}, except that arbitrary
        data\footnote{Commas and semi-colons cannot be escaped and thus
        cannot be used.  This is intended for use with small amounts of
        data rather than large, so dealing with escape sequences seemed
        unnecessary.} is permitted on the right hand side of the
        assignment.

    \item [action] The action the algorithm will take; a full list can be
        found in Section~\ref{actions-in-detail}.

    \item [queueid] Specifies the matching field from the regex which gives
        the queueid, or zero if the log line doesn't contain a queueid.

    \item [rule\_order] is an efficiency measure.  This counter is
        maintained for every rule and incremented each time the rule
        successfully matches.  At the start of each run the program sorts
        the rules in descending rule\_order, and at the end of the run
        updates every rule's rule\_order.  Assuming that the distribution
        of log lines is reasonably consistent between log files, rules
        matching more commonly occurring log lines will be tried before
        rules matching less commonly occurring log lines, lowering the
        program's execution time.  Rule efficiency concerns are discussed
        in section~\ref{rule efficiency}.

    \item [priority] This is the user-configurable companion to
        rule\_order: rules will be tried in order of priority, overriding
        rule\_order.  This allows more specific rules to take precedence
        over more general rules.

\end{description}


\subsection{Example rule}

This example rule matches the message logged by Postfix when it rejects
mail from a sender address where the domain has neither an MX record nor an
A record, i.e.\ mail could not be delivered to the sender's address --- for
full details see~\cite{reject-unknown-sender-domain}.

This rule would match the following log line:

\begin{verbatim}
NOQUEUE: reject: RCPT from example.com[10.1.1.1]: 
  550 <foo@bar.baz>: Sender address rejected: Domain not found;
  from=<foo@bar.baz> to=<info@example.com> proto=SMTP
  helo=<smtp.bar.baz>
\end{verbatim}

% Don't reformat this!
\begin{tabular}[]{ll}

\textbf{Field}      & \textbf{Value}                                    \\
name                & Unknown sender domain                             \\
description         & We do not accept mail from unknown domains        \\
restriction\_name   & reject\_unknown\_sender\_domain                   \\
postfix\_action     & REJECTED                                          \\
program             & postfix/smtpd                                     \\
regex               & \verb!^__RESTRICTION_START__ <(__SENDER__)>: !    \\
                    & \verb!Sender address rejected: Domain not found;! \\
                    & \verb!from=<\5> to=<(__RECIPIENT__)> !            \\
                    & \verb!proto=E?SMTP helo=<(__HELO__)>$!            \\
result\_cols        & recipient = 6; sender = 5                         \\
connection\_cols    & helo = 7                                          \\
result\_data        &                                                   \\
connection\_data    &                                                   \\
action              & REJECTION                                         \\
queueid             & 1                                                 \\
rule\_order         & 0                                                 \\
priority            & 0                                                 \\

\end{tabular}

The various fields are used as follows;

\begin{description}

    \item [name, description, restriction\_name and postfix\_action:] are
        not \newline used by the algorithm, they serve to document the rule
        for the user's benefit.

    \item [program and regex:] If the program in the rule equals the
        program which logged the line the regex will be matched against the
        line; if the match is successful the action will be called, if not
        the next rule will be tried.

    \item [action:] will be executed if the regex matches successfully (see
        section~\ref{actions-in-detail} for full details).

    \item [result\_cols, connection\_cols, result\_data and
        connection\_data:] are \newline used by the action to extract and
        save data matched by the regex.

    \item [queueid:] is used the extract the queueid matched by the regex,
        so that the mail can be found by queueid and actions performed on
        it.

\end{description}

Additional fields are captured by \_\_RESTRICTION\_START\_\_, hence the
fields in result\_cols and connection\_cols start at 5.

        
\subsection{Rule efficiency}

\label{rule efficiency}

XXX WRITE ABOUT\@: DYNAMICS OF THE SYSTEM

XXX PRIORITY, OVERLAPPING RULES, MORE SPECIFIC VS \newline MORE GENERAL

XXX EXPLAIN THE DIPS IN THE GRAPHS WITH LARGE LOG FILES

XXX MENTION THAT THE PROGRAM HAS LINER COMPLEXITY AND SCALES LINEARLY WITH
THE LOG FILE SIZE

Graph~\ref{normal regex vs discard regex} shows execution times gathered
over 93 log files, with and without caching the regex; also shown on the
graph are the file sizes and number of lines in each file.
Graph~\ref{normal regex vs disarded regex factor} shows the percentage
execution time increase when not caching each regex.  Each log file was
parsed 10 times, the first run discarded and the remaining 9 runs averaged.
Logging to the database was disabled for the test runs.
        
        
Graph~\ref{normal vs shuffled vs reversed ordering} plots the execution
time over 93 log files for normal, shuffle and reverse ordering.
Graph~\ref{normal vs shuffled vs reversed ordering factor} shows the
percentage execution time increase for shuffled and reversed over normal.
Each log file was parsed 10 times, the first run discarded and the
remaining 9 runs averaged.  Logging to the database was disabled for the
test runs.

It is hoped that this will prove to have greater effectiveness as the
number of rules grows; currently the effect is negligible in comparison to
caching each compiled regex, but it still provides approximately a 7--10\%
performance increase.

\section{Parsing algorithm}

\label{parsing-algorithm}

While the rules have more lines of code than the algorithm, the rules are
quite simple and each rule is completely independent of its fellows.  The
algorithm is significantly more complicated and highly internally
interdependent.


\subsection{A naive approach}

A high level view of the algorithm could be expressed as:

\begin{enumerate}

    \item Mail enters the system via SMTP or local submission.

    \item If the mail is rejected, log all data and finish.

    \item Follow the progress of the accepted mail until it's either
        delivered, bounced or deleted, log all data, and finish.

\end{enumerate}

Unfortunately it's not that straightforward.


\subsection{Complications encountered}

\label{complications}


\subsubsection{Queueid vs pid}

The mail lacks a queueid until it has been accepted, so log lines must
first be correlated by the smtpd pid, then transition to being correlated
by the queueid.  This is relatively minor, but does require:

\begin{itemize}

    \item Two versions of several functions: \texttt{by\_pid} and
        \texttt{by\_queueid}.

    \item Two state tables to hold the data structure for each connection.

    \item Most importantly: every section of code must know whether it
        needs to lookup the data structures by pid or queueid.

\end{itemize}

\subsubsection{Connection reuse}

Multiple independent mails may be delivered during one connection: this
requires cloning the current data as soon as a mail is accepted, so that
subsequent mails won't trample over each other's data.  This must be done
every time a mail is accepted, as it's impossible to tell in advance which
connections will accept multiple mails.  It is quite easy to overlook this
complication because only a small minority of connections accept more than
one mail. Happily once the mail has been accepted log entries won't be
correlated by pid for that mail any more (its queueid will be used
instead), so there isn't any ambiguity about which mail a given log line
belongs to.\footnote{Unfortunately this statement is not completely
accurate: see section~\ref{timeouts-during-data-phase} for details.
However in general there isn't any ambiguity about which data structure
should be used for a given log line.}  The original connection will be
discarded unsaved when the client disconnects if there haven't been any
rejections.  One unsolved difficulty is
distinguishing between different groups of rejections, e.g.\ when dealing
with the following sequence:

\begin{enumerate}

    \item The client attempts to deliver a mail, but it is rejected.

    \item The client issues the RSET command to reset the session.

    \item The client attempts to deliver another mail, likewise rejected.

\end{enumerate}

There should probably be two different entires in the database resulting
from the above sequence, but currently there will only be one.



\subsubsection{Re-injected mails}

The most difficult complication is that locally addressed mails are not
always delivered directly to a mailbox: sometimes they are accepted for a
local address but need to be delivered to one or more remote addresses due
to aliases.  When this occurs a child mail will be injected into the
postfix queue, but without the explicit logging smtpd or sendmail injected
mails have.  Thus the source is not immediately discernible from the log
line in which the mail first appears; from a strictly chronological reading
of the logs it \textit{usually\/} appears as if the child mail has appeared
from thin air.  Subsequently the parent mail will log the creation of the
child mail:

\texttt{3FF7C4317: to=<username@example.com>, relay=local, \newline 
delay=0, status=sent (forwarded as 56F5B43FD)}

Unfortunately while all log lines from an individual process appear in
chronological order, the order in which log lines from different processes
are interleaved is subject to the vagaries of process scheduling.  In
addition the first log line belonging to the child mail (the log line cited
above properly belongs to the parent mail) is logged by qmgr,\footnote{Qmgr
is the Postfix daemon which manages the mail queue, determining which mails
will have delivery attempted next.} so the order also depends on how busy
qmgr is.\footnote{Postfix is quite paranoid about mail delivery, an
excellent characteristic for an MTA to possess, so it won't log that the
child has been created until it is absolutely certain that the mail has
been written to disk.}

Because of this the parser cannot complain when it encounters a log line
from qmgr for a previously unseen mail; it must flag the mail as coming
from an unknown origin and subsequently clear the flag if and when the
origin of the mail becomes clear.  Obviously the parser could omit checking
of where mails originate from, but the author believes that it is better to
require an explicit source, as bugs are more likely to be exposed.

Process scheduling can have a still more confusing effect: quite often the
child mail will be created, delivered and entirely finished with
\textbf{before} the parent logs the creation line!  Thus mails flagged as
coming from an unknown origin cannot be entered into the database when
their final log line is parsed; instead they must be marked as database
ready and subsequently entered by the parent mail once it has been
identified.  Quite apart from identifying mail injected in an unknown
fashion, or bugs in the parser, unknown origin mails need to be marked as
such because they lack some data present in the parent mail; they must copy
that data from their parent before being entered in the database.
The effect of this complication upon the algorithm is discussed in
section~\ref{tracking-re-injected-mail}.



\newpage
\subsection{Flow chart}

\label{flow-chart}

This flow chart shows the paths the data representing a mail/connection can
take through the parser algorithm.

\includegraphics{logparser-flow-chart.ps}

\subsection{Full algorithm}

\label{full-algorithm}

The intermingling of log entries from different mails immediately rules out
the possibility of handling each mail in isolation; the parser must be
capable of handling multiple mails in parallel, each potentially at a
different stage in its journey, without any interference between mails ---
except in the minority of cases where intra-mail interference is required.
The best way to implement this is to maintain state information for every
unfinished mail and manipulate the appropriate mail correctly for each log
line encountered.  The parser thus requires both a method of mapping log
lines to the correct mail and a method of specifying the action the log
line represents.  The former is achieved by using the smtpd process id to
identify the correct mail during the initial phase, then switching to the
queueid once the mail has been accepted.  The latter uses the action field
of the rule which matched the log line, executing the code in the function
named by the action.  From a high level viewpoint this design shares
significant similarity with how an event dispatch loop works in a GUI\@.

Section~\ref{actions-in-detail} explains the actions in substantive detail;
this section will omit such detail because it would clutter and confuse the
algorithm description.  The data flow chart in section~\ref{flow-chart}
should also be consulted while reading this section.

\subsubsection{Mail enters the system}

\label{mail-enters-the-system}

Everything starts off with a mail entering the system, whether by local
submission via postdrop/sendmail, by SMTP, by re-injection due to
forwarding, or internally generated by Postfix.  Local submission is the
simplest case: a queueid is assigned immediately and the sender address is
logged (action: pickup; flowchart:~2).

SMTP is more complicated: 

\begin{enumerate}
        
    \item First there is a connection from the remote client
        (action: connect; flowchart:~1).

    \item This is followed by rejection of sender, recipients, client
        address, etc. (action: rejection; flowchart:~4); acceptance of one
        or more mails (action: clone; flowchart:~5); or some interleaving
        of both.
        
    \item The client disconnects (action: disconnect; flowchart:~6).  If
        Postfix has rejected any SMTP commands the data will be saved to
        the database; if not there won't be any data to save (any mails
        accepted will already have been cloned so their data is in another
        data structure).

    \item If one or more mails were accepted there will be more log entries
        for those mails later, see section~\ref{mail-delivery}.

\end{enumerate}

Re-injection due to forwarding sadly lacks explicit log lines of its
own;\footnote{Previously discussed in section~\ref{complications},
complication 3.} re-injection is somewhat awkward to explain because it
overlaps both the mail acceptance and mail delivery sections.  See
section~\ref{tracking-re-injected-mail} for a full discussion.

Internally generated mails lack any explicit origin and can only be
detected using heuristics (see
section~\ref{identifying-bounce-notifications} for details).  Bounce
notifications are the primary example of internally generated mails, though
there may be other types.

\subsubsection{Mail delivery}

\label{mail-delivery}

The obvious counterpart to mail entering the system is mail leaving the
system, whether it is by deletion, bouncing, local or remote delivery.  All
four are handled in exactly the same way:

\begin{enumerate}

    \item The sender and recipient addresses will be logged separately
        (action: save\_by\_queueid; flowchart:~10).

    \item Sometimes mail is re-injected and the child mail needs to be
        tracked by the parent mail (action: track; flowchart:~11) ---
        section~\ref{tracking-re-injected-mail} discusses this in detail.

    \item Eventually the mail will be delivered, bounced, or deleted by the
        administrator (action: commit; flowchart:~12).  This is the last
        log line for this particular mail (though it may be indirectly
        referred to if it was re-injected).  If it is neither parent nor
        child of re-injection the data is cleaned up and entered in the
        database (flowchart:~14), then deleted from memory (flowchart:~15).  For
        re-injected mails see section~\ref{tracking-re-injected-mail}.

\end{enumerate}

It should be reiterated that the actions above happen whether the mail is
delivered to a mailbox, piped to a command, delivered to a remote server,
bounced (due to a mail loop, delivery failure, or five day timeout), or
deleted by the administrator.  The exception is what happens after delivery
to the parent or children of mail re-injected due to forwarding, and that
is explained in section~\ref{tracking-re-injected-mail}.

\subsubsection{Tracking re-injected mail}

\label{tracking-re-injected-mail}

It's probably apparent by now that tracking re-injected mails is the single
most complex part of the algorithm.  Section~\ref{complications} has most
of the previous discussion, should you wish to review it.  

The crux of the problem is that re-injected mails appear in the logs
without explicit logging indicating their source.  There are two implicit
indications:

\begin{enumerate}

    \item The indicator which more commonly introduces re-injection is qmgr
        selecting a mail with a previously unseen queueid for delivery
        (action: mail\_picked\_for\_delivery; flowchart:~3), in which case
        a new data structure will be created.  The mail will have been
        flagged as having unknown origins; this flag should be subsequently
        cleared once the origin has been established.  The data is then
        saved.  This may also be an indicator that the mail is a bounce
        notification, see section~\ref{identifying-bounce-notifications}
        for details.

    \item Local delivery re-injects the mail and logs a relayed delivery
        rather than delivering directly to a mailbox or program as it
        usually would (action: track; flowchart:~11).\footnote{Relayed
        delivery is performed by the SMTP client; local delivery means
        local to the server, i.e.\ an address the server accepts mail for.}
        In this case the mail may already have been created (action:
        qmgr\_chooses\_mail; flowchart:~3; described above) and the unknown
        origin flag will be cleared; if not a new data structure will be
        created.  In both cases the new mail is marked as a child of the
        parent.  The log line in question is:

        \texttt{3FF7C4317: to=<username@example.com>, relay=local, \newline 
        delay=0, status=sent (forwarded as 56F5B43FD)}

        This second indicator always occurs for re-injected mail but
        typically it occurs after the first indicator explained above, and
        so rarely introduces re-injection.  This indicator is required to
        tie the parent and child mails together and so is central to the
        process of tracking re-injected mails.

\end{enumerate}

The algorithm for tracking and saving re-injected mail to the database can
finally be described:

\begin{itemize}

    \item If the mail is of unknown origin it is assumed to be a child mail
        whose parent hasn't yet been identified (action: commit;
        flowchart:~16).  Mark the mail as ready for entry in the database
        (flowchart:~17), and wait for the parent to deal with it
        (flowchart:~18).  The mail should not have subsequent log entries;
        only its parent should refer to it.

    \item If the mail is a child mail then it has already been tracked
        (action: commit; flowchart:~19): the data is cleaned up, missing
        data is copied from its parent if necessary, and the child is
        entered in the database (flowchart:~20), then deleted from the
        state tables (flowchart:~21).  The child mail will be removed from
        the parent mail's list of children (flowchart:~22); if this is the
        last child and the parent has also been entered in the database the
        parent will be deleted from the state tables.

    \item The last alternative is that the mail is a parent mail (action:
        commit; flowchart:~23).  Regardless of the state of its children
        the data is cleaned up and entered in the database (flowchart:~24).
        The parent may have children which are waiting to be entered in the
        database (flowchart:~25); for each of those children their data is
        cleaned up and entered in the database, then deleted from the state
        tables.  The parent may also have outstanding children which are
        not yet delivered, in which case (flowchart:~26) the parent must
        wait for those children to be finished with.  As soon as the last
        child is deleted from the state tables the parent will also be
        finished with (flowchart:~27).

\end{itemize}

A parent mail can have multiple children, which may be delivered before or
after the parent mail.  

\subsection{Actions in detail}

\label{actions-in-detail}

Each action is passed the same arguments: 

\begin{description}

    \item [\$line] The log line, with the standard syslog(3) fields parsed,
        removed and made available.
        
    \item [\$rule] The matching rule.
        
    \item [@matches] The fields in the line captured by \$rule's regex.

\end{description}

\begin{description}

    \item [IGNORE] This rule just returns successfully; it is used when a
        line needs to be parsed for completeness but doesn't either provide
        any useful data or require anything to be done.

    \item [CONNECT] Handle a remote client connecting: create a new state
        table entry (indexed by smtpd pid) and save both the client
        hostname and IP address.

    \item [DISCONNECT] Deal with the remote client disconnecting: enter the
        connection in the database, perform any required cleanup, and
        delete the connection from the state tables.

    \item [SAVE\_BY\_QUEUEID] Save the data extracted by the regex to the
        mail identified by the queueid from the log line.

    \item [COMMIT] Enter the data from the mail into the database. Entry
        will be postponed if the mail is a child waiting to be tracked.
        Once entered, the mail will be deleted from the state tables.

    \item [TRACK] Track a mail when it is re-injected for forwarding to
        another mail server; this happens when a local address is aliased
        to a remote address.  TRACK will be called when dealing with the
        parent mail, and will create the child mail if necessary. TRACK
        checks if the child has already been tracked, either with this
        parent or with another parent, and issues appropriate warnings in
        either case.

    \item [REJECTION] Deal with postfix rejecting an SMTP command from the
        remote client: log the rejection with the last accepted mail or the
        connection, as appropriate.

    \item [MAIL\_PICKED\_FOR\_DELIVERY] This action represents Postfix
        picking a mail from the queue to deliver. This action is used for
        both qmgr and cleanup due to out of order log lines; see
        section~\ref{discarding-cleanup-lines} for details.

    \item [PICKUP] Pickup is the service which deals with mail submitted
        locally via /usr/sbin/sendmail. This action creates a new state
        table entry and saves data to it, unless out of order logging has
        caused the cleanup line to be logged first, as discussed in
        section~\ref{pickup-logging-after-cleanup}.

    \item [CLONE] Multiple mails may be accepted on a single connection, so
        each time a mail is accepted the connection's state table entry
        must be cloned; if the original data structure was used the second
        and subsequent mails would corrupt the data structure.

    \item [MAIL\_TOO\_LARGE] When the client tries to send a larger message
        than the local server accepts the mail will be discarded.  See
        TIMEOUT for further discussion; the two are handled in exactly the
        same way.

    \item [TIMEOUT] The connection timed out so the mail currently being
        transferred must be discarded. The mail may have been accepted, in
        which case there's a data structure to dispose of, or it may not in
        which case there's none. See
        section~\ref{timeouts-during-data-phase} for the gory details.

\end{description}

\subsection{Additional complications}

\subsubsection{Identifying bounce notifications}

\label{identifying-bounce-notifications}

Postfix 2.2.x (and presumably previous versions) lacks explicit logging
when bounce notifications are generated; suddenly there will be log entries
for a mail which lacks an obvious source.  There are similarities to the
problem of re-injected mails discussed in
section~\ref{tracking-re-injected-mail}, but unlike the solution described
therein bounce notifications do not eventually have a log line which
identifies their source.  Heuristics must be used to identify bounce
notifications, and those heuristics are:

\begin{enumerate}

    \item The sender address will be $<>$.

    \item Neither smtpd nor pickup will have logged any messages associated
        with the mail, indicating it was generated internally by Postfix,
        not accepted via SMTP or submitted locally by sendmail.

    \item The mail will not have been tagged as a tracked mail (see
        section~\ref{tracking-re-injected-mail}).

    \item The message-id has a specific format: \newline
        \texttt{YYYYMMDDhhmmss.queueid@server.hostname} \newline
        e.g. \texttt{20070321125732.D168138A1@smtp.cs.tcd.ie}

    \item The queueid in the message-id must be the same as the queueid of
        the mail: this is what distinguishes bounce notifications generated
        locally from bounce notifications which are being re-injected as a
        result of aliasing.  In the later case the message-id will be
        unchanged from the original bounce notification, and so even if it
        happens to be in the correct format (e.g.\ if it was generated by
        Postfix on another server) the queueid in the message-id will not
        match the queueid of the mail.

\end{enumerate}

Once a mail has been identified as a bounce notification the unknown origin
flag is cleared and the mail can be entered in the database.

There is a small chance that a mail will be incorrectly identified as a
bounce notification, as the heuristics used may be too broad.  For this to
occur the following conditions would have to be met:

\begin{enumerate}

    \item The mail must have been generated internally by Postfix.

    \item The sender address must be $<>$.

    \item The message-id must have the correct format and match the queueid
        of the mail.  While a mail sent from elsewhere could easily have
        the correct message-id format, the chance that the queueid in the
        message-id would match the queueid of the mail is extremely small.

\end{enumerate}

The most likely cause of mis-identification is if a mail generated
internally by Postfix is identified as a bounce notification when it is a
different type of message; arguably this is a benefit rather than a
drawback, as future mails generated internally by Postfix will be handled
correctly.

\subsubsection{Aborted delivery attempts}

\label{aborted-delivery-attempts}

Some mail clients appear to send the following sequence of commands during
the SMTP session:

\begin{verbatim}
    EHLO client.hostname
    MAIL FROM: <sender@address>
    RCPT TO: <recipient@address>
    RSET
    MAIL FROM: <sender@address>
    RCPT TO: <recipient@address>
    DATA
\end{verbatim}

The client aborts the first delivery attempt after the first recipient is
accepted, then makes a second delivery attempt which it continues with
until the delivery is complete.

Once again Postfix does not log a message making the client's behaviour
clear, so once again heuristics are required to identify when this
behaviour occurs.  In this case a list of all mails accepted during a
connection are saved in the connection state, and the accepted mails
examined when the disconnection line is parsed and the disconnection action
executed.  Each mail is checked for the following characteristics:

\begin{itemize}

    \item Is the mail missing its cleanup log message?  Every mail which
        passes through Postfix will have a cleanup line; lack of a cleanup
        line is a sure sign the mail didn't make it too far.

    \item Were there exactly two smtpd log lines for the mail?  There
        should be a connection line and a mail accepted line.

\end{itemize}

If both checks are successful then the mail is assumed to be one of the
offending bogus mails and is discarded.  There will be no further entries
logged for such mails, so without identifying and discarding them they
accumulate in the state table and will cause clashes if the queueid is
reused.  The mail cannot be entered in the database as the only data
available is the client hostname and IP address, but the database schema
requires many more fields be populated (see section~\ref{database schema}).

\subsubsection{Further aborted delivery attempts}

Some mail clients disconnect abruptly if a second or subsequent recipient
is rejected; they may also disconnect after other errors, but such
disconnections are either unimportant or are handled elsewhere in the
algorithm (section~\ref{timeouts-during-data-phase}).  Sadly Postfix
doesn't log a message saying the mail has been discarded, as should be
expected by now.  The checks to identify this happening are:

\begin{itemize}

    \item Is the mail missing its cleanup log message?  Every mail which
        passes through Postfix will have a cleanup line; lack of a cleanup
        line is a sure sign the mail didn't make it too far.

    \item Were there three or more smtpd log lines for the mail?  There
        should be a connection line and a mail accepted line, followed by
        one or more rejection lines.

    \item Is the last smtpd log line a rejection line?

\end{itemize}

These checks are made during the DISCONNECT action: if all checks are
successful then the mail is assumed to have been discarded when the client
disconnected.  There will be no further entries logged for such mails, so
without identifying and discarding them they accumulate in the state table
and will cause clashes if the queueid is reused.  The mail cannot be
entered in the database as the only data available is the client hostname
and IP address, but the database schema (see section~\ref{database schema})
requires many more fields be populated.

\subsubsection{Timeouts during DATA phase}

\label{timeouts-during-data-phase}

The DATA phase of the SMTP conversation is where the headers and body of the
mail are transferred.  Sometimes there is a timeout or the connection is
lost\footnote{For brevity's sake timeout will be used throughout this
section, but everything applies equally to lost connections.} during the
DATA phase; when this occurs Postfix will discard the mail and the parser
needs to discard the data associated with that mail.  It seems more
intuitive to save the data to the database, but if a timeout occurs there
won't yet have been any data gathered for the mail, so there is none
available to save; the timeout is saved with the connection data instead.

To deal properly with timeouts the parsing algorithm needs to do the
following in the TIMEOUT action:

\begin{enumerate}

    \item Record the timeout and associated data in the connection's
        results.

    \item If no mails have been accepted yet nothing needs to be done; the
        timeout action ends.  The timeout action is dependant on the clone
        action keeping a list of all mails accepted on each connection.

    \item Because of ESMTP pipelining the client can send MAIL FROM, RCPT
        TO and DATA commands in one packet, instead of sending each command
        individually and waiting for the reply before sending subsequent
        commands.  If pipelining is used and Postfix rejects the sender
        address or any of the recipient addresses there is no way for the
        client to tell which command was rejected.  Some clients make no
        attempt to recover and disconnect uncleanly.

        A timeout may thus apply either to an accepted mail or a rejected
        mail.  To distinguish between the two cases the algorithm compares
        the timestamp of the last accepted mail against the timestamp of
        the last line logged by smtpd for that connection.  If the smtpd
        timestamp is later there was a rejection between the accepted mail
        and the timeout, therefore the timeout applies to a rejected mail;
        the timeout has already been recorded so the timeout action
        finishes.  If the mail acceptance timestamp is greater then the
        timeout applies to the just-accepted mail, which will be discarded.

\end{enumerate}

This complication is further complicated by out of order cleanup lines: see
section~\ref{discarding-cleanup-lines} for details.

\subsubsection{Discarding cleanup lines}

\label{discarding-cleanup-lines}

The author has only observed this complication occurring after a timeout,
though there may be other circumstances which trigger it.  Sometimes the
cleanup line is logged after the timeout line; parsing this line causes the
creation of a new state table entry for the queueid in the log line.  This
is incorrect because the line actually belongs to the mail which has just
been discarded, and the next log line for that queueid will be seen when
the queueid is reused, causing a queueid clash and the appropriate warning.

In the case where the cleanup line is still pending the algorithm updates a
global cache of queueids, adding the queueid and the timestamp from the
timeout line.  When the next cleanup line is parsed for that queueid the
cache will be checked, and the line will be deemed part of the discarded
mail and discarded if it meets the following requirements:

\begin{itemize}

    \item The queueid must not have been reused yet, i.e.\ there isn't an
        entry in the state tables for the queueid.

    \item The timestamp of the cleanup line must be within six minutes of
        the mail acceptance timestamp.  Timeouts happen after five minutes,
        so six minutes gives one minute of leeway.

\end{itemize}

The next cleanup line must meet the criteria above for it to be discarded
because not every connection where a timeout occurs will have a cleanup
line logged for it; if the algorithm blindly discarded the next cleanup
line it would in some cases be mistaken.  Whether or not the next cleanup
line is discarded the queueid will be removed from the cache of timeout
queueids when the next pickup line is parsed.

\subsubsection{Pickup logging after cleanup}

\label{pickup-logging-after-cleanup}

Occasionally the pickup line logged when mail is submitted locally via
sendmail appears later in the logfile than the cleanup line for that mail.
This seems to occur during periods of particularly heavy load, so is most
likely due to process scheduling vagaries.  Normally if the queueid given
in the pickup line exists a warning is generated by the pickup action, but
if the following conditions are met it is assumed that the lines are out of
order:

\begin{itemize}

    \item The only program which logged anything for the mail is cleanup.

    \item There is less than a five second difference in the timestamps of
        the cleanup and pickup lines.

\end{itemize}

As always with heuristics there may be circumstances in which these
heuristics incorrectly match, but none have been identified so far.

\section{Parsing coverage}

XXX THIS SECTION NEEDS TO BE WRITTEN

\label{parsing-coverage}

\subsection{Log lines covered}

\label{log-lines-covered}

XXX WRITE THIS

\subsection{Mails covered}

\label{mails-covered}

XXX WRITE THIS

\section{Limitations and possible improvements}

\label{limitations-improvements}

Every piece of software suffers from some limitations and there is almost
always room for improvement.

\subsection{Limitations}

\begin{itemize}

    \item Each new Postfix release requires new rules to be written to cope
        with the new log lines.  Similarly new RBLs, new policy servers and
        new administrator defined rejection messages require new rules.

    \item The program doesn't cope with mails which span log files:
        currently they are reported as unfinished mails at the end of the
        first run and are complained about when encountered during the next
        run because they don't have a source.

    \item It appears that the hostname used in the HELO command is not
        logged if the mail is accepted.\footnote{Tested with Postfix 2.2.10
        and 2.3.7; this may possibly have changed in Postfix 2.4.}  It
        should be reasonably simple to write a policy server which causes
        Postfix to log a warning containing the HELO hostname when the DATA
        command is accepted.

    \item The algorithm does not distinguish between mails where one or
        more mails are rejected and a subsequent mail is accepted; it will
        appear in the database as one mail with lots of rejections followed
        by acceptance.  I don't believe it's possible to make this
        distinction given the data Postfix logs, though it might be
        possible to write a policy server to provide additional
        logging.

    \item The program will not detect parsing the same log file twice,
        resulting in the database containing duplicate entries.

\end{itemize}

\subsection{Possible improvements}

\begin{itemize}

    \item A progress bar would be useful when run interactively, as the
        program takes roughly one minute per 10MB of
        logs.\footnote{Approximately 80\% of the run time is consumed by
        logging to the database.}  Obviously performance is entirely
        dependant on the machine the program is running on.

    \item Write the policy servers referred to in the limitations above.

    \item Save state at the end of each run; load it at the start of the
        next run.  This should allow the program to cope with mails
        spanning log files.  This might also require purging old mails once
        they've been in the state tables for too long, though the presence
        of very old mails indicates a bug in either the implementation or
        the specification.

\end{itemize}

\section{Conclusion}

\label{conclusion}

XXX WRITE THIS


\appendix


\section{Other Postfix log parsers reviewed}

XXX THIS NEEDS TO BE EXPANDED

\label{other-parsers}

\begin{description}

    \item [pflogsumm.pl] \textit{pflogsumm.pl is designed to provide an
        over-view of postfix activity, with just enough detail to give the
        administrator a ``heads up'' for potential trouble spots.\/}
        \newline \url{http://jimsun.linxnet.com/postfix_contrib.html}

    \item [Sawmill Universal Log File Analysis and Reporting] is a
        general purpose commercial product which parses 687 log file
        formats (correct as of 2007/04/15) and produces reports.  I have not
        experimented with it due to its limited data extraction facilities;
        it does have three different sets of data for Postfix (as of
        2007/04/15, one is beta), but they do not appear to be interlinked,
        nor does it save sufficient data for the purposes of this project.
        The source code is available in an obfuscated form only (presumably
        for a fee), and the product is quite expensive, as it requires a
        license for each report which is to be generated. \newline
        \url{http://www.thesawmill.co.uk/formats/postfix.html} \newline
        \url{http://www.thesawmill.co.uk/formats/postfix_ii.html} \newline
        \url{http://www.thesawmill.co.uk/formats/beta_postfix.html}

    \item [Splunk] XXX INVESTIGATE FURTHER \url{http://www.splunk.com}

    \item [Isoqlog] \textit{Isoqlog is an MTA log analysis program written
        in C. It designed to scan qmail, postfix, sendmail and exim logfile
        and produce usage statistics in HTML format for viewing through a
        browser. It produces Top domains output according to Sender,
        Receiver, Total mails and bytes; it keeps your main domain mail
        statistics with regard to Days Top Domain, Top Users values for per
        day, per month and years.\/}  \newline
        \url{http://www.enderunix.org/isoqlog/}

    \item [AWStats] \textit{AWStats is a free powerful and featureful tool
        that generates advanced web, streaming, ftp or mail server
        statistics, graphically.\/}
        
        AWStats will produce simple graphs for many different services, but
        that restricts it to supporting the Lowest Common Denominator: the
        data it will extract from an MTA log file is:
        \texttt{time2, email, email\_r, host, host\_r, method, url, code,
        and bytesd.} \newline
        \url{http://awstats.sourceforge.net/} \newline
        \url{http://awstats.sourceforge.net/awstats.mail.html} \newline
        \url{http://awstats.sourceforge.net/docs/awstats_faq.html#MAIL}

    \item [Log analyser --- throughput monitor] This utility tracks the
        number of events which occurred over a particular time and warns if
        the frequency of events passes a certain threshold.  It's designed
        to provide real time alerts when dictionary attacks, mail loops or
        similar problems occur.  \newline
        \url{http://home.uninet.ee/~ragnar/throughput_monitor/}

    \item [Anteater] \textit{The Anteater project is a Mail Traffic
        Analyser. Anteater supports currently the logformat produced by
        Sendmail and by Postfix. The tool is written in 100\% C++ and is
        very easy to customize. Input, output, and the analysis are modular
        class objects with a clear interface. There are eight useful
        analyse modules, writing the result in plain ASCII or HTML, to
        stdout or to files.\/}

        Anteater hasn't been updated since November 2003, and doesn't have
        any English documentation.
        
        \url{http://anteater.drzoom.ch/}

    \item [Yet Another Advanced Logfile Analyser] uses a plugin based
        system to analyse log files and produce output reports.  The core
        code is merely 91 lines long, as all the parsing and report
        generation is handled by modules.  Using YAALA as a base would be
        only slightly less work as both input and output modules would need
        to be written; it may even be more work to implement the parser
        within the constraints of YAALA\@.

        \url{http://yaala.org/}

    \item [Logparser/Lire] Lire is another general purpose log parser, but
        is developed under the GNU GPL\@.  Like other general purpose log
        parsers it doesn't extract enough data from Postfix logs to be
        worthwhile.  Development also seems to have stalled, with only a
        minor update since October 2004.  \newline
        \url{http://logreport.org/lire/}

    \item [Logrep] \textit{Logrep is a secure multi-platform framework for
        the collection, extraction, and presentation of information from
        various log files.\/}

        The parsing Logrep performs is extremely basic, counting the number
        of different processes executed, from address, to addresses, number
        of recipients, bytes transferred and delay.  \newline
        \url{http://www.itefix.no/phpws/index.php}

\end{description}


\bibliographystyle{logparser-bibliography-style}

XXX ADD RESOURCES/REFERENCES HERE AND THEN MAKE REFERENCES TO THEM FROM WITHIN THE PAPER

\label{bibliography}
\bibliography{logparser-bibliography}

\section{Graphs}

\label{graphs}

\subsection{Execution time vs file size vs number of lines.}
\label{execution time vs file size vs number lines}
\includegraphics{plot-normal-filesize-numlines.ps}


\subsection{Ratio of file size and number of lines to execution time}
\label{execution time vs file size vs number lines factor}
\includegraphics{plot-normal-filesize-numlines-factor.ps}


\subsection{Regex: cached vs discarded}
\label{normal regex vs discard regex}
\includegraphics{plot-cached-discarded.ps}


\subsection{Regex: percentage execution time increase}
\label{normal regex vs disarded regex factor}
\includegraphics{plot-cached-discarded-factor.ps}


\subsection{Normal vs shuffled vs reversed ordering}
\label{normal vs shuffled vs reversed ordering}
\includegraphics{plot-normal-shuffle-reverse.ps}

\subsection{Percentage increase of shuffled and reversed over normal}
\label{normal vs shuffled vs reversed ordering factor}
\includegraphics{plot-normal-shuffle-reverse-factor.ps}


\end{document}
