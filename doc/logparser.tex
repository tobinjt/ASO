% $Id$
\documentclass[a4paper,12pt,draft]{article}

% Useful stuff for math mode.
\usepackage{amstext}
% Include images
\usepackage[final]{graphicx}
% Add the bibliography into the table of contents.
\usepackage[section,numbib]{tocbibind}
% Create a label for the last page.  Might be useful for "page 23/79" or
% something.
\usepackage{lastpage}

% Extra packages recommended by Springer: they change the fonts from
% Computer Modern to something else and tells Latex to use scalable fonts.
\usepackage{type1cm}
%\usepackage{mathptmx}
%\usepackage{helvet}
%\usepackage{courier}
% Extra footnote functionality, including references to earlier footnotes.
\usepackage{footmisc}

% Sort numbers where there are multiple citations.  does not appear to have
% any effect (probably clashes with varioref or hyperref), though it does
% reduce the space between numbers.
\usepackage{cite}

% Acronyms and glossary entries.
% global=true prevents acronyms being expanded twice
\usepackage[global=true,acronym=true,style=altlist,number=none,toc=true]{glossary}
\makeglossary{}
\makeacronym{}

% Produce nicer references, e.g. "section 4.3 on the next page"
% Must be loaded before hyperref.
\usepackage{varioref}
% Warnings rather than errors when references cross page boundaries.
\vrefwarning{}
% Errors when loops are encountered
\vrefshowerrors{}
% Shorten the text used for references with page numbers.
\renewcommand{\reftextfaraway}[1]{%
    [p.~\pageref{#1}]%
}
% Replacement for \ref{}, adds the page number too.
\newcommand{\refwithpage}[1]{%
    \empty{}\vref{#1}%
    %\empty{}\ref{#1} [p.~\pageref{#1}]%
}
% section references, automatically add \textsection
\newcommand{\sectionref}[1]{%
    \textsection{}\vref*{#1}%
    %\textsection{}\refwithpage{#1}%
}
% table references, for consistent formatting.
\newcommand{\tableref}[1]{%
    table~\vref{#1}%
}
% section references, automatically add \textsection
\newcommand{\graphref}[1]{%
    graph~\vref{#1}%
}



% Provides commands to distinguish between pdf and dvi output.
\usepackage{ifpdf}
% When creating a \PDF{} make the table of contents into links to the pages
% (without horrible red borders) and include bookmarks.  The title and
% author do not work - I think either gnuplot or graphviz clobbers it.
% hyperfootnotes need to be disabled to avoid breaking footmisc, but they
% still seem to work, somehow.
\ifpdf{}
    \usepackage[pdftex,hyperfootnotes=false]{hyperref}
\else{}
    \usepackage[dvips,hyperfootnotes=false]{hyperref}
    % This is necessary for wrapping URLs in the bibliography when
    % producing a dvi, but causes problems when generating \PDF{} output.
    \usepackage{breakurl}
\fi{}
\hypersetup{
    pdftitle    = {Parsing Postfix log files},
    pdfauthor   = {John Tobin},
    final       = true,
    pdfborder   = {0 0 0},
}



\renewcommand{\refname}{Bibliography}
% New formatting commands.

% \showgraph{filename}{caption}{label}
\newcommand{\showgraph}[3]{
    \begin{figure}[hbt!]
        \caption{#2}\label{#3}
        \includegraphics{#1}
    \end{figure}
}

%\showtable{filename}{caption}{label}
\newcommand{\showtable}[3]{
    \begin{table}[ht]
        \caption{#2}\label{#3}
        \input{#1}
    \end{table}
}

% A command to format a Postfix daemon's name
\newcommand{\daemon}[1]{%
    \texttt{postfix/#1}%
}

% This is ridiculous, but I cannot put @ in glossary entries, so . . .
\newcommand{\at}[0]{%
    @%
}
% This is ridiculous, but I cannot put @ in glossary entries, so . . .
\newcommand{\backslashat}[0]{%
    \@%
}

\newcommand{\tab}[0]{%
    \hspace*{2em}%
}

% Constant values.
\newcommand{\numberOFlogFILES}[0]{%
    93%
}

\newcommand{\numberOFrules}[0]{%
    169%
}

\newcommand{\numberOFlogLINES}[0]{%
    60,721,709%
}

\newcommand{\numberOFlogLINEShuman}[0]{%
    60.72 million%
}

% The name of the program, so I only have to change it in one place.
\newcommand{\parsername}{\PLP{}}
\newcommand{\parsernames}{\PLP{}'s}

% This keeps hyperref fairly happy about the Abstract entry in the table of
% contents.
\newcounter{dummy}

\begin{document}

% Pull in the acronyms early, so they can be used throughout the text.
% % vim: set textwidth=75 spell :
% Warning: don't use acronyms within definitions, they don't work properly.

\newacronym{PLP}{Postfix Log Parser}{description={
    Postfix Log Parser is the implementation of the algorithm described in
    this paper.
}}

\newacronym{DNSBL}{DNS Blacklist}{description={
    A DNS Blacklist~\cite{Wikipedia-DNSBL} is a simple collaborative
    anti-spam technique used to reject or penalise email sent from mail
    servers reported to have sent large volumes of spam.  To use a DNSBL
    the mail server makes a DNS request incorporating the IP address of the
    client; if the requested hostname exists the client is on the DNSBL,
    and the mail server can decide what course of action to take.
}}

\newacronym{API}{Application Programming Interface}{description={
    One of the fundamental concepts when writing programs is the reuse of
    existing code, so that each program does not reinvent the wheel.  When
    a body of code is intended to be reused repeatedly, the user of this
    code needs to be informed of the functionality provided by the code.
    An API defines the interface provided to the user, and acts as a
    contract between the user and the provider: if the user adheres to the
    API the provider guarantees it will work, while the provider is free to
    change the implementation as long as the API is preserved.
}}

\newacronym{SMTP}{Simple Mail Transfer Protocol}{description={
    SMTP is the protocol which transfers mail from the sender to the
    recipient across the Internet.  It is a simple, human readable, plain
    text protocol, making it quite easy to test and debug problems with it.
    A detailed description of \SMTP{} is beyond the scope of this document:
    the original protocol definition is \RFC{}~821~\cite{RFC821}, updated
    in \RFC{}~2821~\cite{RFC2821}; introductory guides can be found
    at~\cite{smtp-intro-01,smtp-intro-02}.
}}

\newacronym{LMTP}{Local Mail Transfer Protocol}{description={
    LMTP is a protocol derived from SMTP that removes the need for the
    server to maintain a mail delivery queue, instead relying on the client
    to maintain it.  Typically the client would be an MTA, and the server
    would be a delivery agent or a mail store.  Full details are available
    in \cite{lmtp-rfc}.
}}

\newacronym{MTA}{Mail Transfer Agent}{description={
    A Mail Transfer Agent sends and/or receives mail via SMTP.
}}

\newacronym{RFC}{Request For Comments}{description={
    The Request For Comments series is a series of proposals defining
    various protocols and file formats, e.g. SMTP.  The name is somewhat
    misleading: initially the authors were asking for peer review, but
    these documents are now the de facto standards the Internet runs on.
}}

\newacronym{SQL}{Structured Query Language}{description={
    SQL is the standard language used for database querying, modification
    and maintenance.  Some information about SQL, including its history,
    can be found at \cite{Wikipedia-sql}; a good introduction can be found
    at~\cite{sql-for-web-nerds}, others are~\cite{w3schools-sql-tutorial,
    sqlcourse.com}.
}}

\newacronym{SPF}{Sender Policy Framework}{description={
    SPF is introduced in \sectionref{spf introduction} and explained fully
    in~\cite{openspf, Wikipedia-spf}.
}}

\newacronym{ISP}{Internet Service Provider}{description={
    An ISP is a company which sells Internet access to consumers.
}}

\newacronym{LMA}{Log Mail Analyzer}{description={
    One of the other Postfix log parsers reviewed and the only example of
    published prior art~\cite{log-mail-analyser} found by the author.
}}

\newacronym{CSV}{Comma-Separated Value}{description={
    The most basic form of database available, each record is a single line
    in the file and the fields are separated by a special character,
    typically a comma or colon.
}}


\newacronym{FQDN}{Fully Qualified Domain Name}{description={
    An FQDN is a hostname plus domain name, e.g.  \newline
        example.com     is a domain name          \newline
        www             is a hostname             \newline
        www.example.com is a FQDN
}}

\newacronym{DNS}{Domain Name System}{description={
    The DNS converts between hostnames (www.example.com) and IP addresses
    (10.1.2.3).
}}

\newacronym{pid}{Process Identifier}{description={
    There may be multiple copies of any program executing at any one time,
    so the program's name is not suitable as a distinguishing identifier;
    instead each process executing is given a pid which is guaranteed to be
    unique for the lifetime of the process.  Once the process has
    completed, the pid may be reused, as they are drawn from a finite pool.
}}

\newacronym{UCE}{Unsolicited Commercial Email}{description={
    UCE is a more restrictive definition of spam than most people would
    use: it only covers mail that is explicitly commercial, thus excluding
    viruses, Bayesian poisoning mails, backscatter, and those annoying
    chain letters you get from friends.
}}

\newacronym{regex}{Regular Expression}{description={
    Regular expressions are a powerful method of matching patterns against
    text that are explained in detail in~\cite{Wikipedia-regex, perlre}.
}}

% I think this is better than having both regex and regexes in the acronym
% list.  Likewise for pids.
\newcommand{\regexes}{\regex{}es}
\newcommand{\pids}{\pid{}s}

\newacronym{ESMTP}{Extended SMTP}{description={
    ESMTP is Extended SMTP, defined in RFC~1869~\cite{RFC1869}.
}}

\newacronym{HTML}{Hypertext Markup Language}{description={
    HTML is the markup language designed for writing web
    pages~\cite{Wikipedia-html, w3schools-html}.
}}

\newacronym{IP}{Internet Protocol}{description={
    The Internet Protocol~\cite{Wikipedia-ip} is the protocol used to
    communicate between computers on the Internet.  An IP address is a
    unique address assigned to a computer on the Internet, allowing it to
    communicate with other computers on the Internet.  For more information
    see~\cite{Wikipedia-ip-address}.
}}

\newacronym{PDF}{Portable Document Format}{description={
    PDF is the file format created by Adobe Systems in 1993 for document
    exchange.  It aims to be device independent, so documents should look
    the same whether viewed on screen or printed.
}}

\newacronym{LCD}{Lowest Common Denominator}{description={
    LCD is a mathematical term which is used figuratively to mean the least
    useful member of a set of alternatives.
}}

\newacronym{SLCT}{Simple Logfile Clustering Tool}{description={
    SLCT~\cite{slct-paper} is a tool implementing an algorithm designed by
    Risto Vaarandi for identifying, grouping and producing a regex to match
    similar log lines.
}}

\newacronym{ATN}{Augmented Transition Networks}{description={
    Augmented Transition Networks, originally described in~\cite{atns} and
    further in~\cite{nlpip}, are a tool used in Computational Linguistics
    for creating grammars to parse or generate sentences,
}}

\newacronym{CLI}{Command Line Interface}{description={
    An computer interface based on typing commands, rather than using a
    mouse.
}}

\newacronym{NLP}{Natural Language Processing}{description={
    Natural Language Processing~\cite{Wikipedia-nlp} attempts to increase
    our understanding of the languages normally used by humans (e.g.\
    English, Spanish, Japanese), with the goal of writing programs which
    can understand human languages.
}}


\title{Parsing Postfix log files}
\author{John Tobin \\ School of Computer Science and Statistics \\
Trinity College \\ Dublin 2 \\ Ireland \\ tobinjt@cs.tcd.ie}
\maketitle

\refstepcounter{dummy}
\addcontentsline{toc}{section}{Abstract}
\begin{abstract}

    Parsing Postfix log files is much more difficult than it first appears,
    but it \textit{is\/} possible to achieve a high degree of accuracy in
    understanding the log files, and thus accuracy in reconstructing the
    actions taken by Postfix to generate the log files.  This paper
    describes the creation of a parser, documenting the parsing algorithm,
    explaining the difficulties encountered and the solutions developed,
    with reference to an implementation which stores data gleaned from the
    log files in an SQL database.  The gathered data can then be used to
    optimise current anti-spam measures, provide a baseline to test new
    anti-spam measures against, or to produce statistics showing how
    effective those measures are.

\end{abstract}

XXX WEKA FOR DECISION TREE DISPLAY\@.

XXX CHECK Y-AXIS LABELS ON GRAPHS TO ENSURE THEY CAN BE READ, AND WRAP IF
NECESSARY\@.

XXX EXPLAIN THE X-AXIS --- MADS THOUGHT IT WAS UNCLEAR\@.

XXX ``.'' AFTER ETC\@; SHOULD I HAVE A BACKSLASH AFTER THE ``.''?

\newpage
\tableofcontents
\listoffigures
\listoftables

\newpage
\section{Introduction}

\label{introduction}

Most mail server administrators will have performed some basic processing
of the log files produced by their mail server at one time or another,
whether it was to debug a problem, explain to a user why their mail is
being rejected, or check if new anti-spam measures are working.  The more
adventurous will have generated statistics to show how successful each of
their anti-spam measures has been in the last week, and possibly even
generated some graphs to clearly illustrate these statistics to management
or users.\footnote{This was the author's first real foray into processing
Postfix log files.}  Very few will have performed in-depth parsing and
analysis of their log files, where the parsing must correlate the log lines
per-connection or per-queueid rather than processing log lines
independently.  One of the barriers to this kind of processing is the
unstructured nature of Postfix log files, where each log line was added on
an ad hoc basis as a requirement was discovered or new functionality was
added.\footnote{A history of all changes made to Postfix is distributed
with the source code, available from \url{http://www.postfix.org/}} Further
complication arises because the set of rejection messages is not fixed: new
messages can be added by the administrator with custom checks; every
\DNSBL{}\footnote{This document is supplied with a glossary, see
\textsection\ref{Glossary}.} returns a different explanatory message;
policy servers may log different messages depending on the characteristics
of the connection.  There are many ways in which the log lines may differ
between servers, even within the same organisation: servers may be
configured differently, or running different versions of Postfix.  This
paper documents the difficult process of parsing Postfix log files,
presenting \PLP{}, a program that parses Postfix log files and places the
resulting data into a database for later use.  The gathered data can then
be used to optimise current anti-spam measures, provide a baseline to test
new anti-spam measures against, or to produce statistics showing how
effective those measures are.  Numerous other uses are possible for such
data: improving server performance by identifying troublesome destinations
and reconfiguring appropriately; identifying regular high volume uses
(e.g.\ customer newsletters) and restricting those uses to off-peak times;
detecting virus outbreaks that propagate via email; as a base for billing
customers on a shared server.  Preserving the raw data enables users to
develop a multitude of uses far beyond those conceived of by the author.

\vspace{1em}\noindent\textbf{Layout:}

XXX THIS ALL NEEDS TO BE CHECKED AND UPDATED\@.

Section~\ref{background} provides background information useful in
understanding the thesis, parser, and architecture.  It introduces the idea
of using a database schema as an \API{}, providing an interface to the data
gathered that is language-neutral.  The novel separation of rules, actions
and framework is discussed, giving the reasons that approach was taken when
designing the parser.

Section~\ref{state of the art review} reviews both the previously published
research in this area and other available Postfix log parsers, discussing
why they were deemed unsuitable for the task, including why they could not
be improved or expanded upon.

This algorithm requires a database for storing both the rules used when
parsing and the results gleaned from parsing.  The database schema used is
described in \sectionref{database schema}, explaining in detail the tables
used for storing the data gleaned from the log files and the table that
stores the rules.

Section~\ref{rules} discusses the parsing rules in detail, explaining the
purpose and usage of each field in a rule, referring to an example rule and
sample data it matches successfully against.  The pros and cons of
overlapping rules are considered, including techniques for detecting
unintentional overlaps.  Rule efficiency concerns are discussed, in
particular the optimisations used by the algorithm.  The section concludes
with a description of using the tools provided with the parser to generate
new rules (specifically the regex in each rule) from unparsed log lines.

Section~XXX contains the core of the paper, describing
a naive parsing algorithm and the complications initially encountered that
shaped the full algorithm.  A flow chart and a discussion of the emergent
behaviour exhibited by the algorithm accompanies a comprehensive
explanation of the different stages of the initial algorithm.  The
framework that actions and rules fit into is documented, then the actions
taken during execution of the algorithm are described, followed by the
process of adding a new action.  The section concludes with an in-depth
description of the further complications discovered, and their solutions
that complete the parser.

Section~\ref{parsing coverage} analyses the coverage the parser achieves
over a set of \numberOFlogFILES{} log files taken from a mail server
handling mail for over 700 users, averaging 8500 mails per day
(\graphref{Mails received per day}).  Coverage is described both
in terms of the fraction of log lines parsed and the fraction of mails and
connections successfully reconstructed by the parsing algorithm; dealing
with false negatives and a discussion of the difficulties in identifying
false positives is also included.  As part of determining the coverage of
the parser a random sampling of log lines was parsed, and the correctness
of the results manually verified.

Section~\ref{limitations-improvements} lists the limitations of the
algorithm, then suggests some ways of dealing with them, with the goal of
improving parsing and reproduction of the journey a mail takes through
Postfix.

Section~\ref{conclusion} contains the conclusion of the thesis, describing
the results of the research, design and implementation of the parser.

The bibliography contains references to the resources used in developing
the algorithm, writing the program, and preparing this thesis.  Also
listed are some additional resources expected to be helpful in
understanding \SMTP{}, Postfix, anti-spam techniques, or the thesis.

Appendix~\ref{Glossary} provides a glossary of terms used in the thesis.

Appendix~\ref{Acronyms} provides a list of acronyms used in the thesis.


\section{Background}

\label{background}

This section provides background information helpful in understanding the
remainder of the document.  It begins with a discussion of the motivation
underlying the project, followed by some technical information: the use of
a database as an \API{}\@; a brief introduction to \SMTP{}\@; a longer
introduction to Postfix, concentrating on the topics most relevant to this
document, namely Postfix anti-spam restrictions and policy servers.  The
assumptions made in designing and implementing the parser are explained, as
are the conventions used in this document.  Other projects which attempt to
parse Postfix log files are summarised (full details are available in
\sectionref{other-parsers}), finishing with a review of previously
published research in this area.

\subsection{Motivation}

\label{motivation}

This document and the program it describes are part of a larger project to
optimise a mail server's Postfix restrictions, generate statistics and
graphs, and provide a platform on which new restrictions can be trialled
and evaluated to determine if they are beneficial in the fight against
spam.  The program parses Postfix log files and populates a database with
the data gleaned from those log files, providing a consistent and simple
view of the log files which future tools can utilise.  The gathered data
can then be used to optimise current anti-spam measures, provide a baseline
to test new anti-spam measures against, or to produce statistics showing
how effective those measures are.

A snippet of \SQL{} provides a short example of the optimisation possible
using data from the database: determining which Postfix restrictions reject
the highest number of mails:

\begin{verbatim}
SELECT name, description, restriction_name, hits_total
    FROM rules
    WHERE postfix_action = 'REJECTED'
    ORDER BY hits_total DESC;
\end{verbatim}

If the database supports sub-selects percentages can be obtained:
\footnote{\SQL{} note: $||$ is the concatenation operator in SQLite3; if
the database containing the extracted data does not support this syntax,
then simply remove `` $||$ '$\%$'\hspace{1ex}'' from the query --- the
results will be the same, just slightly less visually pleasing.}

\begin{verbatim}
SELECT name, description, restriction_name, hits_total,
        (hits_total * 100.0 /
            (SELECT SUM(hits_total)
                FROM rules
                WHERE postfix_action = 'REJECTED'
            )
        ) || '%' AS percentage
    FROM rules
    WHERE postfix_action = 'REJECTED'
    ORDER BY hits_total DESC;
\end{verbatim}

XXX GIVE SAMPLE OUTPUT FOR THE SNIPPET ABOVE\@.

Another example is determining which restrictions are not effective: this
shows which restrictions had fewer than 100 hits on the last log file
parsed, and the percentage of total rejections each restriction represents.

\begin{verbatim}
SELECT name, description, restriction_name, hits,
        (hits * 100.0 /
            (SELECT SUM(hits)
                FROM rules
                WHERE postfix_action = 'REJECTED'
            )
        ) || '%' AS percentage
    FROM rules
    WHERE postfix_action = 'REJECTED'
        AND hits < 100
    ORDER BY hits ASC;
\end{verbatim}

These database queries yield summary statistics about the efficiency of
spam avoidance techniques; statistics that are far less feasible to assess
directly from log files without prior pre-processing into a database in the
fashion proposed, implemented and tested herein.

\subsection{Database as Application Programming Interface}

\label{database as API}

The database populated by this program provides a simple interface to
Postfix log files.  Although the interface is a database schema, it is in
effect quite similar to any other \API{} provided by shared code: it
insulates both user and provider of the \API{} from changes in the
implementation of the \API{}\@.  The algorithm implemented by the parser
can be improved; support can be added for earlier or later releases of
Postfix; bugs can be fixed or limitations removed from the parser; these
changes will not cause the user to be negatively impacted.  Statistics
and/or graphs can be generated from the database; new restrictions can be
tested and the results inspected; trends in the fight against spam can
emerge from historical data saved in the database; the parser remains the
same as the usage adapts.  Using a database simplifies writing programs
which need to interact with the data in several ways:

\begin{enumerate}

    \item It is already possible to access databases from the vast majority
        of programming languages, allowing a developer to access the data
        gathered using their preferred programming language, rather than
        being restricted to the language the parser is written in.  It is
        often possible to write an interface layer allowing code written in
        one language to be used in another language, but this greatly
        increases the effort required to use the parser.

    \item Databases provide complex querying and sorting functionality to
        the user without requiring large amounts of programming.  All
        databases provide a program, of varying complexity and
        sophistication, which can be used for ad hoc queries with minimal
        investment of time.

    \item Databases are easily extensible, e.g.:

        \begin{itemize}

            \item Other tables can be added to the database, e.g.\ to cache
                historical, summary or computed data.

            \item New columns can be added to the tables used by the
                program, with sufficient DEFAULT clauses or a clever
                TRIGGER or two.\footnote{Please refer to an \SQL{} guide
                for explanations of these terms,
                e.g.~\cite{sql-for-web-nerds}}

            \item A VIEW gives a custom arrangement of data with very
                little effort.

            \item If the database supports it, access can be granted on a
                fine-grained basis, e.g.\ allowing the finance department
                to produce invoices, the helpdesk to run limited queries as
                part of dealing with support calls, and the administrators
                to have full access to the data.

            \item Triggers can be written to perform actions when certain
                events occur.  In pseudo-\SQL{}\@:

\begin{verbatim}
CREATE TRIGGER ON INSERT INTO results
    WHERE sender = 'boss@example.com'
        AND postfix_action = 'REJECTED'
    SEND PANIC EMAIL TO 'postmaster@example.com';
\end{verbatim}

        \end{itemize}


    \item \SQL{} is reasonably standard and many people will already be
        familiar with it; for those unfamiliar with it there are lots of
        readily available resources from which to learn (a good
        introduction to \SQL{} can be found at~\cite{sql-for-web-nerds},
        others are~\cite{w3schools-sql-tutorial, sqlcourse.com}).  Although
        every vendor implements a different dialect of \SQL{}, the basics
        are the same everywhere (analogous to the overall similarities and
        minor differences between Irish English, British English, American
        English and Australian English).  Depending on the database in use
        there may be tools available which reduce or remove the requirement
        to know \SQL{}; for SQLite (the default database used by the
        implementation) there are several available~\cite{sqlite-guis}.

\end{enumerate}

Storing the results in a database will also increase the efficiency of
using those results, as the log files need only be parsed once rather than
each time the data is used; indeed the results may be used by someone with
no access to the original log files.



\subsection{Simple Mail Transfer Protocol}

\label{SMTP background}

The \SMTPlong{}, originally defined in \RFC{}~821~\cite{RFC821} and updated
in \RFC{}~2821~\cite{RFC2821}, is used for transferring mail between the
sending and receiving \MTA{}\@.  It is a simple, human readable, plain text
protocol, making it quite easy to test and debug problems with it.  Despite
the simplicity of the protocol many virus and/or spam sending programs fail
to implement it properly, so requiring strict adherence to the protocol
specification is beneficial in protecting against spam and
viruses.\footnote{\label{footnote:rfc760}Originally all mail servers
adhered to the principle of \textit{Be liberal in what you accept, and
conservative in what you send\/} from \RFC{}~760~\cite{rfc760}, but
unfortunately that principle was written in a friendlier time.  Given the
deluge of spam that mail servers are subjected to daily, a more appropriate
maxim could be: \textit{Require strict adherence to \RFC{}~2821; implement
the strongest restrictions you can; relax the restrictions and adherence
only when legitimate mail is impeded.\/}  it is not as friendly, nor as
catchy, but it more accurately reflects the current situation.} A typical
\SMTP{} conversation resembles the following (the lines starting with a
three digit number are sent by the server, all other lines are sent by the
client):

\begin{verbatim}
220 smtp.example.com ESMTP
HELO client.example.com
250 smtp.example.com
MAIL FROM: <alice@example.com>
250 2.1.0 Ok
RCPT TO: <bob@example.com>
250 2.1.5 Ok
DATA
354 End data with <CR><LF>.<CR><LF>
Message headers and body sent here.
.
250 2.0.0 Ok: queued as D7AFA38BA
QUIT
221 2.0.0 Bye
\end{verbatim}

An example deviation from the protocol:

\begin{verbatim}
220 smtp.example.com ESMTP
HELO client.example.com
250 smtp.example.com
MAIL FROM: Alice N. Other alice@example.com
501 5.1.7 Bad sender address syntax
RCPT TO: Bob in Sales/Marketing bob@example.com
503 5.5.1 Error: need MAIL command
DATA
503 5.5.1 Error: need RCPT command
Message headers and body sent here.
.
502 5.5.2 Error: command not recognized
QUIT
221 2.0.0 Bye
\end{verbatim}

This client is so poorly written that not only does it present the sender
and recipient addresses improperly, it ignores the error messages returned
by the server and carries on regardless.  There are many spam and virus
sending programs which are this deficient --- unfortunately others
(particularly newer programs) were written by competent programmers, or
utilise competently written programs (e.g.\ Postfix or Sendmail on Unix
hosts, Microsoft Outlook on Windows hosts).  Traditionally a mail server
would have done its best to deal with deficient clients, with the intention
of accepting as much mail destined for its users as
possible,\footref{footnote:rfc760} e.g.\ by ignoring the absence of a HELO
command, or accepting sender or recipient addresses which were not enclosed
in \texttt{<>}.  

A detailed description of \SMTP{} is beyond the scope of this document:
introductory guides can be found at~\cite{smtp-intro-01, smtp-intro-02},
the definitive references are~\cite{RFC821, RFC2821}.

\subsection{Postfix}

\label{postfix background}

Postfix is a \MTA{} with the following design aims (in order of
importance): security, flexibility of configuration, scalability, and high
performance.  It features extensive, extensible, optional anti-spam
restrictions, allowing an administrator to deploy those restrictions which
they judge suitable for their site's needs, rather than a fixed set chosen
by Postfix's author.  These restrictions can be selectively applied,
combined and bypassed on a per-client, per-recipient or per-sender basis,
allowing varying levels of stricture and/or permissiveness.  Postfix
leverages simple lookup tables to support arbitrarily complicated
user-defined sequences of restrictions and exceptions, with policy
servers\footnote{Policy servers will be explained in \sectionref{policy
servers}.} as the ultimate in flexibility.  Administrators can also supply
their own rejection messages to make it clear to senders why exactly their
mail was rejected.  Unfortunately this flexibility has a cost: complexity
in the log files generated.  While it is easy to use standard Unix text
processing utilities to determine the fate of an individual email,
following the journey an email takes through Postfix can be quite
difficult.  For the majority of mails the journey is simple and brief, but
the remaining minority can be quite complex (see \sectionref{additional
complications} for details).

Postfix's design follows the Unix philosophy of \textit{Write programs that
do one thing and do it well\/}~\cite{unix-philosophy}, and is separated
into various component programs to perform the tasks required of an
\MTA{}\@: receive mail, send mail, local delivery of mail, etc.\ --- full
details can be found in~\cite{postfix-overview}.  Each log line contains
the name of the Postfix component which produced it, and this information
is used when determining which rules should be used to parse each log line
(see \sectionref{rule characteristics} for details).  This design has
positive security implications also: those components which interact with
other hosts are not privileged,\footnote{Privilege in this case means the
power to perform actions that are limited to the administrator, and not
available to ordinary users.} so bugs in those components will not give an
attacker extra privileges; those components which are privileged do not
interact with other hosts, making it much more difficult for an attacker to
exploit any bugs which may exist in those components.

\subsubsection{Mixing and matching Postfix restrictions}

\label{Mixing and matching Postfix restrictions}

Postfix restrictions are documented fully in~\cite{smtpd_access_readme,
smtpd_per_user_control, policy-servers}; the following is a brief overview
only.

Postfix uses one restriction list (containing zero or more restrictions)
for each stage of the \SMTP{} conversation: client connection, HELO
command, MAIL FROM command, RCPT TO commands, DATA command, and end of
data.  The appropriate restriction list is evaluated for each stage
(evaluation will be explained shortly), though by default the restriction
lists for client connection, HELO and MAIL FROM commands will not be
evaluated until the first RCPT TO command is received, because some clients
do not deal properly with rejections before RCPT TO\@; a benefit of this
delay is that Postfix has more information available when logging
rejections.

Each restriction is evaluated to produce a result of \textit{reject},
\textit{permit}, \textit{dunno\/} or the name of another restriction to be
evaluated.\footnote{Other results are possible as described
in~\cite{smtpd_access_readme, smtpd_per_user_control, policy-servers}.} The
meaning of \textit{permit\/} and \textit{reject\/} is fairly obvious;
\textit{dunno\/} means to stop evaluating the current restriction and
continue processing the remainder of the restriction list, allowing more
specific cases to be used as exceptions to more general cases.  When the
result is the name of another restriction Postfix will evaluate the new
restriction, allowing restrictions to be chosen based on the client \IP{}
address, HELO hostname, sender address, recipient address,
etc.\footnote{E.g.\ the administrator may require that all clients on the
local network have valid \DNS{} entries, to prevent people sending mail
from unknown machines.}  The administrator can define new restrictions as a
list of existing restrictions, allowing arbitrarily long and complex
sequences of lookups, restrictions and exceptions.  Postfix tries to
protect the administrator in as far as is reasonable, e.g.\ the restriction
\texttt{check\_helo\_mx\_access} cannot cause a mail to be accepted,
because the parameter it checks (the hostname given in the HELO command) is
under the control of the remote client.  Despite this, it is possible for
the administrator to make catastrophic mistakes, e.g.\ rejecting mail for
all users --- the administrator must be cognisant of the ramifications of
their configuration changes.  XXX UNIX DOESN'T STOP YOU DOING STUPID THINGS
BECAUSE THAT WOULD ALSO STOP YOU DOING CLEVER THINGS\@.

Postfix uses simple lookup tables as the deciding factor when evaluating
some restrictions, e.g.\ in the example restriction below\newline
\tab{}\texttt{check\_client\_access~cidr:/etc/postfix/client\_access}
\newline The line above can be broken down as follows:

\begin{description}

    \item [check\_client\_access] The name of the restriction to evaluate.

    \item [cidr] The type of the lookup table.

    \item [/etc/postfix/client\_access] The file containing the lookup
        table.

\end{description}

The restriction \texttt{check\_client\_access} checks whether the \IP{}
address of the connected client is found in the file and returned the
associated action if found; the method of searching the file is dependant
on the type of the file (\texttt{cidr} in the example) --- see
\cite{postfix-lookup-tables} for more details.  Other restrictions
determine their result by consulting external sources, e.g.\
\texttt{reject\_rbl\_client dnsbl.example.com} checks whether the \IP{}
address of the client is present in the \DNSBL{}
\texttt{dnsbl.example.com}, rejecting the command if the client is listed.
This description is necessarily brief, for further details
see~\cite{smtpd_access_readme, smtpd_per_user_control, policy-servers}.


\subsubsection{Policy servers}

\label{policy servers}

A \textit{policy server\/}~\cite{policy-servers} is an external program
consulted by Postfix to determine the fate of an \SMTP{} command.  The
policy server is given state information\footnote{Sample state information
is shown in \tableref{Example attributes sent to policy servers}} and
returns a verdict as described in \sectionref{Mixing and matching Postfix
restrictions}.  The policy server can perform more complex checks than
those provided by Postfix: a trivial example is restricting mail from
addresses associated with the payroll system to sending mail on the third
Tuesday after pay day only, to help prevent problems from spam or (worse)
phishing mails with faked sender addresses.\footnote{A phishing mail might
claim that the payroll system had a disastrous disk failure; until the
server is replaced and restored all salary payments will have to be
processed manually, so please reply to this mail with your name, address
and back account details.}

Some widely deployed policy servers:

\begin{itemize}

    \item Checking \SPF{} records~\cite{openspf, Wikipedia-spf}.
        \SPF{}\label{spf introduction} records specify which mail servers
        are allowed to send mail claiming to be from a particular domain.
        The intention is to reduce spam from faked sender addresses,
        backscatter~\cite{postfix-backscatter} and
        joe~jobs~\cite{Wikipedia-joe-job}; however there has been a lot of
        resistance to the proposal because it breaks or vastly complicates
        some features of \SMTP{}, e.g.\ forwarding mail from one company or
        university to another when a user changes jobs.

    \item Greylisting~\cite{greylisting} is a technique that temporarily
        rejects mail when the triple of (sender address, recipient address,
        remote \IP{} address) is unknown; on second and subsequent delivery
        attempts from that triple the mail will be accepted.  The
        assumption is that maintaining a list of failed addresses and
        retrying after a temporary failure is uneconomical for a spam
        sender, but that a legitimate mail server must retry.  Sadly spam
        senders are using increasingly complex and well written programs to
        distribute spam, frequently using an \ISP{} provided \SMTP{} server
        from a compromised machine on the \ISP{}'s network.  Greylisting
        will slowly become less useful, but it does block a large
        percentage of spam mail at the moment; the most effective
        restrictions over the \numberOFlogFILES{} log files used in testing
        the parser are shown in \tableref{Summary of rejections}.
        Greylisting is obviously worth using, at least at the moment,
        particularly when you factor in Greylisting's position as the final
        restriction which a mail must overcome:\footnote{Greylisting is the
        final restriction a mail must overcome in the configuration used on
        the mail server the log files were obtained from; an administrator
        is free to use Greylisting at whichever position in the restriction
        list they feel is most appropriate for their mail system.}
        Greylisting only takes effect for mails which have passed every
        other restriction.

        \begin{table}[ht]
            \caption{Summary of rejections}\label{Summary of rejections}
            \input{build/restriction-table-include.tex}
        \end{table}

    \item Using a scoring system such as
        Policyd-weight~\cite{policyd-weight} where tests accumulate points
        against the sending system --- if the eventual score is higher than
        a threshold the mail is rejected.  Postfix's restrictions are
        binary, and the administrator must manually whitelist clients if
        they are to bypass a restriction; using a threshold which requires
        hitting several restrictions frees the administrator from
        whitelisting clients which fall foul of one restriction only.

    \item Rate limiting on a per-sender, per-client or per-recipient basis
        as performed by Policyd~\cite{policyd}.

\end{itemize}

Example attributes taken from~\cite{policy-servers}:

\begin{table}[ht]

    \caption{Example attributes sent to policy servers}\label{Example
    attributes sent to policy servers}

    \begin{tabular}[]{ll}

        request                 & smtpd\_access\_policy     \\
        protocol\_state         & RCPT                      \\
        protocol\_name          & SMTP                      \\
        helo\_name              & some.domain.tld           \\
        queue\_id               & 8045F2AB23                \\
        sender                  & foo@bar.tld               \\
        recipient               & bar@foo.tld               \\
        recipient\_count        & 0                         \\
        client\_address         & 1.2.3.4                   \\
        client\_name            & another.domain.tld        \\
        reverse\_client\_name   & another.domain.tld        \\
        instance                & 123.456.7                 \\

    \end{tabular}
\end{table}



\subsection{Assumptions}

The algorithm described and the program implementing it make a small number
of (hopefully safe and reasonable) assumptions:

\begin{itemize}

    \item The log files are whole and complete: nothing has been removed,
        either deliberately or accidentally (e.g.\ log rotation gone awry,
        file system filling up, logging system unable to cope with the
        volume of log messages).  On a well run system it is extremely
        unlikely that any of these problems will arise, though it is of
        course possible, particularly when suffering a deluge of spam or a
        mail loop.

    \item Postfix logs sufficient information to make it possible to
        accurately reconstruct the actions it has taken.  There are a
        number of heuristics used when parsing; see
        \sectionref{identifying-bounce-notifications},
        \sectionref{aborted-delivery-attempts} and \sectionref{pickup
        logging after cleanup} for details.

    \item The Postfix queue has not been tampered with, causing unexplained
        appearance or disappearance of mail.  This may happen if the
        administrator deletes mail from the queue without using
        \daemon{postsuper}, or if there is filesystem corruption.

\end{itemize}

In some ways this task is similar to reverse engineering or replicating a
black box program based solely on its inputs and outputs.  Although the
source code is available,\footnote{Reading and understanding the source
code would require a significant investment of time: there are 375,750
lines of code, documentation, etc.\ in Postfix 2.5.1's 17MB of source
code.} there are advantages to treating Postfix as a black box while
developing the parser:

\begin{itemize}

    \item The parser is developed using real world log files rather than
        the idealised log files someone would naturally envisage reading
        the source code.

    \item The source code cannot accurately communicate the variety of
        orderings in which log lines are written to the log file, as
        process scheduling independently interferes with logging and other
        processing.

    \item The parser acts as a second source of information, with the
        information gathered from empirical evidence.  An interesting
        project would be to compare the empirical knowledge inherent in the
        parsing algorithm with the documentation and source code of
        Postfix.

\end{itemize}


\subsection{Parser design}

\label{parser design}

It should be clear from the earlier Postfix background (\sectionref{postfix
background}) that log files produced by Postfix are not fixed; they vary
widely from host to host, depending on the set of restrictions chosen by
the administrator.  With this in mind, one of the parser's design aims was
to make adding new rules as easy as possible, to enable administrators to
properly parse their log files.  To enable this the parser is divided into
three parts:

\begin{description}

    \item [rules] Rules match individual log lines and determine which
        actions will be executed.  Rules provide an easily extensible
        method of associating log lines with actions, and are described in
        detail in \sectionref{rules}.

    \item [actions] Actions are invoked to deal with a log line once it has
        been identified by the rules: actions modify data structures,
        handle complications, and cause data to be saved to the database.
        The actions perform the work of reconstructing the journey a mail
        takes through Postfix.  Full details of actions can be found in
        \sectionref{actions-in-detail} and \sectionref{adding new actions}
        XXX IMPROVE THIS SENTENCE\@.

    \item [framework] The framework is responsible for loading rules,
        managing data structures, finding the rule which matches each log
        line, invoking the correct action, etc\@.  The framework provides
        the structure which actions and rules plug into.  The framework is
        described in detail in \sectionref{framework}.  XXX EXTEND
        FRAMEWORK SECTION IF NECESSARY\@.

\end{description}

It may help to think of the rules and actions as components which plug into
the framework.  

\label{why separate rules, actions and framework?}

XXX MERGE THE NEXT TWO PARAGRAPHS\@.

Decoupling the parsing rules from the associated actions and framework
allows new rules to be written and tested without requiring modifications
to the parser source code (significantly lowering the barrier to entry for
new or casual users who need to parse new log lines), and greatly
simplifies framework, actions and rules.  Decoupling also creates a clear
separation of functionality: rules handle low level details of identifying
log lines and extracting data from a log line; actions handle the higher
level details of following the path a mail takes through Postfix,
assembling the required data before storing it, dealing with complications
arising, etc; the framework provides services to actions and stores data.

Decoupling the actions from the framework simplifies both framework and
actions: the framework provides services to the actions, and does not need
to deal with the complications which arise, or the task of reconstructing a
mail's journey through Postfix; actions benefit from having services
provided by the framework, freeing them to concentrate on the task of
accurately reconstructing each mail's journey through Postfix and dealing
with the complications described in \sectionref{additional complications}.

XXX CLARIFY THE SENTENCE STARTING WITH ``Although this may see'' IN THE
NEXT PARAGRAPH\@.

Separating the rules from the actions and framework makes it possible to
parse new log lines without modifying the core parsing algorithm.  Although
this may seem like a trivial point, is it substantially more difficult to
understand a program's entire parsing algorithm, identify the correct
location to change and make the appropriate changes, versus adding a new
rule with the action to invoke, a \regex{} to match the log lines, and the
specification of the data to extract.  Bear in mind that the changes must
be made without adversely affecting existing parsing, particularly as there
may be edge cases which are not immediately obvious.\footnote{See
\sectionref{yet-more-aborted-delivery-attempts} for a complication which
occurs only four times in \numberOFlogFILES{} log files tested.}  Requiring
changes to the parser's code also complicates upgrades, as the changes must
be preserved during the upgrade, and may clash with changes made by the
developer.\footnote{See \sectionref{adding new actions} for how to add new
actions}  \parsername{} allows the user to add new rules to the database
without changing the parsing algorithm, unless the new log lines to be
parsed require functionality not already provided by the algorithm.  If the
new log lines do require new functionality, new actions can be added to the
parser without modifying existing actions or other parts of the algorithm;
only in the rare case that the new actions require support from other
sections of the code will more extensive changes be required.

XXX CHANGE THIS\@; THE PARSER DESIGN IS NOVEL, DO NOT STRESS THE SIMILARITY
TOO MUCH\@.

There is some similarity between the parser's design and William Wood's
\ATN{}~\cite{atns, nlpip}, a tool used in Computational Linguistics for
creating grammars to parse or generate sentences.  The resemblance between
\ATN{} and the parser is accidental, but it is interesting how two
apparently different approaches share an underlying separation of concerns;
this appears to be a natural division of responsibility and functionality.

% Do Not Reformat!

\begin{tabular}[]{lll}
    \textit{\ATN{}\/}   & \textit{Parser\/} & \textit{Similarity\/}     \\
    Networks            & Algorithm         & Determines the sequence 
                                              of transitions            \\
                        &                   & or actions which 
                                              constitutes a valid       \\
                        &                   & input.                    \\
    Transitions         & Actions           & Save data and impose
                                              conditions the            \\
                        &                   & input must meet to be
                                              considered valid.         \\
    Abbreviations       & Rules             & Responsible for 
                                              classifying input.        \\
\end{tabular}

\subsection{Conventions used in the document}

The words \textit{connection\/} and \textit{mail\/} are often used
interchangeably in this document; in general the word used was chosen based
on the context it appears in.

\subsection{Summary}

This section has provided background information on several topics relevant
to the remainder of the document.  It started with the motivation behind
the project, continuing with explanations of:

\begin{itemize}

    \item Using a database as an \API{}.

    \item \SMTP{}.

    \item Postfix restrictions and policy servers.

    \item Assumptions made when designing and developing the parser.

    \item A description of the parser's novel design.

    \item Conventions used in this document.

    \item An in-depth comparison of this parser with \LMA{}, the parser
        described in previously published research.

    \item A review of the literature previously published in this area.

\end{itemize}


\section{Database schema}
\label{database schema}

The database is an integral part of the parser presented here: it stores
the rules and the data gleaned by applying those rules to Postfix log
files.  Understanding the database schema is important in understanding the
actions of the parser, and essential to developing further applications
which utilise the data gathered.

\subsection{Introduction}

The database schema can be conceptually divided in two: the rules which are
used to parse log files, and the data saved from the parsing of log files.
Rules have the fields required to parse the log lines, extract data to be
saved, and the action to be executed; they also have several fields which
aid the user in understanding what each rule parses.  The rules are
described in detail in \sectionref{rules} but the fields are covered here
in \sectionref{rule attributes}.

The data saved from parsing the logs is also divided into two tables as
described below: connections and results.  The connections table contains a
row for every mail accepted and every connection where there was a
rejection; the individual fields will be described in
\sectionref{connections table}.  The results table has one or more rows per
row in the connections table; the fields will be covered in detail in
\sectionref{results table}.

An important but easily overlooked benefit of storing the rules in the
database is the connection between rules and results: if more information
is required when examining a result, the rules which produced the database
entries are available for inspection (and each result references the rule
which created it).  There is no ambiguity about which rule resulted in a
particular result, eliminating one potential source of confusion.

\subsection{Rules table}

\label{rule attributes}

Rules are discussed in detail in \sectionref{rules}, but the structure of
the rules table is covered here.  Rules are created by the user, not the
parser, and will not be modified by the parser (except for the hits and
hits\_total fields).  Rules parse the individual log lines, extracting data
to be saved in the connections and results tables, and specifying action to
take for that log line.

Each rule defines the following:

\begin{description}

    \item [id] A unique identifier for each rule which other tables can use
        to refer to a specific rule.

    \item [name] A short name for the rule.

    \item [description] Something must have occurred to cause Postfix to
        log each line (e.g.\ a remote client connecting causes a connection
        line to be logged).  This field describes the action causing the
        log lines this rule matches.

    \item [restriction\_name] The restriction which caused the mail to be
        rejected.  Only applicable to rules which have an action of
        \texttt{REJECTION}, other rules should have an empty string.

    \item [postfix\_action] This is the action Postfix must have taken to
        generate this log line.  This field is mostly ignored, but two
        values (IGNORED and INFO) have special meaning, as described below
        in the list of typical values.\label{postfix_action}

        \begin{description}

            \item [ACCEPTED] Postfix has accepted a mail, and will
                subsequently attempt to deliver it.

            \item [BOUNCED] The mail has bounced, due to a mail loop,
                delivery failure, or five day timeout.

            \item [DELETED] The mail was deleted from the queue by an
                administrator.

            \item [DISCARDED] Postfix discarded the mail it was in the
                process of accepting, because it was either larger than the
                limit set by the administrator, or the client timed out or
                disconnected.

            \item [EXPIRED] The mail has been in the queue for five
                days\footnote{As with much of Postfix's behaviour, this is
                the default value but can be changed by the administrator
                if they choose.} without successful delivery.  A bounce
                mail will be generated and sent to the sender address.

            \item [INFO] Represents an unspecified intermediate action that
                the parser is not interested in per se, but which does log
                useful information, supplementing other log lines.

            \item [IGNORED] An action which is not only uninteresting in
                itself, but which also provides no useful data.

            \item [POSTFIX\_RELOAD] The administrator has instructed
                Postfix to start or stop, and all existing \daemon{smtpd}
                processes will be terminated.  This does not negatively
                impact on the logs or mail queued by Postfix for delivery.

            \item [PROCESSING] \daemon{cleanup} is processing a mail ---
                see~\cite{postfix-cleanup} for details of the processing
                performed by \daemon{cleanup}.

            \item [REJECTED] Postfix rejected a command from the remote
                client, causing at least one recipient to be rejected.

            \item [SENT] Postfix has successfully sent a mail.

        \end{description}

        Uninteresting log lines are parsed so that any lines the parser
        isn't capable of handling become immediately obvious errors.

    \item [program] The program (\daemon{smtpd}, \daemon{qmgr}, etc.) whose
        log lines the rule applies to.  This avoids needlessly trying rules
        which won't match the log line, or worse, might match
        unintentionally.  Rules whose program is \texttt{*} will be tried
        against any log lines which aren't parsed by program specific
        rules.

    \item [regex] The \regex{} to match the log line against.  The \regex{}
        will first have several keywords expanded: this simplifies reading
        and writing rules; avoids needless repetition of complex \regex{}
        components; allows the components to be corrected and/or improved
        in one location; and makes each \regex{} largely self-documenting.

        For efficiency the keywords are expanded and every rule's \regex{}
        is compiled before attempting to parse the log file --- otherwise
        each \regex{} would be recompiled each time it was used, resulting
        in a large, data dependent slowdown.  Rule efficiency concerns are
        discussed in \sectionref{rule efficiency}, with the impact of
        compiling and caching \regexes{} covered in \sectionref{Caching
        each regex}.

    \item [result\_cols, connection\_cols] Specifies how the fields in the
        log line will be extracted.  The format is:
        \newline \tab{} \texttt{smtp\_code = 1; recipient = 2, sender = 4;}
        \newline i.e.\ semi-colon or comma separated assignment statements,
        with the variable name on the left and the matching capture from
        the \regex{} on the right hand side.  The list of acceptable
        variable names is:

        \texttt{connection\_cols: client\_hostname, client\_ip, server\_ip,
        \newline \tab{} server\_hostname} and \texttt{helo.\newline}
        \texttt{result\_cols: sender, recipient, smtp\_code, message\_id,
        \newline \tab{} size,} and \texttt{data}

        Additionally \texttt{child} and \texttt{pid} are used respectively
        by the \texttt{TRACK}, \texttt{BOUNCE} and \texttt{SMTPD\_DIED}
        actions.

    \item [result\_data, connection\_data] Sometimes rules need to supply a
        piece of data which isn't present in the log line: e.g.\ setting
        \texttt{smtp\_code} when mail is accepted.  The format and allowed
        variables are the same as for \texttt{result\_cols} and
        \texttt{connection\_cols}, except that arbitrary
        data\footnote{Commas and semi-colons cannot be escaped and thus
        cannot be used.  This is intended for use with small amounts of
        data rather than large amounts in any one rule, so dealing with
        escape sequences was deemed unnecessary.} is permitted on the right
        hand side of the assignment.

    \item [action] The action that will be invoked when this rule matches a
        log line; a full list of actions and the parameters they are
        invoked with can be found in \sectionref{actions-in-detail}.

    \item [queueid] Specifies the matching capture from the \regex{} which
        gives the queueid, or zero if the log line doesn't contain a
        queueid.  Many log lines won't contain a queueid, e.g.\ rejections
        logged before a mail has been accepted (a queueid won't have been
        allocated), or log lines which aren't tied to one particular mail.

    \item [hits] This counter is maintained for every rule and incremented
        each time the rule successfully matches.  At the start of each run
        the program sorts the rules in descending order of hits, and at the
        end of the run updates every rule's hits.  Assuming that the
        distribution of log lines is reasonably consistent between log
        files, rules matching more commonly occurring log lines will be
        tried before rules matching less commonly occurring log lines,
        lowering the program's execution time.  Rule ordering for
        efficiency is discussed in \sectionref{rule ordering for
        efficiency}.

    \item [hits\_total] The total number of hits for this rule over all
        runs of the parser.

    \item [priority] This is the user-configurable companion to hits: rules
        will be tried in order of priority, overriding hits.  This allows
        more specific rules to take precedence over more general rules
        (described in \sectionref{overlapping rules}).

    \item [cluster\_group] A reference to the \texttt{cluster\_group}
        table.  That table is used by the Decision Tree algorithm described
        in a separate document.

\end{description}


\subsection{Connections table}

\label{connections table}

Every accepted mail and every connection where there was a rejection will
have a single entry in the connections table containing the following
fields:

\begin{description}

    \item [id] This field uniquely identifies the row.

    \item [server\_ip] The \IP{} address (IPv4 or IPv6) of the server: the
        local server when receiving mail, the remote server when sending
        mail.

    \item [server\_hostname] The hostname of the server, it will be
        \texttt{unknown} if the \IP{} address could not be resolved to a
        hostname via \DNS{}\@.

    \item [client\_ip] The client \IP{} address (IPv4 or IPv6): the remote
        server when receiving mail, the local server when sending mail.

    \item [client\_hostname] The hostname of the client, it will be
        \texttt{unknown} if the \IP{} address could not be resolved to a
        hostname via \DNS{}\@.

    \item [helo] The hostname used in the HELO command.  The HELO hostname
        occasionally changes during a connection, presumably because spam
        or virus senders think it's a good idea.  By default Postfix only
        logs the HELO hostname when it rejects an \SMTP{} command, but it
        is quite easy to rectify this:

\label{logging helo}

        \begin{enumerate}

            \item Create \texttt{/etc/postfix/log\_helo.pcre}
                containing:\newline \tab{}\texttt{/./~~~~WARN~Logging~HELO}

            \item Modify \texttt{smtpd\_data\_restrictions} in
                \texttt{/etc/postfix/main.cf} to contain\newline
                \tab{}\texttt{check\_helo\_access~/etc/postfix/log\_helo.pcre}

        \end{enumerate}

        Although \texttt{smtpd\_helo\_restrictions} seems like the natural
        place to log the HELO hostname, there won't be a queueid associated
        with the mail for the first recipient, so that log line cannot be
        associated with the correct mail.  There is guaranteed to be a
        queueid when the DATA command has been reached, and thus it will be
        logged by any restrictions taking effect in
        \texttt{smtpd\_data\_restrictions}.  There is no difficulty in
        specifying a HELO-based restriction in
        \texttt{smtpd\_data\_restrictions}, Postfix will perform the check
        correctly.

        Logging the HELO hostname in this fashion also prevents the
        complication described in \sectionref{Mail deleted before delivery
        is attempted} from occurring when, but only in the case where there
        is a single recipient; in that case the recipient address will be
        logged also, but when there are multiple recipients none are
        logged.  It is also possible to warn for every recipient,
        preventing the complication in \sectionref{Mail deleted before
        delivery is attempted} entirely.

    \item [queueid] The queueid of the mail if the connection represents an
        accepted mail, or \texttt{NOQUEUE} otherwise.

    \item [start] The timestamp of the first log line, in seconds since the
        epoch (explained in the glossary, appendix~\refwithpage{Glossary}).

    \item [end] The timestamp of the last log line, in seconds since the
        epoch.

\end{description}

\subsection{Results table}

\label{results table}

Every log line where the associated rule has a \texttt{postfix\_action}
will have an entry in the results table, e.g.\ rejecting an \SMTP{}
command, delivering a mail, or bouncing a mail (see
\sectionref{postfix_action} for typical values of
\texttt{postfix\_action}).  Each row is associated with a single
connection, though there may be many results per connection.

\begin{description}

    \item [connection\_id] A reference to the row in the connections table
        this result is associated with.

    \item [rule\_id] A reference to the entry in the rules table which
        matched the log line and created this result.

    \item [id] A unique identifier for this result.

    \item [warning] Postfix can be configured to log a warning instead of
        enforcing a restriction that would reject an \SMTP{} command --- a
        facility that is quite useful for testing new restrictions.  This
        field will be 1 if the log line parsed was a warning rather than a
        real rejection, or 0 for a real rejection.  This field should be
        ignored if the result is not a rejection.

    \item [smtp\_code] The \SMTP{} code associated with the log line.  In
        general an \SMTP{} code is only present for a rejection or final
        delivery; results initially missing an \SMTP{} code will duplicate
        the \SMTP{} code of other results in the connection.  Some final
        delivery log lines don't contain an \SMTP{} code: in those cases
        the code is faked based on the success or failure represented by
        the log line.

    \item [sender] The sender's email address.  Because multiple mails may
        be delivered during one connection, there may be different sender
        addresses in the results for one connection; however there should
        not be different sender addresses in the results for one email.

    \item [recipient] The recipient address; there may be multiple
        recipient addresses per mail, but only one per result.

    \item [size] The size of the mail; it will only be present for
        delivered mails.

    \item [message\_id] The message-id of the accepted mail, or
        \texttt{NULL} if no mail was accepted.

    \item [data] A field available for anything not covered by other
        fields, e.g.\ the rejection message from an \RBL{}\@.

    \item [timestamp] The time the log line was logged, in seconds since
        the epoch.

\end{description}

\subsection{Conclusion}

The table containing the rules used by the parser and both tables
containing the data extracted from the Postfix logs were described, with
the purpose of each field discussed in detail.  A clear, comprehensible
schema is essential when using the extracted data; it's more important when
using the data than when storing it, because storing the data is a
write-once operation, whereas utilising the data requires frequent
searching, sorting and manipulation of the data to produce customised
reports and/or statistics.


\section{Parsing rules}

\label{rules}

\subsection{Introduction}

This section discusses the rules used in parsing Postfix log files,
starting with rule characteristics, followed by the problems caused by
overlapping rules, and techniques to detect and deal with such rules.  An
example rule and a log line it would match are provided, plus a description
of how the fields in the rule\footnote{The details of the table containing
the rules have already been described in \sectionref{rule attributes}.} are
used when matching a log line and subsequently performing the requested
action.  The section continues with a discussion of rule efficiency
concerns, referring to the graphs in \sectionref{graphs}, and finishes with
an explanation of the algorithm used to generate new \regexes{} from
unparsed lines.

Please refer to \sectionref{parser design} for a discussion of why the
rules and actions have been separated in the parser's design.

\subsection{Rule characteristics}

\label{rule characteristics}

Rule have certain characteristics which may help in understanding the
parser:

\begin{itemize}

    \item Rules are annotated with the name of a Postfix program, and will
        only be used when parsing log lines produced by that
        program.\footnote{There are also generic rules which are used when
        parsing log lines produced by any Postfix program, but only if
        there are also rules specific to that program, and those rules have
        already have been tried and failed on the current log line.}  Any
        given rule will only be used to parse a subset of the log lines,
        and any given log line will only be parsed by a subset of the
        rules.

    \item The first matching rule wins: no further rules are tried against
        that log line, but there is a facility for ordering the rules so
        that more specific rules can be tried first.

    \item Rules are completely self-contained and can be understood in
        isolation, without reference to any other rules.

    \item Rule processing time is a linear function of the number of rules.

\end{itemize}

\label{comparison against context-free grammars}

In context-free grammar terms the parser rules could be described as:

$\text{\textless{}log-line\textgreater{}} \mapsto \text{rule-1} |
\text{rule-2} | \text{rule-3} | \dots | \text{rule-n}$


\subsection{Overlapping rules}

\label{overlapping rules}

The parser does not try to detect overlapping rules;\footnote{It may be
possible to parse each rule's \regex{} and determine if any overlap.  The
author has not attempted to do this: such a project by itself would
probably qualify for a PhD, and may involve solving the Halting
Problem~\cite{Wikipedia-halting-problem} and circumventing the
Church-Turing Thesis~\cite{Wikipedia-church-turing-thesis}.} that
responsibility is left to the author of the rules.  Unintentionally
overlapping rules lead to inconsistent parsing and data extraction because
the order in which rules are tried against each line may change between log
files, and the first matching rule wins.  Overlapping rules are frequently
a requirement, allowing a more specific rule to match some log lines and a
more general rule to match the majority, e.g.\ separating \SMTP{} delivery
to specific sites from \SMTP{} delivery to the rest of the world.  The
algorithm provides a facility for ordering overlapping rules: the priority
field in each rule (defaults to zero).  Rules are sorted by priority,
highest first, and then rules with the same priority are sorted by the
number of successful matches when parsing the previous log file.  Negative
priorities may be useful for catchall rules.

Detecting overlapping rules is difficult, but the following approaches may
be helpful:

\begin{itemize}

    \item Sort by \regex{} and visually inspect the list, e.g.\ with \SQL{}
        similar to: \textbf{select regex from rules order by regex;}

    \item Compare the results of parsing using sorted, shuffled and
        reversed rules.\footnote{See \sectionref{rule efficiency} for more
        details of sorting the rules.}  Parse a number of log files using
        normal sorting, then dump a textual representation of the rules,
        connections and results tables.  Repeat with shuffled and reversed
        rules, starting with a fresh database.  If there are no overlapping
        rules the tables from each run will be identical; differences
        indicate overlapping rules.  Which rules overlap can be determined
        by examining the differences in the tables: each result contains a
        reference to the rule which created it, if the references differ
        between runs the two rules referenced in the differing records
        overlap.  Unfortunately this method cannot prove the absence of
        overlapping rules; it can detect overlapping rules, but only if
        there are log lines in the input files which match more than one
        rule.

\end{itemize}

\subsection{Example rule}

\label{example rule}

This example rule matches the message logged by Postfix when it rejects
mail from a sender address because the appropriate \DNS{} entries are
missing, i.e.\ mail could not be delivered to the sender's address (for
full details see~\cite{reject-unknown-sender-domain}).

The example rule below would match the following log line:

\begin{verbatim}
NOQUEUE: reject: RCPT from example.com[10.1.1.1]: 550
  <foo@example.com>: Sender address rejected: Domain not found;
  from=<foo@example.com> to=<info@example.net>
  proto=SMTP helo=<smtp.example.com>
\end{verbatim}

% Don't reformat this!
\begin{tabular}[]{ll}

\textbf{Field}      & \textbf{Value}                                    \\
name                & Unknown sender domain                             \\
description         & We do not accept mail from unknown domains        \\
restriction\_name   & reject\_unknown\_sender\_domain                   \\
postfix\_action     & REJECTED                                          \\
program             & \daemon{smtpd}                                    \\
regex               & \verb!^__RESTRICTION_START__ <(__SENDER__)>: !    \\
                    & \verb!Sender address rejected: Domain not found;! \\
                    & \verb!from=<\5> to=<(__RECIPIENT__)> !            \\
                    & \verb!proto=E?SMTP helo=<(__HELO__)>$!            \\
result\_cols        & recipient = 6; sender = 5                         \\
connection\_cols    & helo = 7                                          \\
result\_data        &                                                   \\
connection\_data    &                                                   \\
action              & REJECTION                                         \\
queueid             & 1                                                 \\
hits                & 0                                                 \\
hits\_total         & 0                                                 \\
priority            & 0                                                 \\
cluster\_group      & 400                                               \\

\end{tabular}

\vspace{1em}

Additional data will be captured automatically when the \regex{} contains 
\_\_RESTRICTION\_START\_\_, hence the capture numbers in result\_cols and
connection\_cols start at 5.  The various fields are used as follows;

\begin{description}

    \item [name, description, restriction\_name and postfix\_action:] are
        not \newline used by the algorithm, they serve to document the rule
        for the user's benefit.

    \item [program and regex:] If the program in the rule equals the
        program which logged the log line a match using the \regex{} will
        be attempted against the log line; if the match is successful the
        action will be executed, if not the next rule will be tried.  If
        the program-specific rules don't match the log line, the parser
        will fall back to generic rules; if those rules are unsuccessful a
        warning will be issued.

    \item [action:] will be executed if the \regex{} matches successfully
        (see \sectionref{actions-in-detail} for full details).

    \item [result\_cols, connection\_cols, result\_data and
        connection\_data:] are \newline used by the action to extract and
        save data matched by the \regex{}.

    \item [queueid:] The index of the capture in the \regex{} which
        supplies the queueid, or zero if the log line does not contain a
        queueid.  This allows the correct mail can be found by queueid and
        actions performed on it.

    \item [hits, hits\_total and priority:] hits and priority are used in
        ordering the rules (see \sectionref{rule ordering for efficiency});
        hits is set to the number of successful matches at the end of the
        parsing run; hits\_total is the sum of hits over every parsing run,
        but is otherwise unused by the algorithm.

    \item [cluster\_group] The cluster\_group attribute is used by the
        Decision Tree algorithm described in a separate document; the
        parser does not use it in any way.

\end{description}



\subsection{Rule efficiency}

\label{rule efficiency}

Parsing efficiency is an obvious concern when the parser routinely needs to
deal with 75 MB log files containing 300,000 log lines (generated daily on
a mail server handling mail for approximately 700 users --- large scale
mail servers would have much larger log files on a daily basis).  When
generating the data for the graphs included in \sectionref{graphs},
\numberOFlogFILES{} log files (totaling 10.08 GB, \numberOFlogLINEShuman{}
log lines) were each parsed 10 times, the first run discarded, and the
execution time for the remaining 9 runs averaged.  The first run is
discarded for two reasons:

\begin{enumerate}

    \item The execution time will be higher because the log file must be
        read from disk, whereas for subsequent runs the log file will be
        cached in memory by the operating system.

    \item The execution time will also be higher because the rule ordering
        will be sub-optimal compared to subsequent runs.

\end{enumerate}

Saving results to the database was disabled for the test runs, as that
dominates the run time of the program, and the tests are aimed at measuring
the speed of the parser rather than the speed of the database and the disks
the database is stored on.

\subsubsection{Algorithmic complexity}

An important property of a parser is how execution time scales relative to
input size: does it scale linearly, polynomially, or exponentially?
Graph~\refwithpage{execution time vs file size vs number of lines graph}
shows the execution time in seconds, file size in MB and tens of thousands
of log lines per log file.  All three lines run roughly in parallel, giving
a visual impression that the algorithm scales linearly with input size.
This impression is borne out by graph~\refwithpage{execution time vs file
size vs number lines factor} which plots the ratio of file size vs
execution time and ratio of number of log lines vs execution time (higher
is better).\footnote{Table~\refwithpage{execution time vs file size vs
number lines factor table} shows the ratios for different breakdowns of the
log files.}  As the reader can see the ratios are quite tightly banded,
showing that the algorithm scales linearly: the much larger log files
between points 60 and 70 on the X axis in graph~\refwithpage{execution time
vs file size vs number of lines graph} actually cause the ratio to
increase, rather than decrease.  The strange behaviour where larger log
files are parsed more efficiently is explained fully in \sectionref{Why are
there dips in the graphs?}.  

\subsubsection{Rule ordering for efficiency}

\label{rule ordering for efficiency}

Rule ordering was mentioned in \sectionref{rule attributes} and will be
covered in greater detail in this section.  At the time of writing there
are \numberOFrules{} different rules, with the top 10\% matching the vast
majority of the log lines, and the remaining log lines split across the
other 90\% of the rules (as shown in graph~\refwithpage{rule hits graph}).
Assuming that the distribution of log lines is reasonably steady over time,
program efficiency should benefit from trying more frequently matching
rules before those which match less frequently.  To test this hypothesis
three full test runs were performed with different rule orderings:

\begin{description}

    \item [normal]  The most optimal order, according to the hypothesis:
        rules which match most often will be tried first.

    \item [shuffle] Random ordering --- the rules will be shuffled once
        before use and will retain that ordering for the entirety of the
        log file.  Note that the ordering will change every time the parser
        is executed, so 10 different orderings will be generated for each
        log file in the test run.  This is intended to represent an
        unsorted rule set.

    \item [reverse] Hypothetically the worst order: the most frequently
        matching rules will be tried last.

\end{description}

Graphs~\refwithpage{percentage increase of shuffled over normal}
and~\refwithpage{percentage increase of reversed over normal} show the
percentage increase of execution times, with table~\refwithpage{Execution
time increase for different rule orderings} showing the mean increases for
different groupings of log files.  Overall this provides a modest but
worthwhile performance increase of approximately 9\%, for a small
investment in time and programming.

\subsubsection{Caching each regex}

\label{Caching each regex}

Perl compiles the original \regex{} into an internal representation,
optimising the \regex{} to improve the speed of matching, but this
compilation and optimisation takes CPU time; far more CPU time, in fact,
than the actual matching takes.  Perl automatically caches static
\regexes{}, but dynamic \regexes{} need to be explicitly compiled and
cached.  Graph~\refwithpage{normal regex vs discard regex} shows execution
times with and without caching the \regex{}.  Caching the compiled
\regexes{} is obviously far more efficient; graph~\refwithpage{normal regex
vs discarded regex factor} shows the percentage execution time increase
when not caching each \regex{}.

Caching the compiled \regexes{} is quite simple, and is the single most
effective optimisation implemented in the parser.

\subsection{Creating new rules}

\label{creating new rules}

The log files produced by Postfix differ from installation to installation,
because administrators have the freedom to choose the subset of available
restrictions which suits their needs, including using different \RBL{}
services, policy servers, or custom rejection messages.  To facilitate easy
parsing of new log lines, the parser's design separates parsing rules from
parsing actions: adding new actions is difficult, but adding new rules to
parse new rejection messages is trivial (and also occurs much more
frequently).  The implementation provides a program to ease the process of
creating new rules from unparsed log lines, based on the algorithm
developed by Risto Vaarandi~\cite{risto-vaarandi} for his
\SLCT{}~\cite{slct-paper}.  The differences between the two algorithms will
be outlined as part of the general explanation below.

The core of the algorithm is quite simple: log lines are generally created
by substituting variable words into a fixed pattern, and analysis of the
frequency with which each word occurs can be used to determine whether the
word is variable or part of the fixed pattern.  This classification can be
used to group similar log lines and generate a \regex{} to match each group
of log lines.

There are 4 steps in the algorithm:

\begin{description}

    \item [Pre-process the file]  The new algorithm leverages the knowledge
        gained while writing rules and performs a large number of
        substitutions on the log input lines, replacing commonly occurring
        variable terms (e.g.\ email addresses, \IP{} addresses, the
        standard start of rejection messages, etc.) with \regex{} keywords
        which the parser will expand when it loads the rule (see the
        \regex{} entry in \sectionref{rule attributes}).  The purpose of
        this phase is to utilise existing knowledge to create more accurate
        \regexes{}.  The new log lines are written to a temporary file,
        which all subsequent stages use instead of the original input file.

        In the original algorithm the purpose of the preprocessing stage
        was to reduce the memory consumption of the program.  In the first
        pass it generated a hash from a small range of values for each word
        of each log line, incrementing a counter for each hash.  The
        counters will be used in the next pass to filter out words: if the
        word's hash does not have a high frequency, the word itself cannot
        have a high frequency, and there is no need to maintain a counter
        for it, reducing the number of counters and thus the program's
        memory consumption.

    \item [Calculate word frequencies]  The position of words within a log
        line is important: a common word does not indicate similarity
        between log lines unless it occupies the same position within both
        log lines.\footnote{If a variable term within a line contains
        spaces, it will appear to the algorithm as two words rather than
        one.  This will alter the position of subsequent words, so a word
        occurring in different positions in two log lines \textit{may\/}
        indicate similarity, but the algorithm does not attempt to deal
        with this possibility.}  The algorithm maintains a counter for each
        \textit{(word, word's position within the log line)\/} tuple,
        incrementing it each time that word occurs in that position.

        The original algorithm only maintains counters for words whose hash
        result from the previous phase has a high frequency; this reduces
        the number of counters maintained by the algorithm, reducing the
        memory requirements of the algorithm.  The modified algorithm omits
        this check because the majority of unique or infrequently occurring
        words will have been substituted with keywords during the first
        phase, vastly reducing the number of tuples to maintain counters
        for.

    \item [Classify words based on their frequency]  The frequency of each
        \textit{(word, word's position within the log line)\/} tuple is
        checked: if its frequency is greater than the threshold supplied by
        the user (1\% of all log lines is generally a good starting point)
        it is classified as a fixed word, otherwise it is classified as a
        variable term.  If a variable term appears sufficiently often it
        will be classified as a fixed term, but that should be noticed by
        the user when reviewing the new \regexes{}.  Variable terms are
        replaced by \texttt{.+}, which means to match zero or more of any
        character.  

    \item [Build regexes]  The words are reassembled to produce a \regex{}
        matching the log line, and a counter is maintained for each
        \regex{}.  Contiguous sequences of \texttt{.+} in the newly
        reassembled \regexes{} are collapsed to a single \texttt{.+}; any
        resulting duplicate \regexes{} are combined, and their counters
        added.  If the frequency of a \regex{} is lower than the threshold
        supplied by the user the \regex{} is discarded.\footnote{This is a
        second threshold, independent of the threshold used to
        differentiate between fixed and variable words, but once again 1\%
        of input log lines is a good starting point; obviously the
        threshold depends on the number and type of input log lines.}  The
        new \regexes{} are printed for the user to add to the database,
        either as new rules or merged into the \regexes{} of existing
        rules; the counter for each \regex{} is also printed, giving the
        user an indication of how many of the input log lines that \regex{}
        should match.  Discarding \regexes{} will result in some of the
        input log lines not being matched; this utility should be run again
        once the unmatched log lines have been isolated by running the
        parser, including the new \regexes{}, with the same input which
        produced the original set of unparsed log lines.

\end{description}

A second utility is also provided which reads a list of new \regexes{} and
the input given to the first utility.  It tries to match each input log
line against each \regex{}, counting the number of log lines which match
each \regex{}, warning the user if an input log line is matched by more
than one \regex{}, and additionally warning if an input log line is not
matched by any \regex{}.  It displays a summary of how many input log lines
each \regex{} matched, comparing it to the expected number of matches; this
provides the user with an easy method of checking if the \regexes{}
produced by the first utility are correctly matching the input log lines
they are based upon.  A future version of this utility will also group
input log lines by \regex{}, so the user can tweak the \regexes{} if
required.

\subsection{Conclusion}

This section dealt with the rules used in parsing Postfix log files:

\begin{itemize}

    \item The characteristics of the rules were described.

    \item Detecting overlapping rules and dealing with the problems they
        can cause was covered, including a discussion of why overlapping
        rules can be helpful as well as harmful.

    \item An example log line and the rule matching it illustrated a
        description of how the fields in the rule are used both in the
        matching phase and the action that is subsequently executed.

    \item The structure of the database table containing the rules is dealt
        with in \sectionref{rule attributes}, and is not duplicated in this
        section.

    \item The topic of rule efficiency was discussed next, covering the
        effects of caching compiled \regexes{} and optimal ordering of
        rules, with reference to the graphs in
        appendix~\refwithpage{graphs}.

    \item This section finished with an explanation of the algorithm used
        to generate a new \regex{} from unparsed log lines.

\end{itemize}


\section{Parsing algorithm}

\label{parsing-algorithm}

Where the rules are quite simple and each rule is completely independent of
its fellows, the algorithm is significantly more complicated and highly
internally interdependent.  The algorithm deals with all the complications
of parsing, the eccentricities and oddities of Postfix logs, presenting the
resulting data in a normalised, easy to use representation.  The
algorithm's task is to follow the journey each mail takes through Postfix,
combining the data extracted by rules into a coherent whole, saving it in a
useful and consistent form, and performing housekeeping duties.

Please refer to \sectionref{parser design} for a discussion of why the
rules, actions and framework have been separated in the parser's design.
In this section algorithm can be taken to mean the combination of framework
and actions.

\subsection{Introduction}

This section covers the following topics:

\begin{itemize}

    \item A high level overview of the algorithm.

    \item The first set of complications encountered: initially obvious
        difficulties which had to be overcome.

    \item A flow chart showing common paths a mail can take through the
        algorithm, a discussion of the emergent behaviour observed in the
        parser, and an explanation of the paths shown in the flow chart.

    \item A brief description of the framework.

    \item The actions the parser makes available to rules are covered in
        detail.

    \item Additional complications which have arisen during the development
        of this parser are documented, the solutions to those
        complications, and where those solutions are implemented in the
        algorithm

\end{itemize}

\subsection{A high level overview}

From the viewpoint of an individual mail passing through the parser the
experience could be summarised as:

\begin{enumerate}

    \item Mail enters the system via \SMTP{} or local submission.

    \item If the mail is rejected, log all data and finish.

    \item Follow the progress of the accepted mail until it's either
        delivered, bounced or deleted, then log all data, and finish.

\end{enumerate}

The framework's high level overview could be expressed as (indentation
denoting flow of control):

\begin{verbatim}
for each line in the input files:
    for each rule defined by the user:
        if this rule matches the input line:
            perform the action specified by the rule
            skip the remaining rules
            process the next input line
    warn the user that the input line was not parsed
\end{verbatim}

Unfortunately both high level views ignore the many complications
encountered.


\subsection{Complications encountered}

\label{complications}

These complications were encountered early in the parser's implementation
and guided its design and development.

\subsubsection{Queueid vs pid}

The mail lacks a queueid until it has been accepted, so log lines must
first be correlated by the \daemon{smtpd} \pid{}, then transition to being
correlated by the queueid.  This is relatively minor, but does require:

\begin{itemize}

    \item Two versions of several functions: \texttt{by\_pid} and
        \texttt{by\_queueid}.

    \item Two state tables to hold the data structure for each connection.

    \item Most importantly: every section of code must know whether it
        needs to lookup the data structures by \pid{} or queueid.

\end{itemize}

\subsubsection{Connection reuse}

\label{connection reuse}

Multiple independent mails may be delivered during one connection: this
requires the algorithm to clone the current data as soon as a mail is
accepted, so that subsequent mails won't trample over each other's data.
This must be done every time a mail is accepted, as it's impossible to tell
in advance which connections will accept multiple mails.  Happily, once the
mail has been accepted log entries won't be correlated by \pid{} for that
mail any more (its queueid will be used instead), so there isn't any
ambiguity about which mail a given log line belongs
to.\footnote{Unfortunately this statement is not completely accurate: see
\sectionref{timeouts-during-data-phase} for details.  However in general
there isn't any ambiguity about which data structure should be used for a
given log line.}  The original connection will be discarded unsaved when
the client disconnects if it doesn't have any data worth saving, i.e.\ no
rejections.  One unsolved difficulty is distinguishing between different
groups of rejections, e.g.\ when dealing with the following sequence:

\begin{enumerate}

    \item The client attempts to deliver a mail, but it is rejected.

    \item The client issues the RSET command to reset the session.

    \item The client attempts to deliver another mail, likewise rejected.

\end{enumerate}

There should probably be two different entires in the database resulting
from the above sequence, but currently there will only be one.



\subsubsection{Re-injected mails}

The most difficult complication initially encountered is that locally
addressed mails are not always delivered directly to a mailbox: sometimes
they are addressed to and accepted for a local address but need to be
delivered to one or more remote addresses due to aliases.  When this occurs
a child mail will be injected into the Postfix queue, but without the
explicit logging \daemon{smtpd} or \daemon{postdrop} injected mails have.
Thus the source is not immediately discernible from the log line in which
the mail first appears; from a strictly chronological reading of the logs
it usually appears as if the child mail has appeared from thin air.
Subsequently the parent mail will log the creation of the child mail:

\texttt{3FF7C4317: to=<username@example.com>, relay=local, \newline
delay=0, status=sent (forwarded as 56F5B43FD)}

The child mail has been created with queueid \texttt{56F5B43FD}.  Different
delivery methods result in different log lines:

\begin{description}

    \item [Re-injected for forwarding:] forwarded as 56F5B43FD

    \item [Delivered to a remote \SMTP{} server:] 250 Ok: queued as
        BD07D3C49

    \item [Local delivery:] delivered to command:
        /mail/procmail/bin/procmail -p -t /mail/procmail/etc/procmailrc

\end{description}

Unfortunately, while all log lines from an individual process appear in
chronological order, the order in which log lines from different processes
are interleaved is subject to the vagaries of process scheduling.  In
addition, the first log line belonging to the child mail (the log line
cited above belongs to the parent mail) is logged by \daemon{qmgr}, so the
order also depends on how busy \daemon{qmgr} is.\footnote{Postfix is quite
paranoid about mail delivery, an excellent characteristic for an \MTA{} to
possess, so it won't log that the child has been created until it is
absolutely certain that the mail has been written to disk.}

Because of this the parser cannot complain when it encounters a log line
from \daemon{qmgr} for a previously unseen mail; it must flag the mail as
coming from an unknown origin, and subsequently clear the flag if and when
the origin of the mail becomes clear.  Obviously the parser could omit
checking of where mails originate from, but the author believes that it is
better to require an explicit source, as bugs in the parser are more likely
to be exposed; such checks helped to identify the complications described
in \sectionref{discarding cleanup lines} and \sectionref{pickup logging
after cleanup}.

Process scheduling can have a still more confusing effect: quite often the
child mail will be created, delivered and entirely finished with
\textbf{before} the parent logs the creation log line!  Thus, mails flagged
as coming from an unknown origin cannot be entered into the database when
their final log line is parsed; instead they must be marked as ready for
entry and subsequently entered by the parent mail once it has been
identified.

\subsection{Flow chart}

\label{flow-chart}

Figures~\refwithpage{flow chart image part 1} and~\refwithpage{flow chart
image part 2} show the paths the data representing a mail can take through
the parser algorithm.  The flow chart covers the most common paths only;
there are additional, uncommon paths which are excluded for the sake of
clarity; details of the deviations can be found in \sectionref{additional
complications}.

\showgraph{build/logparser-flow-chart-part-1}{Parser flow chart part
1}{flow chart image part 1}

\showgraph{build/logparser-flow-chart-part-2}{Parser flow chart part
2}{flow chart image part 2}

\clearpage

\subsubsection{Mail enters the system}

\label{mail-enters-the-system}

Everything starts off with a mail entering the system, whether by local
submission via \daemon{postdrop} or sendmail, by \SMTP{}, by re-injection
due to forwarding, or internally generated by Postfix.  Local submission is
the simplest case: a queueid is assigned immediately and the sender address
is logged (action: PICKUP\@; flowchart:~2).

\SMTP{} is more complicated:

\begin{enumerate}

    \item First there is a connection from the remote client (action:
        CONNECT\@; flowchart:~1).

    \item This is followed by rejection of sender address, recipient
        addresses, client \IP{} address or hostname, HELO hostname, etc.\
        (action: REJECTION\@; flowchart:~4); acceptance of one or more
        mails (action: CLONE\@; flowchart:~5); or some interleaving of
        both.

    \item The client disconnects (action: DISCONNECT\@; flowchart:~6).  If
        Postfix has rejected any \SMTP{} commands the data will be saved to
        the database; if not there won't be any data to save (any mails
        accepted will already have been cloned so their data is in another
        data structure).

    \item If one or more mails were accepted there will be more log entries
        for those mails later, see \sectionref{mail-delivery}.

\end{enumerate}

Re-injection due to forwarding sadly lacks explicit log lines of its
own;\footnote{Previously discussed in \sectionref{complications},
complication 3.} re-injection is somewhat awkward to explain because it
overlaps both the mail acceptance and mail delivery sections, so discussion
is deferred to \sectionref{tracking re-injected mail}.

Internally generated mails lack any explicit origin in Postfix 2.2.x and
must be detected using heuristics (see
\sectionref{identifying-bounce-notifications} for details).  Bounce
notifications are the primary example of internally generated mails, though
there may be other types.\footnote{Postfix may generate mails to the
administrator when it encounters configuration errors, but such mails are
presumably rare.}

\subsubsection{Mail delivery}

\label{mail-delivery}

The obvious counterpart to mail entering the system is mail leaving the
system, whether by deletion, bouncing, local delivery, or remote delivery.
All four are handled in exactly the same way:

\begin{enumerate}

    \item Postfix will log the sender and recipient addresses separately
        (action: SAVE\_BY\_QUEUEID\@; flowchart:~9).

    \item Sometimes mail is re-injected and the child mail needs to be
        tracked by the parent mail (action: TRACK\@; flowchart:~10) ---
        \sectionref{tracking re-injected mail} discusses this in
        detail.

    \item Eventually the mail will be delivered, bounced, or deleted by the
        administrator (action: COMMIT\@; flowchart:~12).  This is the last
        log line for this particular mail (though it may be indirectly
        referred to if it was re-injected).  If it is neither parent nor
        child of re-injection the data is cleaned up and entered in the
        database (flowchart:~14), then deleted from the state tables.
        Re-injected mails are described in \sectionref{tracking re-injected
        mail}.

\end{enumerate}

It should be reiterated that the actions above happen whether the mail is
delivered to a mailbox, piped to a command, delivered to a remote server,
bounced (due to a mail loop, delivery failure, or five day timeout), or
deleted by the administrator, \textit{unless\/} the mail is either parent
or child of re-injection, as explained in \sectionref{tracking re-injected
mail}.

\subsubsection{Tracking re-injected mail}

\label{tracking re-injected mail}

The crux of the problem is that re-injected mails appear in the logs
without explicit logging indicating their source.  There are two implicit
indications:

\begin{enumerate}

    \item The indicator which more commonly introduces re-injection is when
        \daemon{qmgr} selects a mail with a previously unseen queueid for
        delivery (action: MAIL\_PICKED\_FOR\_DELIVERY\@; flowchart:~3), in
        which case a new data structure will be created for that mail.  The
        mail will be flagged as having unknown origins; this flag should be
        subsequently cleared once the origin has been established.  This
        may also be an indicator that the mail is a bounce notification,
        see \sectionref{identifying-bounce-notifications} for details.

    \item Local delivery re-injects the mail and logs a relayed delivery
        rather than delivering directly to a mailbox or program as it
        usually would (action: TRACK\@; flowchart:~10).\footnote{Relayed
        delivery is performed by the \SMTP{} client; local delivery means
        local to the server, i.e.\ an address the server is final
        destination for.} In this case the mail may already have been
        created (described above) and the unknown origin flag will be
        cleared; if not a new data structure will be created.  In both
        cases the re-injected mail is marked as a child of the original
        mail.  The log line in question is:

        \texttt{3FF7C4317: to=<username@example.com>, relay=local, \newline
        delay=0, status=sent (forwarded as 56F5B43FD)}

        This second indicator always occurs for re-injected mail but
        typically occurs after the first indicator explained above.  This
        is required to tie the parent and child mails together and so is
        central to the process of tracking re-injected mails.

\end{enumerate}

The algorithm for tracking and saving re-injected mail to the database can
finally be described:

\begin{itemize}

    \item If the mail is of unknown origin it is assumed to be a child mail
        whose parent hasn't yet been identified (action: COMMIT\@;
        flowchart:~15).  Mark the mail as ready for entry in the database
        (flowchart:~16), and wait for the parent to deal with it
        (flowchart:~17).  The mail should not have subsequent log entries;
        only its parent should refer to it.

    \item If the mail is a child mail then it has already been tracked
        (action: COMMIT\@; flowchart:~18): as with all other mail, the data
        is cleaned up, the child is entered in the database
        (flowchart:~19), then deleted from the state tables.  The child
        mail will be removed from the parent mail's list of children
        (flowchart:~20); if this is the last child and the parent has also
        been entered in the database the parent will be deleted from the
        state tables.

    \item The last alternative is that the mail is a parent mail (action:
        COMMIT\@; flowchart:~21).  Regardless of the state of its children
        its data is cleaned up and entered in the database (flowchart:~22).
        The parent may have children which are waiting to be entered in the
        database (flowchart:~23); each of those children's data is cleaned
        up and entered in the database, then deleted from the state tables.
        The parent may also have outstanding children which are not yet
        delivered (flowchart:~24), in which case the parent must wait for
        those children to be finished with.  As soon as the last child is
        deleted from the state tables the parent will also be finished with
        (flowchart:~25), and deleted from the state tables.

\end{itemize}

A parent mail can have multiple children, which may be delivered before or
after the parent mail.

\subsubsection{Emergent behaviour}

\label{Emergent behaviour}

The rules and actions exhibit emergent behaviour~\cite{Wikipedia-Emergence}
which is far more complicated than the behaviour explicitly encoded in the
algorithm.  The top-level parser could be written in pseudo-code like so
(with indentation level indicating flow control):

\begin{verbatim}
for each line in the input files:
    for each rule defined by the user:
        if this rule matches the input line:
            perform the action specified by the rule
            skip the remaining rules
            process the next input line
    warn the user that the input line was not parsed
\end{verbatim}

Yet despite the brevity of the algorithm above the (simplified) flow chart
in \sectionref{flow-chart} contains:

\begin{itemize}

    \item Twenty one states (not including explicit branches).

    \item Three entry points (1, 2, 3).

    \item Five exit points (8, 14, 17, 20, 25).

    \item Five loops (4--4, 4--5--4, 5--5, 9--9, 9--10--9).

    \item Four explicit branches (7, 13, 15, 18); these are decisions taken
        by an action, determining what will happen next to a mail.

    \item Four implicit branches, where the transition is determined by the
        next log line which is processed for that mail, e.g.\ state 1 can
        transition to either state 4 or state 5.  (1--\{4,5\},
        4--\{4,5,6\}, 5--\{4,5,6\}, 9--\{9,10,11,12\})

\end{itemize}

Several more states and corresponding branches are omitted for the sake of
clarity (these extra states are caused by solving the complications
described in \sectionref{additional complications}).

The actions introduce some of the complexity seen in the flow chart: all
explicit branches and the transitions between some stages (19--20,
21--22--23--24) in the second half of the flow chart
(figure~\refwithpage{flow chart image part 2}) are encoded within the
actions; the transitions between the other stages (16--17, 24--25) in the
second half arise from the interaction between actions

However the transitions between stages in the first half of the flow chart
(figure~\refwithpage{flow chart image part 1}) are not encoded anywhere in
the parser; the transitions are determined by the ordering of log lines in
the log file.  The complexity of the first 12 stages in the flow chart
emerges from the interaction of simple rules and actions, easing the
process of adding new actions (described in \sectionref{adding new
actions}), as there is no requirement to explicitly insert the action into
an algorithmic version of the flow chart.\footnote{The solution to the
complication in \sectionref{out of order log lines} (out of order log
lines) requires a list of the acceptable combinations of Postfix programs a
mail can pass through; however this does not correspond to the
\textit{parser actions\/} a mail must pass through.  Before that
complication was overcome, i.e.\ for the first half of the development of
this parser, there was nothing in the parser explicitly encoding the states
or paths a mail could take; no other component in the parser has any need
for this flow of data to be specified.}



\subsection{Framework}

\label{framework}

The intermingling of log entries from different mails immediately rules out
the possibility of handling each mail in isolation; the parser must be
capable of handling multiple mails in parallel, each potentially at a
different stage in its journey, without any interference between mails ---
except in the minority of cases where intra-mail interference is required.
The best way to implement this is to maintain state information for every
unfinished mail and manipulate the appropriate mail correctly for each log
line encountered.

This functionality is provided by the framework, which both drives parsing
overall and provides services to the actions it invokes.  The framework is
responsible for loading rules from the database and sanity checking them,
reading log files, matching each rule against the current log line,
invoking the correct action, maintaining state tables, loading and saving
state, displaying a progress indicator, and miscellaneous other tasks.
Actions use services provided by the framework, including storing and
retrieving data in the state tables, extracting and saving data captured by
rules, storage of global data, debugging functions, preparing a mail for
entry in the database, and entering mails in the database.

There is a similarity between this design and the event-driven programming
paradigm commonly used in GUI programs, where one part of the program
responds to events (mouse clicks in a GUI program, log lines being matched
in the parser) and invokes the correct action.

\subsection{Actions in detail}

\label{actions-in-detail}

Each action is passed the same arguments:

\begin{description}

    \item [line] The log line, separated into fields:

        \begin{description}

            \item [timestamp] The time the line was logged at.

            \item [host] The hostname of the server which logged the line.

            \item [program] The name of the program which logged the line.

            \item [pid] The \pid{} of the program which logged the line.

            \item [text] The remainder of the line.

        \end{description}

    \item [rule] The matching rule.

    \item [matches] The fields in the line captured by the rule's \regex{}.

\end{description}

The actions:

\begin{description}

    \item [BOUNCE] Postfix 2.3 and subsequent versions log the creation of
        bounce messages.  This action creates a new mail if necessary; if
        the mail already exists the unknown origin flag will be removed.
        The action also marks the mail as a bounce notification.  To deal
        with complication \sectionref{Further out of order log lines} this
        action checks a cache of recent bounce mails to avoid creating
        bogus bounce mails when lines are logged out of order.

    \item [CLONE] Multiple mails may be accepted on a single connection, so
        each time a mail is accepted the connection's state table entry
        must be cloned and saved in the state tables under its queueid; if
        the original data structure was used then second and subsequent
        mails would overwrite one another's data.

    \item [COMMIT] Enter the data from the mail into the database. Entry
        will be postponed if the mail is a child waiting to be tracked.
        Once entered, the mail will be deleted from the state tables.
        Deletion will be postponed if the mail is the parent of re-injected
        mail.

    \item [CONNECT] Handle a remote client connecting: create a new state
        table entry (indexed by \daemon{smtpd} \pid{}) and save both the
        client hostname and \IP{} address.

    \item [DELETE] Deals with mail deleted using Postfix's administrative
        command \daemon{postsuper}.  This action adds a dummy recipient
        address if required, then invokes the COMMIT action to handle
        adding the mail to the database.  The complication this action
        deals with is described fully in \sectionref{Mail deleted before
        delivery is attempted}.  

    \item [DISCONNECT] Deal with the remote client disconnecting: enter the
        connection in the database, perform any required cleanup, and
        delete the connection from the state tables.  This action deals
        with aborted delivery attempts
        (\sectionref{aborted-delivery-attempts}).

    \item [EXPIRY] If Postfix has not managed to deliver a mail after
        trying for five days it will give up and return the mail to the
        sender.  When this happens the mail will not have a combination of
        Postfix programs which passes the valid combinations check (see
        \sectionref{out of order log lines}).  To ensure that the mail can
        be committed the EXPIRY action sets a flag marking the mail as
        expired; the flag later causes the valid combinations check to be
        skipped, so the mail will be committed.

    \item [IGNORE] This rule just returns successfully; it is used when a
        line needs to be parsed for completeness but doesn't either provide
        any useful data or require anything to be done.

    \item [MAIL\_PICKED\_FOR\_DELIVERY] This action represents Postfix
        picking a mail from the queue to deliver. This action is used for
        both \daemon{qmgr} and \daemon{cleanup} as it needs to deal with
        out of order log lines; see \sectionref{discarding cleanup lines}
        for details.

    \item [MAIL\_TOO\_LARGE] When a client tries to send a message larger
        than the local server accepts, the mail will be discarded, and the
        client informed.  See TIMEOUT for further discussion; the two are
        handled in exactly the same way.

    \item [PICKUP] The PICKUP action corresponds to the \daemon{pickup}
        service dealing with a locally submitted mail.  Out of order log
        entries may have caused the state table entry to already exist (see
        \sectionref{pickup logging after cleanup}); otherwise it is
        created.  The data extracted from the log line is then saved to the
        state table entry.

    \item [POSTFIX\_RELOAD] When Postfix stops or reloads its configuration
        it kills all \daemon{smtpd} processes,\footnote{Possibly other
        programs are killed also, but the parser is only affected by and
        interested in \daemon{smtpd} processes exiting.} requiring any
        active connections to be cleaned up, entered in the database, and
        deleted from the state tables.

    \item [REJECTION] Deal with Postfix rejecting an \SMTP{} command from
        the remote client: log the rejection with a mail if there is a
        queueid in the log line, or with the connection if not.

    \item [SAVE\_BY\_QUEUEID] Find the correct mail based on the queueid in
        the log line, and save the data extracted by the \regex{} to it.

    \item [SMTPD\_DIED] Sometimes a \daemon{smtpd} dies or exits
        unsuccessfully; the active connection for that \daemon{smtpd} must
        be cleaned up, entered in the database, and deleted from the state
        tables.

    \item [SMTPD\_KILLED] Sometimes an \daemon{smtpd} is killed by a
        signal~\cite{Wikipedia-unix-signals}; the active connection for
        that \daemon{smtpd} must be cleaned up, entered in the database,
        and deleted from the state tables.

    \item [SMTPD\_WATCHDOG] \daemon{smtpd} processes have a watchdog timer
        to deal with unusual situations --- after five hours the timer will
        expire and the \daemon{smtpd} will exit.  This occurs very
        infrequently, as there are many other timeouts which should occur
        in the intervening hours: \DNS{} timeouts, timeouts reading data
        from the client, etc.  The active connection for that
        \daemon{smtpd} must be cleaned up, entered in the database, and
        deleted from the state tables.

    \item [TIMEOUT] The connection timed out so the mail currently being
        transferred must be discarded. The mail may have been accepted, in
        which case there's a data structure to dispose of, or it may not in
        which case there isn't.  See
        \sectionref{timeouts-during-data-phase} for the gory details.

    \item [TRACK] Track a mail when it is re-injected for forwarding to
        another mail server; this happens when a local address is aliased
        to a remote address.  TRACK will be called when dealing with the
        parent mail, and will create the child mail if necessary. TRACK
        checks if the child has already been tracked, either by this parent
        or by another parent, and issues appropriate warnings in either
        case.

\end{description}

\subsection{Adding new actions}

\label{adding new actions}

Adding new actions is not as simple as adding new rules, though care has
been taken in the parser's design and implementation to make adding new
actions as painless as possible.  The implementor writes a subroutine which
accepts the standard arguments given to actions, and registers it as an
action\footnote{The new action must be registered before the rules are
loaded, as it is an error for a rule to reference an unregistered action;
this helps catch mistakes made when adding new rules.} by calling the
framework subroutine add\_actions() with the name of the new action
subroutine as a parameter.  No other work is required on the implementor's
part to integrate the action into the parser; all of their attention and
effort can be focused on the correctness of their action.  The only
negative aspect is that the process involves editing the parser source
code, which makes upgrading to a later version of the parser more
difficult, though by no means impossible.  If the author of the new action
wishes, they can take advantage of the parser's object oriented
implementation~\cite{Wikipedia-object-orientation} by subclassing it and
implementing their changes in the derived class, allowing future upgrading
of the parser with greatly reduced chance of conflicts.\footnote{The real
difference between the two approaches is where the new code is placed.  The
simpler option is to change the parser code directly, but those changes
will then have to be made to subsequent versions of the parser, and as the
scope of the changes increases so does the chance of conflict, or mistakes
when copying the action.  The more time consuming option is to write a
subclass containing the new actions and change the program which invokes
the parser so that it uses the subclass rather than the parser; the changes
required to the program invoking the parser are minor and much less likely
to lead to conflicts when upgrading to future versions of the parser.  An
alternative is to submit new actions to the author of the parser for
inclusion in future versions, resulting in two benefits: the new actions do
not need to be maintained separately, and other users of the parser will
avail of the new functionality.} The action may need to extend the list of
valid combinations described in \sectionref{out of order log lines} if the
addition creates a different set of acceptable programs, but this is
unlikely to occur, as it would require parsing log lines from Postfix
components the parser currently ignores.\footnote{The mail server used for
development does not utilise either the \daemon{lmtp} or \daemon{virtual}
delivery agents, so this parser does not have rules to handle log lines
from those components.  Adding new rules to parse those component's log
lines is a simple process, though if their behaviour differs significantly
from the \daemon{smtp} or \daemon{local} delivery agents new actions may be
required.  The mail server in question is a production mail server handling
mail for a university department; the benefit is that the logs used exhibit
the idiosyncrasies and peculiarities a mail server in the wild must deal
with, but the downside is that significantly altering the configuration
just to log messages from a different Postfix component is not an option.}


\subsection{Additional complications}

\label{additional complications}

The complications described in this section are listed in the order in
which they were encountered during development of the parser.  Each of
these complications caused the parser to operate incorrectly, generating
either warning messages or leaving mails in the state table.  The frequency
of occurrence is much higher at the start of the list, with the first
complication occurring several orders of magnitude more frequently than the
last.  When deciding which problem to address next, the most common was
always chosen, as resolving the most common problem would yield the biggest
improvement in the parser, prune the greatest number of mails from
the state tables and error messages, and make the remaining problems more
apparent.


\subsubsection{Identifying bounce notifications}

\label{identifying-bounce-notifications}

Postfix 2.2.x (and presumably previous versions) lacks explicit logging
when bounce notifications are generated; suddenly there will be log entries
for a mail which lacks an obvious source.  There are similarities to the
problem of re-injected mails discussed in \sectionref{tracking re-injected
mail}, but unlike the solution described therein bounce notifications do
not eventually have a log line which identifies their source.  Heuristics
must be used to identify bounce notifications, and those heuristics are:

\begin{enumerate}

    \item The sender address is $<>$.

    \item Neither \daemon{smtpd} nor \daemon{pickup} have logged any
        messages associated with the mail, indicating it was generated
        internally by Postfix, not accepted via \SMTP{} or submitted
        locally by \daemon{postdrop} or sendmail.

    \item The message-id has a specific format: \newline
        \texttt{YYYYMMDDhhmmss.queueid@server.hostname} \newline
        e.g.\ \texttt{20070321125732.D168138A1@smtp.example.com}

    \item The queueid in the message-id must be the same as the queueid of
        the mail: this is what distinguishes bounce notifications generated
        locally from bounce notifications which are being re-injected as a
        result of aliasing.  In the latter case the message-id will be
        unchanged from the original bounce notification, and so even if it
        happens to be in the correct format (e.g.\ if it was generated by
        Postfix on this or another server) the queueid in the message-id
        will not match the queueid of the mail.

\end{enumerate}

Once a mail has been identified as a bounce notification, the unknown
origin flag is cleared and the mail can be cleaned up and entered in the
database.

There is a small chance that a mail will be incorrectly identified as a
bounce notification, as the heuristics used may be too broad.  For this to
occur the following conditions would have to be met:

\begin{enumerate}

    \item The mail must have been generated internally by Postfix.

    \item The sender address must be $<>$.

    \item The message-id must have the correct format and match the queueid
        of the mail.  While a mail sent from elsewhere could easily have
        the correct message-id format, the chance that the queueid in the
        message-id would match the queueid of the mail is extremely small.

\end{enumerate}

If a mail is mis-classified as a bounce message it will almost certainly
have been generated internally by Postfix; arguably mis-classification in
this case is a benefit rather than a drawback, as other mails generated
internally by Postfix will be handled correctly.

Postfix 2.3 (and hopefully subsequent versions) log the creation of a
bounce message.

This check is performed during the COMMIT action.

\subsubsection{Aborted delivery attempts}

\label{aborted-delivery-attempts}

Some mail clients behave unexpectedly during the \SMTP{} dialogue: the
client aborts the first delivery attempt after the first recipient is
accepted, then makes a second delivery attempt for the same recipient which
it continues with until delivery is complete.\footnote{Microsoft Outlook is
one client which behaves in this fashion; no attempt has been made to
identify other clients which act in a similar way.}  An example dialogue
exhibiting this behaviour is presented below (lines starting with a three
digit number are sent by the server, the other lines are sent by the
client):

\begin{verbatim}
220 smtp.example.com ESMTP
EHLO client.example.com
250-smtp.example.com
250-PIPELINING
250-SIZE 15240000
250-ENHANCEDSTATUSCODES
250-8BITMIME
250 DSN
MAIL FROM: <sender@example.com>
250 2.1.0 Ok
RCPT TO: <recipient@example.net>
250 2.1.5 Ok
RSET
250 2.0.0 Ok
RSET
250 2.0.0 Ok
MAIL FROM: <sender@example.com>
250 2.1.0 Ok
RCPT TO: <recipient@example.net>
250 2.1.5 Ok
DATA
354 End data with <CR><LF>.<CR><LF>
The mail transfer is not shown.
250 2.0.0 Ok: queued as 880223FA69
QUIT
221 2.0.0 Bye
\end{verbatim}

Once again Postfix does not log a message making the client's behaviour
clear, so once again heuristics are required to identify when this
behaviour occurs.  In this case a list of all mails accepted during a
connection is saved in the connection state, and the accepted mails are
examined when the disconnection action is executed.  Each mail is checked
for the following: 

\begin{itemize}

    \item Are there two or more \daemon{smtpd} log lines?  Does the second
        result have a postfix\_action attribute of ACCEPTED\@?  The first
        two \daemon{smtpd} log lines will be a connection log line and a
        mail acceptance log line (Postfix logs acceptance as soon as the
        first recipient has been accepted).

    \item Is \daemon{smtpd} the only Postfix component that produced a log
        line for this mail?  Every mail which passes normally through
        Postfix will have a \daemon{cleanup} line, and later a
        \daemon{qmgr} log line; lack of a \daemon{cleanup} line is a sure
        sign the mail didn't make it too far.  

    \item Does the queueid exist in the state tables?  If not it cannot be
        an aborted delivery attempt.

    \item If there are third and subsequent results, are all their
        postfix\_action attributes equal to INFO\@?  If there are any log
        lines after the first two they should be informational lines only.

\end{itemize}

If all the checks above are successful then the mail is assumed to be an
aborted delivery attempt and is discarded.  There will be no further
entries logged for such mails, so without identifying and discarding them
they accumulate in the state table and will cause clashes if the queueid is
reused.  The mail cannot be entered in the database as the only data
available is the client hostname and \IP{} address, but the database schema
requires many more fields be populated (see \sectionref{connections table}
and \sectionref{results table}).  This heuristic is quite restrictive, and
it appears there is little scope for false positives; if there are any
false positives there will be warnings when the next log line for that mail
is parsed.  False negatives are less likely to be detected: there may be
queueid clashes (and warnings) if mails remain in the state tables after
they should have been removed, otherwise the only way to detect false
negatives is to examine the state tables after each parsing run.

This check is performed in the DISCONNECT action; it requires support in
the CLONE action where a list of cloned connections is maintained.

\subsubsection{Further aborted delivery attempts}

Some mail clients disconnect abruptly if a second or subsequent recipient
is rejected; they may also disconnect after other errors, but such
disconnections are either unimportant or are handled elsewhere in the
parser (\sectionref{timeouts-during-data-phase}).  Sadly, Postfix doesn't
log a message saying the mail has been discarded, as should be expected by
now.  The checks to identify this happening are:

\begin{itemize}

    \item Is the mail missing its \daemon{cleanup} log message?  Every mail
        which passes through Postfix will have a \daemon{cleanup} line;
        lack of a \daemon{cleanup} line is a sure sign the mail didn't make
        it too far.

    \item Were there three or more \daemon{smtpd} log lines for the mail?
        There should be a connection line and a mail accepted line,
        followed by one or more rejection lines.

    \item Is the last \daemon{smtpd} log line a rejection line?

\end{itemize}

These checks are made during the DISCONNECT action: if all checks are
successful then the mail is assumed to have been discarded when the client
disconnected.  There will be no further entries logged for such mails, so
without identifying and entering them in the database immediately they
accumulate in the state table and will cause clashes if the queueid is
reused.

\subsubsection{Timeouts during DATA phase}

\label{timeouts-during-data-phase}

The DATA phase of the \SMTP{} conversation is where the headers and body of
the mail are transferred.  Sometimes there is a timeout or the connection
is lost\footnote{For brevity's sake \textit{timeout\/} will be used
throughout this section, but everything applies equally to lost
connections.} during the DATA phase; when this occurs Postfix will discard
the mail and the parser needs to discard the data associated with that
mail.  It seems more intuitive to save the data to the database, but if a
timeout occurs there is no data available to save; the timeout is recorded
with the connection data instead.

To deal properly with timeouts the parsing algorithm needs to do the
following in the TIMEOUT action:

\begin{enumerate}

    \item Record the timeout and associated data in the connection's
        results.

    \item If no mails have been accepted yet nothing needs to be done; the
        TIMEOUT action ends.  

    \item If one or more recipients have been accepted Postfix will have
        allocated a queueid for the incoming mail, and there will be a mail
        in the state tables which needs to be dealt with.
        
\end{enumerate}

A timeout may thus apply either to an accepted mail or a rejected mail.  To
distinguish between the two cases the algorithm compares the timestamp of
the last accepted mail against the timestamp of the last line logged by
\daemon{smtpd} for that connection (the TIMEOUT action is dependant on the
CLONE action keeping a list of all mails accepted on each connection).  If
the mail acceptance timestamp is later then the timeout applies to the
just-accepted mail, which will be discarded.  If the \daemon{smtpd}
timestamp is later there was a rejection between the accepted mail and the
timeout: the action assumes that the timeout applies to a rejected delivery
attempt and finishes.  This assumption is not necessarily correct, because
Postfix may have accepted an earlier recipient and rejected a later one, in
which case the timeout applies to the accepted mail, which should be
discarded.  This has not been a problem in practice, though it may be in
future.

This complication is further complicated by the presence of out of order
\daemon{cleanup} lines: see \sectionref{discarding cleanup lines} for
details.

\subsubsection{Discarding cleanup lines}

\label{discarding cleanup lines}

The author has only observed this complication occurring after a timeout,
though there may be other circumstances which trigger it.  Sometimes the
\daemon{cleanup} line is logged after the timeout line; parsing this line
causes the creation of a new state table entry for the queueid in the log
line.  This is incorrect because the line actually belongs to the mail
which has just been discarded; the next log line for that queueid will be
seen when the queueid is reused for a different mail, causing a queueid
clash and the appropriate warning.

In the case where the \daemon{cleanup} line is still pending during the
TIMEOUT action the algorithm updates a global cache of queueids, adding the
queueid and the timestamp from the timeout line.  When the next
\daemon{cleanup} line is parsed for that queueid the cache will be checked
(during the MAIL\_PICKED\_FOR\_DELIVERY action),
and the line will be deemed part of the discarded mail and discarded if it
meets the following requirements:

\begin{itemize}

    \item The queueid must not have been reused yet, i.e.\ there isn't an
        entry in the state tables for the queueid.

    \item The timestamp of the \daemon{cleanup} line must be within ten
        minutes of the mail acceptance timestamp.  Timeouts happen after
        five minutes, but some data may have been transferred slowly
        (perhaps because either the client or server is suffering from
        network congestion or rate limiting), and empirical evidence shows
        that ten minutes is not unreasonable; hopefully it is a good
        compromise between false positives and false negatives.

\end{itemize}

The next \daemon{cleanup} line must meet the criteria above for it to be
discarded because not every connection where a timeout occurs will have a
\daemon{cleanup} line logged for it; if the algorithm blindly discarded the
next \daemon{cleanup} line after a TIMEOUT it would in some cases be
mistaken.  Whether or not the next \daemon{cleanup} line is discarded the
queueid will be removed from the cache of timeout queueids when the next
\daemon{pickup} line containing that queueid is parsed.

\subsubsection{Pickup logging after cleanup}

\label{pickup logging after cleanup}

Occasionally the \daemon{pickup} line logged when mail is submitted locally
via sendmail appears later in the log file than the \daemon{cleanup} line
for that mail.  This seems to occur during periods of particularly heavy
load, so is most likely due to process scheduling vagaries.  Normally if
the queueid given in the PICKUP line exists in the state tables a warning
is generated by the \daemon{pickup} action, but if the following conditions
are met it is assumed that the lines were out of order:

\begin{itemize}

    \item The only program which has logged anything thus far for the mail
        is \daemon{cleanup}.

    \item There is less than a five second difference between the
        timestamps of the \daemon{cleanup} and \daemon{pickup} lines.

\end{itemize}

As always with heuristics there may be circumstances in which these
heuristics match incorrectly,  but none have been identified so far.

This complication is dealt with during the PICKUP action.

\subsubsection{Smtpd stops logging}

\label{smtpd stops logging}

Occasionally a \daemon{smtpd} will just stop logging, without an
immediately obvious reason.  After poring over logs for some time there are
several reasons for this infrequent occurrence:

\begin{enumerate}

    \item Postfix is stopped or its configuration is reloaded.  When this
        happens all \daemon{smtpd} processes exit, and all entries in the
        connections state table must be cleaned up, entered in the database
        if there is sufficient data, and deleted.  The queueid state table
        is untouched.

    \item Sometimes a \daemon{smtpd} is killed by a signal (sent by Postfix
        for some reason, by the administrator, or by the OS), so its active
        connection must be cleaned up, entered in the database if there is
        sufficient data, and deleted from the connections state table.

    \item Occasionally a \daemon{smtpd} will exit uncleanly, so the active
        connection must be cleaned up, entered in the database if there is
        sufficient data, and deleted from the connections state table.

\end{enumerate}

The above descriptions appear to cover all situations identified thus far
where a \daemon{smtpd} suddenly stops logging.  In addition to removing an
active connection the last accepted mail may need to be discarded, as
detailed in \sectionref{timeouts-during-data-phase}.

These occurrences are handled by the three actions POSTFIX\_RELOAD,
SMTPD\_DIED and SMTPD\_WATCHDOG\@.

\subsubsection{Out of order log lines}

\label{out of order log lines}

Occasionally a log file will have out of order log lines which cannot be
dealt with by the techniques described in \sectionref{tracking re-injected
mail}, \sectionref{discarding cleanup lines} or \sectionref{pickup logging
after cleanup}.  In the \numberOFlogFILES{} log files used for testing this
occurs only five times in 60,721,709 log lines, but for completeness of the
algorithm it should be dealt with.  The five occurrences in the test log
files have the same characteristics: the \daemon{local} log line showing
delivery to a local mailbox occurs after the \daemon{qmgr} log line showing
removal of the mail from the queue because delivery is completed.  This
causes problems: the mail is not complete, so entry into the database
fails; a new mail is created when the \daemon{local} line is parsed and
remains in the state tables; four warnings are issued per pair of out of
order log lines.

The solution to this problem is to examine the list of programs which have
logged lines for each mail, comparing the list against a table of
known-good combinations of programs (this check is performed during the
COMMIT action).  If the mail's combination is found in the valid list the
mail can be entered in the database; if the combination is not found entry
must be postponed and the mail flagged for later entry.  The
SAVE\_BY\_QUEUEID action checks for that flag and retries entry if it's
found; if the additional log lines have caused the mail to reach a valid
combination entry will proceed, otherwise it must be postponed once more.

The list of valid combinations is explained below.  Every mail will
additionally have log entries from \daemon{cleanup} and \daemon{qmgr}; any
mail may also have log entries from \daemon{bounce}, \daemon{postsuper}, or
both.

\begin{description}

    \item [\daemon{local}:] Local delivery of a bounce notification, or
        local delivery of a forwarded or tracked mail.

    \item [\daemon{local}, \daemon{pickup}:] Mail submitted locally on the
        server, delivered locally on the server.

    \item [\daemon{local}, \daemon{pickup}, \daemon{smtp}:] Locally
        submitted mail, \newline both local and remote delivery.

    \item [\daemon{local}, \daemon{smtp}, \daemon{smtpd}:] Mail accepted
        from a remote client, both local and remote delivery.

    \item [\daemon{local}, \daemon{smtpd}:] Mail accepted from a remote
        client, local delivery only.

    \item [\daemon{pickup}, \daemon{smtp}:] Locally submitted mail, remote
        delivery only.

    \item [\daemon{smtp}:] Remote delivery of forwarded or tracked mail, or
        a bounce notification.

    \item [\daemon{smtp}, \daemon{smtpd}:] Mail accepted from a remote
        client, then remotely delivered (typically relaying mail for
        clients on the local network).

    \item [\daemon{smtpd}, \daemon{postsuper}:] Mail accepted from a remote
        client, then deleted by the administrator before any delivery
        attempt was made (the unwanted mail is typically due to a mail loop
        or joe~job).  Notice that \daemon{postsuper} is required, not
        optional, for this combination.

\end{description}

This check applies to accepted mails only, not to rejected mails.

\subsubsection{Yet more aborted delivery attempts}

\label{yet-more-aborted-delivery-attempts}

The aborted delivery attempts described in
\sectionref{aborted-delivery-attempts} occur frequently, but the aborted
delivery attempts described in this section only occur four times in the
\numberOFlogFILES{} log files used for testing.  The symptoms are the same
as in \sectionref{aborted-delivery-attempts}, except that there
\textit{is\/} a \daemon{cleanup} log line; there does not appear to be
anything in the log file to explain why there are no further log messages.
The only way to detect these mails is to periodically scan all mails in the
state tables, deleting any mails displaying the following characteristics:

\begin{itemize}

    \item The timestamp of the last log line for the mail must be 12 hours
        or more earlier than the last log line parsed.

    \item There must be exactly two \daemon{smtpd} and one \daemon{cleanup}
        log entries for the mail, with no additional log entries.

\end{itemize}

12 hours is a somewhat arbitrary time period, but it is far longer than
Postfix would delay delivery of a mail in the queue.\footnote{This may be a
problem if Postfix is not running for an extended period of time.}  The
state tables are scanned for mails matching the characteristics above each
time the end of a log file is reached, and matching mails are deleted.

\subsubsection{Mail deleted before delivery is attempted}

\label{Mail deleted before delivery is attempted}

Postfix logs the recipient address when delivery of a mail is attempted, so
if delivery has yet to be attempted the parser cannot determine the
recipient address or addresses.  This is a problem when mail is arriving
faster than Postfix can attempt delivery, and the administrator deletes
some of the mail (because it's the result of a mail loop, mail bomb, or
joe~job) before Postfix has had a chance to try to deliver it.  In this
case the recipient address will not have been logged, so a dummy recipient
address needs to be added, as every mail is required to have at least one
recipient.  This complication has been observed in few log files, but
typically when it does occur there will be many instances of it, closely
grouped.  The DELETE action is responsible for handling this complication.

\subsubsection{Further out of order log lines}

\label{Further out of order log lines}

This is yet another complication which only occurs during periods of
extremely high load, when process scheduling vagaries and even hard disk
access times cause strange behaviour.  In this complication bounce
notification mails are created, delivered and deleted from the queue,
\textit{before\/} the log line from \daemon{bounce} is logged.  To deal
with this the COMMIT action maintains a cache of recently committed bounce
notification mails, which the BOUNCE action subsequently checks if the
bounce mail isn't already in the state tables. If a mail exists in the
cache under the correct queueid, and its start time is less than ten
seconds before the timestamp of the bounce log line, it is assumed that the
bounce notification mail has already been processed and the BOUNCE action
does not create one.  If the queueid exists it is removed from the cache,
as it has either just been used or it is too old to use in future.  Whether
the BOUNCE action creates a mail or finds one existing in the state tables,
it flags the mail as having been seen by the BOUNCE action; if this flag is
present the COMMIT action will not add the mail to the cache of recent
bounce notification mails.\footnote{This is not required to correctly deal
with the complication, but is an optimisation to reduce memory usage of the
parser; on the occasions the author has observed this action occurring
there have been a huge number of bounce notification mails generated --- if
every bounce notification mail was cached it would dramatically increase
the memory requirements of the parser.}  The cache of bounce notification
mails will be pruned whenever the parser's state is saved, though if the
size of the cache ever becomes a problem it could be pruned periodically to
keep the size in check.

\subsubsection{Mails deleted during delivery}

\label{Mails deleted during delivery}

The administrator can delete mails using \daemon{postsuper}; occasionally
mails in the process of being delivered will be deleted.  This results in
the log lines from the delivery agent (\daemon{local}, \daemon{virtual} or
\daemon{smtp}) appearing in the log file \textit{after\/} the mail has been
removed from the state tables and saved in the database.  The DELETE action
adds deleted mails to a cache, which is checked by the SAVE\_BY\_QUEUEID
action, and the current log line discarded if the following conditions are
met:

\begin{enumerate}

    \item The queueid is not found in the state tables. 
        
    \item The queueid is found in the cache of deleted mails.
        
    \item The timestamp of the log line is within 5 minutes of the end time
        of the mail.

\end{enumerate}

Sadly this solution involves discarding some data, but the complication
only arises eight times (tightly grouped in one log file, which is not in
the group used for testing); if this complication occurred more frequently
it might be desirable to find the mail in the database and add the log
line to it.

\subsection{Conclusion}

This section presented the core of the parser, starting with a very high
level view and the initial complications which arose.  A flow chart showing
the paths a mail may take through the nascent simplified algorithm was
provided, followed by an explanation of those paths, and a discussion of
the parser's emergent behaviour --- the data from the log files creates the
paths in the flow chart, they aren't specified anywhere in the parser.  The
framework which holds the parser together was covered next, after which
came a description of the current actions provided by the parser, and the
algorithm for analysing unparsed log lines to create \regexes{} for new
rules.  Detecting, diagnosing and defeating complications forms the largest
single portion of this document, mirroring the development of the parser,
The complications are described in the order they were overcome, with
subsequent problems affecting fewer mails (often by an order or magnitude),
though the time required to solve problems increased with each successive
problem.



\section{Coverage}

\label{parsing coverage}

\subsection{Introduction}

The discussion of the parser's coverage of Postfix log files is separated
into two parts: log lines covered and mails covered.  The first is
important because the parser should handle all (relevant) log lines it is
given; the second is equally important because the parser must properly
deal with every mail if it is to be useful.  Improving the former is
less intrusive, as it just requires new rules to be written; improving the
latter is much more intrusive as it requires changes to the parser
algorithm, and it can also be much harder to notice a deficiency.

\subsection{Log lines covered}

\label{log-lines-covered}

Parsing a log line is a three stage process:

\begin{enumerate}

    \item Check if there are any rules for the Postfix component which
        produced the log line; if not then skip the log line.

    \item Try each rule until a matching rule is found; if no match is
        issue a warning and move to the next log line.

    \item Execute the action specified by the rule.

\end{enumerate}

Full coverage of log lines requires the following:

\begin{enumerate}

    \item Each Postfix component whose log lines are of interest must have
        at least one rule or its log lines will be silently skipped; in the
        extreme case of zero rules the parser would happily skip every log
        line.  There may be any number of log lines from other programs
        intermingled in the log file, and there are some Postfix programs
        which do not produce any log lines of interest.

    \item There must be a rule to match each different log line produced by
        each program; if a log line is not successfully matched the parser
        will issue a warning.  Rules should be as specific and tightly
        bound as possible to ensure accurate parsing:\footnote{A rule which
        matches zero or more of any character will successfully parse every
        log line, but not in a meaningful way.} most log lines contain
        fixed strings and have a rigid pattern, so this is not a problem.

    \item The appropriate action to take --- discussed in
        \sectionref{mails-covered}.

\end{enumerate}

Full coverage of log lines is easy to achieve yet hard to maintain.  It is
easy to achieve full coverage for a limited set of log files (at the time
of writing the parser is tested with \numberOFrules{} rules, fully parsing
\numberOFlogFILES{} contiguous log files from Postfix 2.2 and 2.3), and new
rules are easy to add.  Maintaining full coverage is hard because other
servers have different restrictions with custom messages, \DNSBL{} messages
change over time, major releases of Postfix change warning messages
(usually adding more information), etc.,\ so over time the log lines drift
and change.  Graph~\refwithpage{rule hits graph} shows the number of hits
for each rule over all \numberOFlogFILES{} log files; it is obvious that a
small number of rules match the vast majority of the lines, and more than
half the rules match fewer than 100 times.

Warnings are issued for any log lines which are not parsed; no warnings are
issued for unparsed log lines while testing with the \numberOFlogFILES{}
test log files, so it can be safely concluded that there are zero false
negatives.  False positives are harder to quantify: short of examining each
of the 60,721,709 log lines and determining which \regex{} parsed it, there
is no way to be sure that every line was parsed by the correct \regex{},
making it impossible to quantify the false positive rate; however a random
sample of 6039 log lines was parsed and the results checked manually to
ensure that the correct \regex{} parsed each log line.\footnote{Each log
line was examined and the correct \regex{} identified from the
\numberOFrules{} rules in the database; the correct \regex{} was then
compared to the \regex{} which was used by the parser.}  The sample was
generated by running the following command:

\verb!    perl -e 'print if (rand 1 < 0.0001)' -n LOG_FILES!

\noindent{}to randomly extract roughly one line in every 10,000.  Although
on initial appearances exercising only 36 rules (from a total of
\numberOFrules{}) when parsing 6039 log lines seems quite low, after
examining graph~\refwithpage{rule hits graph} it becomes apparent that such
a low hit rate is to be expected; the reader should also bear in mind that
even when parsing all \numberOFlogFILES{} log files not all the rules are
exercised (some of the rules are for parsing log lines which only appear in
other log files).

\subsection{Mails covered}

\label{mails-covered}

Coverage of mails is much more difficult to determine accurately than
coverage of log lines.  The parser can dump its state tables in a human
readable form; examining these tables with reference to the log files is
the best way to detect mails which were not handled properly (many of the
complications discussed in \sectionref{additional complications} were
detected in this way).  The parser issues warnings when it detects any
errors, some of which may alert the user to a problem, e.g.\ when a queueid
is reused before the previous mail is fully dealt with, when a queueid or
\pid{} is not found in the state tables,\footnote{There will often be
warnings about a missing queueid or \pid{} in the first few hundred or
thousand log lines because the earlier log lines for those connections or
mails are in the previous log file; loading the saved state from the
previous log file will solve this problem.} or when there are problems
tracking a child mail (see \sectionref{tracking re-injected mail}).  There
should be few or no warnings when parsing, and when finished parsing the
state table should only contain entries for mails which had yet to be
delivered when the log files ended, or were accepted before the log files
began.

At the time of writing the parser is being tested with \numberOFlogFILES{}
log files.  There are 5 warnings produced, but because the parser errs on
the side of producing more warnings rather than fewer, those 5 warnings
represent 3 instances of 1 problem: 3 connections started before the first
log file, so their initial log entries are missing, leading to warnings
when their log lines are parsed.

The state tables contain entries for mails not yet delivered when the
parser finishes execution.  Ideally all they should contain are mails which
are awaiting delivery after the period covered by the log files, though
they may also contain mails whose initial entries are not contained in the
log files.  Any other entries are evidence of a failure in parsing or an
aberration in the log files.  After parsing the \numberOFlogFILES{} test
log files the state tables contain 18 entries, breaking down into:

\begin{itemize}

    \item 1 connection which started only seconds before the log files
        ended and had not yet been fully transferred from client to server.

    \item 1 mail which had been accepted only seconds before the log files
        ended and had not yet been delivered.

    \item 9 mails whose initial log entries were not present in the log
        files.

    \item 7 mails which had yet to be delivered due to repeated failures.

\end{itemize}

There are no mails in the state tables which should not be present, thus it
can be concluded that there are zero false negatives.  Once again,
determining the false positive rate is much harder, as manually checking
the results of parsing 13,850,793 connections and mails accepted, rejected,
bounced or delivered is infeasible.  There is considerable circumstantial
evidence that the false positive rate is quite low:

\begin{itemize}

    \item The parser is quite verbose when complaining about known problems
        (e.g.\ if a mail is tracked twice as described in
        \sectionref{tracking re-injected mail}), and no such warnings are
        produced during the test runs.

    \item Queueids and \pids{} naturally group together log lines belonging
        to one mail or connection respectively; it is extremely unlikely
        that a log line would be associated with the wrong connection.

    \item When dealing with the complications described in
        \sectionref{complications} and \sectionref{additional
        complications} the solutions are as specific and restrictive as
        possible, with the goal of minimising the number of false
        positives.  In addition the solution to the \textit{Out of order
        log files\/} complication described in \sectionref{out of order log
        lines} imposes conditions which each reassembled mail must comply
        with to be acceptable.

    \item Every effort has been made while developing to make the parser as
        precise, demanding and particular as possible.

\end{itemize}

Whereas verifying by inspection that the parser correctly deals with all
60,721,709 lines in the test log files is infeasible, verifying a subset of
those log files is a tractable, if extremely time consuming, task.  A
sample of log lines was obtained by randomly selecting a log file:

\verb!    perl -Mstrict -Mwarnings -MList::Util=shuffle \!\newline
\verb!            -e 'print [shuffle(@ARGV)]->[0];'!

The first 6000 lines of this log file (roughly 0.01\% of the total number
of log lines used in testing) was extracted:

\verb!    sed -n -e '1,6000p' logfile > test-log-segment!

It is important that the log lines used are contiguous so that all log
entries are present for as many of the connections as possible.  This log
segment was parsed with all debugging options enabled, resulting in 167,448
lines of output.\footnote{A mean of 27.908 lines of output per line of
input; each connection has 30 debugging lines, plus 21 debugging lines per
result.  Connections which have been cloned will have the cloned connection
in their debugging output, plus another 33 debugging lines.  Those numbers
are approximate, and may vary $\pm{}$ 2.  There is a linear relationship
between the number of log lines and debugging lines: $33(connections) +
30(accepted~~mails) + 21(results)$.  This formula is an approximation only,
and has not been rigorously verified.}  All 167,448 lines were examined in
conjunction with the input log and a dump of the resulting database,
verifying that for each of the input lines the parser used the correct rule
and executed the correct action, which in turn produced the correct result
and inserted the correct data in the database.  The log segment produced 4
warnings, 10 mails remaining in the state tables, and 1625 connections
correctly entered in the database.

Given the circumstantial and experimental evidence detailed above, the
author is confident that the false positive rate when reconstructing a mail
is exceedingly low, if not approaching zero.

\subsection{Summary}

Parser coverage is divided into two topics in this section: log lines
covered and mails covered.  The former is initially more important, as the
parser must successfully parse every line if it is to be complete, but
subsequently the latter takes precedence because reproducing the path a
mail takes through Postfix is the aim of the parser.  Increasing the
percentage of log lines parsed is relatively simple and non-intrusive:
adding new rules or modifying existing rules is simplified by the
separation of rules, actions and framework.  Improving the logical coverage
is harder, as the actions taken by Postfix must be reconstructed by the
author, and the new sequence of actions integrated into the existing model
without breaking the existing parsing.  Detecting a deficiency in the
parsing algorithm is also significantly harder than detecting unparsed log
lines, as the parser will warn about any unparsed line, whereas discovering
a flaw in the parser requires understanding of the warnings produced and
the mails remaining in the state table.  Rectifying a flaw in the parser
requires an understanding of both the parser and Postfix's log files, and
investigative work to determine the cause of the deficiency, followed by
further examination of the log files in developing a solution.


\section{Limitations and possible improvements}

\label{limitations-improvements}

\subsection{Introduction}

Every piece of software suffers from some limitations and there is almost
always room for improvement.

\subsection{Limitations}

\label{logging helo}

\begin{enumerate}

    \item Each new Postfix release requires new rules to be written to cope
        with the new log lines.  Similarly using a new \DNSBL{}, new policy
        server or new administrator defined rejection messages require new
        rules.

    \item It appears that the hostname used in the HELO command is not
        logged if the mail is accepted.\footnote{Tested with Postfix
        2.2.10, 2.3.11 and 2.4.7; this may possibly have changed in Postfix
        2.5. XXX TEST THIS AGAIN.}  Rectifying this is relatively simple:
        create a restriction which is guaranteed to warn for every accepted
        mail, as follows:

        \begin{enumerate}

            \item Create \texttt{/etc/postfix/log\_helo.pcre}
                containing:\newline \tab{}\texttt{/\^/~~~~WARN~Logging~HELO}

            \item Modify \texttt{smtpd\_data\_restrictions} in
                \texttt{/etc/postfix/main.cf} to contain\newline
                \tab{}\texttt{check\_helo\_access~/etc/postfix/log\_helo.pcre}

        \end{enumerate}

        Although \texttt{smtpd\_helo\_restrictions} seems like the natural
        place to log the HELO hostname, there will not be a queueid
        associated with the mail for the first recipient, so the log line
        cannot be associated with the correct mail.  There is guaranteed
        to be a queueid when the DATA command has been reached, and thus it
        will be logged by any restrictions taking effect in
        \texttt{smtpd\_data\_restrictions}.  There is no difficulty in
        specifying a HELO-based restriction in
        \texttt{smtpd\_data\_restrictions}, Postfix will perform the check
        correctly.

        Logging the HELO hostname in this fashion also prevents the
        complication described in \sectionref{Mail deleted before delivery
        is attempted} from occurring, but only in the case where there is a
        single recipient; in that case the recipient address will be logged
        also, but when there are multiple recipients no addresses are
        logged.  It is also possible to warn for every recipient,
        preventing the complication in \sectionref{Mail deleted before
        delivery is attempted} entirely.

    \item The algorithm does not distinguish between mails where one or
        more mails are rejected and a subsequent mail is accepted; it will
        appear in the database as one mail with lots of rejections followed
        by acceptance (this has already been mentioned in
        \sectionref{connection reuse}).  It does not appear to be possible
        to make this distinction given the data Postfix logs, though it
        might be possible to write a policy server to provide additional
        logging.

    \item The TIMEOUT action uses potentially incorrect heuristics to
        decide whether the timeout applies to an accepted mail or not,
        potentially leaving a mail in the state tables
        (\sectionref{timeouts-during-data-phase}).

    \item The program will not detect parsing the same log file twice,
        resulting in the database containing duplicate entries.

    \item The parser does not distinguish between log files produced by
        different sources when parsing; all results will be saved to the
        same database.  This may be viewed as an advantage, as log files
        from different sources can be combined in the same database, or it
        may be viewed as a limitation as there is no facility to
        distinguish between log files from different sources in the same
        database.  If the results of parsing log files from different
        sources must remain separate, the parser can easily be instructed
        to use a different database to store the results in.

    \item The solution to complication \sectionref{Mails deleted during
        delivery} involves discarding data.

\end{enumerate}

\subsection{Possible improvements}

\begin{itemize}

    \item Investigate and write the policy server referred to in limitation
        3 above.

    \item Improve the solution to complication \sectionref{Mails deleted
        during delivery} so that data is not discarded.

    \item Improve the heuristics used in
        \sectionref{timeouts-during-data-phase}, or develop another
        solution, to avoid incorrectly leaving a mail in the state tables.

\end{itemize}

\subsection{Summary}

This section has covered the limitations of the parser and possible
improvements which may be implemented in the future.



\section{Conclusion}

\label{conclusion}

Parsing Postfix log files appears at first sight to be an almost trivial
task, especially if one has previous experience in parsing log files, but
it turns out to be a much more taxing project than initially expected.  The
variety and breadth of log lines produced by Postfix is quite surprising,
because a quick survey of sample log files gives the impression that the
number of distinct log lines is quite small; this impression is due to the
uneven distribution exhibited by log lines produced in normal operation
(see \graphref{rule hits graph} for a vivid illustration of this).


Given the diverse nature of the log lines and the ease with which
administrators can cause new lines to be logged (\sectionref{postfix
background}), enabling users to easily extend the parser to deal with new
log lines is a design imperative (\sectionref{parser design}).  Despite the
resulting initial increase in complexity the task is quite tractable,
though it does raise efficiency and optimisation questions (answered in
\sectionref{parser efficiency}).  Providing a tool (\sectionref{creating
new rules}) to ease the generation of \regexes{} from unparsed log lines
should greatly help users add rules to parse formerly unparsed log lines.


The implementation not only substantially eases the parsing of new log
lines, it makes adding new actions (\sectionref{adding new actions}) a
relatively simple task.  The simplicity of adding a new action frees the
implementor from worrying about how their action might disrupt parsing of
other log lines or the behaviour of other actions, allowing their
concentration to focus on correctly implementing their new action.


The division of the parser into rules, actions and framework is unusual
because rules are separated so completely from the actions and framework;
although parsers are often divided (parsers based on the combination of
\texttt{lex} and \texttt{yacc}~\cite{lex-and-yacc} 
being an obvious example), the parts are usually quite internally
interdependent, and will be combined into a complete parser by the
compilation process; in contrast \parsername{} keeps the rules and actions
separate until the parser runs.  The actions and framework are not as
completely separated, as the actions depend on services provided by the
framework; however the actions and framework are not tightly integrated: it
would be possible, with some work, to completely separate the two.


The emergent behaviour~\cite{Wikipedia-Emergence} (\sectionref{Emergent
behaviour}) exhibited by the rules and actions is also interesting, and is
discussed after the flow chart (\sectionref{flow-chart}) and explanation of
the paths through the parser.  This emergent behaviour greatly eases the
process of adding new actions (\sectionref{adding new actions}), as the
actions do not have to be inserted into an explicit flow of control; new
actions will naturally find their place in the paths through the parser.

The flow of control in this parser is quite different from that of most
other parsers.  In most traditional parsers the parser has a current state,
and each state has a fixed set of acceptable next states, with unacceptable
states causing parsing to fail --- e.g.\ when a \textbf{C} parser sees the
keyword \textbf{for} it expects to immediately see a left parenthesis, with
any other input causing parsing to fail.\footnote{Comments will already
have been removed by the preprocessor.}  This parser is different: the
rule which matches the next log line for a mail dictates the action that
will be invoked, which is equivalent to the next state in other parsers;
thus the next log line for a mail dictates the next state for that mail.

The real difficulties arise once the parser is successfully dealing with
90\% of the log lines, as the irregularities and complications previously
explained only begin to become apparent once the vast majority of log lines
have been parsed successfully.  Adding new rules to deal with numerous,
infrequently occurring log lines is a simple task, albeit tiresome (though
alleviated by the tool described in \sectionref{adding new actions}),
whereas dealing with mails where information appears to be missing is much
more grueling.  Trawling through the log files, looking for something out
of the ordinary, possibly hundreds or even thousands of lines away from the
last mention of the queueid in question, is extremely time consuming and
error prone.  Sometimes the task is not to spot the line that is unusual,
but to spot that a line normally present is missing, i.e.\ to realise that
one line amongst hundreds is absent.  In all cases the evidence must be
used to construct a hypothesis to explain the irregularities, and that
hypothesis must then be tested in the parser; if successful, the parser
must be modified to deal with the irregularities, without adversely
affecting the existing functionality.  The complications described in
\sectionref{additional complications} were solved in the order they are
described in, and that order closely resembles the frequency with which
they occur; the most frequently occurring complications dominate the
warning messages produced, and so naturally they are the first
complications to be dealt with.

A parser's ability to correctly parse its input is extremely important; the
parser's coverage of its test log files is discussed in \sectionref{parsing
coverage}.  Both its success at parsing individual log lines
(\sectionref{log-lines-covered}) and its correctness in reconstructing each
mail's journey through Postfix (\sectionref{mails-covered}) are described
in detail, including the results of manually verifying the correct parsing
of a subset of the test log files.

\newpage




\appendix

\chapter{State of the Art Review}

\label{state of the art review}

At the start of this project ten Postfix log file parsers were tested, with
the hope of finding a suitable parser to build upon, rather than starting
from scratch.  There are not many Postfix log file parsers available ---
indeed it was quite difficult to find ten parsers to review for this
project --- and the functionality offered ranges from quite basic to much
more mature, depending on the needs of the author of the parser.  None of
those parsers were suitable for this project, so the decision was taken to
write a new parser.  The first parser reviewed is the only previously
published research in this area that the author is aware of; it parses
Postfix log files, but aim of the research is to show that presenting
extracted data in a more easily accessible format is useful to systems
administrators, rather than to improve anti-spam techniques..

The same ten parsers have been reviewed and compared to this project's
finished parser, to show how much effort would have been required to fulfil
the aims and requirements of this project.  It is important to compare and
contrast newly developed algorithms and parsers against those already
available, to accurately judge what improvements, if any, are delivered by
the newcomers.

XXX SHOULD THIS BE IN THE CONCLUSION INSTEAD\@?

There are some important differences between \parsername{} and the parsers
reviewed here:

\begin{enumerate}

    \item None of the parsers reviewed perform the kind of advanced parsing
        required for this project or deal with the complications described
        in \sectionref{complications}.

    \item Only \parsername{} enables parsing of new log lines without
        extensive and intrusive modifications to the parser; \parsernames{}
        architecture is described in \sectionref{why separate rules,
        actions, and framework?}.

    \item The parsers reviewed all produce a report of varying complexity
        and detail, whereas \gls{PLP} does not; it extracts data and leaves
        generation of reports from the data to other programs.  Using an
        \gls{SQL} database simplifies the process of generating such
        reports (discussed in \sectionref{database as API}); some sample
        queries are given in \sectionref{motivation}.  The parser developed
        for this project is designed to enable much more detailed log file
        analysis by providing a stable platform for subsequent programs to
        develop upon.

    \item Most of the reviewed parsers silently ignore log lines they
        cannot handle, whereas \parsername{} complains loudly about every
        single log line it fails to parse.  The exception is AWStats, which
        outputs the percentage of input lines it was unable to parse, but
        does not output the lines themselves.

    \item A minor difference is that most parsers do not handle compressed
        files; both \parsername{} and Splunk handle them transparently,
        without user intervention; Sawmill and Lire can be configured to
        support compressed files, but Sawmill exhibits a dramatic increase
        in parsing time when doing so.  Although this is a minor
        disparity, support for reading compressed log files is quite
        helpful, as it dramatically reduces the disk space required to
        store historical log files.

    \item Some of the parsers reviewed save the extracted data to a data
        store, but the majority discard all data once they have finished
        generating their report, making historical analysis impossible
        without parsing all log files every time.

\end{enumerate}

Each of the reviewed parsers was tested with the \numberOFlogFILES{} test
log files described in \sectionref{parser efficiency}.  The data extracted
by \parsername{} is documented in \sectionref{connections table} and
\sectionref{results table}; for convenience that list is repeated here:
server \gls{IP}, server hostname, client \gls{IP}, client hostname, HELO
hostname, queueid, start time, end time, \gls{SMTP} code, sender,
recipient, size, message ID\@.

\section{Log Mail Analyser}

\label{prior art}

There only appears to be one prior published paper about parsing Postfix
log files: \textit{Log Mail Analyzer: Architecture and Practical
Utilizations\/}~\cite{log-mail-analyser}.  The aim of \gls{LMA} is quite
different from \parsername{}: it attempts to present correlated data from
log files in a form suitable for a systems administrator to search using
the myriad of standard Unix text processing utilities already available.
It produces a \gls{CSV} file and either a MySQL
(\url{http://www.mysql.com/}) or Berkeley DB
(\url{http://www.oracle.com/database/berkeley-db/index.html}) database.
The decision to support both \gls{CSV} and Berkeley DB appears to have been
a serious limitation: XXX EXTEND\@: LIMITATIONS WILL BE EXPLAINED LATER OR
SOMETHING\@.  Very little documentation is provided with \gls{LMA}, though
some documentation is available in~\cite{log-mail-analyser}.  Studying the
source code is informative, though this author had difficulty as the
authors of \gls{LMA} wrote in Italian.

\gls{CSV} is a very simple format where each record is stored in a single
line, with fields separated by a comma or other punctuation symbol.
Problems with \gls{CSV} files include the need to escape separators in the
data stored, providing multiple values for a field (e.g.\ multiple
recipients), and adding new fields.  There is no standard mechanism to
document the fields or the separator, unlike \gls{SQL} databases where
every database includes a schema naming the fields and the type of data
they store (integer, text, timestamp, etc.).  The \gls{CSV} record format
is not documented, but the output file contains a comment giving the
format:\newline{} \texttt{\# Timestamp|Nome Client|IP Client|IP
Server|From|To|Status|Size} \newline{}\gls{LMA} treats lines starting with
\texttt{\#} as comments, but not all \gls{CSV} parsers will.

Berkeley DB only supports storing simple \textbf{(key, value)} pairs,
unlike \gls{SQL} databases that store arbitrary tuples.  In \gls{LMA}'s
main table the key is an integer referred to by secondary tables, and the
value is a \gls{CSV} line containing all of the data for that row.  The
secondary by-sender, by-recipient, by-date, and by-\gls{IP} tables use the
sender/recipient/date/\gls{IP} as the key, and the value is a \gls{CSV}
list of integers referring to the main table.  This effectively
re-implements \gls{SQL} foreign keys, but without the functionality offered
by even the most basic of \gls{SQL} databases (joins, ordering, searches,
etc.).  It also requires custom code to search on some combination of the
above, though the authors of \gls{LMA} did provide some queries: IP-STORY,
FROM-STORY, DAILY-EMAIL, and DAILY-REJECT\@.  Berkeley DB appears to be the
least useful of the three output formats: it does not provide the
functionality of a basic \gls{SQL} database, and unlike \gls{CSV} files it
can not be used with standard Unix text processing tools.

The schema used with the MySQL database is undocumented, but at least it is
possible to discover the schema with an existing \gls{SQL} database, unlike with
Berkeley DB\@; all \gls{SQL} databases embed the schema into the database
and provide commands for displaying it.  Berkeley DB does not embed a
schema, as there is neither requirement nor benefit; it only provides
\textbf{(key, value)} pairs, so any additional structuring of the data is
imposed by the application, thus the application must document this
structure.  MySQL support was not tested because there is no documentation
on the schema required.

Whether a MySQL database or Berkeley DB table is chosen in addition to the
\gls{CSV} output, \gls{LMA} stores the following data: time and date,
client hostname and \gls{IP} address, server \gls{IP} address, sender and
recipient addresses, \gls{SMTP} code, and size (for accepted mails only).
It is unclear which time and date is stored: start time, end time, or
delivery time?  Unlike \parsername{} it does not store the server hostname,
HELO hostname, queueid, start and end times, timestamps for each log line,
or message id (for accepted mails only).  Handling of multiple recipients,
\gls{SMTP} codes, or remote servers\footnote{A single mail may be sent to
multiple remote servers if it was addressed to recipients in different
domains, or Postfix needs to try multiple servers for one or more
recipients.} is not explained; experimental observation shows that multiple
records are added when there are multiple recipients (sadly the records are
not associated or linked in any way), and presumably the same approach is
taken when there are multiple destination servers.

\gls{LMA} requires major changes to the parser code to parse new log lines
or to extract additional data.  The code is structured as a long series of
blocks that each handle all log lines matching a single regex, so parsing
new log lines requires modifying an existing regex or carefully inserting a
new block in the correct place; extracting extra data will require
modifying multiple blocks, regexes, or both.

\gls{LMA} does not deal with any of the complications discussed in
\sectionref{complications}, except for correlating log lines by queueid;
not correlating log lines by pid means it cannot correlate most rejections.
It does not differentiate between different types of rejections, so it is
not suitable for the purposes of this project; the data about which
restriction caused the rejection is discarded, whereas the main goal of
this project is to retain that data to aid optimisation and evaluation of
anti-spam techniques.  \gls{LMA} fails to parse Postfix log files generated
on Solaris hosts because the fields automatically prepended to each log
line differ from those added on Linux hosts; log files from Solaris hosts
(and possibly other operating systems) thus require preprocessing before
parsing by \gls{LMA}.  Once the preprocessing has been performed on the
\numberOFlogFILES{} test log files \gls{LMA} parses the log files without
complaint, although it produced 32 entries in its output file for every
rejection in the input log file; it also missed some 40\% of delivered
mail.  Once these glaring deficiencies were discovered the author did not
waste any more time checking the results.

\gls{LMA} does provide some simple reports: IP-STORY, FROM-STORY,
DAILY-EMAIL and DAILY-REJECT\@.  These reports search the Berkeley DB files
for matching records: the first three extract \gls{CSV} lines for the
specified client \gls{IP} address, sender address, or date respectively.
DAILY-REJECT initially failed with an error message from the Perl
interpreter;\footnote{The error messages were: \newline{}\texttt{Undefined
subroutine \&main::LIST called at queryDB.pl line
372.}\newline{}\texttt{Undefined subroutine \&main::EXTRACT\_FROM\_DB
called at queryDB.pl line 379.}} after making corrections to the code it
worked, extracting the \gls{CSV} lines for the specified day where the
\gls{SMTP} code signifies a rejection.  All of these reports are trivially
simple to produce from the \gls{CSV} file using the standard Unix tool
\texttt{awk}\glsadd{awk}; the most complicated, DAILY-REJECT, is merely:

% perl queryDB.pl -dayreject 2007-01-26 > lma-query

\begin{verbatim}
awk -F\| 'BEGIN { previous = "" };
    $1 ~ /2007-01-26/ && $7 != "250" && $0 != previous 
    { print $0; print " "; previous = $0; }' lma_output.txt
\end{verbatim}

Notes about the command above:

\begin{itemize}

    \item It outputs a line containing only a single space after each
        matching record, to accurately replicate the output of
        DAILY-REJECT\@.

    \item DAILY-REJECT considers all \gls{SMTP} codes except \texttt{250}
        to be rejections; this includes invalid \gls{SMTP} codes such as
        \texttt{0} and \texttt{deferred}, so the awk command does too.
        These invalid \gls{SMTP} codes are most likely present because of
        incorrect parsing by \gls{LMA}.

    \item \gls{LMA} produces 32 output lines in its \gls{CSV} file for
        every single line it should have produced; the command above
        suppresses duplicate sequential lines.

\end{itemize}

There are some differences between the output from DAILY-REJECT and the
\texttt{awk} command; the author did not spend substantial time attempting
to explain these differences.

\begin{enumerate}

    \item The output from DAILY-REJECT is missing some records which are
        present in the \gls{CSV} file; this may be because it uses the
        Berkeley DB files instead, and there may be differences between the
        contents.

    \item Some records output by DAILY-REJECT are truncated: they are
        missing the last | separating fields and the newline following it,
        so the line containing only a single space is concatenated with the
        record.

\end{enumerate}

In summary, \gls{LMA} appears to be a proof of concept, written to
demonstrate the point of their paper (that having this information in an
accessible fashion is useful to systems administrators), rather than a
program designed to be useful in a production environment.

% Literature review notes:
%
% Hard-coded parsing, requiring code changes to add more.  Attempts to
% correlate log lines, saves data to database for data mining purposes.
% Hard to extend/expand/understand.  Appears to only save: date and hour,
% DNS name and \gls{IP} address host, mail server \gls{IP} address, sender,
% receiver and e-mail status (sent, rejected).  Undocumented schema.
% Design decision to use \gls{CSV} as an intermediate format between the
% log file and the database seems to have been restrictive.  Appears to
% require a queueid but majority of log lines (e.g.\ rejections) lack a
% queueid.  Supports whitelisting \gls{IP} addresses when parsing logs, but
% whitelisting when generating reports/data mining would be preferable.
% Supporting Berkeley DB is probably limiting the software - an example is
% the difficulty in searching a pipe-delimited string, so they have
% re-implemented foreign keys with tables keyed by ip address etc.\
% pointing at the main table - this also will not scale well.  There does
% not appear to be any attempt to deal with the complications I have
% encountered: their parsing is not detailed enough to encounter them.  It
% does not run properly; does not create any output; throws up errors.

\section{Pflogsumm}

\begin{quotation}

    pflogsumm is designed to provide an over-view of Postfix activity, with
    just enough detail to give the administrator a ``heads up'' for
    potential trouble spots.

\end{quotation}

\noindent{}\url{http://jimsun.linxnet.com/postfix_contrib.html} \newline{}
(Last checked 2008/11/23.)

Pflogsumm produces a report designed for troubleshooting rather than
in-depth analysis.  It does not support saving any data, and it does not
extract any data that it does not require to produce its report, e.g.\ it
does not extract the HELO hostname, queueid, start and end times,
timestamps for each log line, or message id.  Both the parsing and
reporting are difficult to extend because it is a specialised tool, unlike
the easily extensible design of \parsername{}.  It does not correlate log
lines by queueid or \gls{pid}, and does not need to deal with the
complications encountered during this project (\sectionref{complications}).
Pflogsumm produces a useful report, and successfully parsed the
\numberOFlogFILES{} log files it was tested with.\footnote{The results it reported
were not verified in detail, but it did not report any errors, and has a
very good reputation amongst Postfix users.}
Pflogsumm has many options to include or exclude certain
sections of the report; by default it includes the following:

\begin{itemize}

    \item Total number of mails accepted, delivered, and rejected.  Total
        size of mails accepted and delivered.  Total number of sender and
        recipient addresses and domains.

    \item Per-hour averages and per-day summaries of the number of mails
        received, delivered, deferred, bounced, and rejected.

    \item For received mail: per-domain totals for mails sent, deferred,
        average delay, maximum delay, and bytes delivered.  For received
        mail: per-domain totals for size and number of mails received.

    \item Number and size of mails sent and received for each address.

    \item Summary of why mail delivery was deferred or failed, why mails
        were bounced, why mails were rejected, and warning messages.

\end{itemize}

\section{Sawmill Universal Log File Analysis and Reporting}

\begin{quotation}

    Sawmill is a Postfix log analyzer (it also support 818 other log
    formats). It can process log files in Postfix format, and generate
    dynamic statistics from them, analyzing and reporting events. Sawmill
    can parse Postfix logs, import them into a SQL database (or its own
    built-in database), aggregate them, and generate dynamically filtered
    reports, all through a web interface. Sawmill can perform Postfix
    analysis on any platform, including Window, Linux, FreeBSD, OpenBSD,
    Mac OS, Solaris, other UNIX, and more.

\end{quotation}

\noindent{}\url{http://www.thesawmill.co.uk/formats/postfix.html}
\newline{} \url{http://www.thesawmill.co.uk/formats/postfix_ii.html}
\newline{} \url{http://www.thesawmill.co.uk/formats/beta_postfix.html}
\newline{} (Last checked 2008/11/23.)

Sawmill is a general purpose commercial product that parses 818 log file
formats (as of 2008/11/23) and produces reports from the extracted data.
Its data extraction facilities (described later) are too limited to save
sufficient data for the purposes of this project: although it can extract
three different sets of data from Postfix log files, they are not
interlinked in any way.  The documentation does not suggest that any
attempt is made to correlate log lines by either queueid or pid or to deal
with the difficulties documented in \sectionref{complications}.

Sawmill has three different Postfix log file parsers, extracting three
different sets to data:

\begin{enumerate}

    \item \url{http://www.thesawmill.co.uk/formats/postfix.html} \newline{}
        Fields extracted: from, to, server, UID, relay, status, number of
        recipients, origin hostname, origin \gls{IP}, and virus.  It also
        counts the number of and total size of all mails delivered.  The
        fields \texttt{server}, \texttt{uid}, \texttt{relay}, and
        \texttt{virus} are not explained in the documentation:
        \texttt{server} is probably the hostname or \gls{IP} of the server
        the mail is delivered to; \texttt{relay} might be the delivery
        method: \gls{SMTP}, local delivery, or \gls{LMTP}; \texttt{uid}
        might be the uid of the user submitting mail locally.  Postfix does
        not perform any form of virus checking (though it has many options
        for cooperating with an external virus scanner), so the
        \texttt{virus} field is a mystery.

    \item \url{http://www.thesawmill.co.uk/formats/postfix_ii.html}
        \newline{} Fields extracted: from, to, RBL list, client hostname,
        and client \gls{IP}\@.  It also counts the number and total size of
        all mails delivered.  

    \item \url{http://www.thesawmill.co.uk/formats/beta_postfix.html}
        \newline{} Fields extracted: from, to, client hostname, client
        \gls{IP}, relay hostname, relay \gls{IP}, status, response code,
        RBL list, and message id.  It also counts the number and size of
        all mails delivered, processed, blocked, expired, and bounced.

\end{enumerate}

Even if the three data sets were combined Sawmill would extract less data
than \parsername{}: it omits the HELO hostname, queueid, and start and end
times.  Sawmill does not extract any data about rejections except when the
rejection is caused by a \gls{DNSBL} check (\texttt{RBL list} in the list
of fields).

The source code is available in an obfuscated form, and the product is
quite expensive, requiring a \euros{100} + VAT licence per report
(discounts are available when buying multiple licences); in contrast
\parsername{} is free to use and the code is freely available.  Sawmill is
supplied with thorough and well written documentation; everything the
author searched for was documented, except the MySQL database schema.  A
commercial version of MySQL is required due to MySQL licensing
restrictions, but Sawmill's documentation explains why and includes
instructions on how to compile Sawmill so that it can use a non-commercial
version of MySQL (this was not attempted during the review process).

Sawmill's web interface supports searching on any combination of the fields
it extracts, and searches performed using the web interface produced
accurate results.  The interface for searching is neither as simple to use
nor as informative as the interface provided by Splunk.  The administrative
interface is much easier to use than Splunk's: it took only five minutes to
start parsing a whole directory of log files.  

When tested with the \numberOFlogFILES{} test log files it performed
adequately, though the rate it processed log files at did slow down
noticeably as it progressed.  Sawmill supports reading compressed log files
but it exhibits a dramatic slow down when doing so: it took six hours to
parse the first half of the log files, and twelve hours to parse the next
third; after twenty four hours parsing the remaining sixth it crashed due
to lack of disk space.  On the second parsing attempt the log files were
uncompressed beforehand and parsing took eight hours.

In summary Sawmill suffers from being a general purpose product; it is
probably much more useful when parsing log files where each log line is
self-contained (e.g.\ web server log files), rather than log files
containing interlinked log lines.  It is not suitable as a base for this
parser, as the source code made available is obfuscated and not intended
for modification; in addition the architecture would probably need to be
overhauled or replaced to deal with correlating log lines.

\section{Splunk}

\begin{quotation}

    Splunk is IT Search.

    Search and navigate IT data from applications, servers and network
    devices in real-time. Logs, configurations, messages, traps and alerts,
    scripts, code, metrics and more. If a machine can generate it ---
    Splunk can eat it. It's easy to download and use and it's very
    powerful.

\end{quotation}

\noindent{}\url{http://www.splunk.com/} \newline{}
(Last checked 2008/11/23.)

XXX THE FIRST PARAGRAPH NEEDS TO BE REWRITTEN\@; PICK POINTS, THEN COMPARE
AND CONTRAST\@.

Splunk aims to index all of an organisation's log files, providing a
centralised view capable of searching and correlating diverse log sources.
The web interface supports complicated searches, providing statistics and
graphs in real time, a facility not provided by \parsername{} (report
generation has been deferred to a subsequent program).  Searches can be
based on the fields extracted by Splunk or the full text of the log line.
Splunk allows quite complicated searches, but does not make the raw data
available in an accessible form.  Saved searches can be run periodically
and the results emailed to a recipient or sent to an external program for
further processing (though maybe without the graphs and detailed
statistics); the author was unable to save searches, though that may have
been due to limitations in the free version.  The database is not available
for use by external programs, whereas \parsername{} provides the database
and leaves it to the user to utilise it without limit or restriction.  The
interface is optimised for interactive use rather than automated queries
and it does not appear to be possible to write independent tools to utilise
the Splunk database;.  Some additional Postfix reports are supposedly
available at \url{http://www.splunkbase.com/}, but the author was unable to
find any Postfix reports, or indeed for any other log file types: every
category was empty, even those that the web site claimed had many reports
available.  Many types of graphs can be generated, though most are
variations of a bar or pie chart, except bubble and heatmap graphs.  It is
easy to drill down through the graphs to extract a portion of the data
(e.g.\ select the hour with the largest number of events, then select a
particular host, and finally a specific address), though it is not possible
to search on partial words.  All searches performed using the indexed data
returned reasonable results.

The web interface is quite attractive and simple to use when searching, but
as an administrator it seems unnecessarily difficult to perform simple
tasks.  When testing Splunk it took roughly 30 minutes to figure out how to
add a single log file to be indexed so that it could be searched, with the
downside that the log file was copied into a spool directory before
indexing, doubling the disk space usage.  The next test was to index all
the log files in a particular directory, but after three hours, numerous
futile attempts, and reading all the available documentation, the author
admitted defeat.  Using the \gls{CLI} rather than the web interface was
partially successful: the command \newline{} \tab{} \texttt{splunk find
logs }\textit{log-directory\/}\newline{} added 40 of the
\numberOFlogFILES{} log files to the queue for indexing.  Further attempts
enqueued the same 40 log files, without explaining why the others were
excluded.\footnote{The log files appear to have been indexed once only;
presumably Splunk keeps track of the files it has indexed and discards
requests to index files for a second time.  This may or may not be a useful
feature for \parsername{}.} There did not appear to be an option to ensure
the log files would be processed in the order they were created, though
this may be neither necessary nor beneficial with Splunk.  Subsequently the
author was successful in adding a single file at a time using the
\gls{CLI}:\newline{} \tab{} \texttt{splunk add tail
}\textit{filename\/}\newline{} A simple loop was then enough to add all the
desired log files.  Splunk will periodically check all indexed files for
updates unless they are manually removed from its list; this may or may not
be useful behaviour.  Splunk did not appear to have any difficulty in
indexing the log files, once they had been successfully added to its queue.
\parsername{} parses the logs it is instructed to parse, in the order
given; periodic parsing of logs is a task an administrator can easily
achieve with \texttt{cron(8)} and \texttt{logrotate(8)}.

Copious documentation is made available on \url{http://www.splunk.com/},
but poor organisation and sheer abundance makes it extremely hard to find
useful information.  Searches confusingly tended to return results from old
documentation rather than new.  In general the documentation appears to
have been written by someone intimately acquainted with the software, who
has difficulty understanding how a newcomer would approach tasks or the
questions they would ask.

Splunk supports reading compressed log files without any configuration by
the user.  The free version of Splunk limits the volume of data indexed per
day to 500MB, though a trial Enterprise licence is available that allows
indexing of up to 5GB of data per day.  In 2007 the cheapest licenced
version cost \$5000 plus \$1000 support, and limited the volume of data
indexed per day to 500MB\@.  Prices were removed from the Splunk website
during 2008; now Splunk's sales team must be contacted for a quote.
Typical log file sizes for a small scale mail server are given in
\sectionref{parser efficiency}.

When parsing Postfix log files Splunk parses the standard
syslog\glsadd{syslog} fields at the beginning of the log line, and extracts
any \texttt{key=value} pairs occurring after the standard syslog prologue:
to and from addresses, HELO hostname, and protocol (\gls{SMTP} or
\gls{ESMTP}).  \parsername{} extracts noticeably more data (client and
server \gls{IP} and hostname, queueid, start and end times, timestamps for
each log line, \gls{SMTP} code, and message ID), though it does not make
the full text of the line available (this could be trivially added if
desired, but would greatly increase the size of the resulting database).
The full power of \gls{SQL} is available when searching the data extracted
by \parsername{}, allowing the user to search on arbitrarily complicated
conditions.

Splunk is a generic tool, so it lacks any Postfix specific support over and
above extracting the \texttt{key=value} fields from a log line; most
importantly it makes no attempt to correlate log lines by queueid or
\gls{pid}, or to handle any of the other myriad complications discussed in
\sectionref{complications}.  Its source code is unavailable, so it could
not be used as a base for this project, even if it fulfilled all other
requirements.

\section{Isoqlog}

\begin{quotation}

    Isoqlog is an MTA log analysis program written in C. It designed to
    scan qmail, postfix, sendmail and exim logfile and produce usage
    statistics in HTML format for viewing through a browser. It produces
    Top domains output according to Sender, Receiver, Total mails and
    bytes; it keeps your main domain mail statistics with regard to Days
    Top Domain, Top Users values for per day, per month and years.

\end{quotation}

\noindent{}\url{http://www.enderunix.org/isoqlog/} \newline{}
(Last checked 2009/01/11.)

Isoqlog's report misses most of the information gathered by \parsername{}:
the data extracted is limited to the number of mails sent by each sender,
and it only reports on senders from the domains listed in its configuration
file, making it impossible to produce complete reports.  It ignores all log
lines except those with today's date, so it is impossible to analyse
historical log files, and testing with the \numberOFlogFILES{} test log
files was pointless.  It does maintain a record of data previously
extracted, which the newly extracted data is merged into; the format of the
data store is undocumented.  It does not utilise rejection log lines in any
way, so is unsuitable for the purposes of this project.  Its parsing is
completely inextensible, indeed is almost incomprehensible, relying on
\texttt{scanf(3)}, unexplained fixed offsets, and low level string
manipulation; it is the opposite end of the spectrum to \parsernames{}
parsing.  It does not handle any of the complications discussed in
\sectionref{complications}, does not gather the breadth of data required
for this project, and ignores the majority of log lines produced by
Postfix.

\section{AWStats}

\begin{quotation}

    AWStats is a free powerful and featureful tool that generates advanced
    web, streaming, ftp or mail server statistics, graphically. This log
    analyzer works as a CGI or from command line and shows you all possible
    information your log contains, in few graphical web pages. It uses a
    partial information file to be able to process large log files, often
    and quickly. It can analyze log files from all major server tools like
    Apache log files (NCSA combined/XLF/ELF log format or common/CLF log
    format), WebStar, IIS (W3C log format) and a lot of other web, proxy,
    wap, streaming servers, mail servers and some ftp servers.

\end{quotation}

\noindent{}\url{http://awstats.sourceforge.net/} \newline{}
\url{http://awstats.sourceforge.net/awstats.mail.html} \newline{}
\url{http://awstats.sourceforge.net/docs/awstats_faq.html#MAIL} \newline{}
(Last checked 2009/01/11.)

AWStats will produce simple graphs for many different services, but
supporting many different services without special purpose code limits its
functionality.  The data it will extract from an \gls{MTA} log file is
limited in comparison to \parsername{}: time2, email, email\_r, host,
host\_r, method, url, code, and bytesd.  There is no explanation for any of
those fields in the documentation (\parsername{} provides copious
documentation), so the author could not understand the extracted data, nor
determine what data is missing in comparison to \parsername{}.  AWStats
coerces Postfix log files into Apache\footnote{The Apache web server is the
most popular HTTP server in use over the past 10 years; more information is
available at \url{http://httpd.apache.org/}.} format log files, for
analysis by AWStats' HTTP log file parser.  The converting parser only
deals with a small portion of the log lines generated by Postfix, silently
skipping those it cannot deal with, and does not distinguish between
different types of rejection; it would be a lot of work to extend it to
handle new log lines.  Although it does correlate log lines by queueid (not
by pid), it does not deal with any of the other complications described in
\sectionref{complications}.  AWStats supports saving data but the format of
the saved data is not documented, as far as the author could tell.  It also
supports reading compressed log files, but that functionality was not
tested.

When tested with the \numberOFlogFILES{} test log files AWStats' reported
that it parsed 9,240,075 (88.70\%) of 10,416,129 log lines, skipping
1,176,050 (11.29\%) corrupt log lines; however there are
\numberOFlogLINES{} lines in the \numberOFlogFILES{} log files, so AWStats
parsed only 17.15\% of the input lines, ignoring the remaining 82.85\%.

The graphs it produces give an overview of mails received for the last
calendar month, showing:

\begin{itemize}

    \item The number of mails accepted from each host.

    \item How many mails were received by each recipient.

    \item The average number of mails accepted by the server per-day and
        per-hour.

    \item A summary of the \gls{SMTP} codes used when rejecting delivery
        attempts.

\end{itemize}

AWStats was not a suitable base for this project, because it assumes that
all log files can be rewritten to be compatible with web server log files,
and will contain similar data; coercing Postfix log files into web server
log files, without substantial data loss, would require fully parsing the
Postfix log files, so AWStats would not be required.  It may be possible to
use AWStats graphing capabilities to generate reports, by generating log
files to use as input to AWStats from the data extracted by \parsername{}.

\section{Anteater}

\begin{quotation}

    The Anteater project is a Mail Traffic Analyser. Anteater supports
    currently the logformat produced by Sendmail and by Postfix. The tool
    is written in 100\% C++ and is very easy to customize. Input, output,
    and the analysis are modular class objects with a clear interface.
    There are eight useful analyse modules, writing the result in plain
    ASCII or HTML, to stdout or to files.

\end{quotation}

\noindent{}\url{http://anteater.drzoom.ch/} \newline{}
(Last checked 2009/01/11.)

Anteater does not have any English documentation so it is impossible for
this author to accurately comment on the analysis it performs.  It did not
run successfully when tested, and its parsing would certainly be out of
date as Postfix has evolved considerably since this tool was last updated
(2003/11/06).  As it neither ran successfully nor has documentation the
author can read a detailed review cannot be provided.

The Debian project (\url{http://www.debian.org/}) provides a manual page
with the copy of anteater it distributes, so the author was at least able
to run anteater with the correct arguments; sadly anteater produced zero
for every statistic, presumably because it was unsuccessful in parsing the
log lines.

\section{Yet Another Advanced Logfile Analyser}

\begin{quotation}

    yaala is a very flexible analyser for all kinds of logfiles. It uses
    parsers to extract information from a logfile, an SQL-like query
    language to relate the information to each other and an output-module
    to format the information appropriately.

\end{quotation}

\noindent{}\url{http://yaala.org/} \newline{}
(Last checked 2009/01/11.)

YAALA uses a plugin-based system to analyse log files and produce HTML
output reports, with all the parsing and report generation handled by
modules.  Using YAALA as a base would be only slightly less work than
starting from scratch, as both the input and output modules would need to
be written specially; it may even be more work to implement the parser
within the constraints of YAALA\@.  YAALA supports storing previously
gathered data using Perl's Storable module~\cite{perl-storable}, so other
Perl programs can use Storable to load, examine, and optionally modify the
data; \parsername{} uses a well documented database which is accessible
from the majority of programming languages.  This information was gleaned
from the source code, as the documentation is sadly lacking.

YAALA provides a Postfix parser that extracts the following two types of
fields from specific log lines:

\begin{eqlist}

    \item [Aggregations:] count (not explained), bytes (sum of bytes
        transferred).

    \item [Keyfields:] incoming\_host, outgoing\_host, date, hour, sender,
        recipient, defer\_count, delay.  Which date and hour are stored is
        not documented: start time, end time, delivery time, or another
        time?

\end{eqlist}

\noindent{}YAALA's Postfix parser extracts some of the fields \parsername{}
does: it stores either the \gls{IP} or the hostname for client and server,
not both; it omits the HELO hostname, queueid, \gls{SMTP} code, size of
each accepted mail, start and end times, timestamps for each log line, and
message ID\@.  It extracts some data that \parsername{} does not: the delay
in delivering each mail, and how many times delivery was deferred for each
mail; these could easily be extracted by \parsername{} if desired.  XXX
EXTRACT DELAY AND DEFER ONCE AUTOMATIC EXTRACTION HAS BEEN FINISHED\@.
Unlike \parsername{}, YAALA does not maintain separate counters for each
restriction; this rules out the possibility of using the collected data for
optimisation, testing or understanding of restrictions.  YAALA's Postfix
parser does not deal with the complications explained in
\sectionref{complications}, though it does correlate log lines by queueid.

YAALA provides a mini-language based on \gls{SQL} that is used when
generating reports; sample reports can be seen
at~\url{http://www.yaala.org/samples.html}.  Example query for HTTP proxy
servers: \newline{} \tab{} \texttt{requests BY file WHERE host =\~{}
Google} \newline{} The mini-language is quite limited and cannot be used to
extract data for external use, merely to create reports.  Only data
selected by the query will be saved in the data store; other data will be
discarded, and removed from the data store if already present.

Testing YAALA was unsuccessful because all the select clauses tried
produced a similar error message:
\newline{}\tab{}\texttt{lib/Yaala/Data/Core.pm: Unavailable aggregation
requested:} \newline{}\tab{}\tab{}\texttt{``bytes''. Returning 0.}
\newline{}  The underlying reason for this is that YAALA only parsed 408
(0.11\%) of 360632 log lines in the first log file; it was not tested with
the remainder of the \numberOFlogFILES{} log files.

In summary YAALA provides a Postfix parser that tries to parse the most
common Postfix log lines only, provides reasonably flexible report
generation from the limited data extracted, but has no facilities to
extract data for use in other tools.

\section{Lire}

\begin{quotation}

    As any good system administrator knows, there's a lot more to keep
    track of in an active network than just webservers. Lire is hands down
    the most versatile log analysis software available today. Lire not only
    keeps you informed about your HTTP, FTP, and mail traffic, it also
    reports on your firewalls, your print servers, and your DNS activity.
    The ever growing list of Lire-supported services clearly outstrips any
    other software, in large part thanks to the numerous volunteers who
    have pioneered many new services and features. Lire is a total solution
    for your log analysis needs.

\end{quotation}

\noindent{}\url{http://logreport.org/lire.html} \newline{}
(Last checked 2009/01/11.)

Lire is a general purpose log file parser supporting many different types
of log file.  It takes a similar approach to YAALA, using plugins to parse
different log file types.  The data extracted by its Postfix parser is not
clearly documented; the manual says only:

\begin{quotation}

    The email servers' reports will show you the number of deliveries and
    the volume of email delivered by day, the domains from which you
    receive or send the most emails, the relays most used, etc.

\end{quotation}

\noindent{}Examining the source code reveals that the parser looks for
\texttt{<key>=<value>} pairs in each log line, extracts them, and
correlates the data by queueid.  This approach will find the following
fields: HELO hostname, queueid, \gls{SMTP} code, sender and recipient
addresses, and size of accepted mails.  It is unclear if the parser will
extract any further data.  Lire misses the following fields extracted by
\parsername{}: client and server \gls{IP} and hostname, start and end
times, timestamps of each log line, and message ID\@. 

Lire supports multiple output formats for generated reports (text, HTML,
PDF, and Excel 95) but the reports do not appear to be customisable;
\parsername{} does not produce any reports.  Lire's report contains less
detail than Pflogsumm, and is considerable harder to configure.  Lire
supports saving extracted data for later report generation, but accessing
this data from another application is undocumented; given the source code
it should be possible, with enough time and effort, to understand the
format.  \parsername{} uses an \gls{SQL} database to make accessing the
extracted data as easy as possible.  

Like AWStats and Logrep, Lire attempts to correlate log lines by queueid,
but not by \gls{pid}, so the complete list of recipients for a mail should
be available; however its parser extracts only part of the available data
and makes no attempt to deal with the other complications described in
\sectionref{complications}.  When testing Lire on the \numberOFlogFILES{}
test log files it performed reasonably well: the numbers it reports appear
reasonable, and the subset verified by the author were correct.  Its report
provided summaries of: 

\begin{itemize}

    \item Delivery status and failed deliveries.

    \item Sender and recipient domains and servers.

    \item Number of deliveries and bytes per-day and per-hour.

    \item Recipients by domain.

    \item Deliveries by relays, by size, and by delay.

    \item Delays by server and by domain.

    \item Which pair of correspondents exchanged the highest number of
        emails.

\end{itemize}

Lire would not be a suitable base for this project: it does not extract
enough data; does not deal with rejections in any way; does not make the
extracted data easily available to other programs.  Its parser is
in-extensible but could easily be replaced, however that would require
writing a parser from scratch, so would not be any easier.

\section{Logrep}

\begin{quotation}

    Logrep is a secure multi-platform framework for the collection,
    extraction, and presentation of information from various log files. It
    features HTML reports, multi dimensional analysis, overview pages, SSH
    communication, and graphs, and supports over 30 popular systems
    including Snort, Squid, Postfix, Apache, Sendmail, syslog, ipchains,
    iptables, NT event logs, Firewall-1, wtmp, xferlog, Oracle listener and
    Pix.

\end{quotation}

\noindent{}\url{http://www.itefix.no/i2/index.php} \newline{}
(Last checked 2009/01/11.)

Logrep extracts less than half the fields \parsername{} does:

\begin{itemize}

    \item For mail sent and received: from address, size, and time and
        date.  Which date and hour are stored is not documented: start
        time, end time, delivery time, or another time?

    \item For mail sent: to addresses, \gls{SMTP} code, and delay.

    \item For mail received: the hostname of the sender.

\end{itemize}

It also counts the number of log lines parsed and skipped.  It omits client
\gls{IP} and hostname, server \gls{IP}, HELO hostname, queueid, and message
ID\@.  It extracts the delay for delivered mails, which \parsername{} does
not.  Log lines are correlated based on the queueid (referred to as
sessionname [sic] within Logrep), but not by \gls{pid}.  The parsing is
error prone: empty fields are saved when the log line does not match the
regex, though it appears that they will not overwrite existing data.  Most
notably rejections are completely ignored, making it unsuitable for the
purposes of this project.  It does not try to address any of the
complications in \sectionref{complications} except for correlating by
queueid.

Logrep does not come with any documentation, though some scant
documentation is available on its website (\parsername{} provides copious
documentation).  It requires a web browser to interact with it, so
automated log file processing will be difficult, whereas enabling automated
processing is a key part of \parsernames{} design.  Sadly all the author's
attempts to use Logrep failed, as it was unable to access the log files
selected; this appears to be a bug rather than operator error.  If it was
caused by operator error, the interface needs improvement as the (minimal)
instructions were followed as closely as possible, and multiple attempts
were made.  As parsing failed it was not possible to review the reports
Logrep can generate (available in HTML only), or to examine the
(undocumented) format in which it can save extracted data for subsequent
reuse.

Logrep extracts far less data from Postfix log files than \parsername{},
completely ignores rejections, is effectively undocumented, does not deal
with the more complicated aspects of Postfix log files, and at the time of
writing does not work properly.

\section{Summary}

There are other programs available which perform basic Postfix log file
parsing (some to a greater level of detail than others), but few attempt to
correlate log lines by queueid (none correlate by \gls{pid}) to produce an
overall record of the journey of each mail through Postfix.  None of the
reviewed parsers collect the breadth of information gathered by
\parsername{}, or make it as easy to extend the parser to handle new log
lines.  Most of the parsers generate a report and immediately discard the
data extracted from the log files; those that do not discard the data
typically retain it in a format inaccessible to other tools.  Nearly all of
the parsers reviewed can produce a report of greater or lesser detail and
complexity, unlike \parsername{}.  The quality of the documentation offered
by the subset of parsers that provide some varies from unusable to good;
none of the parsers provide any documentation on the format of their data
stores (if they have one).  Fewer than half of the parsers were capable of
parsing the \numberOFlogFILES{} test log files, and improving or extending
parsing would have been quite a difficult task for any of the parsers.
Table \refwithpage{Summary of parsers' features} provides a summary of the
parsers' features.

The overriding difference between \parsername{} and the other parsers
reviewed herein is that none of them aim for the high level of
understanding of Postfix log files achieved by \parsername{}.


\begin{table}[htb]
    \caption{Summary of parsers' features}
    \empty{}\label{Summary of parsers' features}
    \begin{tabular}{llllll}
        \tabletopline{}%
        Parser          & Parsed test   & Data              & Custom            & Documentation  & Source       \\
                        & log files?    & store?            & reports?          & quality?       & code?        \\
        \tablemiddleline{}%
        \gls{LMA}       & No            & Yes               & No                & Poor           & Yes          \\ 
        Pflogsumm       & Yes           & No                & Partial \dag{}    & Good           & Yes          \\
        Sawmill         & Yes           & Yes               & Searches          & Very good      & \nialpha{}   \\
        Splunk          & Yes           & Yes               & Searches          & Abundant       & No           \\
                        &               &                   & \& reports        & but poor       &              \\
        Isoqlog         & No            & Yes               & No                & Poor           & Yes          \\
        AWStats         & Partially     & Yes               & Partial \dag{}    & Good           & Yes          \\
        Anteater        & No            & No                & No                & Poor           & Yes          \\
        YAALA           & No            & Yes \ddag{}       & Searches          & Poor           & Yes          \\
        Lire            & Yes           & Yes               & Yes               & Reasonable     & Yes          \\
        Logrep          & No            & Yes               & No                & Poor           & Yes          \\
        \parsername{}   & Yes           & Yes \nibeta{}     & No \nichi{}       & XXX            & Yes          \\
        \tablebottomline{}%
    \end{tabular}

    \begin{eqlist}

        \item [\dag{}] Sections can be omitted from a report, but extra
            sections can not be added.

        \item [\ddag{}] YAALA only stores the data required to produce the
            latest report; other data will be discarded.

        \item [\nialpha{}] Sawmill's source code is available in an
            obfuscated form, so that customers can compile it on platforms
            that pre-compiled binaries are not available for.

        \item [\nibeta{}] \parsername{} is the only parser with
            documentation for its data store.

        \item [\nichi{}] \parsername{} defers report generation to
            subsequent programs, but all the necessary data and
            documentation to produce reports is provided.

    \end{eqlist}

\end{table}

\clearpage{}


\bibliographystyle{logparser-bibliography-style}
\bibliography{logparser-bibliography}
\label{bibliography}

\section{Graphs}

\label{graphs}

XXX COMPARE SAVING RESULTS TO RAM DISK VERSUS SAVING TO HARD DISK\@.

\renewcommand{\figurename}{Graph}

\subsection{Introduction}

Graphs are an excellent means of displaying data, transforming a
meaningless stream of numbers into an easily comprehensible form, where
anomalies and patterns are immediately obvious.  These graphs are used to
illustrate the topics discussed in \sectionref{rule efficiency}.  The
graphs in the first section cover parser scalability, demonstrating that
performance scales linearly with input size.  The impact of rule ordering
is shown in \sectionref{rule ordering graphs}, and the anomalous dips and
peaks apparent in some graphs are explained.  The third group of graphs
vividly shows the huge impact that caching compiled \regexes{} has on
parser performance.  Miscellaneous graphs are presented in the final group
of graphs; these graphs are referenced from various places in the test.
This section concludes with a breakdown of the rule hits accumulated during
a single test run of the parser.

\subsection{Parser scalability}

\showgraph{build/plot-normal-filesize-numlines}{Execution time vs file
size vs number of lines}{execution time vs file size vs number of lines
graph}

The Y axis in graph~\refwithpage{execution time vs file size vs number of
lines graph} represents the following:

\begin{enumerate}

    \item The time required, in seconds, to parse the log file.

    \item The size of the log file, in megabytes.

    \item The number of lines in the log file, divided by 10000.

\end{enumerate}

Graph~\refwithpage{execution time vs file size vs number lines factor}
shows the ratio of execution time vs file size and number of lines (higher
is better, it means more bytes or lines processed per second).  The ratios
are quite tightly banded with the exception of log files 22 and 62--68,
where they are noticeably higher; graph~\refwithpage{execution time vs file
size vs number lines factor} and table~\refwithpage{execution time vs file
size vs number lines factor table} show that the parser's execution time
scales linearly with input size.

\showgraph{build/plot-normal-filesize-numlines-factor}{Ratio of file
size and number of lines to execution time}{execution time vs file size vs
number lines factor}

\showtable{build/stats-normal-filesize-line-count-include}{Ratio of file
size \& number of lines to execution time: statistics}{execution time vs
file size vs number lines factor table}

\clearpage

\subsection{Rule ordering}

\label{rule ordering graphs}

\showgraph{build/plot-normal-shuffle-factor}{Percentage increase of
shuffled over normal}{percentage increase of shuffled over normal}

\showgraph{build/plot-normal-reverse-factor}{Percentage increase of
reversed over normal}{percentage increase of reversed over normal}

\showtable{build/stats-normal-shuffle-reverse-include}{Execution time
increase for different rule orderings}{Execution time increase for
different rule orderings}

\subsubsection{Why are there dips in the graphs?}
\label{Why are there dips in the graphs?}

The dips at log files 22 and 62--68 correspond to peaks in log file size in
graph~\refwithpage{execution time vs file size vs number of lines graph},
and peaks in graph~\refwithpage{execution time vs file size vs number lines
factor} (where a peak means that more lines are processed per second, i.e.\
performance is better).  The explanation for this took some time to arrive
at, but it turns out to be reasonably simple.  The large log files in
question were caused by a mail forwarding loop, where the distribution of
log lines is quite different to normal, resulting in different performance
characteristics.

The mail loop was set up by a user modifying his mail forwarding to:
\newline \tab{}\texttt{$\backslash$username, username@domain} \newline This
instructs Postfix to deliver the mail to the local user, and also forward
it to the remote address; this is generally not a problem except that the
remote address in this case is the user's address, creating an infinite
loop.  To prevent this happening Postfix examines the Delivered-To header
in the mail, and if the mail has already been delivered to the current
address it is bounced back to the sender with the error message
\texttt{mail forwarding loop for username@domain}.  Ordinarily this works
well, but unfortunately in this case the user noticed they had not received
any mail in a while and opted to send a test mail to themselves, causing a
loop not caught by Postfix as described below.

\begin{enumerate}

    \item Postfix accepts a mail from username@domain, for username@domain.

    \item Postfix delivers the mail to the local mailbox and
        username@domain, as instructed by the user's forwarding
        instructions. The forwarded mail has a
        \texttt{Delivered-To:~username@domain} header added, and the
        envelope sender address is username@domain.  Log lines are added by
        \daemon{local}, \daemon{qmgr} (twice), \daemon{cleanup} and finally
        \daemon{pickup}.

    \item Postfix accepts the mail for username@domain, but while
        delivering it notices that the \texttt{Delivered-To} header already
        contains the address it's currently delivering to, and therefore
        sends a bounce notification with sender address \textit{$<>$\/} to
        the original sender: username@domain.  Log lines are added by
        \daemon{local}, \daemon{qmgr} (twice) and \daemon{cleanup}.

    \item Postfix accepts the bounce notification and delivers it to both
        the local mailbox and to username@domain, as instructed by the
        user's forwarding instructions.  A
        \texttt{Delivered-To:~username@domain} header is added to the
        forwarded bounce notification, which now has an envelope sender
        address of username@domain.  Log lines are added by \daemon{local},
        \daemon{qmgr} (twice), \daemon{cleanup}, and finally
        \daemon{pickup}.

    \item Postfix accepts the forwarded bounce notification but while
        delivering the mail it notices that the \texttt{Delivered-To}
        header already contains the address currently being delivered to,
        and sends a bounce notification to the sender: username@domain.
        Log lines are added by \daemon{local}, \daemon{qmgr} (twice) and
        \daemon{cleanup}.

    \item Repeat from step two; this will continue indefinitely unless an
        administrator intervenes and deletes the appropriate mails from the
        queue.

\end{enumerate}

The sequence described above occurs extremely rapidly because Postfix does
not have to deliver the mail to an external system, so mails are delivered,
bounced and generated as fast as the disks can keep up, resulting in a huge
volume of logs.

The vast majority of log lines when a mail loop occurs are from Postfix
components which have a small number of rules associated with them, whereas
in general \daemon{smtpd} adds the majority of log lines, and also has the
highest number of rules.  \daemon{smtpd} log lines are distributed across
rules much more evenly than the log lines of \daemon{qmgr}, \daemon{local},
\daemon{cleanup} or \daemon{pickup}, so the average number of rules
required to parse a \daemon{smtpd} log line is much higher that the average
number required to parse other log lines.

These two characteristics combine to reduce the average number or rules
required to parse a log line when there is a mail loop, as shown by the
peaks in graph~\refwithpage{execution time vs file size vs number lines
factor}.  When the rule ordering is reversed the majority of log lines
generated by a mail loop will be parsed with very few rules, whereas
without a mail loop the majority of log lines require a large number of
rules; this leads to a noticeable drop in the average time required to
parse a log line, as shown in graph~\refwithpage{percentage increase of
reversed over normal}.  The number of rules which need to be consulted when
the ordering is shuffled varies between the optimum and nadir, and the
performance varies proportionally.

The difference between logs with a mail loop and logs without can be seen
in table~\refwithpage{Execution time increase for different rule orderings}
showing the increases for the different rule orderings and combinations of
log files:



\subsection{Caching regexes}

\label{Caching regexes}

The following graphs show the impact that not caching compiled \regexes{}
has on parser performance: on typical log files the execution time when not
caching compiled \regexes{} is 500--600\% of the execution time when
caching; reversing the perspective shows that cached execution time is
merely 17--20\% of non-cached execution time.  Caching compiled \regexes{}
is probably the single most effective optimisation possible in the parser's
implementation; given that it only required sixteen extra lines of
reasonably simple code,\footnote{The sixteen lines of code breaks down as
follows: three lines to add caching, eight lines of error checking and
reporting, three lines to optionally disable caching for debugging and
performance measurement, and two lines of comments.} the investment in time
was minimal.

\showgraph{build/plot-cached-discarded}{Regex: cached vs
discarded}{normal regex vs discard regex}

\showgraph{build/plot-cached-discarded-factor}{Regex caching: percentage
execution time increase}{normal regex vs discarded regex factor}

Two large dips can be seen in graph~\refwithpage{normal regex vs discarded
regex factor} at log files 22 and 62--68, corresponding to the spikes in
log file size in graphs~\refwithpage{execution time vs file size vs number
of lines graph} and~\refwithpage{normal regex vs discard regex}.  The
reason for the anomalous log files has already been explained in
\sectionref{Why are there dips in the graphs?}.

The distribution of log lines across rules when there is a mail loop is
much different and the average number of rules consulted per log line is
much lower; this results in far fewer \regex{} compilations per line than
when there isn't a mail loop, and a correspondingly decreased execution
time.  The increases in execution time when not caching \regexes{} for
different combinations of log files are summarised in the table below:

\showtable{build/stats-cached-discarded-include-for-graph}
{Regex caching/discarding with different groups of log files}
{Regex caching/discarding with different groups of log files}


\subsection{Rule hits}
\label{rule hits}

The number of hits per rule is quite unevenly spread, resembling a Power
Law distribution~\cite{powerlaw}.

\showgraph{build/plot-hits}{Hits per rule}{rule hits graph}

As graph~\refwithpage{rule hits graph} is quite difficult to read it has
been separated into three sections: low, middle and high.

\showgraph{build/plot-hits-low}{Hits per rule (low)}{hits per rule low}

It is apparent from the low graph (graph~\refwithpage{hits per rule low})
that some rules have few or no hits; those with zero hits are rules which
were written to parse log files used during development of the parser but
not utilised in the test runs performed for this document.

\showgraph{build/plot-hits-middle}{Hits per rule (middle)}{hits per rule
middle}

\showgraph{build/plot-hits-high}{Hits per rule (high)}{hits per rule
high}

\clearpage



\subsection{Miscellaneous graphs}

\label{Miscellaneous graphs}

\showgraph{build/plot-mails-received}{Mails received per day}{Mails
received per day}

As expected there are far more mails received on weekdays than at weekends.
Note that this graph and the table below show the number of mails received
by \SMTP{} only; in particular the mail loops noticeable in other graphs
do not contribute to these figures.

\showtable{build/mails-received-include-for-graph}{Number of mails received
per day: statistics}{Number of mails received per day: statistics}

\clearpage

\subsection{Conclusion}

The graphs presented in this section illustrate the topics discussed in
\sectionref{rule efficiency}.  The first collection of graphs are about
parser scalability, showing the linear relationship between execution time
and input size.  \sectionref{rule ordering graphs} demonstrates the effect
of rule ordering on execution time, and the unexpected consequences of
specific inputs.  The necessity of caching compiled \regexes{} is attested
to by the third group of graphs, where the difference between caching and
discarding is staggering.  The penultimate section contains a breakdown of
the rule hits accumulated during a single test run of the parser.
Miscellaneous graphs expected to be useful are collected in the final
section.  


% Redefine the command used to produce the glossary title, because the
% default command produces an unnumbered section whereas I want a numbered
% section.
\renewcommand{\glossarytitle}{\section{Glossary}\label{Glossary}}
\printglossary{}
% Redefine the command a second time to produce acronyms instead of a
% glossary.
\renewcommand{\glossarytitle}{\section{Acronyms}\label{Acronyms}}
\printacronym{}

% vim: set filetype=tex :
% The contents of the glossary.
\glossary{name={SQLite3},description={
    \textit{SQLite is a small C library that implements a self-contained,
    embeddable, zero-configuration SQL database engine.\/} SQLite3 is an
    \SQL{} implementation focusing on correctness, simplicity and speed.
    Unlike other \SQL{} implementations it does not require a separate
    server process, greatly simplifying deployment of an application
    utilising it.  More details can be found at~\cite{sqlite-features} or
    \url{http://www.sqlite.org/}.
}}

\glossary{name={Phishing},description={
    Phishing~\cite{Wikipedia-phishing} is an attempt to acquire information
    by masquerading as an entity trusted by the user, e.g.\ a bank.
}}

\glossary{name={Backscatter},description={
    When a spam sender or worm sends mail with forged sender addresses,
    innocent sites are flooded with undeliverable mail notifications; this
    is called backscatter mail.
}}

\glossary{name={Joe~job},description={
    A joe~job is when spam mail is sent with a faked sender address with
    the intention of sullying the good name of the owner of the address.
    joe~jobs are a cause of backscatter, though by no means the only cause.
}}

\glossary{name={Epoch},description={
    Most operating systems store the current time and timestamps of files
    etc.\ as seconds elapsed since the epoch, the beginning of time as far
    as the operating system is concerned.  On Unix and Unix-derived systems
    the epoch is 1970/01/01 00:00:00; on other operating systems it may be
    different.
}}

\glossary{name={NULL},description={
    NULL is a special term used in \SQL{} databases indicating the absence
    of data for the field.
}}

% Postfix components
\glossary{name={bounce},description={
    The bounce daemon is responsible for sending bounce notifications in
    Postfix versions later than 2.2.  The definitive documentation is
    \url{http://www.postfix.org/bounce.8.html}.
}}

\glossary{name={cleanup},description={
    Cleanup processes all incoming mail after it has been accepted and
    before it is delivered.  It removes duplicate recipient addresses,
    inserts missing headers, and optionally rewrites addresses if
    configured to do so.  The definitive documentation is
    \url{http://www.postfix.org/cleanup.8.html}.
}}

\glossary{name={lmtp},description={
    Delivery of mail over \LMTP{} is performed by the lmtp component.  The
    definitive documentation is \url{http://www.postfix.org/lmtp.8.html}.
}}

\glossary{name={local},description={
    Local is the Postfix component responsible for local delivery of mail
    (i.e.\ delivered on the server Postfix is running on), whether it be to
    a user's mailbox or a program such as a mailing list manager or
    procmail (\url{http://www.procmail.org/}).  It also handles aliases and
    processing of a user's \texttt{.forward} file.  The definitive
    documentation is \url{http://www.postfix.org/local.8.html}.
}}

\glossary{name={pickup},description={
    Pickup is the service which deals with mail submitted locally via
    postdrop and sendmail; it passes all submitted mail to cleanup.  The
    definitive documentation is \url{http://www.postfix.org/pickup.8.html}.
}}

\glossary{name={postdrop},description={
    Postdrop is used when submitting mail locally on the server: it creates
    a new mail in the queue and copies its input into the mail.  Subsequent
    delivery of the mail is the responsibility of other Postfix components.
    The definitive documentation is
    \url{http://www.postfix.org/postdrop.1.html}.
}}

\glossary{name={postsuper},description={
    Maintenance task such as deleting mails from the queue, putting mail on
    hold (no further delivery attempts will be made until it is released
    from hold, also by postsuper), and consistency checking of the mail
    queue.  The definitive documentation is available at
    \url{http://www.postfix.org/postsuper.1.html}.
}}

\glossary{name={qmgr},description={
    Qmgr is the Postfix daemon which manages the mail queue, determining
    which mails will be delivered next.  Qmgr orders the mails based on the
    recipient for local mails and the destination server for remote
    addresses, ensuring that it balances the aims of achieving maximum
    concurrency while avoiding overwhelming destinations or wasting time
    and resources on non-responsive destinations.  The definitive
    documentation is \url{http://www.postfix.org/qmgr.8.html}.
}}

\glossary{name={sendmail},description={
    Postfix provides a command that is compatible with the Sendmail
    (\url{http://www.sendmail.org/}) mail submission program that all Unix
    commands which send mail depend on; Postfix sendmail executes postdrop
    to place a new mail in the queue.  The definitive documentation is
    \url{http://www.postfix.org/sendmail.1.html}.
}}

\glossary{name={smtp},description={
    Delivery of mail over \SMTP{} is performed by the smtp component.  The
    definitive documentation is \url{http://www.postfix.org/smtp.8.html}.
}}

\glossary{name={smtpd},description={
    Smtpd is the Postfix program which accepts mail via \SMTP{}, and
    implements all the anti-spam restrictions Postfix provides.  The
    definitive documentation is \url{http://www.postfix.org/smtpd.8.html}.
}}

\glossary{name={virtual},description={
    Virtual is the Postfix component responsible for delivery of mails to
    virtual domains.  With \daemon{local} delivery the destination is
    determined only by the portion of the email address on the left side of
    the \at{}, whereas with \daemon{virtual} delivery the destination is
    determined by the entire email address, e.g.\ if the server considers
    itself responsible for both \textbf{example.org} and
    \textbf{example.net} domains: \daemon{local} considers
    \textbf{john\at{}example.org} and \textbf{john\at{}example.net} to have
    the same mailbox, whereas \daemon{virtual} considers them to have
    different mailboxes.  Virtual delivery is used when a server hosts
    multiple domains where a username may be present in more than one
    domain but represent different users in each.  The definitive
    documentation is
    \url{http://www.postfix.org/virtual.8.html}.
}}

\glossary{name={Bayesian spam filtering},description={
    Bayesian spam filtering is a method of classifying mail based on the
    frequency that the words in the mail have previously appeared in a spam
    corpus and a ham (non-spam) corpus.  A full description is beyond the
    scope of this document, see~\cite{bayesian-filtering, a-plan-for-spam}
    for a detailed explanation.
}}

\glossary{name={Bayesian poisoning},description={
    Bayesian poisoning is the addition of innocuous or unrelated words to a
    spam mail in the hope of defeating Bayesian spam filtering.  E.g.\ the
    word Viagra would be firmly in the spam corpus for most people, but by
    adding the words \textit{schedule}, \textit{meeting}, \textit{moving
    forward\/} and \textit{best business practices\/} to a mail received by
    a manager, the Bayesian spam filter might tip the balance from bad to
    good, if the non-spam words outweigh the spam words.
}}

\glossary{name={$<>$},sort={<>},description={
    $<>$ is the sender address used for mail which should not be replied
    to, e.g.\ bounce notifications.  In \SMTP{} all addresses are enclosed
    in $<>$, so \textit{username\at{}domain\/} becomes
    \textit{$<$username\at{}domain$>$\/}; thus $<>$ is actually an empty
    address, but is always written as $<>$ for clarity.  All mail servers
    must accept mail sent from $<>$, or they are in violation of
    \RFC{}~2821~\cite{RFC2821}.
}}

\glossary{name={queueid},description={
    Each mail in Postfix's queue is assigned a queueid to uniquely identify
    it.  Queueids are assigned from a limited pool, so although they are
    guaranteed to be unique for the lifetime of the mail, given sufficient
    time they will be reused.
}}

\glossary{name={IPv4},description={
    Internet Protocol~\cite{Wikipedia-ipv4} version 4 is the fourth version
    of the Internet Protocol used to interconnect computers on the
    Internet.  It is the first widely deployed version of IP, and has been
    in use for over 25 years.
}}

\glossary{name={IPv6},description={
    Internet Protocol~\cite{Wikipedia-ipv6} version 6 is the latest version
    of the Internet Protocol used to interconnect computers on the
    Internet.  It is the successor to IPv4, bringing with it a greatly
    expanded address space, allowing many more computers to use the
    Internet simultaneously.  IPv4 and IPv6 will coexist for many years to
    come as existing networks transition from the former to the latter.
}}

\glossary{name={hash},description={
    A hashing function transforms a string of characters to a number.
    There are many possible uses for the resulting number: a common usage
    is to maintain a data structure indexed by strings in an efficient
    manner.  A full description is beyond the scope of this paper, further
    information can be found at~\cite{hash-functions}.
}}

\glossary{name={Solaris},description={
    Solaris is a Unix-derived Operating System produced by Sun Microsystems
    (\url{http://www.sun.com/software/solaris/}).
}}

\glossary{name={awk},description={
    AWK is a general purpose programming language that is designed for
    processing text-based data, and is available as a standard utility on
    all Unix systems.
}}

\glossary{name={syslog},description={
    Syslog is the standard logging mechanism on Unix systems: the program
    sends log messages to syslog, then syslog filters and stores the
    messages according to the configuration the administrator has chosen.
}}

\glossary{name={mail bomb},description={
    A mail bomb occurs when an attacker inflicts a huge volume of mail on
    the victim.  At best a mail bomb is irritating to the victim; at worst
    the deluge of mail can be severe enough to: interrupt service for the
    victim and/or other users; cause mail to be rejected because the victim
    has reached a limit (e.g.\ too many mails, too much disk space
    consumed); the victim may accidentally delete other mail while trying
    to cope with the mail bomb.
}}

\glossary{name={mail loop},description={
    Sometimes mail set to one address must be delivered to a different
    address instead, e.g.\ because a person has changed jobs.  A mail loop
    occurs when the recipient addresses constitute a cyclic directed graph;
    the simplest example is when mail for \texttt{foo\at{}example.net} is
    delivered to \texttt{bar\at{}example.com}, and mail for
    \texttt{bar\at{}example.com} is delivered to
    \texttt{bar\at{}example.com}.
}}

% logparser-acronyms.tex has been included in the preamble so that the
% commands are available throughout the text.
\end{document}
