% $Id$
\documentclass[a4paper,12pt,draft]{report}
\pagestyle{headings}

% Useful stuff for math mode.
\usepackage{amstext}
% Include images
\usepackage[final]{graphicx}
% Add the bibliography into the table of contents.
\usepackage[chapter,numbib]{tocbibind}
% Create a label for the last page.  Might be useful for "page 23/79" or
% something.
\usepackage{lastpage}

% Extra footnote functionality, including references to earlier footnotes.
% Removed for now, because it generates a warning:
% "LaTeX Warning: Command \@makecol  has changed."
%\usepackage[bottom]{footmisc}
% The Euro symbol
\usepackage{eurofont}

% Aligned list environments.
\usepackage{eqlist}
% This voodoo below gives us an eqlist environment with bold items.
\newcommand{\bolditem}[1]{%
    \bfseries#1%
}
\newenvironment{boldeqlist}
    {\begin{eqlist}[\renewcommand{\makelabel}{\bolditem}]}
    {\end{eqlist}}

% Tell Latex to use scalable fonts
\usepackage{type1cm}
% Enable nice kerning features.
\usepackage[final]{microtype}

% Extra packages recommended by Springer.
%\usepackage{mathptmx}
%\usepackage{helvet}
%\usepackage{courier}
%\usepackage{makeidx}
%\usepackage{multicol}

% Embed SVN Id etc.
\usepackage{svn}

\newcommand{\parserblurb}[3]{%
    \begin{quotation}
        \noindent{}\textit{#1\/}

        \noindent{}\url{#2}\newline
        Last checked #3.
    \end{quotation}
}

% Produce nicer references like "section 4.3 on the next page"
% Must be loaded before hyperref.
\usepackage{varioref}
% Warnings rather than errors when references cross page boundaries.
\vrefwarning{}
% Errors when loops are encountered
\vrefshowerrors{}
% Shorten the text used for references with page numbers.
\renewcommand{\reftextfaraway}[1]{%
    [p.~\pageref{#1}]%
}
% Replacement for \ref{}, adds the page number too.
\newcommand{\refwithpage}[1]{%
    \empty{}\vref{#1}%
    %\empty{}\ref{#1} [p.~\pageref{#1}]%
}
% section references, automatically add \textsection
\newcommand{\sectionref}[1]{%
    \textsection{}\vref*{#1}%
    %\textsection{}\refwithpage{#1}%
}
% A reference with a space between the label and the reference.
\newcommand{\refwithlabel}[2]{%
    #1~\vref{#2}%
}
% table references, for consistent formatting.
\newcommand{\tableref}[1]{%
    \refwithlabel{table}{#1}%
}
\newcommand{\Tableref}[1]{%
    \refwithlabel{Table}{#1}%
}
% graph references
% SHOULD THESE BE LABELLED WITH graph OR figure?
\newcommand{\graphref}[1]{%
    \refwithlabel{graph}{#1}%
}
\newcommand{\Graphref}[1]{%
    \refwithlabel{Graph}{#1}%
}
% figure references
\newcommand{\figureref}[1]{%
    \refwithlabel{figure}{#1}%
}
\newcommand{\Figureref}[1]{%
    \refwithlabel{Figure}{#1}%
}


% Acronyms and glossary entries.  Useful options:
% * nonumberlist disables the list of page numbers after each entry.
\usepackage[acronym,toc,numberedsection=nolabel,style=list,nonumberlist]{glossaries}
\renewcommand{\glspostdescription}[0]{}
\renewcommand{\glsautoprefix}[1]{glossary:#1}
\newglossary[plg]{postfix}{pin}{pout}{Postfix Daemons}
\makeglossaries{}
\newcommand{\acronyms}[1]{%
    \acronym[s]{#1}%
}
\newcommand{\acronym}[2][]{%
    \gls{#2}#1%
    % If there's a glossary entry for this acronym add it, otherwise
    % do nothing.
    \ifglsentryexists{#2 glossary}%
        {\glsadd{#2 glossary}}%
        {}%
}
% Define a new glossary style that uses eqlist, so that acronyms line up
% nicely.
\newglossarystyle{eqlist}{%
    \glossarystyle{list}%
    \renewenvironment{theglossary}{\begin{eqlist}}{\end{eqlist}}%
    \renewcommand*{\glossaryentryfield}[5]{%
        \item [\textbf{##2}] ##3%
    }%
    \renewcommand{\glsgroupskip}[0]{}%
}
% Define a new glossary style that removes the space between glossary
% groups.
\newglossarystyle{nospacelist}{%
    \glossarystyle{list}%
%    \renewenvironment{theglossary}{\begin{eqlist}}{\end{eqlist}}%
%    \renewcommand*{\glossaryentryfield}[5]{%
%        \item [\textbf{##2}] ##3%
%    }%
    \renewcommand{\glsgroupskip}[0]{}%
}

% Provides commands to distinguish between pdf and dvi output.
\usepackage{ifpdf}
% When creating a \PDF{} make the table of contents into links to the pages
% (without horrible red borders) and include bookmarks.  The title and
% author do not work - I think either gnuplot or graphviz clobbers it.
% hyperfootnotes need to be disabled to avoid breaking footmisc, but they
% still seem to work, somehow.
\ifpdf{}
    \usepackage[pdftex,hyperfootnotes=false,plainpages=false,pdfpagelabels]{hyperref}
\else{}
    \usepackage[dvips,hyperfootnotes=false,plainpages=false,pdfpagelabels]{hyperref}
    % This is necessary for wrapping URLs in the bibliography when
    % producing a dvi, but causes problems when generating \PDF{} output.
    \usepackage{breakurl}
\fi{}
\hypersetup{
    pdftitle    = {Parsing Postfix log files},
    pdfauthor   = {John Tobin},
    final       = true,
    pdfborder   = {0, 0, 0},
}

% Sort numbers where there are multiple citations.  Does not appear to have
% the expected effect (probably clashes with varioref or hyperref), though
% it does reduce the space between numbers.
\usepackage{cite}
% Check for unused references, and how the citation key (e.g.\ slct-paper)
% in the margin beside the reference.
%\usepackage[nomsgs]{refcheck}
% Show where references are used; neither work.
%\usepackage{citeref}
%\usepackage{backref}
%\renewcommand{\refname}{Bibliography}

% Reduce the space between items in a list; this is useful when each item
% is a single line, because then the defaul spacing makes the list look
% very sparse.  This command needs to be the first thing in a list to take
% effect.
\newcommand{\squeezeitems}[0]{%
    \setlength{\itemsep}{0pt}%
    \setlength{\topsep}{0pt}%
    \setlength{\partopsep}{0pt}%
}

% Put each URL in the bibliography on a new line.
\newcommand{\urlprefix}[0]{\newline{}}
% New formatting commands.

% \showgraph{filename}{caption}{label}
\newcommand{\showgraph}[3]{%
    \begin{figure}[hbt!]
        \caption{#2}\label{#3}
        \includegraphics{#1}
    \end{figure}
}

%\showtable{filename}{caption}{label}
\newcommand{\showtable}[3]{
    \begin{table}[ht]
        \caption{#2}\label{#3}
        \input{#1}
    \end{table}
}

\newcommand{\tabletopline}[0]{%
    \hline%
    \noalign{\smallskip}%
}

\newcommand{\tablebottomline}[0]{%
    \noalign{\smallskip}%
    \hline%
}

\newcommand{\tablemiddleline}[0]{%
    \noalign{\smallskip}%
    \hline%
    \noalign{\smallskip}%
}

% A command to format a Postfix daemon's name
\newcommand{\daemon}[1]{%
    \texttt{postfix/#1}%
}

\newcommand{\flowchart}[2]{%
    (action: #1\@; flowchart:~#2)%
}

% Add last checked dates to all URLs
\newcommand{\urlLastChecked}[3][ ]{%
    \url{#2}#1(last checked #3)%
}
\newcommand{\daemonDocURL}[3][]{%
    \hfill{} \newline{} \url{#2}%
    \hfill{} #1 Last checked #3%
}

\newcommand{\tab}[0]{%
    \hspace*{1em}%
}

% Constant values.
\newcommand{\numberOFlogFILES}[0]{%
    93%
}

\newcommand{\numberOFlogFILESall}[0]{%
    522%
}

\newcommand{\numberOFrules}[0]{%
    185%
}

\newcommand{\numberOFrulesMINIMUM}[0]{%
    116%
}

% \numberOFrulesMINIMUM as percentage of \numberOFrules
\newcommand{\numberOFrulesMINIMUMpercentage}[0]{%
    62.70\%%
}

% \numberOFrules as percentage increase of \numberOFrulesMINIMUM
\newcommand{\numberOFrulesMAXIMUMpercentage}[0]{%
    59.48\%%
}

\newcommand{\numberOFlogLINES}[0]{%
    60,721,709%
}

\newcommand{\numberOFlogLINEShuman}[0]{%
    60.72 million%
}

\newcommand{\numberOFactions}[0]{%
    21%
}

\newcommand{\numberOFruleINTERSECTIONS}[0]{%
    17,020%
}

% The name of the program, so I only have to change it in one place.
\newcommand{\parsername}{\acronym{PLP}}
\newcommand{\parsernames}{\acronyms{PLP}}
\newcommand{\parsernamelong}{Postfix Log Parser}

\begin{document}

\title{Parsing Postfix log files}
\author{John Tobin \\ School of Computer Science and Statistics \\
Trinity College \\ Dublin 2 \\ Ireland \\ tobinjt@cs.tcd.ie}
\maketitle

\begin{abstract}

    \phantomsection{}
    \addcontentsline{toc}{chapter}{Abstract}

    Parsing Postfix log files is much more difficult than it first appears,
    but it \textit{is\/} possible to achieve a high degree of accuracy in
    understanding the log files, and thus accuracy in reconstructing the
    actions taken by Postfix to generate the log files.  This paper
    describes the creation of a parser for Postfix log files, documenting
    the architecture developed and the parser implementing it, explaining
    the difficulties encountered and the solutions developed, with
    reference to an implementation that stores data gleaned from the log
    files in an SQL database.  The gathered data can then be used to
    optimise current anti-spam measures, provide a baseline to test new
    anti-spam measures against, or to produce statistics showing how
    effective those measures are.

\end{abstract}

\SVN$Id$
\begin{center}
    \SVNId{}
\end{center}

XXX CHECK ALL ``UNINTERESTING'' RULES TO SEE IF THEY COULD BE USED\@.

XXX SHOULD I SCRAP postfix\_action?

% Pull in the acronyms early, so they can be used throughout the text.
% % vim: set textwidth=75 spell :
% Warning: don't use acronyms within definitions, they don't work properly.

\newacronym{PLP}{Postfix Log Parser}{description={
    Postfix Log Parser is the implementation of the algorithm described in
    this paper.
}}

\newacronym{DNSBL}{DNS Blacklist}{description={
    A DNS Blacklist~\cite{Wikipedia-DNSBL} is a simple collaborative
    anti-spam technique used to reject or penalise email sent from mail
    servers reported to have sent large volumes of spam.  To use a DNSBL
    the mail server makes a DNS request incorporating the IP address of the
    client; if the requested hostname exists the client is on the DNSBL,
    and the mail server can decide what course of action to take.
}}

\newacronym{API}{Application Programming Interface}{description={
    One of the fundamental concepts when writing programs is the reuse of
    existing code, so that each program does not reinvent the wheel.  When
    a body of code is intended to be reused repeatedly, the user of this
    code needs to be informed of the functionality provided by the code.
    An API defines the interface provided to the user, and acts as a
    contract between the user and the provider: if the user adheres to the
    API the provider guarantees it will work, while the provider is free to
    change the implementation as long as the API is preserved.
}}

\newacronym{SMTP}{Simple Mail Transfer Protocol}{description={
    SMTP is the protocol which transfers mail from the sender to the
    recipient across the Internet.  It is a simple, human readable, plain
    text protocol, making it quite easy to test and debug problems with it.
    A detailed description of \SMTP{} is beyond the scope of this document:
    the original protocol definition is \RFC{}~821~\cite{RFC821}, updated
    in \RFC{}~2821~\cite{RFC2821}; introductory guides can be found
    at~\cite{smtp-intro-01,smtp-intro-02}.
}}

\newacronym{LMTP}{Local Mail Transfer Protocol}{description={
    LMTP is a protocol derived from SMTP that removes the need for the
    server to maintain a mail delivery queue, instead relying on the client
    to maintain it.  Typically the client would be an MTA, and the server
    would be a delivery agent or a mail store.  Full details are available
    in \cite{lmtp-rfc}.
}}

\newacronym{MTA}{Mail Transfer Agent}{description={
    A Mail Transfer Agent sends and/or receives mail via SMTP.
}}

\newacronym{RFC}{Request For Comments}{description={
    The Request For Comments series is a series of proposals defining
    various protocols and file formats, e.g. SMTP.  The name is somewhat
    misleading: initially the authors were asking for peer review, but
    these documents are now the de facto standards the Internet runs on.
}}

\newacronym{SQL}{Structured Query Language}{description={
    SQL is the standard language used for database querying, modification
    and maintenance.  Some information about SQL, including its history,
    can be found at \cite{Wikipedia-sql}; a good introduction can be found
    at~\cite{sql-for-web-nerds}, others are~\cite{w3schools-sql-tutorial,
    sqlcourse.com}.
}}

\newacronym{SPF}{Sender Policy Framework}{description={
    SPF is introduced in \sectionref{spf introduction} and explained fully
    in~\cite{openspf, Wikipedia-spf}.
}}

\newacronym{ISP}{Internet Service Provider}{description={
    An ISP is a company which sells Internet access to consumers.
}}

\newacronym{LMA}{Log Mail Analyzer}{description={
    One of the other Postfix log parsers reviewed and the only example of
    published prior art~\cite{log-mail-analyser} found by the author.
}}

\newacronym{CSV}{Comma-Separated Value}{description={
    The most basic form of database available, each record is a single line
    in the file and the fields are separated by a special character,
    typically a comma or colon.
}}


\newacronym{FQDN}{Fully Qualified Domain Name}{description={
    An FQDN is a hostname plus domain name, e.g.  \newline
        example.com     is a domain name          \newline
        www             is a hostname             \newline
        www.example.com is a FQDN
}}

\newacronym{DNS}{Domain Name System}{description={
    The DNS converts between hostnames (www.example.com) and IP addresses
    (10.1.2.3).
}}

\newacronym{pid}{Process Identifier}{description={
    There may be multiple copies of any program executing at any one time,
    so the program's name is not suitable as a distinguishing identifier;
    instead each process executing is given a pid which is guaranteed to be
    unique for the lifetime of the process.  Once the process has
    completed, the pid may be reused, as they are drawn from a finite pool.
}}

\newacronym{UCE}{Unsolicited Commercial Email}{description={
    UCE is a more restrictive definition of spam than most people would
    use: it only covers mail that is explicitly commercial, thus excluding
    viruses, Bayesian poisoning mails, backscatter, and those annoying
    chain letters you get from friends.
}}

\newacronym{regex}{Regular Expression}{description={
    Regular expressions are a powerful method of matching patterns against
    text that are explained in detail in~\cite{Wikipedia-regex, perlre}.
}}

% I think this is better than having both regex and regexes in the acronym
% list.  Likewise for pids.
\newcommand{\regexes}{\regex{}es}
\newcommand{\pids}{\pid{}s}

\newacronym{ESMTP}{Extended SMTP}{description={
    ESMTP is Extended SMTP, defined in RFC~1869~\cite{RFC1869}.
}}

\newacronym{HTML}{Hypertext Markup Language}{description={
    HTML is the markup language designed for writing web
    pages~\cite{Wikipedia-html, w3schools-html}.
}}

\newacronym{IP}{Internet Protocol}{description={
    The Internet Protocol~\cite{Wikipedia-ip} is the protocol used to
    communicate between computers on the Internet.  An IP address is a
    unique address assigned to a computer on the Internet, allowing it to
    communicate with other computers on the Internet.  For more information
    see~\cite{Wikipedia-ip-address}.
}}

\newacronym{PDF}{Portable Document Format}{description={
    PDF is the file format created by Adobe Systems in 1993 for document
    exchange.  It aims to be device independent, so documents should look
    the same whether viewed on screen or printed.
}}

\newacronym{LCD}{Lowest Common Denominator}{description={
    LCD is a mathematical term which is used figuratively to mean the least
    useful member of a set of alternatives.
}}

\newacronym{SLCT}{Simple Logfile Clustering Tool}{description={
    SLCT~\cite{slct-paper} is a tool implementing an algorithm designed by
    Risto Vaarandi for identifying, grouping and producing a regex to match
    similar log lines.
}}

\newacronym{ATN}{Augmented Transition Networks}{description={
    Augmented Transition Networks, originally described in~\cite{atns} and
    further in~\cite{nlpip}, are a tool used in Computational Linguistics
    for creating grammars to parse or generate sentences,
}}

\newacronym{CLI}{Command Line Interface}{description={
    An computer interface based on typing commands, rather than using a
    mouse.
}}

\newacronym{NLP}{Natural Language Processing}{description={
    Natural Language Processing~\cite{Wikipedia-nlp} attempts to increase
    our understanding of the languages normally used by humans (e.g.\
    English, Spanish, Japanese), with the goal of writing programs which
    can understand human languages.
}}

% vim: set filetype=tex :
% The contents of the glossary.
\glossary{name={SQLite3},description={
    \textit{SQLite is a small C library that implements a self-contained,
    embeddable, zero-configuration SQL database engine.\/} SQLite3 is an
    \SQL{} implementation focusing on correctness, simplicity and speed.
    Unlike other \SQL{} implementations it does not require a separate
    server process, greatly simplifying deployment of an application
    utilising it.  More details can be found at~\cite{sqlite-features} or
    \url{http://www.sqlite.org/}.
}}

\glossary{name={Phishing},description={
    Phishing~\cite{Wikipedia-phishing} is an attempt to acquire information
    by masquerading as an entity trusted by the user, e.g.\ a bank.
}}

\glossary{name={Backscatter},description={
    When a spam sender or worm sends mail with forged sender addresses,
    innocent sites are flooded with undeliverable mail notifications; this
    is called backscatter mail.
}}

\glossary{name={Joe~job},description={
    A joe~job is when spam mail is sent with a faked sender address with
    the intention of sullying the good name of the owner of the address.
    joe~jobs are a cause of backscatter, though by no means the only cause.
}}

\glossary{name={Epoch},description={
    Most operating systems store the current time and timestamps of files
    etc.\ as seconds elapsed since the epoch, the beginning of time as far
    as the operating system is concerned.  On Unix and Unix-derived systems
    the epoch is 1970/01/01 00:00:00; on other operating systems it may be
    different.
}}

\glossary{name={NULL},description={
    NULL is a special term used in \SQL{} databases indicating the absence
    of data for the field.
}}

% Postfix components
\glossary{name={bounce},description={
    The bounce daemon is responsible for sending bounce notifications in
    Postfix versions later than 2.2.  The definitive documentation is
    \url{http://www.postfix.org/bounce.8.html}.
}}

\glossary{name={cleanup},description={
    Cleanup processes all incoming mail after it has been accepted and
    before it is delivered.  It removes duplicate recipient addresses,
    inserts missing headers, and optionally rewrites addresses if
    configured to do so.  The definitive documentation is
    \url{http://www.postfix.org/cleanup.8.html}.
}}

\glossary{name={lmtp},description={
    Delivery of mail over \LMTP{} is performed by the lmtp component.  The
    definitive documentation is \url{http://www.postfix.org/lmtp.8.html}.
}}

\glossary{name={local},description={
    Local is the Postfix component responsible for local delivery of mail
    (i.e.\ delivered on the server Postfix is running on), whether it be to
    a user's mailbox or a program such as a mailing list manager or
    procmail (\url{http://www.procmail.org/}).  It also handles aliases and
    processing of a user's \texttt{.forward} file.  The definitive
    documentation is \url{http://www.postfix.org/local.8.html}.
}}

\glossary{name={pickup},description={
    Pickup is the service which deals with mail submitted locally via
    postdrop and sendmail; it passes all submitted mail to cleanup.  The
    definitive documentation is \url{http://www.postfix.org/pickup.8.html}.
}}

\glossary{name={postdrop},description={
    Postdrop is used when submitting mail locally on the server: it creates
    a new mail in the queue and copies its input into the mail.  Subsequent
    delivery of the mail is the responsibility of other Postfix components.
    The definitive documentation is
    \url{http://www.postfix.org/postdrop.1.html}.
}}

\glossary{name={postsuper},description={
    Maintenance task such as deleting mails from the queue, putting mail on
    hold (no further delivery attempts will be made until it is released
    from hold, also by postsuper), and consistency checking of the mail
    queue.  The definitive documentation is available at
    \url{http://www.postfix.org/postsuper.1.html}.
}}

\glossary{name={qmgr},description={
    Qmgr is the Postfix daemon which manages the mail queue, determining
    which mails will be delivered next.  Qmgr orders the mails based on the
    recipient for local mails and the destination server for remote
    addresses, ensuring that it balances the aims of achieving maximum
    concurrency while avoiding overwhelming destinations or wasting time
    and resources on non-responsive destinations.  The definitive
    documentation is \url{http://www.postfix.org/qmgr.8.html}.
}}

\glossary{name={sendmail},description={
    Postfix provides a command that is compatible with the Sendmail
    (\url{http://www.sendmail.org/}) mail submission program that all Unix
    commands which send mail depend on; Postfix sendmail executes postdrop
    to place a new mail in the queue.  The definitive documentation is
    \url{http://www.postfix.org/sendmail.1.html}.
}}

\glossary{name={smtp},description={
    Delivery of mail over \SMTP{} is performed by the smtp component.  The
    definitive documentation is \url{http://www.postfix.org/smtp.8.html}.
}}

\glossary{name={smtpd},description={
    Smtpd is the Postfix program which accepts mail via \SMTP{}, and
    implements all the anti-spam restrictions Postfix provides.  The
    definitive documentation is \url{http://www.postfix.org/smtpd.8.html}.
}}

\glossary{name={virtual},description={
    Virtual is the Postfix component responsible for delivery of mails to
    virtual domains.  With \daemon{local} delivery the destination is
    determined only by the portion of the email address on the left side of
    the \at{}, whereas with \daemon{virtual} delivery the destination is
    determined by the entire email address, e.g.\ if the server considers
    itself responsible for both \textbf{example.org} and
    \textbf{example.net} domains: \daemon{local} considers
    \textbf{john\at{}example.org} and \textbf{john\at{}example.net} to have
    the same mailbox, whereas \daemon{virtual} considers them to have
    different mailboxes.  Virtual delivery is used when a server hosts
    multiple domains where a username may be present in more than one
    domain but represent different users in each.  The definitive
    documentation is
    \url{http://www.postfix.org/virtual.8.html}.
}}

\glossary{name={Bayesian spam filtering},description={
    Bayesian spam filtering is a method of classifying mail based on the
    frequency that the words in the mail have previously appeared in a spam
    corpus and a ham (non-spam) corpus.  A full description is beyond the
    scope of this document, see~\cite{bayesian-filtering, a-plan-for-spam}
    for a detailed explanation.
}}

\glossary{name={Bayesian poisoning},description={
    Bayesian poisoning is the addition of innocuous or unrelated words to a
    spam mail in the hope of defeating Bayesian spam filtering.  E.g.\ the
    word Viagra would be firmly in the spam corpus for most people, but by
    adding the words \textit{schedule}, \textit{meeting}, \textit{moving
    forward\/} and \textit{best business practices\/} to a mail received by
    a manager, the Bayesian spam filter might tip the balance from bad to
    good, if the non-spam words outweigh the spam words.
}}

\glossary{name={$<>$},sort={<>},description={
    $<>$ is the sender address used for mail which should not be replied
    to, e.g.\ bounce notifications.  In \SMTP{} all addresses are enclosed
    in $<>$, so \textit{username\at{}domain\/} becomes
    \textit{$<$username\at{}domain$>$\/}; thus $<>$ is actually an empty
    address, but is always written as $<>$ for clarity.  All mail servers
    must accept mail sent from $<>$, or they are in violation of
    \RFC{}~2821~\cite{RFC2821}.
}}

\glossary{name={queueid},description={
    Each mail in Postfix's queue is assigned a queueid to uniquely identify
    it.  Queueids are assigned from a limited pool, so although they are
    guaranteed to be unique for the lifetime of the mail, given sufficient
    time they will be reused.
}}

\glossary{name={IPv4},description={
    Internet Protocol~\cite{Wikipedia-ipv4} version 4 is the fourth version
    of the Internet Protocol used to interconnect computers on the
    Internet.  It is the first widely deployed version of IP, and has been
    in use for over 25 years.
}}

\glossary{name={IPv6},description={
    Internet Protocol~\cite{Wikipedia-ipv6} version 6 is the latest version
    of the Internet Protocol used to interconnect computers on the
    Internet.  It is the successor to IPv4, bringing with it a greatly
    expanded address space, allowing many more computers to use the
    Internet simultaneously.  IPv4 and IPv6 will coexist for many years to
    come as existing networks transition from the former to the latter.
}}

\glossary{name={hash},description={
    A hashing function transforms a string of characters to a number.
    There are many possible uses for the resulting number: a common usage
    is to maintain a data structure indexed by strings in an efficient
    manner.  A full description is beyond the scope of this paper, further
    information can be found at~\cite{hash-functions}.
}}

\glossary{name={Solaris},description={
    Solaris is a Unix-derived Operating System produced by Sun Microsystems
    (\url{http://www.sun.com/software/solaris/}).
}}

\glossary{name={awk},description={
    AWK is a general purpose programming language that is designed for
    processing text-based data, and is available as a standard utility on
    all Unix systems.
}}

\glossary{name={syslog},description={
    Syslog is the standard logging mechanism on Unix systems: the program
    sends log messages to syslog, then syslog filters and stores the
    messages according to the configuration the administrator has chosen.
}}

\glossary{name={mail bomb},description={
    A mail bomb occurs when an attacker inflicts a huge volume of mail on
    the victim.  At best a mail bomb is irritating to the victim; at worst
    the deluge of mail can be severe enough to: interrupt service for the
    victim and/or other users; cause mail to be rejected because the victim
    has reached a limit (e.g.\ too many mails, too much disk space
    consumed); the victim may accidentally delete other mail while trying
    to cope with the mail bomb.
}}

\glossary{name={mail loop},description={
    Sometimes mail set to one address must be delivered to a different
    address instead, e.g.\ because a person has changed jobs.  A mail loop
    occurs when the recipient addresses constitute a cyclic directed graph;
    the simplest example is when mail for \texttt{foo\at{}example.net} is
    delivered to \texttt{bar\at{}example.com}, and mail for
    \texttt{bar\at{}example.com} is delivered to
    \texttt{bar\at{}example.com}.
}}

% vim: set filetype=tex :
% Postfix components
\newglossaryentry{bounce}{name={bounce},type={postfix},description={
    The bounce daemon is responsible for generating bounce notifications in
    Postfix version 2.3 and later.
    \daemonDocURL{http://www.postfix.org/bounce.8.html}{2009/02/23}.
}}

\newglossaryentry{cleanup}{name={cleanup},type={postfix},description={
    The cleanup daemon processes all incoming mail after it has been
    accepted and before it is delivered.  It removes duplicate recipient
    addresses, inserts missing headers, and rewrites addresses if
    configured to do so.
    \daemonDocURL{http://www.postfix.org/cleanup.8.html}{2009/02/23}.
}}

\newglossaryentry{lmtp}{name={lmtp},type={postfix},description={
    Delivers mail using the \acronym{LMTP} protocol.
    \daemonDocURL{http://www.postfix.org/lmtp.8.html}{2009/02/23}.
}}

\newglossaryentry{local}{name={local},type={postfix},description={
    The Postfix component responsible for local delivery of mail (i.e.\
    mail delivered on the server Postfix is running on); this includes
    alias expansion, processing of a user's \texttt{.forward} file, and
    delivery of the mail, whether to a user's mailbox or a program such as
    a mailing list manager.
    \daemonDocURL{http://www.postfix.org/local.8.html}{2009/02/23}.
}}

\newglossaryentry{pickup}{name={pickup},type={postfix},description={
    Pickup is the daemon that deals with mail submitted locally via
    \daemon{postdrop}, passing the mail on to \daemon{cleanup} for further
    processing.
    \daemonDocURL{http://www.postfix.org/pickup.8.html}{2009/02/23}.
}}

\newglossaryentry{postdrop}{name={postdrop},type={postfix},description={
    Postdrop is used when submitting mail locally on the server: it copies
    its input into a newly created mail in the queue, for processing by
    \daemon{pickup} and subsequent delivery.
    \daemonDocURL{http://www.postfix.org/postdrop.1.html}{2009/02/23}.
}}

\newglossaryentry{postsuper}{name={postsuper},type={postfix},description={
    Used by the administrator for maintenance tasks such as deleting mails
    from the queue, putting mail on hold and later releasing it, and
    consistency checking of the mail queue.
    \daemonDocURL{http://www.postfix.org/postsuper.1.html}{2009/02/23}.
}}

\newglossaryentry{qmgr}{name={qmgr},type={postfix},description={
    Qmgr is the Postfix daemon that manages the mail queue, determining
    which mails will be delivered next.  Qmgr groups mail based on the
    recipient for local mails and the destination server for remote
    addresses, ensuring that it achieves maximum concurrency without
    overwhelming destinations or wasting resources on non-responsive
    destinations.
    \daemonDocURL{http://www.postfix.org/qmgr.8.html}{2009/02/23}.
}}

\newglossaryentry{sendmail}{name={sendmail},type={postfix},description={
    A Postfix component that is compatible with the Sendmail mail
    submission program which all Unix commands that need to send mail use;
    it executes \daemon{postdrop} to place a new mail in the queue.
    \daemonDocURL{http://www.postfix.org/sendmail.1.html}{2009/02/23}.
}}

\newglossaryentry{smtp}{name={smtp},type={postfix},description={
    Delivers mail using the \acronym{SMTP} protocol.
    \daemonDocURL{http://www.postfix.org/smtp.8.html}{2009/02/23}.
}}

\newglossaryentry{smtpd}{name={smtpd},type={postfix},description={
    The Postfix component that accepts mail via \acronym{SMTP}, and
    implements most of the anti-spam restrictions Postfix provides.
    \daemonDocURL{http://www.postfix.org/smtpd.8.html}{2009/02/23}.
}}

\newglossaryentry{virtual}{name={virtual},type={postfix},description={
    The Postfix component responsible for delivery of mails to virtual
    domains.  When \daemon{local} delivers mail, the destination is
    determined only by the portion of the email address on the left side of
    the \texttt{@}, whereas when \daemon{virtual} delivers mail, the
    destination is determined by the entire email address.  For example, if
    the server is responsible for both the \texttt{example.org} and
    \texttt{example.net} domains: \daemon{local} would deliver mail for
    \texttt{john@example.org} and \texttt{john@example.net} to the same
    mailbox, whereas \daemon{virtual} would deliver mail for those
    addresses to different mailboxes.  Virtual delivery is used where the
    local part of an address may be present in multiple domains, and each
    must be delivered to different users.
    \daemonDocURL{http://www.postfix.org/virtual.8.html}{2009/02/23}.
}}

\glsaddall[types={postfix}]

\newpage
\tableofcontents
\listoffigures
\listoftables

\newpage

% WSUIPA fonts.
\input{ipamacs}

\section{Introduction}

\label{introduction}

Most mail server administrators will have performed some basic processing
of the log files produced by their mail server at one time or another,
whether it was to debug a problem, explain to a user why their mail is
being rejected, or check if new anti-spam measures are working.  The more
adventurous will have generated statistics to show how successful each of
their anti-spam measures has been in the last week, and possibly even
generated some graphs to clearly illustrate these statistics to management
or users.\footnote{This was the author's first real foray into processing
Postfix log files.}  Very few will have performed in-depth parsing and
analysis of their log files, where the parsing must correlate the log lines
per-connection or per-queueid rather than processing log lines
independently.  One of the barriers to this kind of processing is the
unstructured nature of Postfix log files, where each log line was added on
an ad hoc basis as a requirement was discovered or new functionality was
added.\footnote{A history of all changes made to Postfix is distributed
with the source code, available from \url{http://www.postfix.org/}} Further
complication arises because the set of rejection messages is not fixed: new
messages can be added by the administrator with custom checks; every
\DNSBL{}\footnote{This document is supplied with a glossary, see
\textsection\ref{Glossary}.} returns a different explanatory message;
policy servers may log different messages depending on the characteristics
of the connection.  There are many ways in which the log lines may differ
between servers, even within the same organisation: servers may be
configured differently, or running different versions of Postfix.  This
paper documents the difficult process of parsing Postfix log files,
presenting \PLP{}, a program that parses Postfix log files and places the
resulting data into a database for later use.  The gathered data can then
be used to optimise current anti-spam measures, provide a baseline to test
new anti-spam measures against, or to produce statistics showing how
effective those measures are.  Numerous other uses are possible for such
data: improving server performance by identifying troublesome destinations
and reconfiguring appropriately; identifying regular high volume uses
(e.g.\ customer newsletters) and restricting those uses to off-peak times;
detecting virus outbreaks that propagate via email; as a base for billing
customers on a shared server.  Preserving the raw data enables users to
develop a multitude of uses far beyond those conceived of by the author.

\vspace{1em}\noindent\textbf{Layout:}

XXX THIS ALL NEEDS TO BE CHECKED AND UPDATED\@.

Section~\ref{background} provides background information useful in
understanding the thesis, parser, and architecture.  It introduces the idea
of using a database schema as an \API{}, providing an interface to the data
gathered that is language-neutral.  The novel separation of rules, actions
and framework is discussed, giving the reasons that approach was taken when
designing the parser.

Section~\ref{state of the art review} reviews both the previously published
research in this area and other available Postfix log parsers, discussing
why they were deemed unsuitable for the task, including why they could not
be improved or expanded upon.

This algorithm requires a database for storing both the rules used when
parsing and the results gleaned from parsing.  The database schema used is
described in \sectionref{database schema}, explaining in detail the tables
used for storing the data gleaned from the log files and the table that
stores the rules.

Section~\ref{rules} discusses the parsing rules in detail, explaining the
purpose and usage of each field in a rule, referring to an example rule and
sample data it matches successfully against.  The pros and cons of
overlapping rules are considered, including techniques for detecting
unintentional overlaps.  Rule efficiency concerns are discussed, in
particular the optimisations used by the algorithm.  The section concludes
with a description of using the tools provided with the parser to generate
new rules (specifically the regex in each rule) from unparsed log lines.

Section~XXX contains the core of the paper, describing
a naive parsing algorithm and the complications initially encountered that
shaped the full algorithm.  A flow chart and a discussion of the emergent
behaviour exhibited by the algorithm accompanies a comprehensive
explanation of the different stages of the initial algorithm.  The
framework that actions and rules fit into is documented, then the actions
taken during execution of the algorithm are described, followed by the
process of adding a new action.  The section concludes with an in-depth
description of the further complications discovered, and their solutions
that complete the parser.

Section~\ref{parsing coverage} analyses the coverage the parser achieves
over a set of \numberOFlogFILES{} log files taken from a mail server
handling mail for over 700 users, averaging 8500 mails per day
(\graphref{Mails received per day}).  Coverage is described both
in terms of the fraction of log lines parsed and the fraction of mails and
connections successfully reconstructed by the parsing algorithm; dealing
with false negatives and a discussion of the difficulties in identifying
false positives is also included.  As part of determining the coverage of
the parser a random sampling of log lines was parsed, and the correctness
of the results manually verified.

Section~\ref{limitations-improvements} lists the limitations of the
algorithm, then suggests some ways of dealing with them, with the goal of
improving parsing and reproduction of the journey a mail takes through
Postfix.

Section~\ref{conclusion} contains the conclusion of the thesis, describing
the results of the research, design and implementation of the parser.

The bibliography contains references to the resources used in developing
the algorithm, writing the program, and preparing this thesis.  Also
listed are some additional resources expected to be helpful in
understanding \SMTP{}, Postfix, anti-spam techniques, or the thesis.

Appendix~\ref{Glossary} provides a glossary of terms used in the thesis.

Appendix~\ref{Acronyms} provides a list of acronyms used in the thesis.


\section{Background}

\label{background}

This section provides background information helpful in understanding the
remainder of the document.  It begins with a discussion of the motivation
underlying the project, followed by some technical information: the use of
a database as an \API{}\@; a brief introduction to \SMTP{}\@; a longer
introduction to Postfix, concentrating on the topics most relevant to this
document, namely Postfix anti-spam restrictions and policy servers.  The
assumptions made in designing and implementing the parser are explained, as
are the conventions used in this document.  Other projects which attempt to
parse Postfix log files are summarised (full details are available in
\sectionref{other-parsers}), finishing with a review of previously
published research in this area.

\subsection{Motivation}

\label{motivation}

This document and the program it describes are part of a larger project to
optimise a mail server's Postfix restrictions, generate statistics and
graphs, and provide a platform on which new restrictions can be trialled
and evaluated to determine if they are beneficial in the fight against
spam.  The program parses Postfix log files and populates a database with
the data gleaned from those log files, providing a consistent and simple
view of the log files which future tools can utilise.  The gathered data
can then be used to optimise current anti-spam measures, provide a baseline
to test new anti-spam measures against, or to produce statistics showing
how effective those measures are.

A snippet of \SQL{} provides a short example of the optimisation possible
using data from the database: determining which Postfix restrictions reject
the highest number of mails:

\begin{verbatim}
SELECT name, description, restriction_name, hits_total
    FROM rules
    WHERE postfix_action = 'REJECTED'
    ORDER BY hits_total DESC;
\end{verbatim}

If the database supports sub-selects percentages can be obtained:
\footnote{\SQL{} note: $||$ is the concatenation operator in SQLite3; if
the database containing the extracted data does not support this syntax,
then simply remove `` $||$ '$\%$'\hspace{1ex}'' from the query --- the
results will be the same, just slightly less visually pleasing.}

\begin{verbatim}
SELECT name, description, restriction_name, hits_total,
        (hits_total * 100.0 /
            (SELECT SUM(hits_total)
                FROM rules
                WHERE postfix_action = 'REJECTED'
            )
        ) || '%' AS percentage
    FROM rules
    WHERE postfix_action = 'REJECTED'
    ORDER BY hits_total DESC;
\end{verbatim}

XXX GIVE SAMPLE OUTPUT FOR THE SNIPPET ABOVE\@.

Another example is determining which restrictions are not effective: this
shows which restrictions had fewer than 100 hits on the last log file
parsed, and the percentage of total rejections each restriction represents.

\begin{verbatim}
SELECT name, description, restriction_name, hits,
        (hits * 100.0 /
            (SELECT SUM(hits)
                FROM rules
                WHERE postfix_action = 'REJECTED'
            )
        ) || '%' AS percentage
    FROM rules
    WHERE postfix_action = 'REJECTED'
        AND hits < 100
    ORDER BY hits ASC;
\end{verbatim}

These database queries yield summary statistics about the efficiency of
spam avoidance techniques; statistics that are far less feasible to assess
directly from log files without prior pre-processing into a database in the
fashion proposed, implemented and tested herein.

\subsection{Database as Application Programming Interface}

\label{database as API}

The database populated by this program provides a simple interface to
Postfix log files.  Although the interface is a database schema, it is in
effect quite similar to any other \API{} provided by shared code: it
insulates both user and provider of the \API{} from changes in the
implementation of the \API{}\@.  The algorithm implemented by the parser
can be improved; support can be added for earlier or later releases of
Postfix; bugs can be fixed or limitations removed from the parser; these
changes will not cause the user to be negatively impacted.  Statistics
and/or graphs can be generated from the database; new restrictions can be
tested and the results inspected; trends in the fight against spam can
emerge from historical data saved in the database; the parser remains the
same as the usage adapts.  Using a database simplifies writing programs
which need to interact with the data in several ways:

\begin{enumerate}

    \item It is already possible to access databases from the vast majority
        of programming languages, allowing a developer to access the data
        gathered using their preferred programming language, rather than
        being restricted to the language the parser is written in.  It is
        often possible to write an interface layer allowing code written in
        one language to be used in another language, but this greatly
        increases the effort required to use the parser.

    \item Databases provide complex querying and sorting functionality to
        the user without requiring large amounts of programming.  All
        databases provide a program, of varying complexity and
        sophistication, which can be used for ad hoc queries with minimal
        investment of time.

    \item Databases are easily extensible, e.g.:

        \begin{itemize}

            \item Other tables can be added to the database, e.g.\ to cache
                historical, summary or computed data.

            \item New columns can be added to the tables used by the
                program, with sufficient DEFAULT clauses or a clever
                TRIGGER or two.\footnote{Please refer to an \SQL{} guide
                for explanations of these terms,
                e.g.~\cite{sql-for-web-nerds}}

            \item A VIEW gives a custom arrangement of data with very
                little effort.

            \item If the database supports it, access can be granted on a
                fine-grained basis, e.g.\ allowing the finance department
                to produce invoices, the helpdesk to run limited queries as
                part of dealing with support calls, and the administrators
                to have full access to the data.

            \item Triggers can be written to perform actions when certain
                events occur.  In pseudo-\SQL{}\@:

\begin{verbatim}
CREATE TRIGGER ON INSERT INTO results
    WHERE sender = 'boss@example.com'
        AND postfix_action = 'REJECTED'
    SEND PANIC EMAIL TO 'postmaster@example.com';
\end{verbatim}

        \end{itemize}


    \item \SQL{} is reasonably standard and many people will already be
        familiar with it; for those unfamiliar with it there are lots of
        readily available resources from which to learn (a good
        introduction to \SQL{} can be found at~\cite{sql-for-web-nerds},
        others are~\cite{w3schools-sql-tutorial, sqlcourse.com}).  Although
        every vendor implements a different dialect of \SQL{}, the basics
        are the same everywhere (analogous to the overall similarities and
        minor differences between Irish English, British English, American
        English and Australian English).  Depending on the database in use
        there may be tools available which reduce or remove the requirement
        to know \SQL{}; for SQLite (the default database used by the
        implementation) there are several available~\cite{sqlite-guis}.

\end{enumerate}

Storing the results in a database will also increase the efficiency of
using those results, as the log files need only be parsed once rather than
each time the data is used; indeed the results may be used by someone with
no access to the original log files.



\subsection{Simple Mail Transfer Protocol}

\label{SMTP background}

The \SMTPlong{}, originally defined in \RFC{}~821~\cite{RFC821} and updated
in \RFC{}~2821~\cite{RFC2821}, is used for transferring mail between the
sending and receiving \MTA{}\@.  It is a simple, human readable, plain text
protocol, making it quite easy to test and debug problems with it.  Despite
the simplicity of the protocol many virus and/or spam sending programs fail
to implement it properly, so requiring strict adherence to the protocol
specification is beneficial in protecting against spam and
viruses.\footnote{\label{footnote:rfc760}Originally all mail servers
adhered to the principle of \textit{Be liberal in what you accept, and
conservative in what you send\/} from \RFC{}~760~\cite{rfc760}, but
unfortunately that principle was written in a friendlier time.  Given the
deluge of spam that mail servers are subjected to daily, a more appropriate
maxim could be: \textit{Require strict adherence to \RFC{}~2821; implement
the strongest restrictions you can; relax the restrictions and adherence
only when legitimate mail is impeded.\/}  it is not as friendly, nor as
catchy, but it more accurately reflects the current situation.} A typical
\SMTP{} conversation resembles the following (the lines starting with a
three digit number are sent by the server, all other lines are sent by the
client):

\begin{verbatim}
220 smtp.example.com ESMTP
HELO client.example.com
250 smtp.example.com
MAIL FROM: <alice@example.com>
250 2.1.0 Ok
RCPT TO: <bob@example.com>
250 2.1.5 Ok
DATA
354 End data with <CR><LF>.<CR><LF>
Message headers and body sent here.
.
250 2.0.0 Ok: queued as D7AFA38BA
QUIT
221 2.0.0 Bye
\end{verbatim}

An example deviation from the protocol:

\begin{verbatim}
220 smtp.example.com ESMTP
HELO client.example.com
250 smtp.example.com
MAIL FROM: Alice N. Other alice@example.com
501 5.1.7 Bad sender address syntax
RCPT TO: Bob in Sales/Marketing bob@example.com
503 5.5.1 Error: need MAIL command
DATA
503 5.5.1 Error: need RCPT command
Message headers and body sent here.
.
502 5.5.2 Error: command not recognized
QUIT
221 2.0.0 Bye
\end{verbatim}

This client is so poorly written that not only does it present the sender
and recipient addresses improperly, it ignores the error messages returned
by the server and carries on regardless.  There are many spam and virus
sending programs which are this deficient --- unfortunately others
(particularly newer programs) were written by competent programmers, or
utilise competently written programs (e.g.\ Postfix or Sendmail on Unix
hosts, Microsoft Outlook on Windows hosts).  Traditionally a mail server
would have done its best to deal with deficient clients, with the intention
of accepting as much mail destined for its users as
possible,\footref{footnote:rfc760} e.g.\ by ignoring the absence of a HELO
command, or accepting sender or recipient addresses which were not enclosed
in \texttt{<>}.  

A detailed description of \SMTP{} is beyond the scope of this document:
introductory guides can be found at~\cite{smtp-intro-01, smtp-intro-02},
the definitive references are~\cite{RFC821, RFC2821}.

\subsection{Postfix}

\label{postfix background}

Postfix is a \MTA{} with the following design aims (in order of
importance): security, flexibility of configuration, scalability, and high
performance.  It features extensive, extensible, optional anti-spam
restrictions, allowing an administrator to deploy those restrictions which
they judge suitable for their site's needs, rather than a fixed set chosen
by Postfix's author.  These restrictions can be selectively applied,
combined and bypassed on a per-client, per-recipient or per-sender basis,
allowing varying levels of stricture and/or permissiveness.  Postfix
leverages simple lookup tables to support arbitrarily complicated
user-defined sequences of restrictions and exceptions, with policy
servers\footnote{Policy servers will be explained in \sectionref{policy
servers}.} as the ultimate in flexibility.  Administrators can also supply
their own rejection messages to make it clear to senders why exactly their
mail was rejected.  Unfortunately this flexibility has a cost: complexity
in the log files generated.  While it is easy to use standard Unix text
processing utilities to determine the fate of an individual email,
following the journey an email takes through Postfix can be quite
difficult.  For the majority of mails the journey is simple and brief, but
the remaining minority can be quite complex (see \sectionref{additional
complications} for details).

Postfix's design follows the Unix philosophy of \textit{Write programs that
do one thing and do it well\/}~\cite{unix-philosophy}, and is separated
into various component programs to perform the tasks required of an
\MTA{}\@: receive mail, send mail, local delivery of mail, etc.\ --- full
details can be found in~\cite{postfix-overview}.  Each log line contains
the name of the Postfix component which produced it, and this information
is used when determining which rules should be used to parse each log line
(see \sectionref{rule characteristics} for details).  This design has
positive security implications also: those components which interact with
other hosts are not privileged,\footnote{Privilege in this case means the
power to perform actions that are limited to the administrator, and not
available to ordinary users.} so bugs in those components will not give an
attacker extra privileges; those components which are privileged do not
interact with other hosts, making it much more difficult for an attacker to
exploit any bugs which may exist in those components.

\subsubsection{Mixing and matching Postfix restrictions}

\label{Mixing and matching Postfix restrictions}

Postfix restrictions are documented fully in~\cite{smtpd_access_readme,
smtpd_per_user_control, policy-servers}; the following is a brief overview
only.

Postfix uses one restriction list (containing zero or more restrictions)
for each stage of the \SMTP{} conversation: client connection, HELO
command, MAIL FROM command, RCPT TO commands, DATA command, and end of
data.  The appropriate restriction list is evaluated for each stage
(evaluation will be explained shortly), though by default the restriction
lists for client connection, HELO and MAIL FROM commands will not be
evaluated until the first RCPT TO command is received, because some clients
do not deal properly with rejections before RCPT TO\@; a benefit of this
delay is that Postfix has more information available when logging
rejections.

Each restriction is evaluated to produce a result of \textit{reject},
\textit{permit}, \textit{dunno\/} or the name of another restriction to be
evaluated.\footnote{Other results are possible as described
in~\cite{smtpd_access_readme, smtpd_per_user_control, policy-servers}.} The
meaning of \textit{permit\/} and \textit{reject\/} is fairly obvious;
\textit{dunno\/} means to stop evaluating the current restriction and
continue processing the remainder of the restriction list, allowing more
specific cases to be used as exceptions to more general cases.  When the
result is the name of another restriction Postfix will evaluate the new
restriction, allowing restrictions to be chosen based on the client \IP{}
address, HELO hostname, sender address, recipient address,
etc.\footnote{E.g.\ the administrator may require that all clients on the
local network have valid \DNS{} entries, to prevent people sending mail
from unknown machines.}  The administrator can define new restrictions as a
list of existing restrictions, allowing arbitrarily long and complex
sequences of lookups, restrictions and exceptions.  Postfix tries to
protect the administrator in as far as is reasonable, e.g.\ the restriction
\texttt{check\_helo\_mx\_access} cannot cause a mail to be accepted,
because the parameter it checks (the hostname given in the HELO command) is
under the control of the remote client.  Despite this, it is possible for
the administrator to make catastrophic mistakes, e.g.\ rejecting mail for
all users --- the administrator must be cognisant of the ramifications of
their configuration changes.  XXX UNIX DOESN'T STOP YOU DOING STUPID THINGS
BECAUSE THAT WOULD ALSO STOP YOU DOING CLEVER THINGS\@.

Postfix uses simple lookup tables as the deciding factor when evaluating
some restrictions, e.g.\ in the example restriction below\newline
\tab{}\texttt{check\_client\_access~cidr:/etc/postfix/client\_access}
\newline The line above can be broken down as follows:

\begin{description}

    \item [check\_client\_access] The name of the restriction to evaluate.

    \item [cidr] The type of the lookup table.

    \item [/etc/postfix/client\_access] The file containing the lookup
        table.

\end{description}

The restriction \texttt{check\_client\_access} checks whether the \IP{}
address of the connected client is found in the file and returned the
associated action if found; the method of searching the file is dependant
on the type of the file (\texttt{cidr} in the example) --- see
\cite{postfix-lookup-tables} for more details.  Other restrictions
determine their result by consulting external sources, e.g.\
\texttt{reject\_rbl\_client dnsbl.example.com} checks whether the \IP{}
address of the client is present in the \DNSBL{}
\texttt{dnsbl.example.com}, rejecting the command if the client is listed.
This description is necessarily brief, for further details
see~\cite{smtpd_access_readme, smtpd_per_user_control, policy-servers}.


\subsubsection{Policy servers}

\label{policy servers}

A \textit{policy server\/}~\cite{policy-servers} is an external program
consulted by Postfix to determine the fate of an \SMTP{} command.  The
policy server is given state information\footnote{Sample state information
is shown in \tableref{Example attributes sent to policy servers}} and
returns a verdict as described in \sectionref{Mixing and matching Postfix
restrictions}.  The policy server can perform more complex checks than
those provided by Postfix: a trivial example is restricting mail from
addresses associated with the payroll system to sending mail on the third
Tuesday after pay day only, to help prevent problems from spam or (worse)
phishing mails with faked sender addresses.\footnote{A phishing mail might
claim that the payroll system had a disastrous disk failure; until the
server is replaced and restored all salary payments will have to be
processed manually, so please reply to this mail with your name, address
and back account details.}

Some widely deployed policy servers:

\begin{itemize}

    \item Checking \SPF{} records~\cite{openspf, Wikipedia-spf}.
        \SPF{}\label{spf introduction} records specify which mail servers
        are allowed to send mail claiming to be from a particular domain.
        The intention is to reduce spam from faked sender addresses,
        backscatter~\cite{postfix-backscatter} and
        joe~jobs~\cite{Wikipedia-joe-job}; however there has been a lot of
        resistance to the proposal because it breaks or vastly complicates
        some features of \SMTP{}, e.g.\ forwarding mail from one company or
        university to another when a user changes jobs.

    \item Greylisting~\cite{greylisting} is a technique that temporarily
        rejects mail when the triple of (sender address, recipient address,
        remote \IP{} address) is unknown; on second and subsequent delivery
        attempts from that triple the mail will be accepted.  The
        assumption is that maintaining a list of failed addresses and
        retrying after a temporary failure is uneconomical for a spam
        sender, but that a legitimate mail server must retry.  Sadly spam
        senders are using increasingly complex and well written programs to
        distribute spam, frequently using an \ISP{} provided \SMTP{} server
        from a compromised machine on the \ISP{}'s network.  Greylisting
        will slowly become less useful, but it does block a large
        percentage of spam mail at the moment; the most effective
        restrictions over the \numberOFlogFILES{} log files used in testing
        the parser are shown in \tableref{Summary of rejections}.
        Greylisting is obviously worth using, at least at the moment,
        particularly when you factor in Greylisting's position as the final
        restriction which a mail must overcome:\footnote{Greylisting is the
        final restriction a mail must overcome in the configuration used on
        the mail server the log files were obtained from; an administrator
        is free to use Greylisting at whichever position in the restriction
        list they feel is most appropriate for their mail system.}
        Greylisting only takes effect for mails which have passed every
        other restriction.

        \begin{table}[ht]
            \caption{Summary of rejections}\label{Summary of rejections}
            \input{build/restriction-table-include.tex}
        \end{table}

    \item Using a scoring system such as
        Policyd-weight~\cite{policyd-weight} where tests accumulate points
        against the sending system --- if the eventual score is higher than
        a threshold the mail is rejected.  Postfix's restrictions are
        binary, and the administrator must manually whitelist clients if
        they are to bypass a restriction; using a threshold which requires
        hitting several restrictions frees the administrator from
        whitelisting clients which fall foul of one restriction only.

    \item Rate limiting on a per-sender, per-client or per-recipient basis
        as performed by Policyd~\cite{policyd}.

\end{itemize}

Example attributes taken from~\cite{policy-servers}:

\begin{table}[ht]

    \caption{Example attributes sent to policy servers}\label{Example
    attributes sent to policy servers}

    \begin{tabular}[]{ll}

        request                 & smtpd\_access\_policy     \\
        protocol\_state         & RCPT                      \\
        protocol\_name          & SMTP                      \\
        helo\_name              & some.domain.tld           \\
        queue\_id               & 8045F2AB23                \\
        sender                  & foo@bar.tld               \\
        recipient               & bar@foo.tld               \\
        recipient\_count        & 0                         \\
        client\_address         & 1.2.3.4                   \\
        client\_name            & another.domain.tld        \\
        reverse\_client\_name   & another.domain.tld        \\
        instance                & 123.456.7                 \\

    \end{tabular}
\end{table}



\subsection{Assumptions}

The algorithm described and the program implementing it make a small number
of (hopefully safe and reasonable) assumptions:

\begin{itemize}

    \item The log files are whole and complete: nothing has been removed,
        either deliberately or accidentally (e.g.\ log rotation gone awry,
        file system filling up, logging system unable to cope with the
        volume of log messages).  On a well run system it is extremely
        unlikely that any of these problems will arise, though it is of
        course possible, particularly when suffering a deluge of spam or a
        mail loop.

    \item Postfix logs sufficient information to make it possible to
        accurately reconstruct the actions it has taken.  There are a
        number of heuristics used when parsing; see
        \sectionref{identifying-bounce-notifications},
        \sectionref{aborted-delivery-attempts} and \sectionref{pickup
        logging after cleanup} for details.

    \item The Postfix queue has not been tampered with, causing unexplained
        appearance or disappearance of mail.  This may happen if the
        administrator deletes mail from the queue without using
        \daemon{postsuper}, or if there is filesystem corruption.

\end{itemize}

In some ways this task is similar to reverse engineering or replicating a
black box program based solely on its inputs and outputs.  Although the
source code is available,\footnote{Reading and understanding the source
code would require a significant investment of time: there are 375,750
lines of code, documentation, etc.\ in Postfix 2.5.1's 17MB of source
code.} there are advantages to treating Postfix as a black box while
developing the parser:

\begin{itemize}

    \item The parser is developed using real world log files rather than
        the idealised log files someone would naturally envisage reading
        the source code.

    \item The source code cannot accurately communicate the variety of
        orderings in which log lines are written to the log file, as
        process scheduling independently interferes with logging and other
        processing.

    \item The parser acts as a second source of information, with the
        information gathered from empirical evidence.  An interesting
        project would be to compare the empirical knowledge inherent in the
        parsing algorithm with the documentation and source code of
        Postfix.

\end{itemize}


\subsection{Parser design}

\label{parser design}

It should be clear from the earlier Postfix background (\sectionref{postfix
background}) that log files produced by Postfix are not fixed; they vary
widely from host to host, depending on the set of restrictions chosen by
the administrator.  With this in mind, one of the parser's design aims was
to make adding new rules as easy as possible, to enable administrators to
properly parse their log files.  To enable this the parser is divided into
three parts:

\begin{description}

    \item [rules] Rules match individual log lines and determine which
        actions will be executed.  Rules provide an easily extensible
        method of associating log lines with actions, and are described in
        detail in \sectionref{rules}.

    \item [actions] Actions are invoked to deal with a log line once it has
        been identified by the rules: actions modify data structures,
        handle complications, and cause data to be saved to the database.
        The actions perform the work of reconstructing the journey a mail
        takes through Postfix.  Full details of actions can be found in
        \sectionref{actions-in-detail} and \sectionref{adding new actions}
        XXX IMPROVE THIS SENTENCE\@.

    \item [framework] The framework is responsible for loading rules,
        managing data structures, finding the rule which matches each log
        line, invoking the correct action, etc\@.  The framework provides
        the structure which actions and rules plug into.  The framework is
        described in detail in \sectionref{framework}.  XXX EXTEND
        FRAMEWORK SECTION IF NECESSARY\@.

\end{description}

It may help to think of the rules and actions as components which plug into
the framework.  

\label{why separate rules, actions and framework?}

XXX MERGE THE NEXT TWO PARAGRAPHS\@.

Decoupling the parsing rules from the associated actions and framework
allows new rules to be written and tested without requiring modifications
to the parser source code (significantly lowering the barrier to entry for
new or casual users who need to parse new log lines), and greatly
simplifies framework, actions and rules.  Decoupling also creates a clear
separation of functionality: rules handle low level details of identifying
log lines and extracting data from a log line; actions handle the higher
level details of following the path a mail takes through Postfix,
assembling the required data before storing it, dealing with complications
arising, etc; the framework provides services to actions and stores data.

Decoupling the actions from the framework simplifies both framework and
actions: the framework provides services to the actions, and does not need
to deal with the complications which arise, or the task of reconstructing a
mail's journey through Postfix; actions benefit from having services
provided by the framework, freeing them to concentrate on the task of
accurately reconstructing each mail's journey through Postfix and dealing
with the complications described in \sectionref{additional complications}.

XXX CLARIFY THE SENTENCE STARTING WITH ``Although this may see'' IN THE
NEXT PARAGRAPH\@.

Separating the rules from the actions and framework makes it possible to
parse new log lines without modifying the core parsing algorithm.  Although
this may seem like a trivial point, is it substantially more difficult to
understand a program's entire parsing algorithm, identify the correct
location to change and make the appropriate changes, versus adding a new
rule with the action to invoke, a \regex{} to match the log lines, and the
specification of the data to extract.  Bear in mind that the changes must
be made without adversely affecting existing parsing, particularly as there
may be edge cases which are not immediately obvious.\footnote{See
\sectionref{yet-more-aborted-delivery-attempts} for a complication which
occurs only four times in \numberOFlogFILES{} log files tested.}  Requiring
changes to the parser's code also complicates upgrades, as the changes must
be preserved during the upgrade, and may clash with changes made by the
developer.\footnote{See \sectionref{adding new actions} for how to add new
actions}  \parsername{} allows the user to add new rules to the database
without changing the parsing algorithm, unless the new log lines to be
parsed require functionality not already provided by the algorithm.  If the
new log lines do require new functionality, new actions can be added to the
parser without modifying existing actions or other parts of the algorithm;
only in the rare case that the new actions require support from other
sections of the code will more extensive changes be required.

XXX CHANGE THIS\@; THE PARSER DESIGN IS NOVEL, DO NOT STRESS THE SIMILARITY
TOO MUCH\@.

There is some similarity between the parser's design and William Wood's
\ATN{}~\cite{atns, nlpip}, a tool used in Computational Linguistics for
creating grammars to parse or generate sentences.  The resemblance between
\ATN{} and the parser is accidental, but it is interesting how two
apparently different approaches share an underlying separation of concerns;
this appears to be a natural division of responsibility and functionality.

% Do Not Reformat!

\begin{tabular}[]{lll}
    \textit{\ATN{}\/}   & \textit{Parser\/} & \textit{Similarity\/}     \\
    Networks            & Algorithm         & Determines the sequence 
                                              of transitions            \\
                        &                   & or actions which 
                                              constitutes a valid       \\
                        &                   & input.                    \\
    Transitions         & Actions           & Save data and impose
                                              conditions the            \\
                        &                   & input must meet to be
                                              considered valid.         \\
    Abbreviations       & Rules             & Responsible for 
                                              classifying input.        \\
\end{tabular}

\subsection{Conventions used in the document}

The words \textit{connection\/} and \textit{mail\/} are often used
interchangeably in this document; in general the word used was chosen based
on the context it appears in.

\subsection{Summary}

This section has provided background information on several topics relevant
to the remainder of the document.  It started with the motivation behind
the project, continuing with explanations of:

\begin{itemize}

    \item Using a database as an \API{}.

    \item \SMTP{}.

    \item Postfix restrictions and policy servers.

    \item Assumptions made when designing and developing the parser.

    \item A description of the parser's novel design.

    \item Conventions used in this document.

    \item An in-depth comparison of this parser with \LMA{}, the parser
        described in previously published research.

    \item A review of the literature previously published in this area.

\end{itemize}


\chapter{State of the Art Review}

\label{state of the art review}

At the start of this project ten Postfix log file parsers were tested, with
the hope of finding a suitable parser to build upon, rather than starting
from scratch.  There are not many Postfix log file parsers available ---
indeed it was quite difficult to find ten parsers to review for this
project --- and the functionality offered ranges from quite basic to much
more mature, depending on the needs of the author of the parser.  None of
those parsers were suitable for this project, so the decision was taken to
write a new parser.  The first parser reviewed is the only previously
published research in this area that the author is aware of; it parses
Postfix log files, but aim of the research is to show that presenting
extracted data in a more easily accessible format is useful to systems
administrators, rather than to improve anti-spam techniques..

The same ten parsers have been reviewed and compared to this project's
finished parser, to show how much effort would have been required to fulfil
the aims and requirements of this project.  It is important to compare and
contrast newly developed algorithms and parsers against those already
available, to accurately judge what improvements, if any, are delivered by
the newcomers.

XXX SHOULD THIS BE IN THE CONCLUSION INSTEAD\@?

There are some important differences between \parsername{} and the parsers
reviewed here:

\begin{enumerate}

    \item None of the parsers reviewed perform the kind of advanced parsing
        required for this project or deal with the complications described
        in \sectionref{complications}.

    \item Only \parsername{} enables parsing of new log lines without
        extensive and intrusive modifications to the parser; \parsernames{}
        architecture is described in \sectionref{why separate rules,
        actions, and framework?}.

    \item The parsers reviewed all produce a report of varying complexity
        and detail, whereas \gls{PLP} does not; it extracts data and leaves
        generation of reports from the data to other programs.  Using an
        \gls{SQL} database simplifies the process of generating such
        reports (discussed in \sectionref{database as API}); some sample
        queries are given in \sectionref{motivation}.  The parser developed
        for this project is designed to enable much more detailed log file
        analysis by providing a stable platform for subsequent programs to
        develop upon.

    \item Most of the reviewed parsers silently ignore log lines they
        cannot handle, whereas \parsername{} complains loudly about every
        single log line it fails to parse.  The exception is AWStats, which
        outputs the percentage of input lines it was unable to parse, but
        does not output the lines themselves.

    \item A minor difference is that most parsers do not handle compressed
        files; both \parsername{} and Splunk handle them transparently,
        without user intervention; Sawmill and Lire can be configured to
        support compressed files, but Sawmill exhibits a dramatic increase
        in parsing time when doing so.  Although this is a minor
        disparity, support for reading compressed log files is quite
        helpful, as it dramatically reduces the disk space required to
        store historical log files.

    \item Some of the parsers reviewed save the extracted data to a data
        store, but the majority discard all data once they have finished
        generating their report, making historical analysis impossible
        without parsing all log files every time.

\end{enumerate}

Each of the reviewed parsers was tested with the \numberOFlogFILES{} test
log files described in \sectionref{parser efficiency}.  The data extracted
by \parsername{} is documented in \sectionref{connections table} and
\sectionref{results table}; for convenience that list is repeated here:
server \gls{IP}, server hostname, client \gls{IP}, client hostname, HELO
hostname, queueid, start time, end time, \gls{SMTP} code, sender,
recipient, size, message ID\@.

\section{Log Mail Analyser}

\label{prior art}

There only appears to be one prior published paper about parsing Postfix
log files: \textit{Log Mail Analyzer: Architecture and Practical
Utilizations\/}~\cite{log-mail-analyser}.  The aim of \gls{LMA} is quite
different from \parsername{}: it attempts to present correlated data from
log files in a form suitable for a systems administrator to search using
the myriad of standard Unix text processing utilities already available.
It produces a \gls{CSV} file and either a MySQL
(\url{http://www.mysql.com/}) or Berkeley DB
(\url{http://www.oracle.com/database/berkeley-db/index.html}) database.
The decision to support both \gls{CSV} and Berkeley DB appears to have been
a serious limitation: XXX EXTEND\@: LIMITATIONS WILL BE EXPLAINED LATER OR
SOMETHING\@.  Very little documentation is provided with \gls{LMA}, though
some documentation is available in~\cite{log-mail-analyser}.  Studying the
source code is informative, though this author had difficulty as the
authors of \gls{LMA} wrote in Italian.

\gls{CSV} is a very simple format where each record is stored in a single
line, with fields separated by a comma or other punctuation symbol.
Problems with \gls{CSV} files include the need to escape separators in the
data stored, providing multiple values for a field (e.g.\ multiple
recipients), and adding new fields.  There is no standard mechanism to
document the fields or the separator, unlike \gls{SQL} databases where
every database includes a schema naming the fields and the type of data
they store (integer, text, timestamp, etc.).  The \gls{CSV} record format
is not documented, but the output file contains a comment giving the
format:\newline{} \texttt{\# Timestamp|Nome Client|IP Client|IP
Server|From|To|Status|Size} \newline{}\gls{LMA} treats lines starting with
\texttt{\#} as comments, but not all \gls{CSV} parsers will.

Berkeley DB only supports storing simple \textbf{(key, value)} pairs,
unlike \gls{SQL} databases that store arbitrary tuples.  In \gls{LMA}'s
main table the key is an integer referred to by secondary tables, and the
value is a \gls{CSV} line containing all of the data for that row.  The
secondary by-sender, by-recipient, by-date, and by-\gls{IP} tables use the
sender/recipient/date/\gls{IP} as the key, and the value is a \gls{CSV}
list of integers referring to the main table.  This effectively
re-implements \gls{SQL} foreign keys, but without the functionality offered
by even the most basic of \gls{SQL} databases (joins, ordering, searches,
etc.).  It also requires custom code to search on some combination of the
above, though the authors of \gls{LMA} did provide some queries: IP-STORY,
FROM-STORY, DAILY-EMAIL, and DAILY-REJECT\@.  Berkeley DB appears to be the
least useful of the three output formats: it does not provide the
functionality of a basic \gls{SQL} database, and unlike \gls{CSV} files it
can not be used with standard Unix text processing tools.

The schema used with the MySQL database is undocumented, but at least it is
possible to discover the schema with an existing \gls{SQL} database, unlike with
Berkeley DB\@; all \gls{SQL} databases embed the schema into the database
and provide commands for displaying it.  Berkeley DB does not embed a
schema, as there is neither requirement nor benefit; it only provides
\textbf{(key, value)} pairs, so any additional structuring of the data is
imposed by the application, thus the application must document this
structure.  MySQL support was not tested because there is no documentation
on the schema required.

Whether a MySQL database or Berkeley DB table is chosen in addition to the
\gls{CSV} output, \gls{LMA} stores the following data: time and date,
client hostname and \gls{IP} address, server \gls{IP} address, sender and
recipient addresses, \gls{SMTP} code, and size (for accepted mails only).
It is unclear which time and date is stored: start time, end time, or
delivery time?  Unlike \parsername{} it does not store the server hostname,
HELO hostname, queueid, start and end times, timestamps for each log line,
or message id (for accepted mails only).  Handling of multiple recipients,
\gls{SMTP} codes, or remote servers\footnote{A single mail may be sent to
multiple remote servers if it was addressed to recipients in different
domains, or Postfix needs to try multiple servers for one or more
recipients.} is not explained; experimental observation shows that multiple
records are added when there are multiple recipients (sadly the records are
not associated or linked in any way), and presumably the same approach is
taken when there are multiple destination servers.

\gls{LMA} requires major changes to the parser code to parse new log lines
or to extract additional data.  The code is structured as a long series of
blocks that each handle all log lines matching a single regex, so parsing
new log lines requires modifying an existing regex or carefully inserting a
new block in the correct place; extracting extra data will require
modifying multiple blocks, regexes, or both.

\gls{LMA} does not deal with any of the complications discussed in
\sectionref{complications}, except for correlating log lines by queueid;
not correlating log lines by pid means it cannot correlate most rejections.
It does not differentiate between different types of rejections, so it is
not suitable for the purposes of this project; the data about which
restriction caused the rejection is discarded, whereas the main goal of
this project is to retain that data to aid optimisation and evaluation of
anti-spam techniques.  \gls{LMA} fails to parse Postfix log files generated
on Solaris hosts because the fields automatically prepended to each log
line differ from those added on Linux hosts; log files from Solaris hosts
(and possibly other operating systems) thus require preprocessing before
parsing by \gls{LMA}.  Once the preprocessing has been performed on the
\numberOFlogFILES{} test log files \gls{LMA} parses the log files without
complaint, although it produced 32 entries in its output file for every
rejection in the input log file; it also missed some 40\% of delivered
mail.  Once these glaring deficiencies were discovered the author did not
waste any more time checking the results.

\gls{LMA} does provide some simple reports: IP-STORY, FROM-STORY,
DAILY-EMAIL and DAILY-REJECT\@.  These reports search the Berkeley DB files
for matching records: the first three extract \gls{CSV} lines for the
specified client \gls{IP} address, sender address, or date respectively.
DAILY-REJECT initially failed with an error message from the Perl
interpreter;\footnote{The error messages were: \newline{}\texttt{Undefined
subroutine \&main::LIST called at queryDB.pl line
372.}\newline{}\texttt{Undefined subroutine \&main::EXTRACT\_FROM\_DB
called at queryDB.pl line 379.}} after making corrections to the code it
worked, extracting the \gls{CSV} lines for the specified day where the
\gls{SMTP} code signifies a rejection.  All of these reports are trivially
simple to produce from the \gls{CSV} file using the standard Unix tool
\texttt{awk}\glsadd{awk}; the most complicated, DAILY-REJECT, is merely:

% perl queryDB.pl -dayreject 2007-01-26 > lma-query

\begin{verbatim}
awk -F\| 'BEGIN { previous = "" };
    $1 ~ /2007-01-26/ && $7 != "250" && $0 != previous 
    { print $0; print " "; previous = $0; }' lma_output.txt
\end{verbatim}

Notes about the command above:

\begin{itemize}

    \item It outputs a line containing only a single space after each
        matching record, to accurately replicate the output of
        DAILY-REJECT\@.

    \item DAILY-REJECT considers all \gls{SMTP} codes except \texttt{250}
        to be rejections; this includes invalid \gls{SMTP} codes such as
        \texttt{0} and \texttt{deferred}, so the awk command does too.
        These invalid \gls{SMTP} codes are most likely present because of
        incorrect parsing by \gls{LMA}.

    \item \gls{LMA} produces 32 output lines in its \gls{CSV} file for
        every single line it should have produced; the command above
        suppresses duplicate sequential lines.

\end{itemize}

There are some differences between the output from DAILY-REJECT and the
\texttt{awk} command; the author did not spend substantial time attempting
to explain these differences.

\begin{enumerate}

    \item The output from DAILY-REJECT is missing some records which are
        present in the \gls{CSV} file; this may be because it uses the
        Berkeley DB files instead, and there may be differences between the
        contents.

    \item Some records output by DAILY-REJECT are truncated: they are
        missing the last | separating fields and the newline following it,
        so the line containing only a single space is concatenated with the
        record.

\end{enumerate}

In summary, \gls{LMA} appears to be a proof of concept, written to
demonstrate the point of their paper (that having this information in an
accessible fashion is useful to systems administrators), rather than a
program designed to be useful in a production environment.

% Literature review notes:
%
% Hard-coded parsing, requiring code changes to add more.  Attempts to
% correlate log lines, saves data to database for data mining purposes.
% Hard to extend/expand/understand.  Appears to only save: date and hour,
% DNS name and \gls{IP} address host, mail server \gls{IP} address, sender,
% receiver and e-mail status (sent, rejected).  Undocumented schema.
% Design decision to use \gls{CSV} as an intermediate format between the
% log file and the database seems to have been restrictive.  Appears to
% require a queueid but majority of log lines (e.g.\ rejections) lack a
% queueid.  Supports whitelisting \gls{IP} addresses when parsing logs, but
% whitelisting when generating reports/data mining would be preferable.
% Supporting Berkeley DB is probably limiting the software - an example is
% the difficulty in searching a pipe-delimited string, so they have
% re-implemented foreign keys with tables keyed by ip address etc.\
% pointing at the main table - this also will not scale well.  There does
% not appear to be any attempt to deal with the complications I have
% encountered: their parsing is not detailed enough to encounter them.  It
% does not run properly; does not create any output; throws up errors.

\section{Pflogsumm}

\begin{quotation}

    pflogsumm is designed to provide an over-view of Postfix activity, with
    just enough detail to give the administrator a ``heads up'' for
    potential trouble spots.

\end{quotation}

\noindent{}\url{http://jimsun.linxnet.com/postfix_contrib.html} \newline{}
(Last checked 2008/11/23.)

Pflogsumm produces a report designed for troubleshooting rather than
in-depth analysis.  It does not support saving any data, and it does not
extract any data that it does not require to produce its report, e.g.\ it
does not extract the HELO hostname, queueid, start and end times,
timestamps for each log line, or message id.  Both the parsing and
reporting are difficult to extend because it is a specialised tool, unlike
the easily extensible design of \parsername{}.  It does not correlate log
lines by queueid or \gls{pid}, and does not need to deal with the
complications encountered during this project (\sectionref{complications}).
Pflogsumm produces a useful report, and successfully parsed the
\numberOFlogFILES{} log files it was tested with.\footnote{The results it reported
were not verified in detail, but it did not report any errors, and has a
very good reputation amongst Postfix users.}
Pflogsumm has many options to include or exclude certain
sections of the report; by default it includes the following:

\begin{itemize}

    \item Total number of mails accepted, delivered, and rejected.  Total
        size of mails accepted and delivered.  Total number of sender and
        recipient addresses and domains.

    \item Per-hour averages and per-day summaries of the number of mails
        received, delivered, deferred, bounced, and rejected.

    \item For received mail: per-domain totals for mails sent, deferred,
        average delay, maximum delay, and bytes delivered.  For received
        mail: per-domain totals for size and number of mails received.

    \item Number and size of mails sent and received for each address.

    \item Summary of why mail delivery was deferred or failed, why mails
        were bounced, why mails were rejected, and warning messages.

\end{itemize}

\section{Sawmill Universal Log File Analysis and Reporting}

\begin{quotation}

    Sawmill is a Postfix log analyzer (it also support 818 other log
    formats). It can process log files in Postfix format, and generate
    dynamic statistics from them, analyzing and reporting events. Sawmill
    can parse Postfix logs, import them into a SQL database (or its own
    built-in database), aggregate them, and generate dynamically filtered
    reports, all through a web interface. Sawmill can perform Postfix
    analysis on any platform, including Window, Linux, FreeBSD, OpenBSD,
    Mac OS, Solaris, other UNIX, and more.

\end{quotation}

\noindent{}\url{http://www.thesawmill.co.uk/formats/postfix.html}
\newline{} \url{http://www.thesawmill.co.uk/formats/postfix_ii.html}
\newline{} \url{http://www.thesawmill.co.uk/formats/beta_postfix.html}
\newline{} (Last checked 2008/11/23.)

Sawmill is a general purpose commercial product that parses 818 log file
formats (as of 2008/11/23) and produces reports from the extracted data.
Its data extraction facilities (described later) are too limited to save
sufficient data for the purposes of this project: although it can extract
three different sets of data from Postfix log files, they are not
interlinked in any way.  The documentation does not suggest that any
attempt is made to correlate log lines by either queueid or pid or to deal
with the difficulties documented in \sectionref{complications}.

Sawmill has three different Postfix log file parsers, extracting three
different sets to data:

\begin{enumerate}

    \item \url{http://www.thesawmill.co.uk/formats/postfix.html} \newline{}
        Fields extracted: from, to, server, UID, relay, status, number of
        recipients, origin hostname, origin \gls{IP}, and virus.  It also
        counts the number of and total size of all mails delivered.  The
        fields \texttt{server}, \texttt{uid}, \texttt{relay}, and
        \texttt{virus} are not explained in the documentation:
        \texttt{server} is probably the hostname or \gls{IP} of the server
        the mail is delivered to; \texttt{relay} might be the delivery
        method: \gls{SMTP}, local delivery, or \gls{LMTP}; \texttt{uid}
        might be the uid of the user submitting mail locally.  Postfix does
        not perform any form of virus checking (though it has many options
        for cooperating with an external virus scanner), so the
        \texttt{virus} field is a mystery.

    \item \url{http://www.thesawmill.co.uk/formats/postfix_ii.html}
        \newline{} Fields extracted: from, to, RBL list, client hostname,
        and client \gls{IP}\@.  It also counts the number and total size of
        all mails delivered.  

    \item \url{http://www.thesawmill.co.uk/formats/beta_postfix.html}
        \newline{} Fields extracted: from, to, client hostname, client
        \gls{IP}, relay hostname, relay \gls{IP}, status, response code,
        RBL list, and message id.  It also counts the number and size of
        all mails delivered, processed, blocked, expired, and bounced.

\end{enumerate}

Even if the three data sets were combined Sawmill would extract less data
than \parsername{}: it omits the HELO hostname, queueid, and start and end
times.  Sawmill does not extract any data about rejections except when the
rejection is caused by a \gls{DNSBL} check (\texttt{RBL list} in the list
of fields).

The source code is available in an obfuscated form, and the product is
quite expensive, requiring a \euros{100} + VAT licence per report
(discounts are available when buying multiple licences); in contrast
\parsername{} is free to use and the code is freely available.  Sawmill is
supplied with thorough and well written documentation; everything the
author searched for was documented, except the MySQL database schema.  A
commercial version of MySQL is required due to MySQL licensing
restrictions, but Sawmill's documentation explains why and includes
instructions on how to compile Sawmill so that it can use a non-commercial
version of MySQL (this was not attempted during the review process).

Sawmill's web interface supports searching on any combination of the fields
it extracts, and searches performed using the web interface produced
accurate results.  The interface for searching is neither as simple to use
nor as informative as the interface provided by Splunk.  The administrative
interface is much easier to use than Splunk's: it took only five minutes to
start parsing a whole directory of log files.  

When tested with the \numberOFlogFILES{} test log files it performed
adequately, though the rate it processed log files at did slow down
noticeably as it progressed.  Sawmill supports reading compressed log files
but it exhibits a dramatic slow down when doing so: it took six hours to
parse the first half of the log files, and twelve hours to parse the next
third; after twenty four hours parsing the remaining sixth it crashed due
to lack of disk space.  On the second parsing attempt the log files were
uncompressed beforehand and parsing took eight hours.

In summary Sawmill suffers from being a general purpose product; it is
probably much more useful when parsing log files where each log line is
self-contained (e.g.\ web server log files), rather than log files
containing interlinked log lines.  It is not suitable as a base for this
parser, as the source code made available is obfuscated and not intended
for modification; in addition the architecture would probably need to be
overhauled or replaced to deal with correlating log lines.

\section{Splunk}

\begin{quotation}

    Splunk is IT Search.

    Search and navigate IT data from applications, servers and network
    devices in real-time. Logs, configurations, messages, traps and alerts,
    scripts, code, metrics and more. If a machine can generate it ---
    Splunk can eat it. It's easy to download and use and it's very
    powerful.

\end{quotation}

\noindent{}\url{http://www.splunk.com/} \newline{}
(Last checked 2008/11/23.)

XXX THE FIRST PARAGRAPH NEEDS TO BE REWRITTEN\@; PICK POINTS, THEN COMPARE
AND CONTRAST\@.

Splunk aims to index all of an organisation's log files, providing a
centralised view capable of searching and correlating diverse log sources.
The web interface supports complicated searches, providing statistics and
graphs in real time, a facility not provided by \parsername{} (report
generation has been deferred to a subsequent program).  Searches can be
based on the fields extracted by Splunk or the full text of the log line.
Splunk allows quite complicated searches, but does not make the raw data
available in an accessible form.  Saved searches can be run periodically
and the results emailed to a recipient or sent to an external program for
further processing (though maybe without the graphs and detailed
statistics); the author was unable to save searches, though that may have
been due to limitations in the free version.  The database is not available
for use by external programs, whereas \parsername{} provides the database
and leaves it to the user to utilise it without limit or restriction.  The
interface is optimised for interactive use rather than automated queries
and it does not appear to be possible to write independent tools to utilise
the Splunk database;.  Some additional Postfix reports are supposedly
available at \url{http://www.splunkbase.com/}, but the author was unable to
find any Postfix reports, or indeed for any other log file types: every
category was empty, even those that the web site claimed had many reports
available.  Many types of graphs can be generated, though most are
variations of a bar or pie chart, except bubble and heatmap graphs.  It is
easy to drill down through the graphs to extract a portion of the data
(e.g.\ select the hour with the largest number of events, then select a
particular host, and finally a specific address), though it is not possible
to search on partial words.  All searches performed using the indexed data
returned reasonable results.

The web interface is quite attractive and simple to use when searching, but
as an administrator it seems unnecessarily difficult to perform simple
tasks.  When testing Splunk it took roughly 30 minutes to figure out how to
add a single log file to be indexed so that it could be searched, with the
downside that the log file was copied into a spool directory before
indexing, doubling the disk space usage.  The next test was to index all
the log files in a particular directory, but after three hours, numerous
futile attempts, and reading all the available documentation, the author
admitted defeat.  Using the \gls{CLI} rather than the web interface was
partially successful: the command \newline{} \tab{} \texttt{splunk find
logs }\textit{log-directory\/}\newline{} added 40 of the
\numberOFlogFILES{} log files to the queue for indexing.  Further attempts
enqueued the same 40 log files, without explaining why the others were
excluded.\footnote{The log files appear to have been indexed once only;
presumably Splunk keeps track of the files it has indexed and discards
requests to index files for a second time.  This may or may not be a useful
feature for \parsername{}.} There did not appear to be an option to ensure
the log files would be processed in the order they were created, though
this may be neither necessary nor beneficial with Splunk.  Subsequently the
author was successful in adding a single file at a time using the
\gls{CLI}:\newline{} \tab{} \texttt{splunk add tail
}\textit{filename\/}\newline{} A simple loop was then enough to add all the
desired log files.  Splunk will periodically check all indexed files for
updates unless they are manually removed from its list; this may or may not
be useful behaviour.  Splunk did not appear to have any difficulty in
indexing the log files, once they had been successfully added to its queue.
\parsername{} parses the logs it is instructed to parse, in the order
given; periodic parsing of logs is a task an administrator can easily
achieve with \texttt{cron(8)} and \texttt{logrotate(8)}.

Copious documentation is made available on \url{http://www.splunk.com/},
but poor organisation and sheer abundance makes it extremely hard to find
useful information.  Searches confusingly tended to return results from old
documentation rather than new.  In general the documentation appears to
have been written by someone intimately acquainted with the software, who
has difficulty understanding how a newcomer would approach tasks or the
questions they would ask.

Splunk supports reading compressed log files without any configuration by
the user.  The free version of Splunk limits the volume of data indexed per
day to 500MB, though a trial Enterprise licence is available that allows
indexing of up to 5GB of data per day.  In 2007 the cheapest licenced
version cost \$5000 plus \$1000 support, and limited the volume of data
indexed per day to 500MB\@.  Prices were removed from the Splunk website
during 2008; now Splunk's sales team must be contacted for a quote.
Typical log file sizes for a small scale mail server are given in
\sectionref{parser efficiency}.

When parsing Postfix log files Splunk parses the standard
syslog\glsadd{syslog} fields at the beginning of the log line, and extracts
any \texttt{key=value} pairs occurring after the standard syslog prologue:
to and from addresses, HELO hostname, and protocol (\gls{SMTP} or
\gls{ESMTP}).  \parsername{} extracts noticeably more data (client and
server \gls{IP} and hostname, queueid, start and end times, timestamps for
each log line, \gls{SMTP} code, and message ID), though it does not make
the full text of the line available (this could be trivially added if
desired, but would greatly increase the size of the resulting database).
The full power of \gls{SQL} is available when searching the data extracted
by \parsername{}, allowing the user to search on arbitrarily complicated
conditions.

Splunk is a generic tool, so it lacks any Postfix specific support over and
above extracting the \texttt{key=value} fields from a log line; most
importantly it makes no attempt to correlate log lines by queueid or
\gls{pid}, or to handle any of the other myriad complications discussed in
\sectionref{complications}.  Its source code is unavailable, so it could
not be used as a base for this project, even if it fulfilled all other
requirements.

\section{Isoqlog}

\begin{quotation}

    Isoqlog is an MTA log analysis program written in C. It designed to
    scan qmail, postfix, sendmail and exim logfile and produce usage
    statistics in HTML format for viewing through a browser. It produces
    Top domains output according to Sender, Receiver, Total mails and
    bytes; it keeps your main domain mail statistics with regard to Days
    Top Domain, Top Users values for per day, per month and years.

\end{quotation}

\noindent{}\url{http://www.enderunix.org/isoqlog/} \newline{}
(Last checked 2009/01/11.)

Isoqlog's report misses most of the information gathered by \parsername{}:
the data extracted is limited to the number of mails sent by each sender,
and it only reports on senders from the domains listed in its configuration
file, making it impossible to produce complete reports.  It ignores all log
lines except those with today's date, so it is impossible to analyse
historical log files, and testing with the \numberOFlogFILES{} test log
files was pointless.  It does maintain a record of data previously
extracted, which the newly extracted data is merged into; the format of the
data store is undocumented.  It does not utilise rejection log lines in any
way, so is unsuitable for the purposes of this project.  Its parsing is
completely inextensible, indeed is almost incomprehensible, relying on
\texttt{scanf(3)}, unexplained fixed offsets, and low level string
manipulation; it is the opposite end of the spectrum to \parsernames{}
parsing.  It does not handle any of the complications discussed in
\sectionref{complications}, does not gather the breadth of data required
for this project, and ignores the majority of log lines produced by
Postfix.

\section{AWStats}

\begin{quotation}

    AWStats is a free powerful and featureful tool that generates advanced
    web, streaming, ftp or mail server statistics, graphically. This log
    analyzer works as a CGI or from command line and shows you all possible
    information your log contains, in few graphical web pages. It uses a
    partial information file to be able to process large log files, often
    and quickly. It can analyze log files from all major server tools like
    Apache log files (NCSA combined/XLF/ELF log format or common/CLF log
    format), WebStar, IIS (W3C log format) and a lot of other web, proxy,
    wap, streaming servers, mail servers and some ftp servers.

\end{quotation}

\noindent{}\url{http://awstats.sourceforge.net/} \newline{}
\url{http://awstats.sourceforge.net/awstats.mail.html} \newline{}
\url{http://awstats.sourceforge.net/docs/awstats_faq.html#MAIL} \newline{}
(Last checked 2009/01/11.)

AWStats will produce simple graphs for many different services, but
supporting many different services without special purpose code limits its
functionality.  The data it will extract from an \gls{MTA} log file is
limited in comparison to \parsername{}: time2, email, email\_r, host,
host\_r, method, url, code, and bytesd.  There is no explanation for any of
those fields in the documentation (\parsername{} provides copious
documentation), so the author could not understand the extracted data, nor
determine what data is missing in comparison to \parsername{}.  AWStats
coerces Postfix log files into Apache\footnote{The Apache web server is the
most popular HTTP server in use over the past 10 years; more information is
available at \url{http://httpd.apache.org/}.} format log files, for
analysis by AWStats' HTTP log file parser.  The converting parser only
deals with a small portion of the log lines generated by Postfix, silently
skipping those it cannot deal with, and does not distinguish between
different types of rejection; it would be a lot of work to extend it to
handle new log lines.  Although it does correlate log lines by queueid (not
by pid), it does not deal with any of the other complications described in
\sectionref{complications}.  AWStats supports saving data but the format of
the saved data is not documented, as far as the author could tell.  It also
supports reading compressed log files, but that functionality was not
tested.

When tested with the \numberOFlogFILES{} test log files AWStats' reported
that it parsed 9,240,075 (88.70\%) of 10,416,129 log lines, skipping
1,176,050 (11.29\%) corrupt log lines; however there are
\numberOFlogLINES{} lines in the \numberOFlogFILES{} log files, so AWStats
parsed only 17.15\% of the input lines, ignoring the remaining 82.85\%.

The graphs it produces give an overview of mails received for the last
calendar month, showing:

\begin{itemize}

    \item The number of mails accepted from each host.

    \item How many mails were received by each recipient.

    \item The average number of mails accepted by the server per-day and
        per-hour.

    \item A summary of the \gls{SMTP} codes used when rejecting delivery
        attempts.

\end{itemize}

AWStats was not a suitable base for this project, because it assumes that
all log files can be rewritten to be compatible with web server log files,
and will contain similar data; coercing Postfix log files into web server
log files, without substantial data loss, would require fully parsing the
Postfix log files, so AWStats would not be required.  It may be possible to
use AWStats graphing capabilities to generate reports, by generating log
files to use as input to AWStats from the data extracted by \parsername{}.

\section{Anteater}

\begin{quotation}

    The Anteater project is a Mail Traffic Analyser. Anteater supports
    currently the logformat produced by Sendmail and by Postfix. The tool
    is written in 100\% C++ and is very easy to customize. Input, output,
    and the analysis are modular class objects with a clear interface.
    There are eight useful analyse modules, writing the result in plain
    ASCII or HTML, to stdout or to files.

\end{quotation}

\noindent{}\url{http://anteater.drzoom.ch/} \newline{}
(Last checked 2009/01/11.)

Anteater does not have any English documentation so it is impossible for
this author to accurately comment on the analysis it performs.  It did not
run successfully when tested, and its parsing would certainly be out of
date as Postfix has evolved considerably since this tool was last updated
(2003/11/06).  As it neither ran successfully nor has documentation the
author can read a detailed review cannot be provided.

The Debian project (\url{http://www.debian.org/}) provides a manual page
with the copy of anteater it distributes, so the author was at least able
to run anteater with the correct arguments; sadly anteater produced zero
for every statistic, presumably because it was unsuccessful in parsing the
log lines.

\section{Yet Another Advanced Logfile Analyser}

\begin{quotation}

    yaala is a very flexible analyser for all kinds of logfiles. It uses
    parsers to extract information from a logfile, an SQL-like query
    language to relate the information to each other and an output-module
    to format the information appropriately.

\end{quotation}

\noindent{}\url{http://yaala.org/} \newline{}
(Last checked 2009/01/11.)

YAALA uses a plugin-based system to analyse log files and produce HTML
output reports, with all the parsing and report generation handled by
modules.  Using YAALA as a base would be only slightly less work than
starting from scratch, as both the input and output modules would need to
be written specially; it may even be more work to implement the parser
within the constraints of YAALA\@.  YAALA supports storing previously
gathered data using Perl's Storable module~\cite{perl-storable}, so other
Perl programs can use Storable to load, examine, and optionally modify the
data; \parsername{} uses a well documented database which is accessible
from the majority of programming languages.  This information was gleaned
from the source code, as the documentation is sadly lacking.

YAALA provides a Postfix parser that extracts the following two types of
fields from specific log lines:

\begin{eqlist}

    \item [Aggregations:] count (not explained), bytes (sum of bytes
        transferred).

    \item [Keyfields:] incoming\_host, outgoing\_host, date, hour, sender,
        recipient, defer\_count, delay.  Which date and hour are stored is
        not documented: start time, end time, delivery time, or another
        time?

\end{eqlist}

\noindent{}YAALA's Postfix parser extracts some of the fields \parsername{}
does: it stores either the \gls{IP} or the hostname for client and server,
not both; it omits the HELO hostname, queueid, \gls{SMTP} code, size of
each accepted mail, start and end times, timestamps for each log line, and
message ID\@.  It extracts some data that \parsername{} does not: the delay
in delivering each mail, and how many times delivery was deferred for each
mail; these could easily be extracted by \parsername{} if desired.  XXX
EXTRACT DELAY AND DEFER ONCE AUTOMATIC EXTRACTION HAS BEEN FINISHED\@.
Unlike \parsername{}, YAALA does not maintain separate counters for each
restriction; this rules out the possibility of using the collected data for
optimisation, testing or understanding of restrictions.  YAALA's Postfix
parser does not deal with the complications explained in
\sectionref{complications}, though it does correlate log lines by queueid.

YAALA provides a mini-language based on \gls{SQL} that is used when
generating reports; sample reports can be seen
at~\url{http://www.yaala.org/samples.html}.  Example query for HTTP proxy
servers: \newline{} \tab{} \texttt{requests BY file WHERE host =\~{}
Google} \newline{} The mini-language is quite limited and cannot be used to
extract data for external use, merely to create reports.  Only data
selected by the query will be saved in the data store; other data will be
discarded, and removed from the data store if already present.

Testing YAALA was unsuccessful because all the select clauses tried
produced a similar error message:
\newline{}\tab{}\texttt{lib/Yaala/Data/Core.pm: Unavailable aggregation
requested:} \newline{}\tab{}\tab{}\texttt{``bytes''. Returning 0.}
\newline{}  The underlying reason for this is that YAALA only parsed 408
(0.11\%) of 360632 log lines in the first log file; it was not tested with
the remainder of the \numberOFlogFILES{} log files.

In summary YAALA provides a Postfix parser that tries to parse the most
common Postfix log lines only, provides reasonably flexible report
generation from the limited data extracted, but has no facilities to
extract data for use in other tools.

\section{Lire}

\begin{quotation}

    As any good system administrator knows, there's a lot more to keep
    track of in an active network than just webservers. Lire is hands down
    the most versatile log analysis software available today. Lire not only
    keeps you informed about your HTTP, FTP, and mail traffic, it also
    reports on your firewalls, your print servers, and your DNS activity.
    The ever growing list of Lire-supported services clearly outstrips any
    other software, in large part thanks to the numerous volunteers who
    have pioneered many new services and features. Lire is a total solution
    for your log analysis needs.

\end{quotation}

\noindent{}\url{http://logreport.org/lire.html} \newline{}
(Last checked 2009/01/11.)

Lire is a general purpose log file parser supporting many different types
of log file.  It takes a similar approach to YAALA, using plugins to parse
different log file types.  The data extracted by its Postfix parser is not
clearly documented; the manual says only:

\begin{quotation}

    The email servers' reports will show you the number of deliveries and
    the volume of email delivered by day, the domains from which you
    receive or send the most emails, the relays most used, etc.

\end{quotation}

\noindent{}Examining the source code reveals that the parser looks for
\texttt{<key>=<value>} pairs in each log line, extracts them, and
correlates the data by queueid.  This approach will find the following
fields: HELO hostname, queueid, \gls{SMTP} code, sender and recipient
addresses, and size of accepted mails.  It is unclear if the parser will
extract any further data.  Lire misses the following fields extracted by
\parsername{}: client and server \gls{IP} and hostname, start and end
times, timestamps of each log line, and message ID\@. 

Lire supports multiple output formats for generated reports (text, HTML,
PDF, and Excel 95) but the reports do not appear to be customisable;
\parsername{} does not produce any reports.  Lire's report contains less
detail than Pflogsumm, and is considerable harder to configure.  Lire
supports saving extracted data for later report generation, but accessing
this data from another application is undocumented; given the source code
it should be possible, with enough time and effort, to understand the
format.  \parsername{} uses an \gls{SQL} database to make accessing the
extracted data as easy as possible.  

Like AWStats and Logrep, Lire attempts to correlate log lines by queueid,
but not by \gls{pid}, so the complete list of recipients for a mail should
be available; however its parser extracts only part of the available data
and makes no attempt to deal with the other complications described in
\sectionref{complications}.  When testing Lire on the \numberOFlogFILES{}
test log files it performed reasonably well: the numbers it reports appear
reasonable, and the subset verified by the author were correct.  Its report
provided summaries of: 

\begin{itemize}

    \item Delivery status and failed deliveries.

    \item Sender and recipient domains and servers.

    \item Number of deliveries and bytes per-day and per-hour.

    \item Recipients by domain.

    \item Deliveries by relays, by size, and by delay.

    \item Delays by server and by domain.

    \item Which pair of correspondents exchanged the highest number of
        emails.

\end{itemize}

Lire would not be a suitable base for this project: it does not extract
enough data; does not deal with rejections in any way; does not make the
extracted data easily available to other programs.  Its parser is
in-extensible but could easily be replaced, however that would require
writing a parser from scratch, so would not be any easier.

\section{Logrep}

\begin{quotation}

    Logrep is a secure multi-platform framework for the collection,
    extraction, and presentation of information from various log files. It
    features HTML reports, multi dimensional analysis, overview pages, SSH
    communication, and graphs, and supports over 30 popular systems
    including Snort, Squid, Postfix, Apache, Sendmail, syslog, ipchains,
    iptables, NT event logs, Firewall-1, wtmp, xferlog, Oracle listener and
    Pix.

\end{quotation}

\noindent{}\url{http://www.itefix.no/i2/index.php} \newline{}
(Last checked 2009/01/11.)

Logrep extracts less than half the fields \parsername{} does:

\begin{itemize}

    \item For mail sent and received: from address, size, and time and
        date.  Which date and hour are stored is not documented: start
        time, end time, delivery time, or another time?

    \item For mail sent: to addresses, \gls{SMTP} code, and delay.

    \item For mail received: the hostname of the sender.

\end{itemize}

It also counts the number of log lines parsed and skipped.  It omits client
\gls{IP} and hostname, server \gls{IP}, HELO hostname, queueid, and message
ID\@.  It extracts the delay for delivered mails, which \parsername{} does
not.  Log lines are correlated based on the queueid (referred to as
sessionname [sic] within Logrep), but not by \gls{pid}.  The parsing is
error prone: empty fields are saved when the log line does not match the
regex, though it appears that they will not overwrite existing data.  Most
notably rejections are completely ignored, making it unsuitable for the
purposes of this project.  It does not try to address any of the
complications in \sectionref{complications} except for correlating by
queueid.

Logrep does not come with any documentation, though some scant
documentation is available on its website (\parsername{} provides copious
documentation).  It requires a web browser to interact with it, so
automated log file processing will be difficult, whereas enabling automated
processing is a key part of \parsernames{} design.  Sadly all the author's
attempts to use Logrep failed, as it was unable to access the log files
selected; this appears to be a bug rather than operator error.  If it was
caused by operator error, the interface needs improvement as the (minimal)
instructions were followed as closely as possible, and multiple attempts
were made.  As parsing failed it was not possible to review the reports
Logrep can generate (available in HTML only), or to examine the
(undocumented) format in which it can save extracted data for subsequent
reuse.

Logrep extracts far less data from Postfix log files than \parsername{},
completely ignores rejections, is effectively undocumented, does not deal
with the more complicated aspects of Postfix log files, and at the time of
writing does not work properly.

\section{Summary}

There are other programs available which perform basic Postfix log file
parsing (some to a greater level of detail than others), but few attempt to
correlate log lines by queueid (none correlate by \gls{pid}) to produce an
overall record of the journey of each mail through Postfix.  None of the
reviewed parsers collect the breadth of information gathered by
\parsername{}, or make it as easy to extend the parser to handle new log
lines.  Most of the parsers generate a report and immediately discard the
data extracted from the log files; those that do not discard the data
typically retain it in a format inaccessible to other tools.  Nearly all of
the parsers reviewed can produce a report of greater or lesser detail and
complexity, unlike \parsername{}.  The quality of the documentation offered
by the subset of parsers that provide some varies from unusable to good;
none of the parsers provide any documentation on the format of their data
stores (if they have one).  Fewer than half of the parsers were capable of
parsing the \numberOFlogFILES{} test log files, and improving or extending
parsing would have been quite a difficult task for any of the parsers.
Table \refwithpage{Summary of parsers' features} provides a summary of the
parsers' features.

The overriding difference between \parsername{} and the other parsers
reviewed herein is that none of them aim for the high level of
understanding of Postfix log files achieved by \parsername{}.


\begin{table}[htb]
    \caption{Summary of parsers' features}
    \empty{}\label{Summary of parsers' features}
    \begin{tabular}{llllll}
        \tabletopline{}%
        Parser          & Parsed test   & Data              & Custom            & Documentation  & Source       \\
                        & log files?    & store?            & reports?          & quality?       & code?        \\
        \tablemiddleline{}%
        \gls{LMA}       & No            & Yes               & No                & Poor           & Yes          \\ 
        Pflogsumm       & Yes           & No                & Partial \dag{}    & Good           & Yes          \\
        Sawmill         & Yes           & Yes               & Searches          & Very good      & \nialpha{}   \\
        Splunk          & Yes           & Yes               & Searches          & Abundant       & No           \\
                        &               &                   & \& reports        & but poor       &              \\
        Isoqlog         & No            & Yes               & No                & Poor           & Yes          \\
        AWStats         & Partially     & Yes               & Partial \dag{}    & Good           & Yes          \\
        Anteater        & No            & No                & No                & Poor           & Yes          \\
        YAALA           & No            & Yes \ddag{}       & Searches          & Poor           & Yes          \\
        Lire            & Yes           & Yes               & Yes               & Reasonable     & Yes          \\
        Logrep          & No            & Yes               & No                & Poor           & Yes          \\
        \parsername{}   & Yes           & Yes \nibeta{}     & No \nichi{}       & XXX            & Yes          \\
        \tablebottomline{}%
    \end{tabular}

    \begin{eqlist}

        \item [\dag{}] Sections can be omitted from a report, but extra
            sections can not be added.

        \item [\ddag{}] YAALA only stores the data required to produce the
            latest report; other data will be discarded.

        \item [\nialpha{}] Sawmill's source code is available in an
            obfuscated form, so that customers can compile it on platforms
            that pre-compiled binaries are not available for.

        \item [\nibeta{}] \parsername{} is the only parser with
            documentation for its data store.

        \item [\nichi{}] \parsername{} defers report generation to
            subsequent programs, but all the necessary data and
            documentation to produce reports is provided.

    \end{eqlist}

\end{table}

\clearpage{}

\chapter{Parser Architecture}

\label{parser architecture}

To avoid cluttering the explanation of the parser architecture with the
details involved in implementing a parser for Postfix log files, the two
topics have been separated.  This chapter presents the architecture
developed for this project, beginning with the overall architecture and
design, followed by the three components of the architecture: Framework,
Actions, and Rules.  This chapter centers on the theoretical,
implementation-independent aspects of the architecture; the practical
difficulties of writing a parser for Postfix log files are covered in
detail in \sectionref{Postfix Parser Implementation}.

\section{Architecture Overview}

\label{parser design}

It should be clear from the earlier Postfix background (\sectionref{postfix
background}) that log files produced by Postfix may vary widely from host
to host, depending on the set of restrictions chosen by the administrator.
With this in mind, one of the architecture's design aims was to make adding
new rules to parse new inputs as effortless as possible, to enable
administrators to properly parse their own log files.  The solution
developed is to divide the architecture into three parts: Framework,
Actions, and Rules.  Each will be discussed separately, but first an
overview:

\begin{boldeqlist}

    \item [Framework]  The framework is the structure that actions and
        rules plug into.  It manages the parsing process, providing shared
        data storage, loading and validation of rules, storage of results,
        and other support functions.

    \item [Actions] Each action performs the work required to deal with a
        single \textit{category\/} of inputs, e.g.\ rejecting a delivery
        attempt.  Actions are invoked to process an input once it has been
        recognised by a rule.

    \item [Rules]  Rules are responsible for classifying inputs: each rule
        recognises one input \textit{variant\/} (a single input category
        may have many input variants).  Each rule also specifies the action
        to be invoked when an input has been recognised; rules thus provide
        an extensible method of associating inputs with actions.

\end{boldeqlist}

For each input, the framework tries each rule in turn until it finds a rule
that recognises the input, then invokes the action specified by that rule.
If the input is not recognised by any of the rules, the framework issues a
warning; the framework can continue parsing after this, although for some
inputs stopping immediately may be preferable.

Decoupling the parsing rules from their associated actions allows new rules
to be written and tested without requiring modifications to the parser
source code, significantly lowering the barrier to entry for casual users
who need to parse new inputs, e.g.\ part-time systems administrators
attempting to combat and reduce spam; it also allows companies to develop
user-extensible parsers without divulging their source code.  Decoupling
the framework, actions, and rules simplifies all three and creates a clear
separation of functionality: the framework manages the parsing process and
provides services to the actions; actions benefit from having services
provided by the framework, freeing them to concentrate on the task of
accurately and correctly processing the inputs and the information provided
by rules; rules are responsible for classifying inputs, and extracting data
from those inputs for processing by actions.

Separating the rules from the actions and framework makes it possible to
parse new inputs without modifying the core parsing algorithm.  Adding a
new rule with the action to invoke and a regex to recognise the inputs is
trivial in comparison to understanding an entire parser, identifying the
correct location to change, and making the appropriate changes.  Additions
to a parser must be made without adversely affecting existing parsing,
including any edge cases that are not immediately obvious; an edge case
that occurs only four times in \numberOFlogFILES{} log files is described
in Yet More Aborted Delivery Attempts
(\sectionref{yet more aborted delivery attempts}).  More intrusive changes
are more likely to introduce a bug, so reducing the extent of the changes
is important.  Requiring changes to a parser's source code also complicates
upgrades of the parser, because the changes must be preserved during the
upgrade, and they may clash with changes made by the developer.  This
architecture allows the user to add new rules without having to edit a
parser, unless the new inputs cannot be processed by the existing actions.
If the new inputs do require new functionality, new actions can be added to
the parser without having to modify existing actions; only in the rare
event that the new actions need to cooperate with existing actions will
more extensive changes be required.

Some similarity exists between this architecture and William Wood's
\acronym{ATN}~\cite{atns,nlpip}, used in Computational Linguistics for
creating grammars to parse or generate sentences.  The resemblance between
the two (shown in \tableref{Similarities between ATN and this
architecture}) is accidental, but clearly the two different approaches
share a similar division of responsibilities, despite having different
semantics.

% Do Not Reformat!

\begin{table}[thbp]
    \caption{Similarities between ATN and this architecture}
    \empty{}\label{Similarities between ATN and this architecture}
    \begin{tabular}[]{lll}
        \tabletopline{}%
        \acronym{ATN}   & Architecture  & Similarity                  \\
        \tablemiddleline{}%
        Networks        & Parser        & Determines the sequence
                                          of transitions              \\
                        &               & or actions that
                                          constitutes a valid input.  \\
        Transitions     & Actions       & Assemble data and
                                          impose conditions the       \\
                        &               & input must meet to be
                                          accepted as valid.          \\
        Abbreviations   & Rules         & Responsible for
                                          classifying input.          \\
        \tablebottomline{}%
    \end{tabular}
\end{table}

The architecture can be thought of as implementing transduction: it takes
data in one form and transforms it to another form; \parsername{}
transforms log files to a database.

Unlike traditional parsers such as those used when compiling a programming
language, this architecture does not require a fixed grammar specification
that inputs must adhere to.  The architecture is capable of dealing with
interleaved inputs, out of order inputs, and ambiguous inputs where
heuristics must be applied --- all have arisen and been successfully
accommodated in \parsername{}.  This architecture is ideally suited to
parsing inputs where the input is not fully understood or does not conform
to a fixed grammar: the architecture warns about unparsed inputs and other
errors, but continues parsing as best it can, allowing the developer of a
new parser to decide which deficiencies are most important and the order to
address them in, rather than being forced to fix the first error that
arises.

\section{Framework}

\label{framework in architecture}

The framework manages the parsing process and provides support functions,
freeing the programmers writing actions to concentrate on writing
productive code.  It links actions and rules, allowing either to be
improved independently of the other, and allows new rules to be written
without needing changes to the source code of a parser.  The framework is
the core of the architecture and is deliberately quite simple: the rules
deal with the variation in inputs, and the actions deal with the
intricacies and complications encountered during parsing.  The function
that finds the rule recognising the input and invokes the requested action
can be expressed in pseudo-code as:

\begin{verbatim}
INPUT:
for each input {
    for each rule defined by the user {
        if this rule recognises the input {
            perform the action specified by the rule
            next INPUT
        }
    }
    warn the user that the input was not parsed
}
\end{verbatim}

Most parsers will require the same basic functionality from the framework;
it is responsible for managing the parsing process from start to finish,
which will generally involve the following:

\begin{description}

    \item [Register actions]  Each action needs to be registered with the
        framework so that the framework knows about it; the list of
        registered actions will also be used when validating rules.

    \item [Load and validate rules]  The framework loads the rules from
        wherever they are stored: a simple file, a database, or possibly
        even a web server or other network service --- though that would
        have serious security implications.  It validates each rule to
        catch problems as early in the parsing process as possible; the
        checks will be implementation-specific to some extent, but will
        generally include some of the following:

        \begin{itemize}

            \squeezeitems{}

            \item Ensuring the action specified by the rule has been
                registered with the framework.

            \item Checking for conflicts in the data to be extracted, e.g.\
                setting the same variable twice.

            \item Checking that the regex in the rule is valid.

            \item Some optimisation steps may also be performed during
                loading of rules, as discussed in \sectionref{parser
                efficiency}.

        \end{itemize}

    \item [Convert physical inputs to logical inputs] Each rule recognises
        a single input at a time: there is no facility for rules to consume
        more input or push unused input back onto the input stream,
        although actions may use cascaded parsing (explained in
        \sectionref{actions in architecture}) to effectively push input
        back onto the input stream.  A physical input (i.e.\ a single line
        read from the input stream) may contain multiple or partial logical
        inputs, and the framework must pre-process these to provide a
        logical input for the rules to recognise.  This simplifies the
        rules and actions considerably, at the cost of added complexity in
        the framework; during the design phase it was decided that it was
        easier to deal with the problem of parsing multiple or partial
        inputs once, rather than dealing with it in every rule and action.
        This is trivial for Postfix log files because they have a
        one-to-one mapping between physical and logical inputs; mapping
        between physical and logical inputs may be more difficult for other
        types of input.  Some input types may require pre-processing
        equivalent to parsing the physical inputs; in those cases the
        framework should behave similarly to other architectures: present
        the entire input stream at once, and discard the portion recognised
        by the successful rule.

    \item [Classify the input]  The pseudo-code above shows how rules are
        successively tried until one is found that recognises the current
        input.  That pseudo-code is very simple: there may be efficiency
        concerns (\sectionref{parser efficiency}), rule conditions
        (\sectionref{attaching conditions to rules}), or rule priorities
        (\sectionref{rules in architecture}) that complicate the process.

    \item [Invoke actions]  Once a rule has been found that recognises the
        current input, the specified action must be invoked.  The framework
        marshals the data extracted by the rule, invokes the action, and
        repeats the parsing process if cascaded parsing (see
        \sectionref{actions in architecture}) is used.

    \item [Shared storage]  Parsers commonly need to save some state
        information about the input being parsed, e.g.\ a compiler tracking
        which variables are in lexical scope as it moves from one lexical
        block to another.  The framework provides shared storage to deal
        with this and any other storage needs the actions have.  Actions
        may need to exchange data to correctly parse the input, e.g.\
        setting or clearing flags, maintaining a list of previously used
        identifiers, or ensuring at a higher level that the input being
        parsed meets requirements.

    \item [Save and load state]  The architecture can save its current
        state, i.e.\ the shared storage it provides for actions, and reload
        it later, so that information is not lost between parsing runs.
        \parsername{} uses this functionality because mails may take some
        time to deliver and thus have their log lines split between log
        files; a compiler might store data structures it builds as it
        parses different files.

    \item [Specialised support functions]  Actions may need support or
        utility functions; the framework may be a good location for support
        functions, but if another way exists to make those functions
        available to all actions it may be preferable to use it instead,
        maintaining a clear separation of concerns.

\end{description}

\section{Actions}

\label{actions in architecture}

Each action is a separate procedure written to process a particular
category of input, e.g.\ rejecting a delivery attempt.  There may be many
input variants within one input category; in general each action will
handle one input category, with each rule recognising one input variant.
It is anticipated that parsers based on this architecture will have many
actions, and each action may be specified by many rules, with the aim of
having simple rules and tightly focused actions.  An action may need to
process different input variants in slightly different ways, but large
changes in processing indicate the need for a new action and a new category
of input; if an action becomes overly complicated it starts to turn into a
monolithic parser, with too much logic contained in a single procedure.

The ability to easily add special purpose actions to deal with difficulties
and new requirements that are discovered during parser development is one
of the strengths of this architecture.  Instead of writing a single
monolithic function that processes every input and must be modified to
support any new behaviour, with all the attendant risks of adversely
affecting the existing parsing, when a new requirement arises an
independent action can be written to satisfy it.  Sometimes the new action
will require the cooperation of other actions, e.g.\ to set or check a
flag, so actions are not always self-contained, but there will still be a
far lower degree of coupling and interdependency than in a monolithic
parser.

During development of \parsername{} it became apparent that in addition to
the obvious variety in log lines there were many complications to be
overcome.  Some were the result of deficiencies in Postfix's logging, and
some of those deficiencies were rectified by later versions of Postfix,
e.g.\ identifying bounce notifications (\sectionref{identifying bounce
notifications}); others were due to the vagaries of process scheduling,
client behaviour, and administrative actions.  All were successfully
accommodated in \parsername{}: adding new actions was enough to overcome
several of the complications; others required modifications to a single
existing action to work around a difficulty; the remainder were resolved by
adapting existing actions to cooperate and exchange extra data (via the
framework), changing their behaviour as appropriate based on that extra
data.  Every architecture should aim to make the easy things easy and the
hard things possible; the successful implementation of \parsername{}
demonstrates that this architecture achieves that aim.

Actions may return a modified input line to the framework that will be
parsed as if read from the input stream, allowing for a simplified version
of cascaded parsing~\cite{cascaded-parsing}.  This powerful facility allows
several rules and actions to parse a single input, potentially simplifying
both rules and actions.  A simple example is to have one rule and action
removing comments from inputs, so that other rules and actions do not
have to handle comments at all; obviously if comment characters can be
escaped or embedded in quoted strings the implementation must be careful
not to remove those.  For some inputs this kind of pre-processing can
greatly simplify parsing, echoing the simplification provided by the
framework presenting rules and actions with logical inputs rather than
physical inputs.  A more complex example use of cascaded parsing is
evaluating simple arithmetic expressions, where sub-expressions enclosed in
parentheses must be evaluated first; cascaded parsing can be used to parse
and evaluate the sub-expressions, substituting the result into the original
expression for subsequent re-evaluation.  Actions do not need to be
specially registered with the framework or be declared in a particular way
to use cascaded parsing: actions that do not use cascaded parsing will
return nothing, those that do will simply return a string to be parsed.

\section{Rules}

\label{rules in architecture}

Rules are responsible for categorising inputs: each rule should recognise
one and only one input variant; an input category with multiple input
variants should have multiple rules, one for each variant.  Rules will
typically use a regex when recognising inputs, but other approaches may
prove useful for some applications, e.g.\ comparing fixed strings to the
input, or checking the length of the input; for the remainder of this
thesis it will be assumed that a regex is used.  Each rule must specify at
least the regex to recognise inputs and the action to invoke when
recognition is successful, but implementations are free to add any other
attributes they require; \sectionref{rules table} describes the attributes
used in \parsername{}, and some generally useful attributes will be
discussed later in this section.

Classifying inputs using the rules is simple: the first
rule to recognise the input determines the action that will be invoked;
there is no backtracking to try alternate rules; no attempt is made to pick
a \textit{best\/} rule.  \sectionref{attaching conditions to rules}
contains an example which requires that the rules are used in a specific
order to correctly parse the input, so a mechanism is needed to allow the
author of the rules to specify that ordering.  The framework should sort
the rules according to a priority attribute specified by the rule author,
allowing fine-grained control over the order that rules are used in.  The
priority attribute may be implemented as a number, or as a range of values,
e.g.\ low, medium, and high, or in a different fashion entirely if it suits
the implementation.  Rule ordering for efficiency is a separate topic that
is covered in \sectionref{rule ordering for efficiency}; overlapping rules
are discussed in \sectionref{overlapping rules in architecture}.

In \acronym{CFG} terms the rules could be described as:

$\text{\textless{}input\textgreater{}}~\mapsto{}~\text{rule-1}~|~\text{rule-2}~|~\text{rule-3}~|~\dots~|~\text{rule-n}$

This is not entirely correct because the rules are not truly context free:
rule conditions (described in \sectionref{attaching conditions to rules})
restrict which rules will be used to recognise each input, imposing a
context of sorts.

\subsection{Adding New Rules}

\label{adding new rules in architecture}

The framework issues a warning for each unparsed input, so it is clearly
evident when the ruleset needs to be augmented.  Parsing new inputs is
achieved in one of three ways:

\begin{enumerate}

    \item Modify an existing rule's regex, because the new input is part of
        an existing variant.

    \item Write a new rule that pairs an existing action with a new regex,
        adding a new variant to an existing category.

    \item Create a new category of inputs and a new action to process
        inputs from the new category, and write a new rule pairing the new
        action with a new regex.

\end{enumerate}

Decoupling the rules from the actions and framework enables other rule
management approaches to be used, e.g.\ instead of manually editing
existing rules or adding new rules, machine learning techniques could be
used to automate the process.  If this approach was taken the choice of
machine learning technique would be constrained by the size of typical data
sets (see \sectionref{parser efficiency}).  Techniques requiring the full
data set when training would be impractical; Instance Based
Learning~\cite{instance-based-learning} techniques that automatically
determine which inputs from the training set are valuable and which inputs
can be discarded might reduce the data required to a manageable size.  A
parser could also dynamically create new rules in response to certain
inputs, e.g.\ parsing a subroutine declaration could cause a rule to be
created that parses calls to that subroutine, checking that the arguments
used agree with the subroutine's signature.  These avenues of research and
development have not been pursued by the author, but the architecture
allows them to easily be undertaken independently.

\subsection{Attaching Conditions To Rules}

\label{attaching conditions to rules}

\label{rule conditions in architecture}

Rules can have conditions attached that will be evaluated by the framework
before attempting to use a rule to recognise an input: if the condition is
true the rule will be used, if not the rule will be skipped.  Conditions
can be as simple or complex as the parser requires, though naturally as the
complexity rises so too does the difficulty in understanding how everything
interacts.  The framework has to evaluate each condition, so as the
complexity of conditions increases so will the complexity of the code
required to evaluate them.  Beyond a certain level of complexity,
conditions should probably be written in a proper programming language,
e.g.\ taking advantage of dynamic languages' support for evaluating code at
run-time, or embedding a language like Lua.  If an implementation is going
to use conditions so complex that they will require a programming language
capable of implementing a Turing machine, the design may need to be
re-thought, including the decision to use this architecture --- there may
be other architectures more suitable.

Conditions that examine the input will be the easiest to understand,
because they can be understood in isolation; they do not depend on
variables set by actions or other rules.  Conditions that examine the input
can be complex if required, but simple conditions can be quite useful too,
e.g.\ every Postfix log line contains the name of the Postfix component
that produced it, so every rule used in \parsername{} has a condition
specifying the component whose log lines it recognises, reducing the number
of rules that will be used when classifying an input (see \sectionref{rules
in implementation} for details) and increasing the chance that the input
will be correctly recognised.

Conditions can also check the value of variables that have been set by
either actions or rules; it is easier to understand how a variable's value
will be used and changed if it is set by rules only rather than by actions,
because the chain of checking and setting can be followed from rule to
rule.  The downside to actions setting variables used in rule conditions is
action at a distance: understanding when a rule's condition will be true or
false requires understanding not just every other rule but also every
action.  The framework probably does not need to support a high level of
complexity and flexibility when rules are setting variables; however, if
the framework supports complex conditions, that code can probably be easily
extended to support complex variable assignments too.  The level of
complexity the framework supports when evaluating conditions and setting
variables has two costs that must be taken into account when designing a
parser: the difficulty of implementation, and the difficulty of
understanding writing correct rules; the flexibility provided by complex
conditions may be outweighed by the difficulty in understanding the
interactions between rules that use complex conditions.

An example of how rule conditions can be useful is parsing C-style
comments, which start with \texttt{/*} and end with \texttt{*/}; the start
and end tokens can be on one line, or may have many lines between them.
\Tableref{Rules to parse C-style comments} shows the regexes, conditions,
and state changes of the four rules required to parse C-style comments.
These rules are somewhat simplified, e.g.\ rules one and two will recognise
the comment start token even if it is within a quoted string, which is
incorrect; the rules could be extended with new rules to parse quoted
strings, excluding comments from within strings.

Rules one and two will be used when the parser is parsing code, not
comments: rule one recognises a comment that is contained within one line
and leaves the parser's state unchanged; rule two recognises the start of a
comment and changes the parser's state to parsing comments.  Rules three
and four will be used when the parser is parsing a comment: rule three
recognises the end of a comment and switches the parser's state back to
parsing code; rule four recognises a comment line without an end token and
keeps the parser's state unchanged.  It is important that the rules are
applied in the order listed in \tableref{Rules to parse C-style comments}
because rule 2 overlaps with rule 1, so rule 1 must be used first to
correctly parse comments that are contained within one line;
\sectionref{rules in architecture} has explained how this is achieved.
\sectionref{overlapping rules in architecture} will discuss the benefits
and difficulties of using overlapping rules.

\begin{table}[thbp]
    \caption{Rules to parse C-style comments}
    \empty{}\label{Rules to parse C-style comments}
    \begin{tabular}{llll}
        \tabletopline{}%
        No.   & Regex             & Condition             & Variable Changes    \\
        \tablemiddleline{}%
        1     & \verb!/\*.*?\*/!  & expecting == code     & expecting = code      \\
        2     & \verb!/\*.*!      & expecting == code     & expecting = comment   \\
        3     & \verb!.*\*/!      & expecting == comment  & expecting = code      \\
        4     & \verb!.*!         & expecting == comment  & expecting = comment   \\
        \tablebottomline{}%
    \end{tabular}
\end{table}

\subsection{Overlapping Rules}

\label{overlapping rules in architecture}

When adding new rules, the rule author must be aware that the new rule may
overlap with one or more existing rules, i.e.\ some inputs could be parsed
by more than one rule.  Unintentionally overlapping rules lead to
inconsistent parsing and data extraction because the first rule to
recognise the input wins, and the order in which rules are tried against
each input might change between parser invocations.  Overlapping rules are
frequently a requirement, allowing a more specific rule to recognise some
inputs and a more general rule to recognise the remainder, e.g.\ separating
\acronym{SMTP} delivery to specific sites from \acronym{SMTP} delivery to
the rest of the world.  Allowing overlapping rules simplifies both the
general rule and the more specific rule; additionally rules from different
sources can be combined with a minimum of prior cooperation or modification
required.  Overlapping rules should have a priority attribute to specify
their relative ordering; negative priorities may be useful for catchall
rules.  The architecture does not try to detect overlapping rules: that
responsibility is left to the author of the rules.

Overlapping rules can be detected by visual inspection, or a program may be
written to analyse the rule's regexes.  Traditional regexes are equivalent
in computational power to \acronym{FA} and can be converted to
\acronym{FA}, so regex overlap can be detected by finding a non-empty
intersection of two \acronym{FA}\@.  Perl 5.10 regexes~\cite{perlre} are
more powerful than traditional regexes: it is possible to match correctly
balanced brackets nested to an arbitrary depth, e.g.\
\verb!/^[^<>]*(<(?:(?>[^<>]+)|(?1))*>)[^<>]*$/!  matches
\verb!z<123<pq<>rs>j<r>ml>s!.  Matching balanced brackets requires the
regex engine to maintain state on a stack, so Perl 5.10 regexes are
equivalent in computational power to \acronym{PDA}; detecting overlap may
require calculating the intersection of two \acronym{PDA} instead of two
\acronym{FA}.  \acronym{PDA} intersection is not closed, i.e.\ the result
cannot always be implemented using a \acronym{PDA}, so intersection may be
intractable for some combinations, e.g.:
$a^{*}b^{n}c^{n}~\cap~a^{n}b^{n}c^{*}~\rightarrow~a^{n}b^{n}c^{n}$.
Detecting overlap amongst $n$ regexes requires calculating
$\frac{n\left(n-1\right)}{2}$ intersections, resulting in
$O\left(n^{2}x\right)$ complexity, where $x$ is the cost of calculating
\acronym{FA} or \acronym{PDA} intersection.  This is certainly not a task
to be performed every time a parser runs: naive detection of overlap
amongst \parsernames{} \numberOFrules{} rules would require calculating
\numberOFruleINTERSECTIONS{} intersections.

When detecting overlap any conditions attached to rules must be taken into
account, because two rules whose regexes overlap may have conditions
attached that prevent the rules overlapping.  A less naive approach to
overlap detection would first check for overlapping conditions amongst
rules, and then check for overlap between the regexes of each pair of rules
with overlapping conditions.  Depending on the conditions employed a rule
may overlap with multiple rules, and rule overlap is not transitive, e.g.\
given these three artificial conditions:

\begin{enumerate}

    % Reduce inter-item spacing because each item is a single line.
    \squeezeitems{}

    \item total $<$ 10

    \item total $>$ 20

    \item total $<$ 30

\end{enumerate}

\noindent{}The first and second conditions do not overlap, but the third
condition overlaps with both the first and second conditions.  When rules
are paired based on how their conditions overlap, the complexity of
detecting overlap amongst $n$ rules is $O\left(n^{2}y+|o|x\right)$, where:

\begin{tabular}[]{rcl}

    $x$ & = & cost of checking for overlap between two regexes    \\
    $y$ & = & cost of checking for overlap between two conditions \\
    $o$ & = & set of pairs of rules with overlapping conditions   \\

\end{tabular}

In the worst case no overlap will be found between conditions, so all pairs
or regexes must be checked for overlap and $|o|$ above will be equal to
$n^{2}$.  For this approach to be more efficient than the naive approach
$y$ must be significantly lower than $x$.  If $y$ is higher than $x$ then
the checks for overlap should be performed in the opposite order: check for
regex overlap first, then check for condition overlap only between pairs of
rules with overlapping regexes.

Once conditions pass a certain level of complexity determining if two
conditions overlap becomes intractable, because it requires so much
knowledge of other rules and possibly even actions.  E.g.\ given two rules
with conditions \verb!verbose == true! and \verb!silent == true!, logically
these rules should not overlap, yet there is nothing to stop both variables
being set to true by one or more rules or actions.  If variables used in
conditions can be set by actions, determining if two conditions overlap is
impossible: the Halting Problem show that it is impossible for one program
to answer questions about another program's behaviour if the inquiring
program is implemented on a computational machine of the equal or lesser
power to the program being inquired into.

\subsection{Pathological Rules}

It is possible to define pathological regexes, which fall into two main
categories: regexes that match inputs they should not, and regexes that
consume excessive amounts of CPU time during matching.  Defining a regex
that matches inputs it should not is trivial: \verb!/^/! matches the start
of every input.  This regex would be found by a tool that detects
overlapping rules, and would easily be noticed by visual inspection, but
more complex regexes would be harder to spot.  Regexes that match more than
they should are a problem not because of excessive resource usage, but
because they may prevent the correct rule from recognising the input.  If
an adaptive ordering system is used that prioritises rules that frequently
recognise inputs (see \sectionref{rule ordering for efficiency}), then a
rule with a regex that matches inputs it should not may be promoted up
through the list, displacing an increasing number of correct rules as it
moves.

Usually excessive CPU time is consumed when a regex fails to match, and the
regex engine backtracks many times because of alteration or nested
quantifiers; successful matching is generally quite fast with such regexes,
so problematic regexes may go unnoticed for some time.  E.g.\ with most
regex engines matching double quoted strings with \verb!"([^"\\]+|\\.)*"!
is very fast when there is a match, but when the match fails its
computational complexity for a string of length $n$ is $O(2^{n})$;
see~\cite{mastering-regular-expressions} for in-depth discussion of nested
quantifiers, backtracking, alteration, and capturing groups.  Pathological
regexes that consume excessive CPU time can be difficult to detect, both by
visual inspection and by machine inspection, but if a regex is converted to
a \acronym{FA} or the internal representation used by the regex engine, it
may be possible to determine if nested quantifiers or other troublesome
constructs are present.  Modern regex engines have addressed many of these
problems, e.g.\ the regex to match double quoted strings given above fails
immediately with Perl 5.10, regardless of the input length, because the
regex engine looks for both of the required double quotes first.  Similarly
Perl 5.10's regex engine optimises alterations starting with literal text
into a trie, which has matching time proportional to the length of the
alternatives, rather than the number of alternatives.  Perl regexes can use
\verb!(?>pattern)!, which matches \verb!pattern! the first time the regex
engine passes over it, and does not change what it originally matched if
the regex engine backtracks over it, so troublesome regexes can use this to
reduce the impact of backtracking; Prolog users will notice a similarity to
the \verb'!' (cut) operator.  A presentation showing some of Perl 5.10's
new regex features is available at
\urlLastChecked{http://www.regex-engineer.org/slides/perl510_regex.html}{2009/03/03}.

Conditions can vary in complexity from simple equality through to a
Turing-equivalent language, so enumerating pathological conditions is
difficult if not pointless.  Conditions that examine the input using
regexes can suffer from the same problems as rule regexes; conditions that
check variables in uncomplicated ways may exhibit unexpected or incorrect
behaviour, but are unlikely to exhibit pathological behaviour.  Deciding if
a more complex condition's behaviour is pathological or simply a bug is
difficult and to some extent is a matter of opinion.  When this
architecture has received more widespread usage consensus should be reached
on the topic of pathological conditions.

\section{Summary}

This chapter has presented the parser architecture developed for this
project.  It started with a high level view of the architecture, describing
how it achieves its design aim of being easily extensible for users, and
the advantages that being easily extensible brings to parser authors too.
The three main components of the architecture were documented in detail,
explaining each component's responsibilities and the functionality it
provides, plus any difficulties associated with the components.  The
framework provides several support functions and manages the parsing
process, enabling simple rules and actions to be written.  The actions are
simple to understand, because the architecture does not impose any
structure or requirements upon them: parser authors are free to do anything
they want within an action.  The architecture's support for cascaded
parsing was described in the actions section, with an example to illustrate
how it can be useful for general parsing.  The rules section was the
longest, because although the rules appear conceptually simple ---
recognise an input and specify the action that handles that input --- they
have subtle behaviour that needs to be clearly explained.  When extending a
ruleset a decision needs to be taken about whether the input should be
recognised by extending an existing rule, by adding a new rule to an
existing rule category, or by adding a new rule category, rule, and action.
Rules can have conditions attached to them, restricting the set of rules
used to recognise an input; the complexity of the conditions used greatly
influences the difficulty of writing a correct ruleset or understanding and
extending an existing ruleset.  Overlapping rules are frequently a
requirement in a parser, and their use can greatly simplify some rules, but
they can be a source of bugs because they can recognise inputs
unexpectedly.  The framework does not try to detect overlapping rules,
because overlap amongst rules may be valid and is quite often intentional;
that responsibility falls to the author of the rules.  The difficulty
involved in detecting overlap is proportional to the complexity of a
ruleset's regexes and conditions, and may be possible for a human yet
intractable or impossible for a program.  The rules section finishes with a
discussion of pathological rules, concentrating on pathological regexes.

\chapter{Postfix Parser Implementation}

\label{Postfix Parser Implementation}

XXX WRITE AN INTRODUCTION\@.

XXX UPDATE REFERENCES WHEN ARCHITECTURE AND IMPLEMENTATION ARE MORE FLESHED
OUT\@.

XXX FIGURE OUT WHEN TO USE ACTIONS AND WHEN TO USE ALGORITHM, ETC\@.

XXX THIS MIGHT BE USEFUL SOMEWHERE\@:

XXX SHOULD I SAY ``cleaned up'' EVERYWHERE OR NOT\@?

XXX WHEN FINISHED, CHECK ALL REFERENCES TO ENSURE THEY POINT AT THE CORRECT
CHAPTER\@.

Whereas the rules are quite simple and each rule is completely independent
of the other rules, the algorithm is significantly more complicated and
highly internally interdependent.  The algorithm deals with all the
complications of parsing, the eccentricities and oddities of Postfix log
files, presenting the resulting data in a normalised, simple to use
representation.  The algorithm's task is to follow the journey each mail
takes through Postfix, combining the data extracted by rules into a
coherent whole, saving it in a useful and consistent form, and performing
housekeeping duties.

Please refer to \sectionref{parser design} for a discussion of why the
rules, actions, and framework have been separated in the parser's design.
In this section algorithm can be taken to mean the combination of framework
and actions.

The intermingling of log entries from different mails immediately rules out
the possibility of handling each mail in isolation; the parser must be able
to handle multiple mails in parallel, each potentially at a different stage
in its journey, without any interference between mails --- except in the
minority of cases where intra-mail interference is required.  The best way
to implement this is to maintain state information for every unfinished
mail and manipulate the appropriate mail correctly for each log line
encountered.

There is a similarity between this design and the event-driven programming
paradigm commonly used in GUI programs, where one part of the program
responds to events (mouse clicks in a GUI program, log lines being matched
in the parser) and invokes the correct action.


\section{Assumptions}

\parsername{} makes a small number of (hopefully safe and reasonable)
assumptions:

\begin{itemize}

    \item The log files are whole and complete: nothing has been removed,
        either deliberately or accidentally (e.g.\ log file rotation gone
        awry, file system filling up, logging system unable to cope with
        the volume of log messages).  On a well run system it is extremely
        unlikely that any of these problems will arise, though it is of
        course possible, particularly when suffering a deluge of spam or a
        mail loop.

    \item Postfix logs enough information to make it possible to accurately
        reconstruct the actions it has taken.  There are several heuristics
        used when parsing; see
        \sectionref{identifying-bounce-notifications},
        \sectionref{aborted-delivery-attempts} and \sectionref{pickup
        logging after cleanup} for details.

    \item The Postfix queue has not been tampered with, causing unexplained
        appearance or disappearance of mail.  This may happen if the
        administrator deletes mail from the queue without using
        \daemon{postsuper}, or if there is filesystem corruption.

\end{itemize}

In some ways this task is similar to reverse engineering or replicating a
black box program based solely on its inputs and outputs.  There are
advantages to treating Postfix as a black box during parser development:

\begin{itemize}

    \item Reading and understanding the source code would require a
        significant investment of time: Postfix 2.5.5 has 17MB of source
        code.  

    \item The parser is developed using real world log files rather than
        the idealised log files someone would naturally envisage reading
        the source code.

    \item The source code cannot accurately communicate the variety of
        orderings in which log lines are written to the log file, as
        process scheduling independently interferes with logging and other
        processing.

    \item The parser acts as a second source of information about Postfix's
        operation, using information gathered from empirical evidence.  A
        separate project could compare the empirical knowledge inherent in
        the parsing algorithm with the documentation and source code of
        Postfix to see how closely the two agree.

\end{itemize}



\section{Database Schema}
\label{database schema}

XXX ADD A DIAGRAM OF THE DATABASE SCHEMA\@.

The database is an integral part of the parser presented here: it stores
the rules and the data gleaned by applying those rules to Postfix log
files.  Understanding the database schema is important in understanding the
actions of the parser, and essential to developing further applications
which utilise the data gathered.

The database schema can be conceptually divided in two: the rules used to
parse log files, and the data saved from the parsing of log files.  Rules
have the fields required to parse the log lines, extract data to be saved,
and specify the action to be executed; they also have several fields that
aid the user in understanding the meaning of the log lines parsed by each
rule.  The rules are described in detail in \sectionref{rules in
implementation} but the fields are covered in \sectionref{rule attributes}.

The data saved from parsing the log files is also divided into two tables
as described below: connections and results.  The connections table
contains a row for every mail accepted and every connection where there was
a rejection; the individual fields will be described in
\sectionref{connections table}.  There will be at least one row in the
results table for each row in the connections table; the fields will be
covered in detail in \sectionref{results table}.

An important but easily overlooked benefit of storing the rules in the
database is the link between rules and results: if more information is
required when examining a result, the rule that produced the result is
available for inspection: each result references the rule that created it.
There is no ambiguity about which rule resulted in a particular result,
eliminating one potential source of confusion.

A clear, comprehensible schema is essential when using the extracted data;
it is more important when using the data than when storing it, because
storing the data is a write-once operation, whereas utilising the data
requires frequent searching, sorting, and manipulation of the data to
produce customised reports or statistics.

\subsection{Using A Database To Provide An Application Programming Interface}

\label{database as API}

The database populated by this program provides a simple interface to
Postfix log files.  Although the interface is a database schema rather than
a set of functions provided by a library, it performs the same function as
any other \acronym{API}: it provides a stable interface, allowing code on
either side of the interface to be changed without adverse effects, as long
as the interface is maintained.  The parser can be improved to handle
additional complications, support earlier or later releases of Postfix,
bugs can be fixed or limitations removed.  Programs that use the database
can range from the simple examples in \sectionref{motivation} to far more
complex data mining tools.

Using a database simplifies writing programs that need to interact with the
data in several ways:

\begin{enumerate}

    \item Facilities are provided to access databases from most programming
        languages, allowing a developer to access the data gathered using
        their preferred programming language, rather than being restricted
        to the language the parser is written in.  It is often possible to
        write an interface layer allowing code written in one language to
        be used in another language, but this greatly increases the effort
        required to use the parser.

    \item Databases provide complex querying and sorting functionality to
        the user without requiring large amounts of programming.  All
        databases provide a program, of varying complexity and
        sophistication, that can be used for ad hoc queries with minimal
        investment of time.

    \item Databases are easily extensible, e.g.:

        \begin{itemize}

            \item New columns can be added to the tables used by the
                program, using DEFAULT clauses or TRIGGERS to populate
                them.

            \item A VIEW gives a custom arrangement of data with minimal
                effort.

            \item If the database supports it, access can be granted on a
                fine-grained basis, e.g.\ allowing the finance department
                to produce invoices, the helpdesk to run limited queries as
                part of dealing with support calls, and the administrators
                to have full access to the data.

            \item Triggers can be written to perform actions when certain
                events occur.  In pseudo-\acronym{SQL}\@:

\begin{verbatim}
CREATE TRIGGER ON INSERT INTO results
    WHERE sender = 'boss@example.com'
        AND postfix_action = 'REJECTED'
    SEND PANIC EMAIL TO 'postmaster@example.com';
\end{verbatim}

            \item Other tables can be added to the database, e.g.\ to cache
                historical, summary, or computed data.

        \end{itemize}


    \item \acronym{SQL} is reasonably standard and many people will already
        be familiar with it; for those unfamiliar with it there are lots of
        readily available resources from which to learn (a good
        introduction to \acronym{SQL} can be found at
        \urlLastChecked{http://philip.greenspun.com/sql/}{2009/02/23},
        others are
        \urlLastChecked{http://www.w3schools.com/sql/default.asp}{2009/02/23},
        \urlLastChecked{http://sqlcourse.com/}{2009/02/23}).  Although
        every vendor implements a different dialect of \acronym{SQL}, the
        basics are the same everywhere (analogous to the overall
        similarities and minor differences amongst Irish English, British
        English, American English, and Australian English).  Depending on
        the database in use there may be tools available that reduce or
        remove the requirement to know \acronym{SQL}; several are available
        for \gls{SQLite}, the default database used by the parser
        implementation.

\end{enumerate}

Storing the results in a database will also increase the efficiency of
using those results, because the log files need only be parsed once rather
than each time the data is used; indeed the results may be used by someone
with no access to the original log files.

\subsection{Rules Table}

\label{rule attributes}

Rules are discussed in detail in \sectionref{rules in implementation}, but
the structure of the rules table is covered here.  Rules are created by the
user, not the parser, and will not be modified by the parser (except for
the hits and hits\_total fields).  Rules parse the individual log lines,
extracting data to be saved in the connections and results tables, and
specifying the action to take for that log line.

The following attributes are defined for each rule:

\begin{eqlist}

    \item [id] A unique identifier for each rule that other tables can use
        when referring to a specific rule.

    \item [name] A short name for the rule.

    \item [description] Something must have occurred to cause Postfix to
        log each line (e.g.\ a remote client connecting causes a connection
        line to be logged).  This field describes the event causing the log
        lines this rule matches.

    \item [restriction\_name] The restriction that caused the mail to be
        rejected.  Only applicable to rules that have an action of
        \texttt{REJECTION}, other rules should have an empty string.

    \item [postfix\_action] The action Postfix must have taken to generate
        this log line; the possible actions are described later in
        \sectionref{list of postfix actions in implementation}.

    \item [program] The program (\daemon{smtpd}, \daemon{qmgr}, etc.) whose
        log lines the rule applies to.  This avoids needlessly trying rules
        that will not match the log line, or worse, might match
        unintentionally.  Rules whose program is \texttt{*} will be tried
        against any log lines that are not parsed by program specific
        rules.

    \item [regex] The regex to match the log line against.  The regex will
        have several keywords expanded when the rules are loaded: this
        simplifies reading and writing rules; avoids needless repetition of
        complex regex components; allows the components to be corrected
        and/or improved in one location; enables automatic extraction and
        saving of data; and makes each regex largely self-documenting.  For
        efficiency the keywords are expanded and every rule's regex is
        compiled before attempting to parse the log file --- otherwise each
        regex would be recompiled each time it was used, resulting in a
        large, data dependent slowdown.  Efficiency concerns are discussed
        in \sectionref{parser efficiency}, with caching compiled regexes
        covered in \sectionref{Caching compiled regexes}.

    \item [connection\_data] Sometimes rules need to save data that is not
        present in the log line: e.g.\ setting \texttt{client\_ip} when a
        mail is being delivered to another server.  The format is:
        \newline{} \tab{} \texttt{ client\_hostname = localhost,}
        \newline{} \tab{} \tab{} \texttt{client\_ip
        = 127.0.0.1} \newline{} i.e.\ semi-colon or comma separated
        assignment statements, with the variable name on the left and the
        value on the right hand side.\footnote{Commas and semi-colons
        cannot be escaped and thus cannot be used.  This feature is
        intended for use with small amounts of data rather than large
        amounts in any one rule, so dealing with escape sequences was
        deemed unnecessary.}  Any field in the connections table can be set
        in this way.

    \item [result\_data] The result table equivalent of
        \texttt{connection\_data}.

    \item [action] The action that will be invoked when this rule matches a
        log line; a full list of actions and the parameters they are
        invoked with can be found in \sectionref{actions in detail in
        implementation}.

    \item [hits] This counter is maintained for every rule and incremented
        each time the rule successfully matches.  At the start of each run
        the parser sorts the rules in descending order of hits, and at the
        end of the run it updates every rule's hits field in the database.
        Assuming that the distribution of log lines is reasonably
        consistent across log files, rules matching more commonly occurring
        log lines will be tried before rules matching less commonly
        occurring log lines, reducing the parser's execution time.  Rule
        ordering for efficiency is discussed in \sectionref{rule ordering
        for efficiency}.

    \item [hits\_total] The total number of hits for this rule over all
        runs of the parser.

    \item [priority] This is the user-configurable companion to hits: when
        the list of rules is sorted, priority overrides hits.  This allows
        more specific rules to take precedence over more general rules
        (described in \sectionref{overlapping rules in architecture}).

%    \item [cluster\_group] A reference to the \texttt{cluster\_group}
%        table.  That table is used by the Decision Tree algorithm
%        described in a separate document.

\end{eqlist}

\subsection{Connections Table}

\label{connections table}

Every accepted mail and every connection where there was a rejection will
have a single entry in the connections table containing the following
fields:

\begin{eqlist}

    \item [id] This field uniquely identifies the row.

    \item [server\_ip] The \acronym{IP} address (IPv4 or IPv6) of the
        server: the local address when receiving mail, the remote address
        when sending mail.

    \item [server\_hostname] The hostname of the server, it will be
        \texttt{unknown} if the \acronym{IP} address could not be resolved
        to a hostname via DNS\@.

    \item [client\_ip] The client \acronym{IP} address (IPv4 or IPv6): the
        remote address when receiving mail, the local address when sending
        mail.

    \item [client\_hostname] The hostname of the client, it will be
        \texttt{unknown} if the \acronym{IP} address could not be resolved
        to a hostname via DNS\@.

    \item [helo] The hostname used in the HELO command.  The HELO hostname
        occasionally changes during a connection, presumably because spam
        or virus senders think it is a good idea.  By default Postfix only
        logs the HELO hostname when it rejects an \acronym{SMTP} command,
        but it is quite simple to rectify this, as described in
        \sectionref{logging helo}.

    \item [queueid] The queueid of the mail if the connection represents an
        accepted mail, or \texttt{NOQUEUE} otherwise.

    \item [start] The timestamp of the first log line, in seconds since the
        epoch.\glsadd{Epoch}

    \item [end] The timestamp of the last log line, in seconds since the
        epoch.

\end{eqlist}

\subsection{Results Table}

\label{results table}

Every log line a row in the results table, unless the
\texttt{postfix\_action} field of the rule which parsed it was INFO or
IGNORE\@.  Each row in the results table is associated with a single
connection, and there may be many results per connection.

\begin{eqlist}

    \item [connection\_id] The id of the row in the connections table this
        result is associated with.

    \item [rule\_id] The id of the entry in the rules table that matched
        the log line and created this result.

    \item [id] A unique identifier for this result.

    \item [warning] Postfix can be configured to log a warning instead of
        enforcing a restriction that would reject an \acronym{SMTP} command
        --- a mechanism that is quite useful for testing new restrictions.
        This field will be 1 if the log line parsed was a warning rather
        than a real rejection, or 0 for a real rejection.  This field
        should be ignored if the result is not a rejection, i.e.\ the
        action field of the associated rule is not \texttt{REJECTION}.

    \item [smtp\_code] The \acronym{SMTP} code associated with the log
        line.  In general an \acronym{SMTP} code is only present in a log
        line for a rejection or final delivery; results initially missing
        an \acronym{SMTP} code will duplicate the \acronym{SMTP} code of
        other results in that connection.  Some final delivery log lines do
        not contain an \acronym{SMTP} code: in those cases the code is
        specified by the rule, based on the success or failure represented
        by the log line.

    \item [sender] The sender's email address.  Because multiple mails may
        be delivered during one connection, there may be different sender
        addresses in the results for one connection; however there should
        not be different sender addresses in the results for one mail.
        Mails sent over the same connection can be distinguished by their
        queueid.

    \item [recipient] The recipient address; there may be multiple
        recipient addresses per mail or connection.

    \item [size] The size of the mail; it will only be present for
        delivered mails.

    \item [message\_id] The message-id of the accepted mail, or
        \texttt{NULL} if no mail was accepted.

    \item [data] A field available for anything not covered by other
        fields, e.g.\ the rejection message from a \acronym{DNSBL}\@.

    \item [timestamp] The time at which the log line was logged, in seconds
        since the epoch.\glsadd{Epoch}

\end{eqlist}



\section{Rules}

\label{rules in implementation}

XXX NEED AN INTRODUCTION\@.


\subsection{Overlapping Rules}

\label{overlapping rules in architecture}

XXX IS THERE ANY PREVIOUS RESEARCH IN THIS AREA\@?

XXX DFA COMPARISON\@.

The parser does not try to detect overlapping rules; that responsibility is
left to the author of the rules.  Unintentionally overlapping rules lead to
inconsistent parsing and data extraction because the order in which rules
are tried against each line may change between log files, and the first
matching rule wins.  Overlapping rules are frequently a requirement,
allowing a more specific rule to match some log lines and a more general
rule to match the majority, e.g.\ separating \acronym{SMTP} delivery to
specific sites from \acronym{SMTP} delivery to the rest of the world.  The
algorithm provides a mechanism for ordering overlapping rules: the priority
field in each rule (defaults to zero).  Negative priorities may be useful
for catchall rules.

Detecting overlapping rules is difficult, but the following approaches may
be helpful:

\begin{itemize}

    \item Sort by regex and visually inspect the list, e.g.\ with
        \acronym{SQL} similar to: \textbf{select regex from rules order by
        regex;}

    \item Compare the results of parsing using sorted, shuffled, and
        reversed rules (described in \sectionref{rule ordering for
        efficiency}).  Parse several log files using optimal ordering, then
        dump a textual representation of the rules, connections, and
        results tables.  Repeat with shuffled and reversed ordering,
        starting with a fresh database.  If there are no overlapping rules
        the tables from each run will be identical; differences indicate
        overlapping rules.  The rules that overlap can be determined by
        examining the differences in the tables: each result contains a
        reference to the rule which created it, if the references differ
        between runs the two rules referenced in the differing results
        overlap.  Unfortunately this method cannot prove the absence of
        overlapping rules; it can detect overlapping rules, but only if
        there are log lines in the log files that are matched by more than
        one rule.

\end{itemize}

\subsection{Example Rule}

\label{example rule in implementation}

This example rule matches the message logged by Postfix when it rejects
mail from a sender address because the domain in the sender address does
not have an A, AAAA, or MX DNS entry, i.e.\ mail could not be delivered
to the sender's address (for full details
see~\cite{reject-unknown-sender-domain}).

Example log line matched by this rule:

% RFC 3330 says that 192.0.2.0/24 is reserved for example use.

\begin{verbatim}
NOQUEUE: reject: RCPT from example.com[192.0.2.1]: 550
  <foo@example.com>: Sender address rejected: Domain not found;
  from=<foo@example.com> to=<info@example.net>
  proto=SMTP helo=<smtp.example.com>
\end{verbatim}

% do not reformat this!
\begin{tabular}[]{ll}

\textbf{Field}      & \textbf{Value}                                    \\
name                & Unknown sender domain                             \\
description         & We do not accept mail from unknown domains        \\
restriction\_name   & reject\_unknown\_sender\_domain                   \\
postfix\_action     & REJECTED                                          \\
program             & \daemon{smtpd}                                    \\
regex               & \verb!^__RESTRICTION_START__ <(__SENDER__)>: !    \\
                    & \verb!Sender address rejected: Domain not found;! \\
                    & \verb!from=<\k<sender>> to=<(__RECIPIENT__)> !    \\
                    & \verb!proto=E?SMTP helo=<(__HELO__)>$!            \\
result\_data        &                                                   \\
connection\_data    &                                                   \\
action              & DELIVERY\_REJECTED                                \\
hits                & 0                                                 \\
hits\_total         & 0                                                 \\
priority            & 0                                                 \\
%cluster\_group      & 400                                               \\

\end{tabular}

\vspace{1em}

Fields matched by the regex will be automatically saved to the results and
connections tables.

\begin{description}

    \item [name, description, restriction\_name, and postfix\_action:] are
        not \newline{} used by the parser, they serve to document the rule
        for the user's benefit.

    \item [program and regex:] If the program in the rule equals the
        program in the log line, a match using the rule's regex will be
        attempted against the log line; if the match is successful the
        action will be executed, if not the next rule will be tried.  If
        the program-specific rules do not match the log line, the parser
        will fall back to generic rules; if those rules are unsuccessful a
        warning will be issued.

    \item [action:] The action to be executed if the regex matches
        successfully.  See \sectionref{actions in detail in implementation}
        for details of the actions implemented by the parser,
        \sectionref{actions in architecture} for the role of actions in the
        parser architecture.

    \item [result\_data and connection\_data:] are used to provide XXX

    \item [hits, hits\_total, and priority:] hits and priority are used in
        ordering the rules (see \sectionref{rule ordering for efficiency});
        hits is set to the number of successful matches at the end of the
        parsing run; hits\_total is the sum of hits over every parsing run,
        but is otherwise unused by the parser.

%    \item [cluster\_group] The cluster\_group attribute is used by the
%        Decision Tree algorithm described in a separate document; the
%        parser does not use it in any way.

\end{description}


\subsection{Creating New Rules}

\label{creating new rules in implementation}

The log files produced by Postfix differ from installation to installation,
because administrators have the freedom to choose the subset of available
restrictions which suits their needs, including using different
\acronym{DNSBL} services, policy servers, or custom rejection messages.  To
facilitate parsing new log lines, the parser's design separates parsing
rules from parsing actions: adding new actions can be difficult, but adding
new rules to parse new rejection messages is trivial, and occurs much more
frequently.  A program is provided to ease the process of creating new
rules from unparsed log lines, based on the algorithm developed by Risto
Vaarandi for his \acronym{SLCT}~\cite{slct-paper}.  The differences between
the two algorithms will be outlined as part of the general explanation
below.

The core of the \acronym{SLCT} algorithm is quite simple: log lines are
generally created by substituting variable words into a fixed pattern, and
analysis of the frequency with which each word occurs can be used to
determine whether the word is variable or part of the fixed pattern.  This
classification can be used to group similar log lines and generate a regex
to match each group of log lines.

There are 4 steps in the algorithm:

\begin{description}

    \item [Pre-process the file]  The new algorithm leverages the knowledge
        gained when writing rules and performs a large number of
        substitutions on the input log lines, replacing commonly occurring
        variable terms (e.g.\ email addresses, \acronym{IP} addresses, the
        standard start of rejection messages, etc.) with regex keywords
        that the parser will expand when it loads the rule (see the
        regex entry in \sectionref{rule attributes}).  The purpose of
        this step is to utilise existing knowledge to create more accurate
        regexes; it replaces a large number of variable words with fixed
        words, improving the subsequent classification of words as fixed or
        variable.  The new log lines are written to a temporary file, which
        all subsequent stages use instead of the original input file.

        In the original algorithm the purpose of the preprocessing stage
        was to reduce the memory consumption of the program.  In the first
        pass it generated a hash~\cite{hash-functions} from a small range
        of values for each word of each log line, incrementing a counter
        for each hash.  The counters will be used in the next pass to
        filter out words: if the word's hash does not have a high
        frequency, the word itself cannot have a high frequency, so there
        is no need to maintain a counter for it, reducing the number of
        counters required and thus the program's memory consumption.

    \item [Calculate word frequencies]  The position of words within a log
        line is important: a word occurring in two log lines does not
        indicate similarity unless it occupies the same position within
        both log lines.\footnote{If a variable term within a line contains
        spaces, it will appear to the algorithm as two or more words rather
        than one.  This will alter the position of subsequent words, so a
        word occurring in different positions in two log lines
        \textit{may\/} indicate similarity, but the algorithm does not
        attempt to deal with this possibility.}  The algorithm maintains a
        counter for each \texttt{(word, word's position within the log
        line)} tuple, incrementing it each time that word occurs in that
        position.

        The original algorithm only maintains counters for words whose hash
        result from the previous step has a high frequency; this reduces
        the number of counters maintained by the algorithm, reducing the
        memory requirements of the algorithm at a cost of increased CPU
        usage.
        
        As time goes on the amount of memory typically available increases
        and the requirement to reduce memory requirements decreases, hence
        the modified algorithm omits the hashing step and maintains
        counters for all tokens. In addition most of the infrequently
        occurring words will have been substituted with keywords during the
        first step, vastly reducing the number of tuples to maintain
        counters for.

    \item [Classify words based on their frequency]  The frequency of every
        tuple \texttt{(word, word's position within the log line)} is
        checked: if its frequency is greater than the threshold supplied by
        the user (1\% of all log lines is generally a good starting point)
        it is classified as a fixed word, otherwise it is classified as a
        variable term.  If a variable term appears sufficiently often it
        will be misclassified as a fixed term, but that should be noticed
        by the user when reviewing the new regexes.  Variable terms are
        replaced by \texttt{.+}, which means to match zero or more of any
        character.  XXX SHOULD I USE \verb![\S]+! INSTEAD\@?

    \item [Build regexes]  The words are reassembled to produce a regex
        matching the log line, and a counter is maintained for each
        regex.  Contiguous sequences of \texttt{.+} in the newly
        reassembled regexes are collapsed to a single \texttt{.+}; any
        resulting duplicate regexes are combined, and their counters
        added together.  If a regex's counter is lower than the
        threshold supplied by the user the regex is discarded; this
        second threshold is independent of the threshold used to
        differentiate between fixed and variable words, but once again 1\%
        of log lines is a good starting point.  The new regexes are
        displayed for the user to add to the database, either as new rules
        or merged into the regexes of existing rules; the counter for
        each regex is also displayed, giving the user an indication of
        how many of the log lines that regex should match.  Discarding
        regexes will result in some of the log lines not being matched;
        when the rules have been augmented with the new regexes, the
        original log files should be parsed again, and any remaining
        unparsed log lines used as input to this utility.

\end{description}

XXX MERGE logs2regexes WITH check-regexes.

A second utility is also provided that reads a list of new regexes and
the input given to the first utility.  It tries to match each input log
line against each regex, counting the number of log lines that match
each regex, warning the user if an input log line is matched by more
than one regex, and additionally warning if an input log line is not
matched by any regex.  It displays a summary of how many input log lines
each regex matched, comparing it to the expected number of matches; this
provides the user with an easy method of checking if the regexes
produced by the first utility are correctly matching the input log lines
they are based upon.  A future version of this utility will also group
input log lines by regex, so the user can tweak the regexes if
required.

These utilities are not intended to create perfect regexes, but they
greatly reduce the effort required to parse new or different log lines.

\section{Actions}

\label{actions in implementation}

XXX INTRODUCTION\@.

The actions reconstruct the journey a mail takes through Postfix.  Details
of the actions available in the Postfix parser can be found in
\sectionref{actions in detail in implementation}, and \sectionref{adding
new actions in implementation} describes the process of adding new actions.

In the Postfix log parser developed for this project there are
\numberOFactions{} actions and \numberOFrules{} rules, with an uneven
distribution of rules to actions as shown in \graphref{Distribution of
rules per action}.  Unsurprisingly, the action with the most associated
rules is \texttt{DELIVERY\_REJECTED}, the action that handles Postfix
rejecting a mail delivery attempt; it is followed by \texttt{SAVE\_DATA},
the action responsible for handling informative log lines, supplementing
the data gathered from other log lines.  The third most common action is,
perhaps surprisingly, \texttt{UNINTERESTING}: this action does nothing when
executed, allowing uninteresting log lines to be parsed without causing any
effects (it does not imply that the input is ungrammatical or unparsed).
Generally rules specifying the \texttt{UNINTERESTING} action parse log
lines that are not associated with a specific mail, e.g.\ notices about
configuration files changing.  The remaining actions have only one or two
associated rules: some actions are required to address a deficiency in the
log files, or a complication that arises during parsing;  other actions
will only ever have one log line variant, e.g.\ all log lines showing that
a remote client has connected are matched by a single rule and handled by
the \texttt{CONNECT} action.

Using the \texttt{CONNECT} action as an example: it creates a new data
structure in memory for the new client connection, saving the data
extracted by the rule into it; this data will be entered into the database
when the mail delivery attempt is complete.  If a data structure already
exists for the new connection it is treated as a symptom of a bug, and the
action issues a warning containing the full contents of the existing data
structure, plus the log line that has just been parsed.

Each action is passed the same arguments:

%\begin{eqlist}[\def\makelabel#1{\bfseries#1:}]

XXX IMPROVE HOW THIS LOOKS

\begin{eqlist}

    \item [line] The log line, separated into fields:

        \begin{eqlist}

            \item [timestamp] The time the line was logged at.

            \item [host] The hostname of the server that logged the line.

            \item [program] The name of the program that logged the line.

            \item [pid] The \acronym{pid} of the program that logged the
                line.

            \item [text] The remainder of the line.

        \end{eqlist}

    \item [rule] The matching rule.

    \item [matches] The data extracted from the line by the rule's regex.

\end{eqlist}


\subsection{Actions in Detail}

\label{actions in detail in implementation}

Each action saves all data extracted by the rule's regex, if there is
enough information in the log line to identify the correct connection to
save the data to.

XXX MAKE ALL ACTION DESCRIPTIONS FLOW BETTER\@.  

\begin{description}

    \item [BOUNCE] Postfix 2.3 and subsequent versions log the creation of
        bounce messages.  This action creates a new mail if necessary; if
        the mail already exists the unknown origin flag will be removed.
        The action also marks the mail as a bounce notification.  To deal
        with complication \sectionref{Bounce notification mails delivered
        before their creation is logged} this action checks a cache of
        recent bounce mails to avoid creating bogus bounce mails when log
        lines are out of order.

    \item [CLONE] Multiple mails may be accepted on a single connection, so
        each time a mail is accepted the connection's state table entry
        must be cloned and saved in the state tables under its queueid; if
        the original data structure was used then second and subsequent
        mails would overwrite one another's data.

    \item [COMMIT] Enter the data from the mail into the database. Entry
        will be postponed if the mail is a child waiting to be tracked.
        Once entered, the mail will be deleted from the state tables.
        Deletion will be postponed if the mail is the parent of re-injected
        mail (\sectionref{Re-injected mails}).

    \item [CONNECT] Handle a remote client connecting: create a new state
        table entry (indexed by \daemon{smtpd} \acronym{pid}) and save both
        the client hostname and \acronym{IP} address.

    \item [DELETE] Deals with mail deleted using Postfix's administrative
        command \daemon{postsuper}.  This action adds a dummy recipient
        address if required, then invokes the COMMIT action to handle
        adding the mail to the database.  The complication this action
        deals with is described fully in \sectionref{Mail deleted before
        delivery is attempted}.  

    \item [DELIVERY\_REJECTED] Deals with Postfix rejecting an
        \acronym{SMTP} command from the remote client: log the rejection
        with a mail if there is a queueid in the log line, or with the
        connection if not.

    \item [DISCONNECT] Deal with the remote client disconnecting: enter the
        connection in the database, perform any required cleanup, and
        delete the connection from the state tables.  This action deals
        with aborted delivery attempts
        (\sectionref{aborted-delivery-attempts}).

    \item [EXPIRY] If Postfix has not managed to deliver a mail after
        trying for five days it will give up and return the mail to the
        sender.  When this happens the mail will not have a combination of
        Postfix programs which passes the valid combinations check (see
        \sectionref{out of order log lines}).  To ensure that the mail can
        be committed the EXPIRY action sets a flag marking the mail as
        expired; the flag later causes the valid combinations check to be
        skipped, so the mail will be committed.

    \item [MAIL\_QUEUED] This action represents Postfix picking a mail from
        the queue to deliver. This action is used for both \daemon{qmgr}
        and \daemon{cleanup} as it needs to deal with out of order log
        lines; see \sectionref{discarding cleanup log lines} for details.
        XXX EXPLAIN THIS BETTER\@.

    \item [MAIL\_TOO\_LARGE] When a client tries to send a mail larger than
        the local server accepts it will be discarded and the client
        informed of the problem.  This is handled in exactly the same way
        as the TIMEOUT action, further details are given there.

    \item [PICKUP] The PICKUP action corresponds to the \daemon{pickup}
        service dealing with a locally submitted mail.  Out of order log
        entries may have caused the state table entry to already exist (see
        \sectionref{pickup logging after cleanup}); otherwise it is
        created.  The data extracted from the log line is then saved to the
        state table entry.

    \item [POSTFIX\_RELOAD] When Postfix stops or reloads its configuration
        it kills all \daemon{smtpd} processes,\footnote{Possibly other
        programs are killed also, but the parser is only affected by and
        interested in \daemon{smtpd} processes exiting.} requiring any
        active connections to be cleaned up, entered in the database, and
        deleted from the state tables.

    \item [SAVE\_DATA] Find the correct mail based on the queueid in
        the log line, and save the data extracted by the regex to it.

    \item [SMTPD\_DIED] Sometimes a \daemon{smtpd} dies or exits
        unsuccessfully; the active connection for that \daemon{smtpd} must
        be cleaned up, entered in the database, and deleted from the state
        tables.

    \item [SMTPD\_KILLED] Sometimes an \daemon{smtpd} is killed by a
        signal; the active connection for
        that \daemon{smtpd} must be cleaned up, entered in the database,
        and deleted from the state tables.

    \item [SMTPD\_WATCHDOG] \daemon{smtpd} processes have a watchdog timer
        to deal with unusual situations --- after five hours the timer will
        expire and the \daemon{smtpd} will exit.  This occurs very
        infrequently, as there are many other timeouts that should occur
        in the intervening hours: DNS timeouts, timeouts reading data
        from the client, etc\@.  The active connection for that
        \daemon{smtpd} must be cleaned up, entered in the database, and
        deleted from the state tables.

    \item [TIMEOUT] The connection timed out so the mail currently being
        transferred must be discarded. The mail may have been accepted, in
        which case there's a data structure to dispose of, or it may not in
        which case there is not.  See
        \sectionref{timeouts-during-data-phase} for the gory details.

    \item [TRACK] Track a mail when it is re-injected for forwarding to
        another mail server; this happens when a local address is aliased
        to a remote address.  TRACK will be called when dealing with the
        parent mail, and will create the child mail if necessary. TRACK
        checks if the child has already been tracked, either by this parent
        or by another parent, and issues appropriate warnings in either
        case.

    \item [UNINTERESTING] This rule just returns successfully; it is used
        when a line needs to be parsed for completeness but does not either
        provide any useful data or require anything to be done.

\end{description}

The distribution of rules per action is shown in \graphref{Distribution of
rules per action}.

\showgraph{build/graph-action-distribution}{Distribution of rules per
action}{Distribution of rules per action}

\subsection{Postfix Actions}

\label{list of postfix actions in implementation}

XXX WHY DO I NEED HAVE postfix\_actions AT ALL\@?

SEE TABLE (PROBABLY ON NEXT PAGE): THERE IS NEARLY A 1--1 MAPPING BETWEEN
action AND postfix\_action.  NEW ACTIONS COULD BE ADDED TO FORCE A 1--1
MAPPING, E.G. COMMIT + DISCARDED COULD BECOME A DISCARD ACTION\@.
MAIL\_QUEUED HAVING INFO AND PROCESSING CAN PROBABLY BE DEALT WITH, THE
BIGGEST DIFFICULTY WILL BE WITH SAVE\_DATA\@.

\begin{tabular}{lll}

1 & BOUNCE & BOUNCED \\
1 & CLONE & ACCEPTED \\
3 & COMMIT & DISCARDED \\
1 & COMMIT & INFO \\
1 & CONNECT & INFO \\
1 & DELETE & DELETED \\
64 & DELIVERY\_REJECTED & REJECTED \\
1 & DISCONNECT & INFO \\
1 & EXPIRY & EXPIRED \\
1 & MAIL\_QUEUED & INFO \\
1 & MAIL\_QUEUED & PROCESSING \\
1 & MAIL\_TOO\_LARGE & DISCARDED \\
1 & PICKUP & INFO \\
2 & POSTFIX\_RELOAD & POSTFIX\_RELOAD \\
13 & SAVE\_DATA & BOUNCED \\
38 & SAVE\_DATA & INFO \\
1 & SAVE\_DATA & REJECTED \\
5 & SAVE\_DATA & SENT \\
2 & SMTPD\_DIED & IGNORED \\
1 & SMTPD\_WATCHDOG & IGNORED \\
2 & TIMEOUT & DISCARDED \\
1 & TRACK & SENT \\
42 & UNINTERESTING & IGNORED \\

\end{tabular}

postfix\_action IS USED IN A FEW ACTIONS, BUT IT COULD PROBABLY BE EASILY
REMOVED, OR MAYBE I SHOULD KEEP TRACK OF actions INSTEAD\@?

XXX IF I KEEP THESE I NEED TO EXPLAIN WHY I HAVE THEM\@.

\begin{description}

    \item [ACCEPTED] Postfix has accepted a mail, and will
        subsequently attempt to deliver it.

    \item [BOUNCED] The mail has bounced, because of a mail loop,
        delivery failure, or five day timeout.\footnote{As with
        much of Postfix's behaviour, five days is the default value
        but can be changed by the administrator if they choose.}

    \item [DELETED] The mail was deleted from the queue by an
        administrator.

    \item [DISCARDED] Postfix discarded the mail it was in the
        process of accepting, because it was either larger than the
        size limit set by the administrator, or the client timed
        out or disconnected.

    \item [EXPIRED] The mail has been in the queue for five days without
        successful delivery.  A bounce mail will be generated and sent to
        the sender address.

    \item [INFO] Represents an unspecified intermediate action that
        the parser is not interested in per se, but that does log
        useful information, supplementing other log lines.

    \item [IGNORED] An action that is not only uninteresting in
        itself, but that also provides no useful data.

    \item [POSTFIX\_RELOAD] The administrator has instructed
        Postfix to start or stop, and all \daemon{smtpd} processes
        will be terminated.  This does not negatively affect the
        log files or mail queued by Postfix for delivery.

    \item [PROCESSING] \daemon{cleanup} is processing a mail.
        XXX ADD MORE DETAIL HERE

    \item [REJECTED] Postfix rejected a command from the remote
        client, causing at least one recipient to be rejected.

    \item [SENT] Postfix has successfully sent a mail.

\end{description}

\subsection{Adding New Actions}

\label{adding new actions in implementation}

XXX HALF OF THIS CONTENT IS IN FOOTNOTES, WHICH SHOULD BE INLINED\@.

Adding new actions is not as easy as adding new rules, though care has been
taken in the parser's design and implementation to make adding new actions
as painless as possible.  The implementor writes a subroutine that accepts
the standard arguments given to actions, and registers it as an action by
calling the framework subroutine add\_actions() with the name of the new
action as a parameter.  The new action must be registered before the rules
are loaded, because it is an error for a rule to specify an unregistered
action; this helps catch mistakes made when adding new rules.  No other
work is required from the implementor to integrate the action into the
parser; all of their attention and effort can be focused on correctly
implementing their action.  The only negative aspect is that the process
involves editing the parser source code, which makes upgrading to a later
version of the parser more difficult, though by no means impossible.  If
the author of the new action wishes, they can take advantage of the
parser's object oriented implementation by subclassing it and implementing
their changes in the derived class, allowing future upgrades of the parser
with greatly reduced chance of conflicts.\footnote{The real difference
between the two approaches is where the new code is placed.  The simpler
option is to change the parser code directly, but those changes will then
have to be made to subsequent versions of the parser, and as the scope of
the changes increases so does the chance of conflict, or mistakes when
copying the action.  The more time consuming option is to write a subclass
containing the new actions and change the program that invokes the parser
so that it uses the subclass rather than the parser; the changes required
to the program invoking the parser are minor and much less likely to lead
to conflicts when upgrading to future versions of the parser.  An
alternative is to submit new actions to the author of the parser for
inclusion in future versions, resulting in two benefits: the new actions do
not need to be maintained separately, and other users of the parser can
avail of the new functionality.} The action may need to extend the list of
valid combinations described in \sectionref{out of order log lines} if the
new action creates a different set of acceptable programs, but this is
unlikely to occur, as it would require parsing log lines from Postfix
components the parser currently ignores.\footnote{The mail server used for
development does not utilise either the \daemon{lmtp} or \daemon{virtual}
delivery agents, so this parser does not have rules to handle log lines
from those components.  Adding new rules to parse those component's log
lines is a simple process, though if their behaviour differs significantly
from the \daemon{smtp} or \daemon{local} delivery agents new actions may be
required.  The mail server in question is a production mail server handling
mail for a university department; the benefit of using this server is that
the log files used exhibit the idiosyncrasies and peculiarities a mail
server in the wild must deal with, but the downside is that significantly
altering the configuration just to log messages from a different Postfix
component is not an option.}


\section{Framework}

XXX TO BE WRITTEN\@.


\section{Parser Flow Chart}

\label{flow-chart}

XXX MIGHT NEED A SMALL INTRO\@.

From the viewpoint of an individual mail passing through the parser the
experience could be summarised as:

\begin{enumerate}

    \item Mail enters the system via \acronym{SMTP} or local submission.

    \item If the mail is rejected, save all data and finish.

    \item Follow the progress of the accepted mail until it is either
        delivered, bounced, or deleted, then save all data, and finish.

\end{enumerate}

\Figureref{flow chart image} shows the most common paths the data
representing a mail can take through the parser; the complications
described in \sectionref{complications} are excluded for the sake of
clarity.

\showgraph{build/logparser-flow-chart-part-1}{Parser flow chart}{flow chart
image}

\label{mail-enters-the-system}

Everything starts off with a mail entering the system, whether by local
submission via \daemon{postdrop}, by \acronym{SMTP}, by re-injection due to
forwarding, or internally generated by Postfix.  Local submission is the
simplest case: a queueid is assigned immediately and the sender address is
logged (action: PICKUP\@; flowchart:~2).  Re-injection due to forwarding
sadly lacks explicit log lines of its own; it is explained in
\sectionref{Re-injected mails}.

Internally generated mails lack any explicit origin in Postfix 2.2.x and
must be detected using heuristics as described in
\sectionref{identifying-bounce-notifications}.  Bounce notifications are
the primary example of internally generated mails, though there may be
other types.  Postfix may generate mails to the administrator when it
encounters configuration errors, but such mails are presumably rare.

\acronym{SMTP} is more complicated than local submission:

\begin{enumerate}

    \item First there is a connection from the remote client (action:
        CONNECT\@; flowchart:~1).

    \item This is followed by rejection of sender address, recipient
        addresses, client \acronym{IP} address or hostname, HELO hostname,
        etc.\ (action: REJECTION\@; flowchart:~4); acceptance of one or
        more mails (action: CLONE\@; flowchart:~5); or some interleaving of
        both.

    \item The client disconnects (action: DISCONNECT\@; flowchart:~6).  If
        Postfix has rejected any \acronym{SMTP} commands the data will be
        saved to the database; if not there will not be any data to save
        (any mails accepted will already have been cloned so their data is
        in another data structure).

    \item If one or more mails were accepted there will be more log lines
        for those mails later, see \sectionref{mail-delivery}.

\end{enumerate}

\label{mail-delivery}

The obvious counterpart to mail entering the system is mail leaving the
system, whether by deletion, bouncing, local delivery, or remote delivery.
All four are handled in exactly the same way:

\begin{enumerate}

    \item Postfix will log the sender and recipient addresses separately
        (action: SAVE\_DATA\@; flowchart:~9).

    \item Sometimes mail is re-injected and the child mail needs to be
        tracked by the parent mail (action: TRACK\@; flowchart:~10) ---
        \sectionref{tracking re-injected mail} discusses this in
        detail.

    \item Eventually the mail will be delivered, bounced, or deleted by the
        administrator (action: COMMIT\@; flowchart:~12).  This is the last
        log line for this particular mail (though it may be indirectly
        referred to if it was re-injected).  If it is neither parent nor
        child of re-injection the data is cleaned up and entered in the
        database (flowchart:~14), then deleted from the state tables.
        Re-injected mails are described in \sectionref{tracking re-injected
        mail}.

\end{enumerate}

It should be reiterated that the actions above happen whether the mail is
delivered to a mailbox, piped to a command, delivered to a remote server,
bounced (due to a mail loop, delivery failure, or five day timeout), or
deleted by the administrator, \textit{unless\/} the mail is either parent
or child of re-injection, as explained in \sectionref{tracking re-injected
mail}.

\section{Complications Encountered}

\label{complications}

XXX IMPROVE THE INTRO\@: ``needs more of a lead-in''.

XXX SHOULD I HAVE A LIST OF HOW MANY COMPLICATIONS ARE HANDLED BY EACH
ACTION\@?

The complications described in this section are listed in the order in
which they were encountered during development of the parser.  Each of
these complications caused the parser to operate incorrectly, generating
either warning messages or leaving mails in the state table.  The frequency
of occurrence is much higher at the start of the list, with the first
complication occurring several orders of magnitude more frequently than the
last.  When deciding which problem to address next, the most common was
always chosen, as resolving the most common problem would yield the biggest
improvement in the parser, prune the greatest number of mails from the
state tables and error messages, and make the remaining problems more
apparent.  The first three complications were encountered early in the
parser's implementation and guided its design and development.

\subsection{Queueid Vs Pid}

The mail lacks a queueid until it has been accepted, so log lines must
first be correlated by the \daemon{smtpd} \acronym{pid}, then transition to
being correlated by the queueid.  This is relatively minor, but does
require:

\begin{itemize}

    \item Two versions of several functions: \texttt{by\_pid} and
        \texttt{by\_queueid}.

    \item Two state tables to hold the data structure for each connection
        and mail.

    \item Most importantly: every section of code must know whether it
        needs to lookup the data structures by \acronym{pid} or queueid.

\end{itemize}

\subsection{Connection Reuse}

\label{connection reuse}

Multiple independent mails may be delivered across one connection: this
requires the algorithm to clone the current data as soon as a mail is
accepted, so that subsequent mails will not trample over each other's data.
This must be done every time a mail is accepted, as it is impossible to
tell in advance which connections will accept multiple mails.  Once the
mail has been accepted its log lines will not be correlated by
\acronym{pid} any more, its queueid will be used instead (except when
timeouts occur during the data phase
\sectionref{timeouts-during-data-phase}).  If the original connection has
any useful data (e.g.\ rejections) it will be saved to the database when
the client disconnects.  One unsolved difficulty is distinguishing between
different groups of rejections, e.g.\ when dealing with the following
sequence:

\begin{enumerate}

    \item The client attempts to deliver a mail, but it is rejected.

    \item The client issues the RSET command to reset the \acronym{SMTP}
        session.

    \item The client attempts to deliver another mail, likewise rejected.

\end{enumerate}

There should ideally be two separate entries in the database resulting from
the above sequence, but currently there will only be one.



\subsection{Re-injected Mails}

\label{Re-injected mails}

\label{tracking re-injected mail}

The most difficult complication initially encountered is that locally
addressed mails are not always delivered directly to a mailbox: sometimes
they are addressed to and accepted for a local address but need to be
delivered to one or more remote addresses due to aliases.  When this
occurs a child mail will be injected into the Postfix queue, but without
the explicit logging that \daemon{smtpd} or \daemon{postdrop} injected
mails have.  Thus the source of the mail is not immediately discernible
from the log line in which the mail first appears; from a strictly
chronological reading of the log files it usually appears as if the child
mail has appeared from thin air.  Subsequently the parent mail will log the
creation of the child mail, e.g.\ parent mail 3FF7C4317 creates child mail
56F5B43FD\@:

\texttt{3FF7C4317: to=<username@example.com>, relay=local, \newline{}
\tab{} delay=0, status=sent (forwarded as 56F5B43FD)}

Unfortunately, while all log lines from an individual process appear in
chronological order, the order in which log lines from different processes
are interleaved is subject to the vagaries of process scheduling.  In
addition, the first log line belonging to the child mail (the log line
cited above belongs to the parent mail) is logged by \daemon{qmgr}, so the
order also depends on how soon \daemon{qmgr} processes the new mail.

Because of the uncertain order the parser cannot complain when it
encounters a log line from \daemon{qmgr} for a previously unseen mail;
instead it must flag the mail as coming from an unknown origin, and
subsequently clear the flag if and when the origin of the mail becomes
clear.  Obviously the parser could omit checking where mails originate
from, but requiring an explicit source helps to expose bugs in the parser;
such checks helped to identify the complications described in
\sectionref{discarding cleanup log lines} and \sectionref{pickup logging
after cleanup}.

Process scheduling can have a still more confusing effect: quite often the
child mail will be created, delivered, and entirely finished with
\textbf{before} the parent logs the creation log line!  Thus, mails flagged
as coming from an unknown origin cannot be entered into the database when
their final log line is parsed; instead they must be marked as ready for
entry and subsequently entered once the parent mail has been identified.

XXX MERGE THE REMAINDER OF THIS SECTION INTO THE PRIOR MATERIAL\@.

The crux of this complication is that re-injected mails appear in the log
files without explicit logging indicating their source.  There are two
implicit indications:

\begin{enumerate}

    \item The indicator which more commonly introduces re-injection is when
        \daemon{qmgr} selects a mail with a previously unseen queueid for
        delivery, in which case a new data structure will be created for
        that mail.  The mail will be flagged as having unknown origins;
        this flag will subsequently be cleared once the origin has been
        established.  This may also be an indicator that the mail is a
        bounce notification, see
        \sectionref{identifying-bounce-notifications} for details.

    \item Local delivery re-injects the mail and logs a relayed delivery
        rather than delivering directly to a mailbox or program as it
        usually would.\footnote{Relayed delivery is performed by the
        \acronym{SMTP} client; local delivery means local to the server,
        i.e.\ an address the server is final destination for.}  In this
        case the mail may already have been created (described above) and
        the unknown origin flag will be cleared; if not a new data
        structure will be created.  In both cases the re-injected mail is
        marked as a child of the original mail.  The log line in question
        is:

        \texttt{3FF7C4317: to=<username@example.com>, relay=local,
        \newline{} \tab{} delay=0, status=sent (forwarded as 56F5B43FD)}

        This always occurs for re-injected mail but typically occurs after
        the first indicator.  This log line is required to tie the parent
        and child mails together and so is central to the process of
        tracking re-injected mails.

\end{enumerate}

The algorithm for tracking and saving re-injected mail to the database can
finally be described:

\begin{itemize}

    \item If the mail is of unknown origin it is assumed to be a child mail
        whose parent has not yet been identified.  Mark the mail as ready
        for entry in the database and wait for the parent to deal with it.
        The mail should not have any subsequent log lines; only its parent
        should refer to it.

    \item If the mail is a child mail then it has already been tracked: as
        with all other mail, the data is cleaned up, the child is entered
        in the database, and then deleted from the state tables.  The child
        mail will be removed from the parent mail's list of children; if
        this is the last child and the parent has already been entered in
        the database, the parent will also be deleted from the state
        tables.

    \item The last alternative is that the mail is a parent mail.
        Regardless of the state of its children its data is cleaned up and
        entered in the database.  The parent may have children that are
        waiting to be entered in the database; the data for each of those
        children is cleaned up and entered in the database, then deleted
        from the state tables.  The parent may also have outstanding
        children which are not yet delivered, in which case the parent must
        be retained in the state tables until those children are finished
        with.  As soon as the last child is deleted from the state tables
        the parent will also be deleted from the state tables.

\end{itemize}

A parent mail can have multiple children, which may be delivered before or
after the parent mail.


\subsection{Identifying Bounce Notifications}

\label{identifying-bounce-notifications}

Postfix 2.2.x (and presumably previous versions) does not generate a log
line when it generates a bounce notification; suddenly there will be log
entries for a mail that lacks an obvious source.  There are similarities to
the problem of identifying re-injected mails discussed in
\sectionref{tracking re-injected mail}, but unlike the solution described
therein bounce notifications do not eventually have a log line that
identifies their source.  Heuristics must be used to identify bounce
notifications, and those heuristics are:

\begin{enumerate}

    \item The sender address is \verb!<>!.\glsadd{<>}

    \item Neither \daemon{smtpd} nor \daemon{pickup} have logged any
        messages associated with the mail, indicating it was generated
        internally by Postfix, not accepted via \acronym{SMTP} or submitted
        locally by \daemon{postdrop}.

    \item The message-id has a specific format: \newline{}
        \tab{} \texttt{YYYYMMDDhhmmss.queueid@server.hostname} \newline{}
        e.g.\ \texttt{20070321125732.D168138A1@smtp.example.com}

    \item The queueid in the message-id must be the same as the queueid of
        the mail: this is what distinguishes bounce notifications generated
        locally from bounce notifications which are being re-injected as a
        result of aliasing (described in \sectionref{Re-injected mails}).
        In the latter case the message-id will be unchanged from the
        original bounce notification, and so even if it happens to be in
        the correct format (e.g.\ if it was generated by Postfix on this or
        another server) it will not match the queueid of the mail.

\end{enumerate}

Once a mail has been identified as a bounce notification, the unknown
origin flag is cleared and the mail can be entered in the database.

There is a small chance that a mail will be incorrectly identified as a
bounce notification, as the heuristics used may be too broad.  For this to
occur the following conditions would have to be met:

\begin{enumerate}

    \item The mail must have been generated internally by Postfix.

    \item The sender address must be \verb!<>!.\glsadd{<>}

    \item The message-id must have the correct format and match the queueid
        of the mail.  While a mail sent from elsewhere could easily have
        the correct message-id format, the chance that the queueid in the
        message-id would match the queueid of the mail is extremely small.

\end{enumerate}

If a mail is mis-classified as a bounce message it will almost certainly
have been generated internally by Postfix; arguably mis-classification in
this case is a benefit rather than a drawback, as other mails generated
internally by Postfix will be handled correctly.  Postfix 2.3 (and
hopefully subsequent versions) log the creation of a bounce message.

This check is performed during the COMMIT action.

\subsection{Aborted Delivery Attempts}

\label{aborted-delivery-attempts}

Some mail clients behave unexpectedly during the \acronym{SMTP} dialogue:
the client aborts the first delivery attempt after the first recipient is
accepted, then makes a second delivery attempt for the same recipient which
it continues with until delivery is complete.  Microsoft Outlook is one
client that behaves in this fashion; other clients may act in a similar
way.  An example dialogue exhibiting this behaviour is presented below
(lines starting with a three digit number are sent by the server, the other
lines are sent by the client):

\begin{verbatim}
220 smtp.example.com ESMTP
EHLO client.example.com
250-smtp.example.com
250-PIPELINING
250-SIZE 15240000
250-ENHANCEDSTATUSCODES
250-8BITMIME
250 DSN
MAIL FROM: <sender@example.com>
250 2.1.0 Ok
RCPT TO: <recipient@example.net>
250 2.1.5 Ok
RSET
250 2.0.0 Ok
RSET
250 2.0.0 Ok
MAIL FROM: <sender@example.com>
250 2.1.0 Ok
RCPT TO: <recipient@example.net>
250 2.1.5 Ok
DATA
354 End data with <CR><LF>.<CR><LF>
The mail transfer is not shown.
250 2.0.0 Ok: queued as 880223FA69
QUIT
221 2.0.0 Bye
\end{verbatim}

Once again Postfix does not log a message making the client's behaviour
clear, so once again heuristics are required to identify when this
behaviour occurs.  In this case a list of all mails accepted during a
connection is saved in the connection state, and the accepted mails are
examined when the disconnection action is executed.  Each mail is checked
for the following: 

\begin{itemize}

    \item XXX IS THIS CORRECT NOW\@?\newline{}  Are there two or more
        \daemon{smtpd} log lines?  Does the second result have a
        postfix\_action attribute of ACCEPTED\@?  The first two
        \daemon{smtpd} log lines will be a connection log line and a mail
        acceptance log line (Postfix logs acceptance as soon as the first
        recipient has been accepted).

    \item Is \daemon{smtpd} the only Postfix component that produced a log
        line for this mail?  Every mail which passes normally through
        Postfix will have a \daemon{cleanup} line, and later a
        \daemon{qmgr} log line; lack of a \daemon{cleanup} line is a sure
        sign the mail did not make it too far.  

    \item Does the queueid exist in the state tables?  If not it cannot be
        an aborted delivery attempt.

    \item If there are third and subsequent results, are all their
        postfix\_action attributes equal to INFO\@?  If there are any log
        lines after the first two they should be informational only.

\end{itemize}

If all the checks above are successful then the mail is assumed to be an
aborted delivery attempt and is discarded.  There will be no further
entries logged for such mails, so without identifying and discarding them
they accumulate in the state table and will cause clashes if the queueid is
reused.  The mail cannot be entered in the database as the only data
available is the client hostname and \acronym{IP} address, but the database
schema requires many more fields be populated (see \sectionref{connections
table} and \sectionref{results table}).  This heuristic is quite
restrictive, and it appears there is little scope for false positives; if
there are any false positives there will be warnings when the next log line
for that mail is parsed.  False negatives are less likely to be detected:
there may be queueid clashes (and warnings) if mails remain in the state
tables after they should have been removed, otherwise the only way to
detect false negatives is to examine the state tables after each parsing
run.

This check is performed in the DISCONNECT action; it requires support in
the CLONE action where a list of cloned connections is maintained.


\newpage{} % XXX

\subsection{Further Aborted Delivery Attempts}

XXX MAYBE FIND OUT WHAT CLIENTS DISCONNECT\@?  Some mail clients disconnect
abruptly if a second or subsequent recipient is rejected; they may also
disconnect after other errors, but such disconnections are either
unimportant or are handled elsewhere in the parser
(\sectionref{timeouts-during-data-phase}).  Sadly, Postfix does not log a
message saying the mail has been discarded, as should be expected by now.
The checks to identify this happening are:

\begin{itemize}

    \item Is the mail missing its \daemon{cleanup} log line?  Every mail
        which passes through Postfix will have a \daemon{cleanup} line;
        lack of a \daemon{cleanup} line is a sure sign the mail did not
        make it too far.

    \item Were there three or more \daemon{smtpd} log lines for the mail?
        There should be a connection log line and a mail accepted log line,
        followed by one or more rejection log lines.

    \item Is the last \daemon{smtpd} log line a rejection line?

\end{itemize}

If all checks are successful then the mail is assumed to have been
discarded when the client disconnected.  There will be no further entries
logged for such mails, so without identifying and entering them in the
database immediately they accumulate in the state table and will cause
clashes if the queueid is reused.

These checks are made during the DISCONNECT action.

\subsection{Timeouts During DATA Phase}

\label{timeouts-during-data-phase}

The DATA phase of the \acronym{SMTP} conversation is where the headers and
body of the mail are transferred.  Sometimes there is a timeout or the
connection is lost\footnote{For brevity's sake \textit{timeout\/} will be
used throughout this section, but everything applies equally to lost
connections.} during the DATA phase; when this occurs Postfix will discard
the mail and the parser needs to discard the data associated with that
mail.  It seems more intuitive to save the mail's data to the database, but
if a timeout occurs there is no data available to save; the timeout is
recorded with the connection data instead, which is saved.

To deal properly with timeouts the parsing algorithm needs to do the
following in the TIMEOUT action:

\begin{enumerate}

    \item Record the timeout and associated data in the connection's
        results.

    \item If no mails have been accepted yet nothing needs to be done; the
        TIMEOUT action ends.  

    \item If one or more recipients have been accepted Postfix will have
        allocated a queueid for the incoming mail, and there will be a mail
        in the state tables that needs to be dealt with.

\end{enumerate}

XXX MAKE THIS PARAGRAPH CLEARER\@.  A timeout may thus apply either to an
accepted mail or a rejected mail.  To distinguish between the two cases the
algorithm compares the timestamp of the last accepted mail against the
timestamp of the last line logged by \daemon{smtpd} for that connection
(the TIMEOUT action is dependant on the CLONE action keeping a list of all
mails accepted on each connection).  If the mail acceptance timestamp is
later then the timeout applies to the just-accepted mail, which will be
discarded.  If the \daemon{smtpd} timestamp is later there was a rejection
between the accepted mail and the timeout: the action assumes that the
timeout applies to a rejected delivery attempt and finishes.  This
assumption is not necessarily correct, because Postfix may have accepted an
earlier recipient and rejected a later one, in which case the timeout
applies to the accepted mail, which should be discarded.  This has not been
a problem in practice, though it may be in future.  This complication is
further complicated by the presence of out of order \daemon{cleanup} log
lines: see \sectionref{discarding cleanup log lines} for details.

This complication is dealt with in the TIMEOUT action.

\subsection{Discarding Cleanup Log Lines}

\label{discarding cleanup log lines}

The author has only observed this complication occurring after a timeout,
though there may be other circumstances that trigger it.  Sometimes the
\daemon{cleanup} line for a mail being accepted is logged after the timeout
line; parsing this line causes the MAIL\_QUEUED action to create a new
state table entry for the queueid in the log line.  This is incorrect
because the line actually belongs to the mail that has just been discarded;
the next log line for that queueid will be seen when the queueid is reused
for a different mail, causing a queueid clash and the appropriate warning.

When the \daemon{cleanup} line is still pending during the TIMEOUT action,
the action updates a global list of queueids, adding the queueid and the
timestamp from the log line.  When the next \daemon{cleanup} line is parsed
for that queueid the list will be checked (during the MAIL\_QUEUED action),
and the log line will be deemed part of the mail where the timeout occurred
and discarded if it meets the following requirements:

\begin{itemize}

    \item The queueid must not have been reused yet, i.e.\ it does not have
        an entry in the state tables.

    \item The timestamp of the \daemon{cleanup} log line must be within ten
        minutes of the mail acceptance timestamp.  Timeouts happen after
        five minutes, but some data may have been transferred slowly
        (perhaps because either the client or server is suffering from
        network congestion or rate limiting), and empirical evidence shows
        that ten minutes is not unreasonable; hopefully it is a good
        compromise between false positives (log lines incorrectly
        discarded) and false negatives (new state table entries incorrectly
        created).

\end{itemize}

The next \daemon{cleanup} line must meet the criteria above for it to be
discarded because some, but not all connections where a timeout occurs will
have an associated \daemon{cleanup} line logged; if the algorithm blindly
discarded the next \daemon{cleanup} line after a timeout it would sometimes
be mistaken.  When the next \daemon{pickup} line containing that queueid is
parsed the queueid will be removed from the cache of timeout queueids,
regardless of whether it meets the criteria above.

This complication is handled by the TIMEOUT and MAIL\_QUEUED actions.

\subsection{Pickup Logging After Cleanup}

\label{pickup logging after cleanup}

When mail is submitted locally, \daemon{pickup} accepts the new mail and
generates a log line showing the source.  Occasionally this log line will
occur later in the log file than the \daemon{cleanup} log line, so the
PICKUP action will find that a state table entry exists for that queueid.
Normally if the queueid given in the PICKUP line exists in the state tables
a warning is generated by the \daemon{pickup} action, but if the following
conditions are met it is assumed that the log lines were out of order:

\begin{itemize}

    \item The only program which has logged anything thus far for the mail
        is \daemon{cleanup}.

    \item There is less than a five second difference between the
        timestamps of the \daemon{cleanup} and \daemon{pickup} log lines.

\end{itemize}

As always with heuristics there may be circumstances in which these
heuristics match incorrectly, but none have been identified so far.  This
complication seems to occur during periods of particularly heavy load, so
is most likely caused by process scheduling vagaries.  

This complication is dealt with during the PICKUP action.

\subsection{Smtpd Stops Logging}

\label{smtpd stops logging}

Occasionally a \daemon{smtpd} will just stop logging, without an
immediately obvious reason.  After poring over log files for some time
several reasons have been found for this infrequent occurrence:

\begin{enumerate}

    \item Postfix is stopped or its configuration is reloaded.  When this
        happens all \daemon{smtpd} processes exit, so all entries in the
        connections state table must be cleaned up, entered in the database
        if there is sufficient data, and deleted.

    \item Sometimes a \daemon{smtpd} is killed by a signal (sent by Postfix
        for some reason, by the administrator, or by the OS), so its active
        connection must be cleaned up, entered in the database if there is
        sufficient data, and deleted from the connections state table.

    \item Occasionally a \daemon{smtpd} will exit uncleanly, so the active
        connection must be cleaned up, entered in the database if there is
        sufficient data, and deleted from the connections state table.

    \item Every Postfix process uses a watchdog which kills the process if
        it is not reset for a considerable period of time (five hours by
        default).  This safeguard prevents Postfix processes from running
        indefinitely and consuming resources if a failure causes them to
        enter a stuck state.  XXX IMPROVE THIS SOMEHOW\@.

\end{enumerate}

The circumstances above account for all occasions identified thus far where
a \daemon{smtpd} suddenly stops logging.  In addition to removing an active
connection the last accepted mail may need to be discarded, as detailed in
\sectionref{timeouts-during-data-phase}; otherwise the queueid state table
is untouched.

These occurrences are handled by the three actions POSTFIX\_RELOAD,
SMTPD\_DIED, and SMTPD\_WATCHDOG\@.

\subsection{Out of Order Log Lines}

\label{out of order log lines}

Occasionally a log file will have out of order log lines which cannot be
dealt with by the techniques described in \sectionref{tracking re-injected
mail}, \sectionref{discarding cleanup log lines} or \sectionref{pickup
logging after cleanup}.  In the \numberOFlogFILES{} log files used for
testing this occurs only five times in 60,721,709 log lines, but for parser
correctness it must be dealt with.  The five occurrences have the same
characteristics: the \daemon{local} log line showing delivery to a local
mailbox occurs after the \daemon{qmgr} log line showing removal of the mail
from the queue because delivery is completed.  This causes problems: the
data in the state tables for the mail is not complete, so entering it into
the database fails; a new mail is created when the \daemon{local} line is
parsed and remains in the state tables; four warnings are issued per pair
of out of order log lines.

The COMMIT action examines the list of programs that have produced a log
lines for each mail, comparing the list against a table of known-good
program combinations.  If the mail's combination is found in the table the
mail can be entered in the database; if the combination is not found entry
must be postponed and the mail flagged for later entry.  The SAVE\_DATA
action checks for that flag; if the additional log lines have caused the
mail to reach a valid combination then entry in the database will proceed,
otherwise it must be postponed once more.

The list of valid combinations is explained below.  Every mail will
additionally have log line from \daemon{cleanup} and \daemon{qmgr}; any
mail may also have log line from \daemon{bounce}, \daemon{postsuper}, or
both.

XXX WOULD IT BE BETTER TO BEGIN THE EXPLANATIONS ON THE FOLLOWING LINE\@?

\begin{description}

    \item [\daemon{local}:] Local delivery of a bounce notification, or
        local delivery of a re-injected mail.

    \item [\daemon{local}, \daemon{pickup}:] Mail submitted locally on the
        server, delivered locally on the server.

    \item [\daemon{local}, \daemon{pickup}, \daemon{smtp}:] Mail submitted
        locally \newline{} on the server, for both local and remote
        delivery.

    \item [\daemon{local}, \daemon{smtp}, \daemon{smtpd}:] Mail accepted
        from a remote client, for both local and remote delivery.

    \item [\daemon{local}, \daemon{smtpd}:] Mail accepted from a remote
        client, for local delivery only.

    \item [\daemon{pickup}, \daemon{smtp}:] Mail submitted locally on the
        server, for remote delivery only.

    \item [\daemon{smtp}:] Remote delivery of either a re-injected mail or
        a bounce notification.

    \item [\daemon{smtp}, \daemon{smtpd}:] Mail accepted from a remote
        client, then remotely delivered (typically relaying mail for
        clients on the local network to addresses outside the local
        network).

    \item [\daemon{smtpd}, \daemon{postsuper}:] Mail accepted from a remote
        client, then deleted by the administrator before any delivery
        attempt was made (the unwanted mail is typically due to a mail loop
        or joe~job).  Notice that \daemon{postsuper} is required, not
        optional, for this combination.

\end{description}

This check applies to accepted mails only, not to rejected mails.  This
check is performed during the COMMIT action.

\subsection{Yet More Aborted Delivery Attempts}

\label{yet-more-aborted-delivery-attempts}

The aborted delivery attempts described in
\sectionref{aborted-delivery-attempts} occur frequently, but the aborted
delivery attempts described in this section only occur four times in the
\numberOFlogFILES{} log files used for testing.  The symptoms are the same
as in \sectionref{aborted-delivery-attempts}, except that there
\textit{is\/} a \daemon{cleanup} log line; there does not appear to be
anything in the log file to explain why there are no further log lines.
The only way to detect these mails is to periodically scan all mails in the
state tables, deleting any mails displaying the following characteristics:

\begin{itemize}

    \item The timestamp of the last log line for the mail must be 12 hours
        or more earlier than the last log line parsed from the current log
        file.

    \item XXX IS THIS CORRECT NOW\@?  CAN THERE BE MORE smtpd LOG LINES\@?
        There must be exactly two \daemon{smtpd} and one \daemon{cleanup}
        log lines for the mail, with no additional log lines.

\end{itemize}

12 hours is a somewhat arbitrary time period, but it is far longer than
Postfix would delay delivery of a mail in the queue (unless Postfix is not
running for an extended period of time).  The state tables are scanned for
mails matching the characteristics above each time the end of a log file is
reached, and matching mails are deleted.

\subsection{Mail Deleted Before Delivery is Attempted}

\label{Mail deleted before delivery is attempted}

Postfix logs the recipient address when delivery of a mail is attempted, so
if delivery has yet to be attempted the parser cannot determine the
recipient address or addresses.  This is a problem when mail is arriving
faster than Postfix can attempt delivery, and the administrator deletes
some of the mail (because it is the result of a mail loop\glsadd{mail
loop}, mail bomb\glsadd{mail bomb}, or joe~job\glsadd{Joe job}) before
Postfix has had a chance to try to deliver it.  In this case the recipient
address will not have been logged, so a dummy recipient address needs to be
added, as every mail is required by the database schema
(\sectionref{results table}) to have at least one recipient.  This
complication has been observed in XXX/\numberOFlogFILES{} log files, but
typically when it does occur there will be many instances of it, closely
grouped.

The DELETE action is responsible for handling this complication.

\subsection{Bounce Notification Mails Are Delivered Before Their Creation
Is Logged}

\label{Bounce notification mails delivered before their creation is logged}

This is yet another complication that only occurs during periods of
extremely high load, when process scheduling and even hard disk access
times cause strange behaviour.  In this complication bounce notification
mails are created, delivered, and deleted from the queue, \textit{before\/}
the log line from \daemon{bounce} that explains their source is logged.  To
deal with this the COMMIT action maintains a cache of recently committed
bounce notification mails, which the BOUNCE action subsequently checks if
the bounce mail is not already in the state tables. If the queueid exists
in the cache, and its start time is less than ten seconds before the
timestamp of the bounce log line, it is assumed that the bounce
notification mail has already been processed and the BOUNCE action does not
create one.  If the queueid exists it is removed from the cache, because it
has either just been used or it is too old to use in future.  Whether the
BOUNCE action creates a mail or finds an existing mail in the state tables,
it flags the mail as having been seen by the BOUNCE action; if this flag is
present the COMMIT action will not add the mail to the cache of recent
bounce notification mails.\footnote{This is not required to correctly deal
with the complication, but is an optimisation to reduce the parser's memory
usage; on the occasions the author has observed this complication occurring
there have been a huge number of bounce notification mails generated --- if
every bounce notification mail was cached it would dramatically increase
the memory requirements of the parser.}  The cache of bounce notification
mails will be pruned whenever the parser's state is saved, though if the
size of the cache ever becomes a problem it could be pruned periodically to
keep the size in check.  XXX EXPLAIN PRUNING\@.

\subsection{Mails Deleted During Delivery}

\label{Mails deleted during delivery}

The administrator can delete mails using \daemon{postsuper}; occasionally
mails that are in the process of being delivered will be deleted.  This
results in the log lines from the delivery agent (\daemon{local},
\daemon{virtual} or \daemon{smtp}) appearing in the log file
\textit{after\/} the mail has been removed from the state tables and saved
in the database.  The DELETE action adds deleted mails to a cache, which is
checked by the SAVE\_DATA action, and the current log line discarded if the
following conditions are met:

\begin{enumerate}

    \item The queueid is not found in the state tables. 

    \item The queueid is found in the cache of deleted mails.

    \item The timestamp of the log line is within 5 minutes of the final
        timestamp of the mail.

\end{enumerate}

Sadly this solution involves discarding some data, but the complication
only arises eight times in quick succession in one log file, which is not
in the \numberOFlogFILES{} log files used for testing; if this complication
occurred more frequently it might be desirable to find the mail in the
database and add the log line to it.

\section{Limitations and Possible Improvements}

\label{limitations and improvements in implementation}

Every piece of software suffers from some limitations and there is almost
always room for improvement.  Below are the limitations and possible
improvements that have been identified in the parser.

\subsection{Limitations}

\label{logging helo}

\begin{enumerate}

    \item Each new Postfix release requires new rules to be written or
        existing rules modified to cope with the new or changed log lines.
        Similarly using a new \acronym{DNSBL}, a new policy server, or new
        administrator-defined rejection messages require new rules.

    \item The hostname used in the HELO command is not logged if the mail
        is accepted.\footnote{Tested with Postfix 2.2.10, 2.3.11, and
        2.4.7; this may possibly have changed in Postfix 2.5. XXX TEST THIS
        AGAIN.}  Rectifying this is relatively simple: create a restriction
        that is guaranteed to warn for every accepted mail, as follows:

        \begin{enumerate}

            \item Create \texttt{/etc/postfix/log\_helo.pcre}
                containing:\newline{}
                \tab{}\texttt{/\^/~~~~WARN~Logging~HELO}

            \item Modify \texttt{smtpd\_data\_restrictions} in
                \texttt{/etc/postfix/main.cf} to contain:\newline{}
                \tab{}\texttt{check\_helo\_access~pcre:/etc/postfix/log\_helo.pcre}

        \end{enumerate}

        Although \texttt{smtpd\_helo\_restrictions} seems like the natural
        place to log the HELO hostname, there will not be a queueid
        associated with the mail when \texttt{smtpd\_helo\_restrictions} is
        evaluated for the first recipient, so the log line cannot be
        associated with the correct mail.  There is guaranteed to be a
        queueid when the DATA command has been reached, and thus the
        queueid will be logged by any restrictions taking effect in
        \texttt{smtpd\_data\_restrictions}.  There is no difficulty in
        specifying a HELO-based restriction in
        \texttt{smtpd\_data\_restrictions}, Postfix will perform the check
        correctly.

        Logging the HELO hostname in this fashion also partially prevents
        the complication described in \sectionref{Mail deleted before
        delivery is attempted} from occurring, but only when there is a
        single recipient; in that case the recipient address will be logged
        also, but when there are multiple recipients no addresses are
        logged.  It is also possible to warn for every recipient,
        preventing the complication in \sectionref{Mail deleted before
        delivery is attempted} entirely.  XXX ADD A PARAGRAPH EXPLAINING
        HOW\@.

    \item The parser does not separate mails where one or more mails are
        rejected and a subsequent mail is accepted; it will appear in the
        database as one mail with lots of rejections followed by acceptance
        (this has already been mentioned in \sectionref{connection reuse}).
        It does not appear to be possible to make this distinction given
        the data Postfix logs, though it might be possible to write a
        policy server to provide additional logging.

    \item The parser will not detect that it is parsing the same log file
        twice, resulting in the database containing duplicate entries.

    \item The parser does not distinguish between log files produced by
        different sources when parsing; all results will be saved to the
        same database.  This may be viewed as an advantage, as log files
        from different sources can be combined in the same database, or it
        may be viewed as a limitation as there is no facility to
        distinguish between log files from different sources in the same
        database.  If the results of parsing log files from different
        sources must remain separate, the parser can easily be instructed
        to use a different database to store the results in.

    \item The solution to complication \sectionref{Mails deleted during
        delivery} involves discarding data.

\end{enumerate}

\subsection{Possible Improvements}

\begin{itemize}

    \item Investigate and write the policy server referred to in limitation
        3 above.

    \item Improve the solution to complication \sectionref{Mails deleted
        during delivery} so that data is not discarded.

    \item Improve the heuristics used in
        \sectionref{timeouts-during-data-phase}, or develop another
        solution, to avoid incorrectly leaving a mail in the state tables.

\end{itemize}


\section{Summary}

XXX TO BE WRITTEN\@.  START WITH THE OLD CONTENT BELOW\@.

This section presented the core of the parser, starting with a very high
level view and the initial complications that arose.  A flow chart showing
the paths a mail may take through the nascent simplified algorithm was
provided, followed by an explanation of those paths, and a discussion of
the parser's emergent behaviour --- the data from the log files creates the
paths in the flow chart, they are not specified anywhere in the parser.
The framework which holds the parser together was covered next, after which
came a description of the current actions provided by the parser, and the
algorithm for analysing unparsed log lines to create regexes for new rules.
Detecting, diagnosing, and defeating complications forms the largest single
portion of this section, mirroring the development of the parser.  The
complications are described in the order they were overcome, with
subsequent problems affecting fewer mails (often by an order or magnitude),
though the time required to solve problems increased with each successive
problem.



\chapter{Results}

XXX NEED AN INTRODUCTION\@.

XXX COMPARE SAVING RESULTS TO RAM DISK VERSUS SAVING TO HARD DISK\@.

XXX COMPARE PARSING + ACTIONS TO JUST PARSING\@.

XXX USE THE DATA COLLECTED ABOUT NUMBER OF RULES TRIED\@.  MAYBE INCLUDE IN
ORDERINGS SECTION\@?

XXX COMPARE BEST AND WORST POSSIBLE RESULTS WITH REAL RESULTS\@.

XXX REUSE THE CONTENT FROM THE PAPER FOR THIS SECTION\@.

\section{Parser Efficiency}

\label{parser efficiency}

Parsing efficiency is an obvious concern when the parser routinely needs to
parse large log files.  The server that generated the log files used in
testing this parser accepts approximately 10,000 mails for 700 users a day;
\graphref{Mails received per day} shows that as expected there are far more
mails received on weekdays than at weekends.   Median log file size is
50MB, containing 285,000 log lines --- large scale mail servers would have
much larger log files.  Note that \graphref{Mails received per day} and
\tableref{Number of mails received per day: statistics} show the number of
mails received by \gls{SMTP} only; in particular the mail loops noticeable
in other graphs do not contribute to these figures.

\showgraph{build/graph-mails-received}{Number of mails received via SMTP
per day}{Mails received per day}

\showtable{build/include-mails-received-table}{Number of mails received via
SMTP per day}{Number of mails received per day: statistics}

When generating the timing data used in this section, \numberOFlogFILES{}
log files (totaling 10.08 GB, \numberOFlogLINEShuman{} log lines) were each
parsed 10 times and the parsing times averaged.  Saving results to the
database was disabled for the test runs, as that dominates the run time of
the program, and the tests are aimed at measuring the speed of the parser
rather than the speed of the database and the disks the database is stored
on.  The computer used for test runs was a Dell Optiplex 745, equipped
with: 

\begin{tabular}[]{ll}

    CPU         & One dual core 2.40GHz Intel\textregistered{}
                    Core\texttrademark{}2 CPU, \\
                & with 32KB L1 cache and 4MB L2 cache. \\
    RAM         & 2GB 667 MHz DDR RAM\@. \\
    Hard disk   & One Seagate Barracuda 7200 RPM 250GB SATA disk. \\

\end{tabular}

Parsing all \numberOFlogFILES{} log files in one run took
\input{build/include-timing-run-duration}.  The computer was dedicated to
the task of gathering statistics from test runs, and was not used for any
other purpose while test runs were ongoing; any services not necessary for
running the tests were disabled.

\clearpage{}

\subsection{Architecture Scalability: Input Size}

An important property of a parser is how parsing time scales relative to
input size: does it scale linearly, polynomially, or exponentially?
\Graphref{parsing time vs file size vs number of log lines graph} shows the
parsing time in seconds, file size in MB, and tens of thousands of log
lines for each of the \numberOFlogFILES{} log files.  All three lines run
roughly in parallel, giving the impression that the algorithm scales
linearly with input size.  This impression is borne out by
\graphref{parsing time vs file size vs number of log lines factor} which
plots both the ratio of file size vs parsing time, and the ratio of number
of log lines vs parsing time (higher is better in both cases);
\tableref{parsing time vs file size vs number of log lines factor table}
shows the same ratios for different groups of log files.  The ratios are
quite tightly banded, providing empirical evidence that the algorithm
scales linearly.  The ratio increases (i.e.\ improves) for log files 22 and
62--68 despite their larger than average size (shown in \graphref{parsing
time vs file size vs number of log lines graph}); this is explained fully
in \sectionref{Explaining the peaks in log file size}.  

\showtable{build/include-file-size-and-number-of-log-lines-vs-parsing-time}{Ratio
of file size and number of log lines to parsing time}{parsing time vs file
size vs number of log lines factor table}

\showgraph{build/graph-input-size-vs-parsing-time}{Parsing time, file size,
and number of log lines}{parsing time vs file size vs number of log lines
graph}

\showgraph{build/graph-input-size-vs-parsing-time-ratio}{Ratio of file size
and number of log lines to parsing time}{parsing time vs file size vs
number of log lines factor}

\subsection{Explaining the peaks in log file size}

\label{Explaining the peaks in log file size}

\Graphref{parsing time vs file size vs number of log lines graph} shows log
file size varying between 30--100MB, except for log files 22 (200MB) and
62--68 (400--900MB).  These large log files were caused by a user
unintentionally creating a mail forwarding loop, resulting in a huge number
of mail deliveries and up to an order of magnitude more log lines than is
usual.  The user specified his mail forwarding as: \newline{}
\tab{}\texttt{$\backslash$username, username@domain} \newline{} This
instructs Postfix to deliver the mail to the local user, and also forward a
copy to the remote address; this is generally not a problem except that the
remote address in this instance is the address the mail was originally sent
to, creating an infinite loop.  To prevent this happening Postfix examines
the Delivered-To header in the mail, and if the mail has already been
delivered to the current address it is bounced back to the sender with the
error message \newline{} \tab{} \texttt{mail forwarding loop for
username@domain}\newline{}  Ordinarily this works well, but unfortunately
in this instance the user noticed they had not received any mail in a while
and opted to send a test mail to themselves, causing a loop not caught by
Postfix:

XXX REWRITE THIS SECTION SO THAT IT IS CORRECT\@: POINT 4 SAYS THAT THE
ENVELOPE SENDER CHANGES, IS THAT CORRECT\@?

\begin{enumerate}

    \item Postfix accepts a mail from username@domain, for username@domain.

    \item Postfix delivers the mail to the local mailbox and
        username@domain, as instructed by the user's forwarding
        instructions. The forwarded mail has a
        \texttt{Delivered-To:~username@domain} header added, and the
        envelope sender address is username@domain.

    \item Postfix accepts the mail for username@domain, but during delivery
        it notices that the \texttt{Delivered-To} header already contains
        the address it is currently delivering to, and therefore sends a
        bounce notification with sender address \texttt{<>}\glsadd{<>} to
        the original sender: username@domain.

    \item Postfix accepts the bounce notification and delivers it to both
        the local mailbox and to username@domain, as instructed by the
        user's forwarding instructions.  A
        \texttt{Delivered-To:~username@domain} header is added to the
        forwarded bounce notification, XXX which now has an envelope sender
        address of username@domain.   XXX IS THIS CORRECT\@?

    \item Postfix accepts the forwarded bounce notification but during mail
        delivery it notices that the \texttt{Delivered-To} header already
        contains the address currently being delivered to, and sends a
        bounce notification to the sender: username@domain.

    \item Repeat from step two; this will continue indefinitely unless an
        administrator intervenes and deletes the appropriate mails from the
        queue.

\end{enumerate}

The sequence described above occurs extremely rapidly because Postfix does
not have to deliver the mail to an external system, so mails are delivered,
bounced, and generated as fast as the disks can keep up, resulting in a
huge volume of log lines.

The different distribution of log lines caused by a mail forwarding loop
has unexpected effects on parsing time, described in detail in
\sectionref{rule ordering for efficiency}, \sectionref{scalability as the
number of rules rises}, and \sectionref{Caching compiled regexes}.

\subsection{Rule ordering for efficiency}

\label{rule ordering for efficiency}

At the time of writing there are \numberOFrules{} different rules: the top
10\% match the vast majority of log lines, with the remaining log lines
split across the other 90\% of the rules (as shown in \graphref{rule hits
graph}).  Assuming that the distribution of log lines is reasonably
consistent over time, parser efficiency should benefit from trying more
frequently matching rules before those that match less frequently.  To
test this hypothesis three full test runs were performed with different
rule orderings:

\begin{eqlist}

    \item [optimal]  The most optimal order, according to the hypothesis:
        rules that match most often will be tried first.

    \item [shuffle] This is intended to represent a randomly ordered rule
        set.  The rules will be shuffled once before use and will retain
        that ordering until the parser exits.  Note that the ordering will
        change every time the parser is executed, so 10 different rule
        orderings will be generated for each log file in the test run.  

    \item [reverse] Hypothetically the worst order: the most frequently
        matching rules will be tried last.

\end{eqlist}

\Graphref{Parsing time relative to shuffled ordering graph} shows the
parsing times of optimal and reverse orderings relative to shuffled
ordering; the mean relative parsing times for different groupings of log
files are given in \tableref{Parsing time relative to shuffled ordering
table}.  Overall this optimisation provides a modest but worthwhile
performance increase of approximately 10\%, for a small investment in time
and programming.

\showgraph{build/graph-optimal-and-reverse-vs-shuffle}{Parsing time
relative to shuffled ordering}{Parsing time relative to shuffled ordering
graph}

\showtable{build/include-optimal-and-reverse-vs-shuffle}{Parsing time
relative to shuffled ordering}{Parsing time relative to shuffled ordering
table}

Differences in rule ordering have less effect on parsing time when parsing
log files 22 and 62--68, due to the different distribution of log lines in
those log files (see \sectionref{Explaining the peaks in log file size}).
When a mail loop occurs the vast majority of log lines are from Postfix
components that have few rules associated with them, whereas in general the
most log lines are from Postfix components that have a high number of
associated rules; this results in the average number of rules required to
parse a log line from log files 22 or 62--68 being considerably lower than
usual.  The ratio of number of log lines versus parsing time in
\graphref{parsing time vs file size vs number of log lines factor} shows a
noticeable improvement at log files 22 and 62--68, where a higher ratio
means more log lines parsed per unit of time; similarly \graphref{Parsing
time relative to shuffled ordering graph} shows a much smaller difference
between the orderings at log files 22 and 62--68.

\subsection{Scalability as the number of rules rises}

\label{scalability as the number of rules rises}

How any architecture scales as the number of rules increases is important,
but it is particularly important in this architecture because it is
expected that the typical parser will have a large number of rules.  There
are \numberOFrules{} rules in the full Postfix ruleset, whereas the minimum
number of rules required to parse the \numberOFlogFILES{} log files used
when generating the results in this thesis is \numberOFrulesMINIMUM{},
\numberOFrulesMINIMUMpercentage{} of the full ruleset.  The full ruleset is
larger because the Postfix parser is tested with \numberOFlogFILESall{} log
files, to improve its parsing and ensure it correctly parses logs generated
by later Postfix versions.  A second set of statistics was generated using
the minimum ruleset and compared to the statistics generated using the full
ruleset: the percentage parsing time increase when using the full ruleset
instead of the minimal ruleset for optimal, shuffled, and reversed
orderings is shown in \graphref{Percentage parsing time increase of maximum
ruleset over minimum ruleset}.

\showgraph{build/graph-full-ruleset-vs-minimum-ruleset}{Percentage parsing
time increase of maximum ruleset over minimum ruleset}{Percentage parsing
time increase of maximum ruleset over minimum ruleset}

It is clear from \graphref{Percentage parsing time increase of maximum
ruleset over minimum ruleset} that the increased number of rules noticeably
increases parsing time with reverse ordering, and to a lesser extent with
shuffled ordering.  The optimal ordering (where the most frequently
matching rules are tried first) shows a mean increase of
\input{build/include-full-ruleset-vs-minimum-ruleset} in parsing time for a
\numberOFrulesMAXIMUMpercentage{} increase in the number of rules.  Log
files 22 and 62--68 show much smaller increases for shuffled and reversed
ordering than other log files do, because the majority of log lines in
those log files are parsed by Postfix components with a small number of
parsing rules, so removing unnecessary rules has little effect on the total
number of rules used.  These results show that the architecture scales very
well as the number of rules increases, and that optimally sorting the rules
is an important optimisation contributing to this scalability.

\subsection{Caching compiled regexs}

\label{Caching compiled regexes}

Before the perl interpreter attempts to match a regex against a piece of
text, the regex is compiled into an internal representation and
optimised to improve the speed of matching.  This compilation and
optimisation takes CPU time ---in many cases it takes far more CPU time
than the actual matching.  If the interpreter is certain that a regex
will not change it will automatically cache the compilation results for
later use.   The results of compiling a dynamically generated regex can
be cached and used in preference to the original regex, but it is the
responsibility of the programmer to do this.  The Postfix parser loads the
rules it uses from a database, and thus each rule's regex is compiled
and the result cached to reduce parsing time.

\Graphref{Increase in parsing time when not caching compiled regexes graph}
shows the impact that not caching compiled regexes has on parser
performance: on typical log files the parsing time when not caching
compiled regexes is 400--600\% of the parsing time when caching.
Caching compiled regexes is probably the single most effective
optimisation possible in the parser's implementation, and was quite simple
to implement.

\showgraph{build/graph-cached-regexes-vs-discarded-regexes}{Increase in
parsing time when not caching compiled regexes}{Increase in parsing time
when not caching compiled regexes graph}

Two large dips can be seen in \graphref{Increase in parsing time when not
caching compiled regexes graph} at log files 22 and 62--68, corresponding
to the spikes in log file size in \graphref{parsing time vs file size vs
number of log lines graph}.  The distribution of log lines across rules
when there is a mail loop is much different to the norm, and the average
number of rules consulted per log line is much lower: this results in far
fewer regex compilations per line when there is a mail loop, and a
corresponding decrease in the average parsing time for a single log line.
The outcome is that caching compiled regexes is proportionally less
important when the log file contents were created by a mail loop.  The
increases in parsing time when not caching compiled regexes for different
groups of log files are summarised in \tableref{Increase in parsing time
when not caching compiled regexes table}.

\showtable{build/include-cached-regexes-vs-discarded-regexes} {Increase in
parsing time when not caching compiled regexes} {Increase in parsing time
when not caching compiled regexes table}


\subsection{Summary}

This section began with parser scalability, showing the linear relationship
between parsing time and input size.  Demonstrates the effect of rule
ordering on parsing time, and the unexpected consequences of specific
inputs.  The necessity of caching compiled regexes is attested to by the
third group of graphs, where the difference between caching and not caching
is staggering.  The penultimate section contains a breakdown of the rule
hits accumulated during a single test run of the parser.  Miscellaneous
graphs expected to be useful are collected in the final section.  

\section{Coverage}

XXX I THINK THIS SECTION NEEDS SUBSTANTIAL WORK, IF NOT A RE-WRITE\@; START
WITH THE PAPER'S CONTENT\@.

\label{parsing coverage}

The discussion of the parser's coverage of Postfix log files is separated
into two parts: log lines covered and mails covered.  The first is
important because the parser should handle all (relevant) log lines it is
given; the second is equally important because the parser must properly
deal with every mail if it is to be useful.  Improving the former is
less intrusive, as it just requires new rules to be written; improving the
latter is much more intrusive as it requires changes to the parser
algorithm, and it can also be much harder to notice a deficiency.

\subsection{Log lines covered}

\label{log-lines-covered}

Parsing a log line is a three stage process:

\begin{enumerate}

    \item Check if there are any rules for the Postfix component that
        produced the log line; if not then skip the log line.

    \item Try each rule until a matching rule is found; if no match is
        issue a warning and move to the next log line.

    \item Execute the action specified by the rule.

\end{enumerate}

Full coverage of log lines requires the following:

\begin{enumerate}

    \item Each Postfix component whose log lines are of interest must have
        at least one rule or its log lines will be silently skipped; in the
        extreme case of zero rules the parser would happily skip every log
        line.  There may be any number of log lines from other programs
        intermingled in the log file, and there are some Postfix programs
        that do not produce any log lines of interest.

    \item There must be a rule to match each different log line produced by
        each program; if a log line is not successfully matched the parser
        will issue a warning.  Rules should be as specific and tightly
        bound as possible to ensure accurate parsing:\footnote{A rule that
        matches zero or more of any character will successfully parse every
        log line, but not in a meaningful way.} most log lines contain
        fixed strings and have a rigid pattern, so this is not a problem.

    \item The appropriate action to take --- discussed in
        \sectionref{mails-covered}.

\end{enumerate}

Full coverage of log lines is easy to achieve yet hard to maintain.  It is
easy to achieve full coverage for a limited set of log files (at the time
of writing the parser is tested with \numberOFrules{} rules, fully parsing
\numberOFlogFILESall{} contiguous log files from Postfix 2.2 through to
Postfix 2.5), and new rules are easy to add.  Maintaining full coverage is
hard because other servers have different restrictions with custom
messages, \gls{DNSBL} messages change over time, major releases of Postfix
change warning messages (usually adding more information), etc.,\ so over
time the log lines drift and change.  \Graphref{rule hits graph} shows the
number of hits for each rule over all \numberOFlogFILES{} log files.  The
number of hits per rule is quite unevenly spread, resembling a Power Law
distribution --- it is obvious that a small number of rules match the vast
majority of the log lines, and more than half the rules match fewer than
100 times.

\showgraph{build/graph-hits}{Hits per rule}{rule hits graph}

Warnings are issued for any log lines that are not parsed; no warnings are
issued for unparsed log lines while testing with the \numberOFlogFILES{}
test log files, so it can be safely concluded that there are zero false
negatives.  False positives are harder to quantify: short of examining each
of the 60,721,709 log lines and determining which regex parsed it, there
is no way to be sure that every line was parsed by the correct regex,
making it impossible to quantify the false positive rate; however a random
sample of 6,039 log lines was parsed and the results checked manually to
ensure that the correct regex parsed each log line.\footnote{Each log
line was examined and the correct regex identified from the
\numberOFrules{} rules in the database; the correct regex was then
compared to the regex that was used by the parser.}  The sample was
generated by running the following command:

\verb!    perl -e 'print if (rand 1 < 0.0001)' -n LOG_FILES!

\noindent{}to randomly extract roughly one line in every 10,000.  Although
on initial appearances exercising only 36 rules (from a total of
\numberOFrules{}) when parsing 6039 log lines seems quite low, after
examining \graphref{rule hits graph} it becomes apparent that such
a low hit rate is to be expected; the reader should also bear in mind that
even when parsing all \numberOFlogFILES{} log files not all the rules are
exercised (some of the rules are for parsing log lines that only appear in
other log files).

\subsection{Mails covered}

\label{mails-covered}

Coverage of mails is much more difficult to determine accurately than
coverage of log lines.  The parser can dump its state tables in a human
readable form; examining these tables with reference to the log files is
the best way to detect mails that were not handled properly (many of the
complications discussed in \sectionref{complications} were
detected in this way).  The parser issues warnings when it detects any
errors, some of which may alert the user to a problem, e.g.\ when a queueid
is reused before the previous mail is fully dealt with, when a queueid or
\gls{pid} is not found in the state tables,\footnote{There will often be
warnings about a missing queueid or \gls{pid} in the first few hundred or
thousand log lines because the earlier log lines for those connections or
mails are in the previous log file; loading the saved state from the
previous log file will solve this problem.} or when there are problems
tracking a child mail (see \sectionref{tracking re-injected mail}).  There
should be few or no warnings when parsing, and when finished parsing the
state table should only contain entries for mails that had yet to be
delivered when the log files ended, or were accepted before the log files
began.

At the time of writing the parser is being tested with \numberOFlogFILES{}
log files.  There are 5 warnings produced, but because the parser errs on
the side of producing more warnings rather than fewer, those 5 warnings
represent 3 instances of 1 problem: 3 connections started before the first
log file, so their initial log lines are missing, leading to warnings when
their remaining log lines are parsed.

The state tables contain entries for mails not yet delivered when the
parser finishes execution.  Ideally all they should contain are mails that
are awaiting delivery after the period covered by the log files, though
they may also contain mails whose initial entries are not contained in the
log files.  Any other entries are evidence of a failure in parsing or an
aberration in the log files.  After parsing the \numberOFlogFILES{} test
log files the state tables contain 18 entries, breaking down into:

\begin{itemize}

    \item 1 connection that started only seconds before the log files
        ended and had not yet been fully transferred from client to server.

    \item 1 mail that had been accepted only seconds before the log files
        ended and had not yet been delivered.

    \item 9 mails whose initial log lines were not present in the log
        files.

    \item 7 mails that had yet to be delivered due to repeated failures.

\end{itemize}

There are no mails in the state tables which should not be present, thus it
can be concluded that there are zero false negatives.  Once again,
determining the false positive rate is much harder, as manually checking
the results of parsing 13,850,793 connections and mails accepted, rejected,
bounced or delivered is infeasible.  There is considerable circumstantial
evidence that the false positive rate is quite low:

\begin{itemize}

    \item The parser is quite verbose when complaining about known problems
        (e.g.\ if a mail is tracked twice as described in
        \sectionref{tracking re-injected mail}), and no such warnings are
        produced during the test runs.

    \item Queueids and \glspl{pid} naturally group together log lines
        belonging to one mail or connection respectively; it is extremely
        unlikely that a log line would be associated with the wrong
        connection.

    \item When dealing with the complications described in
        \sectionref{complications} the solutions are as specific and
        restrictive as possible, with the goal of minimising the number of
        false positives.  In addition the solution to the \textit{Out of
        order log files\/} complication described in \sectionref{out of
        order log lines} imposes conditions that each reassembled mail must
        comply with to be acceptable.

    \item Every effort has been made while developing to make the parser as
        precise, demanding, and particular as possible.

\end{itemize}

Whereas verifying by inspection that the parser correctly deals with all
60,721,709 log lines in the test log files is infeasible, verifying a
subset of those log files is a tractable, if extremely time consuming,
task.  A sample of log lines was obtained by randomly selecting a log file:

\verb!    perl -Mstrict -Mwarnings -MList::Util=shuffle \!\newline{}
\verb!            -e 'print [shuffle(@ARGV)]->[0];'!

The first 6000 log lines of this log file (roughly 0.01\% of the total
number of log lines used in testing) was extracted:

\verb!    sed -n -e '1,6000p' logfile > test-log-segment!

It is important that the log lines used are contiguous so that all log
entries are present for as many of the connections as possible.  This log
segment was parsed with all debugging options enabled, resulting in 167,448
lines of output.\footnote{A mean of 27.908 lines of output per line of
input; each connection has 30 debugging lines, plus 21 debugging lines per
result.  Connections which have been cloned will have the cloned connection
in their debugging output, plus another 33 debugging lines.  Those numbers
are approximate, and may vary $\pm{}$ 2.  There is a linear relationship
between the number of log lines and debugging lines: $33(connections) +
30(accepted~~mails) + 21(results)$.  This formula is an approximation only,
and has not been rigorously verified.}  All 167,448 lines were examined in
conjunction with the input log file and a dump of the resulting database,
verifying that for each of the log lines the parser used the correct rule
and executed the correct action, which in turn produced the correct result
and inserted the correct data in the database.  The log file segment
produced 4 warnings, 10 mails remaining in the state tables, and 1625
connections correctly entered in the database.

Given the circumstantial and experimental evidence detailed above, the
author is confident that the false positive rate when reconstructing a mail
is exceedingly low, if not approaching zero.

\section{Summary}

XXX EXTEND TO SUMMARISE PARSER EFFICIENCY TOO\@.

XXX REWRITE COVERAGE SUMARY IF NECESSARY\@.

Parser coverage is divided into two topics in this section: log lines
covered and mails covered.  The former is initially more important, as the
parser must successfully parse every line if it is to be complete, but
subsequently the latter takes precedence because reproducing the path a
mail takes through Postfix is the aim of the parser.  Increasing the
percentage of log lines parsed is relatively simple and non-intrusive:
adding new rules or modifying existing rules is simplified by the
separation of rules, actions, and framework.  Improving the logical coverage
is harder, as the actions taken by Postfix must be reconstructed by the
author, and the new sequence of actions integrated into the existing model
without breaking the existing parsing.  Detecting a deficiency in the
parsing algorithm is also significantly harder than detecting unparsed log
lines, as the parser will warn about any unparsed line, whereas discovering
a flaw in the parser requires understanding of the warnings produced and
the mails remaining in the state table.  Rectifying a flaw in the parser
requires an understanding of both the parser and Postfix's log files, and
investigative work to determine the cause of the deficiency, followed by
further examination of the log files in developing a solution.


\section{Conclusion}

\label{conclusion}

Parsing Postfix log files appears at first sight to be an almost trivial
task, especially if one has previous experience in parsing log files, but
it turns out to be a much more taxing project than initially expected.  The
variety and breadth of log lines produced by Postfix is quite surprising,
because a quick survey of sample log files gives the impression that the
number of distinct log lines is quite small; this impression is due to the
uneven distribution exhibited by log lines produced in normal operation
(see \graphref{rule hits graph} for a vivid illustration of this).


Given the diverse nature of the log lines and the ease with which
administrators can cause new lines to be logged (\sectionref{postfix
background}), enabling users to easily extend the parser to deal with new
log lines is a design imperative (\sectionref{parser design}).  Despite the
resulting initial increase in complexity the task is quite tractable,
though it does raise efficiency and optimisation questions (answered in
\sectionref{parser efficiency}).  Providing a tool (\sectionref{creating
new rules}) to ease the generation of \regexes{} from unparsed log lines
should greatly help users add rules to parse formerly unparsed log lines.


The implementation not only substantially eases the parsing of new log
lines, it makes adding new actions (\sectionref{adding new actions}) a
relatively simple task.  The simplicity of adding a new action frees the
implementor from worrying about how their action might disrupt parsing of
other log lines or the behaviour of other actions, allowing their
concentration to focus on correctly implementing their new action.


The division of the parser into rules, actions and framework is unusual
because rules are separated so completely from the actions and framework;
although parsers are often divided (parsers based on the combination of
\texttt{lex} and \texttt{yacc}~\cite{lex-and-yacc} 
being an obvious example), the parts are usually quite internally
interdependent, and will be combined into a complete parser by the
compilation process; in contrast \parsername{} keeps the rules and actions
separate until the parser runs.  The actions and framework are not as
completely separated, as the actions depend on services provided by the
framework; however the actions and framework are not tightly integrated: it
would be possible, with some work, to completely separate the two.


The emergent behaviour~\cite{Wikipedia-Emergence} (\sectionref{Emergent
behaviour}) exhibited by the rules and actions is also interesting, and is
discussed after the flow chart (\sectionref{flow-chart}) and explanation of
the paths through the parser.  This emergent behaviour greatly eases the
process of adding new actions (\sectionref{adding new actions}), as the
actions do not have to be inserted into an explicit flow of control; new
actions will naturally find their place in the paths through the parser.

The flow of control in this parser is quite different from that of most
other parsers.  In most traditional parsers the parser has a current state,
and each state has a fixed set of acceptable next states, with unacceptable
states causing parsing to fail --- e.g.\ when a \textbf{C} parser sees the
keyword \textbf{for} it expects to immediately see a left parenthesis, with
any other input causing parsing to fail.\footnote{Comments will already
have been removed by the preprocessor.}  This parser is different: the
rule which matches the next log line for a mail dictates the action that
will be invoked, which is equivalent to the next state in other parsers;
thus the next log line for a mail dictates the next state for that mail.

The real difficulties arise once the parser is successfully dealing with
90\% of the log lines, as the irregularities and complications previously
explained only begin to become apparent once the vast majority of log lines
have been parsed successfully.  Adding new rules to deal with numerous,
infrequently occurring log lines is a simple task, albeit tiresome (though
alleviated by the tool described in \sectionref{adding new actions}),
whereas dealing with mails where information appears to be missing is much
more grueling.  Trawling through the log files, looking for something out
of the ordinary, possibly hundreds or even thousands of lines away from the
last mention of the queueid in question, is extremely time consuming and
error prone.  Sometimes the task is not to spot the line that is unusual,
but to spot that a line normally present is missing, i.e.\ to realise that
one line amongst hundreds is absent.  In all cases the evidence must be
used to construct a hypothesis to explain the irregularities, and that
hypothesis must then be tested in the parser; if successful, the parser
must be modified to deal with the irregularities, without adversely
affecting the existing functionality.  The complications described in
\sectionref{additional complications} were solved in the order they are
described in, and that order closely resembles the frequency with which
they occur; the most frequently occurring complications dominate the
warning messages produced, and so naturally they are the first
complications to be dealt with.

A parser's ability to correctly parse its input is extremely important; the
parser's coverage of its test log files is discussed in \sectionref{parsing
coverage}.  Both its success at parsing individual log lines
(\sectionref{log-lines-covered}) and its correctness in reconstructing each
mail's journey through Postfix (\sectionref{mails-covered}) are described
in detail, including the results of manually verifying the correct parsing
of a subset of the test log files.

\newpage




\appendix

\bibliographystyle{logparser-bibliography-style}
\bibliography{logparser-bibliography}

% Add some glossary entries that should be present, but lack an appropriate
% place in the text to mark them.
\glsadd{queueid}
\renewcommand{\glossarypostamble}{\label{Glossary}}
\printglossary[style=nospacelist]{}
\renewcommand{\glossarypostamble}{\label{Acronyms}}
\printglossary[type=\acronymtype,style=eqlist]{}
\renewcommand{\glossarypostamble}{\label{Postfix Daemons}}
\printglossary[type=postfix,style=nospacelist]{}

\end{document}
