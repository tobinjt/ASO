\documentclass[a4paper,12pt,draft]{article}

% Better typesetting of URLs.
\usepackage{url}
\usepackage[final]{graphicx}

\begin{document}

\title{Parsing Postfix log files}
\author{John Tobin \\ Computer Science Department \\ Trinity College \\
Dublin 2 \\ Ireland \\ tobinjt@cs.tcd.ie}
\date{}
\maketitle

\section{Abstract}

Most mail server administrators will have performed some basic processing
of Postfix logs at one time or another, whether it was to figure out why
something isn't working, why mail is being rejected, or to check whether
their new anti-spam measures are working.  The more adventurous will have
generated some statistics to show how many hits each of their anti-spam
measures have gotten in the last week, and possibly even generated some
graphs to clearly illustrate the point to management or complaining
users\footnote{This was the author's first real foray into processing
Postfix logs}.  Very few will have performed in-depth parsing and analysis
of their logs, where the parsing needs to be per-connection or per-queueid,
rather than per-line.  One of the barriers to this type of processing is
the unstructured nature of Postfix's logs: each logging line was added on
an ad hoc basis as Wietse found a requirement or added new functionality.
This paper attempts to document the difficult process of parsing Postfix
logs, and presents a program which parses logs and places the resulting
data into a database for further processing.


\section{Background}

Postfix is a highly configurable, high performance, secure and scalable
Mail Transport Agent (MTA),  It features extensive anti-spam restrictions,
allowing a mail administrator to deploy those restrictions which they judge
suitable for their needs, rather than a fixed set chosen by Postfix's
author.  These restrictions can be applied and combined on a per-host,
per-recipient or per-sender basis, allowing clients/recipients to be
whitelisted with respect to specific restrictions.  Postfix leverages
simple lookup tables to support arbitrarily complicated user-defined
sequences of checks and exceptions, with the ultimate in flexibility being
the facility to consult an external process which can implement whatever
logic is required; if you wish you can restrict some users to sending mail
on the third Tuesday after pay day only, if that is of any use to you.
Administrators can also supply their own rejection messages, to make it
clear to senders why exactly their mail was rejected.  Unfortunately this
flexibility has a cost: complexity in the logs generated.  While it is easy
to use \texttt{grep(1)} to determine the fate of an individual email, and
even to extend that into generating some useful graphs, following the
progress of an email through Postfix can be quite difficult.  The logs tend
to follow a 90\%/10\% pattern: 90\% of the time the logging is simple, and
the other 10\% of the logging requires 90\% of the code.

This paper and the program it describes are part of a larger project to
optimise your Postfix restrictions, generate statistics and graphs, and
provide a platform on which new restrictions can be trialled and evaluated
to see if they are worth using in the fight against spam.  The program
parses Postfix logs and populates a database with the data gleamed from
those logs, providing a consistent and simple view of your logs which
future tools can utilise.  


\section{Parsing details}

The complexity and variation in Postfix's logs requires similar flexibility
in the parser; decoupling the parsing rules from the associated actions
allows the rules to be loaded from an external source without modifications
to the parser source code (significantly lowering the barrier to entry for
new users), and simplifies both parser and rules.  The separation also
allows for a XXX REWRITE separation of functionality: rules handle low
level details of identifying a line, whereas the parser handles the higher
level details of following the path a mail takes through postfix,
assembling the required information, etc.

\subsection{Rules}

Rules are relatively simple, defining the following for each rule:

\begin{description}

    \item [Name] A short name for the rule.

    \item [Description] A longer, more detailed description of the rule.

    \item [Program] The program (smtp, smtpd, qmgr, etc) which the rule
        applies to.  This avoids needlessly trying rules which won't
        match, or worse, might match unintentionally.

    \item [Regex] The regex to compare the log line against.  The regex
        will first have several keywords expanded; this simplifies writing
        rules and avoids needless repetition of complex regexs.

    \item [Result\_cols and connection\_cols] This is how we extract the
        fields in the log line matched by the regex.  The format is:
        \newline 
        hostname = 1; helo = 2; sender = 4; \newline
        i.e. semi-colon separated assignment statements, with the column
        name on the left and the match from the regex (\$1, \$2 etc) on the
        right hand side (without \$).

    \item [Action] The action for the parser to take.

    \item [Queueid] The index of the field in regex giving the queueid.

    \item [Rule\_order] This is used to sort the rules, so that rules
        matching more commonly occurring log lines will be tried first, and
        is updated every time the program is run.

    \item [Priority] This is the user-configurable companion to
        rule\_order: rules with a higher priority will be tried first,
        overriding rule\_order, allowing more specific rules to take
        precedence over more general rules.

    \item [Result]  The result the log line represents: rejection,
        acceptance, info, etc.

    \item [Result\_data and connection\_data] Sometimes rules need to fake
        a piece of data which isn't present in the log line: the most
        common would be smtp\_code when mail is accepted.  The format is
        the same as result\_cols and connection\_cols.

\end{description}


\subsection{Algorithm}

Lets start with a flow chart describing the program:

XXX FLOW CHART GOES HERE

%\includegraphics{logparser-flow-chart.ps}

A high level view of the algorithm could be expressed as:

\begin{enumerate}

    \item Mail enters the system via SMTP or local submission.
    \item If the mail is rejected, log all data, and finish with the mail.
    \item Follow the progress of the mail until it's either delivered or
        bounced, log all data, and finish with the mail.

\end{enumerate}

Unfortunately it's not that easy.

\subsubsection{Complications}

\begin{enumerate}

    \item The mail lacks a queueid until the mail has been accepted, so log
        lines must first be correlated by the smtpd pid, then transition to
        being correlated by the queueid.  This is relatively minor, but
        does require two versions of several functions,
        \texttt{per\_pid} and \texttt{per\_queueid}.

    \item Multiple independent mails may be delivered during one
        connection; this requires cloning the current data, so that
        subsequent mails won't trample over each other.  This must be done
        every time a mail is accepted, as it's impossible to tell in
        advance which connections will accept multiple mails.  Happily once
        the mail has been accepted log entries won't be correlated by pid
        for that mail any more, so there isn't any ambiguity about which
        mail a given log line belongs to.  One possible (as yet unseen)
        difficulty is distinguishing between rejects which belong to the
        current mail, and rejects which belong to previous, unaccepted,
        mails.

    \item The most difficult complication is that mails are not always
        delivered directly to a mailbox: sometimes they are aliased and
        need to be delivered to an address (or addresses) on another
        server.  When this occurs a child mail will be injected into the
        postfix queue, but without the explicit logging smtpd or sendmail
        injected mails have, so the source is not immediately discernible;
        from a strictly linear reading of the logs it \textit{usually}
        appears as if the child mail has appeared from thin air.
        Subsequently the parent mail will log the creation of the child
        mail:

        \texttt{3FF7C4317: to=<username@cs.tcd.ie>, relay=local, \newline delay=0,
        status=sent (forwarded as 56F5B43FD)}

        Unfortunately while all log lines from an individual process appear
        in the logs in monotonic order, the order in which log lines from
        different processes are interleaved is subject to the vagaries of
        process scheduling.  In addition the first log line belonging to
        the child mail (the log line cited above properly belongs to the
        parent mail) is logged by qmgr, so depending on how busy qmgr (and
        the server generally) is, the parent's creation log line may appear
        before or after (generally after) the child's log line from qmgr.
        This has the effect that the parser cannot complain when it
        encounters a log line from qmgr for a previously unseen mail; it
        must mark the mail as potentially fake and subsequently clear the
        faked flag if and when the origin of the mail becomes clear.
        Obviously the parser could omit checking of where mails originate
        from, but I feel that it is better to require an explicit source,
        as bugs are more likely to be exposed.

        Process scheduling can have a still more confusing effect: quite
        often the child mail will be created, delivered and removed
        \textbf{before} the parent logs the creation line!  Thus mails
        marked as faked cannot be committed, instead they must be marked as
        commit ready and subsequently committed by the parent mail.  Quite
        apart from identifying mail injected in an unknown fashion, or bugs
        in the parser, faked mails need to be marked as such because they
        lack some data present in the parent mail, and must copy that data
        from their parent.


\end{enumerate}

XXX TODO what happens to the original connection?? XXX

XXX TODO is it possible to distinguish between different rejected mails on
one connection??  XXX

\appendix



\end{document}
