\documentclass{beamer}
%\mode<presentation>
%{
%  \usetheme{Warsaw}
%  % or ...
%
%  \setbeamercovered{transparent}
%  % or whatever (possibly just delete it)
%}
\useoutertheme{infolines}

\usepackage[english]{babel}
% or whatever

\usepackage[latin1]{inputenc}
% or whatever

\usepackage{times}
\usepackage[T1]{fontenc}
% Or whatever. Note that the encoding and the font should match. If T1
% does not look nice, try deleting the line with the fontenc.

\title{A User-Extensible and Adaptable Parser Architecture}

\author{John Tobin \and Carl Vogel}
% - Give the names in the same order as the appear in the paper.
% - Use the \inst{?} command only if the authors have different
%   affiliation.

\institute[Trinity College]
{
  School of Computer Science and Statistics\\
  Trinity College, University of Dublin
}

\date[SGAI 2008]{Twenty-eighth SGAI International Conference on Artificial Intelligence}

\subject{Theoretical Computer Science}

% If you have a file called "university-logo-filename.xxx", where xxx
% is a graphic format that can be processed by latex or pdflatex,
% resp., then you can add a logo as follows:

% \pgfdeclareimage[height=0.5cm]{university-logo}{university-logo-filename}
% \logo{\pgfuseimage{university-logo}}



%% Delete this, if you do not want the table of contents to pop up at
%% the beginning of each subsection:
%\AtBeginSubsection[]
%{
%  \begin{frame}<beamer>{Outline}
%    \tableofcontents[currentsection,currentsubsection]
%  \end{frame}
%}


% If you wish to uncover everything in a step-wise fashion, uncomment
% the following command: 

%\beamerdefaultoverlayspecification{<+->}


\begin{document}

\begin{frame}
    \titlepage{}
\end{frame}

% XXX MAYBE REMOVE THIS
\begin{frame}{Outline}
    \tableofcontents{}
    % You might wish to add the option [pausesections]
\end{frame}


% Structuring a talk is a difficult task and the following structure
% may not be suitable. Here are some rules that apply for this
% solution: 

% - Exactly two or three sections (other than the summary).
% - At *most* three subsections per section.
% - Talk about 30s to 2min per frame. So there should be between about
%   15 and 30 frames, all told.

% - A conference audience is likely to know very little of what you
%   are going to talk about. So *simplify*!
% - In a 20min talk, getting the main ideas across is hard
%   enough. Leave out details, even if it means being less precise than
%   you think necessary.
% - If you omit details that are vital to the proof/implementation,
%   just say so once. Everybody will be happy with that.

\section{Background}

\begin{frame}{Background}

    \begin{description}

        \item [Aim] To improve SMTP-time anti-spam restriction / measures /
            techniques.

        \item [Method] Parse Postfix log files and analyse the data
            extracted.

    \end{description}

    This requires a parser for Postfix log files.

\end{frame}

\section{XXX}

\begin{frame}{XXX}

    \begin{itemize}

        \item Postfix log lines change over time and from site to site, so
            it must be easy for the end user to parse new lines.

        \item The processing required for a type of line (e.g.\ remote
            server connecting to send mail) rarely changes, but it can be
            quite complex.

        \item By separating parsing of lines from processing of lines, we
            can make parsing new lines as easy as possible, while still
            allowing for complicated processing when necessary.

    \end{itemize}

\end{frame}


\section{Architecture}

\begin{frame}{Architecture}

    \begin{description}

        \item [Framework] The framework handles loading rules and providing
            services to actions, e.g.\ data storage.  The framework tries
            each rule in turn until one matches, then it invokes the action
            specified by the rule.

        \item [Actions] Each action performs the processing required, e.g.\
            XXX GIVE SOME EXAMPLES\@.

        \item [Rules] Each rule matches one type of log line, e.g.\ the log
            line resulting from client connecting, a mail successfully
            delivered, or a mail rejected by an anti-spam measure.

    \end{description}

\end{frame}

\section{Features}

\begin{frame}{Features}

    \begin{itemize}

        \item Cascaded parsing.

        \item The parser does not need to be recompiled when rules are
            changed.

        \item Different rulesets can easily be used.

        \item Rules can be added by actions while the parser is running,
            allowing dynamically changing parsing, e.g.\ macros.

        \item XXX MORE FEATURES\@?

    \end{itemize}

\end{frame}


\section{Efficiency}

\begin{frame}{Efficiency}

    \begin{itemize}

        \item Rules are sorted by how frequently they match.

        \item Architecture scales linearly as the number of rules
            increases.

        \item Performance is \ldots{} XXX 

        \item XXX MORE\@?

    \end{itemize}

\end{frame}

\section{Results}

\begin{frame}{Results.}

    \begin{itemize}

        \item 100\% coverage of log lines.

        \item 100\% coverage of all mails accepted, delivered, or rejected.

        \item XXX rules.

        \item XXX MORE\@?

    \end{itemize}

\end{frame}



\section*{Summary}

\begin{frame}{Summary}

    \begin{itemize}

        \item The \alert{first main message} of your talk in one or two lines.

        \item The \alert{second main message} of your talk in one or two lines.

        \item Perhaps a \alert{third message}, but not more than that.

    \end{itemize}

\end{frame}

\end{document}
