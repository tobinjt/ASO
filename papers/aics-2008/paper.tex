% $Id: logparser.tex 545 2008-05-01 20:00:22Z tobinjt $
\documentclass[twocolumn,a4paper,11pt,draft]{article}

% Useful stuff for math mode.
\usepackage{amstext}
% Wrapping of URLs; this doesn't work in the bibliography, but the breakurl
% package does.
%\usepackage{url}
% The name of the program, so I only have to change it in one place.
\newcommand{\parsername}{\PLP{}}
\newcommand{\parsernames}{\PLP{}'s}
% Include images
\usepackage[final]{graphicx}
% Change how nested enumerate environments are labelled.
\renewcommand{\labelenumii}{\roman{enumii}:}
\renewcommand{\refname}{Bibliography}
% Add the bibliography into the table of contents.
\usepackage[section,numbib]{tocbibind}
% Provides commands to distinguish between pdf and dvi output.
\usepackage{ifpdf}

\usepackage[acronym=true,style=altlist,number=none,toc=true]{glossary}
\makeglossary{}
\makeacronym{}

% When creating a \PDF{} make the table of contents into links to the pages
% (without horrible red borders) and include bookmarks.  The title and
% author don't work - I think either gnuplot or graphviz clobbers it.
% hyperfootnotes need to be disabled to avoid breaking footmisc, but they
% still seem to work, somehow.
\ifpdf{}
    \usepackage[pdftex,hyperfootnotes=false]{hyperref}
\else{}
    \usepackage[dvips,hyperfootnotes=false]{hyperref}
\fi{}
\hypersetup{
    pdftitle    = {Parsing Postfix log files},
    pdfauthor   = {John Tobin},
    final       = true,
    pdfborder   = {0 0 0},
%    breaklinks  = true,
}
% This is necessary for URLs in the bibliography; I dunno why the url
% package doesn't work; don't include it when generating \PDF{} output.
\ifpdf{}
\else{}
    \usepackage{breakurl}
\fi{}
\usepackage{lastpage}

% Extra footnote functionality, including references to earlier footnotes.
\usepackage{footmisc}

% \showgraph{filename}{caption}{label}
\newcommand{\showgraph}[3]{
    \begin{figure}[hbt!]
        \caption{#2}\label{#3}
        \includegraphics{#1}
    \end{figure}
}

%\showtable{filename}{caption}{label}
\newcommand{\showtable}[3]{
    \begin{table}[ht]
        \caption{#2}\label{#3}
        \input{#1}
    \end{table}
}

% Replacement for \ref{}, adds the page number too.
\newcommand{\refwithpage}[1]{%
    \empty{}\ref{#1} [page~\pageref{#1}]%
}
% section references, automatically add \textsection
\newcommand{\sectionref}[1]{%
    \textsection{}\refwithpage{#1}%
}

% A command to format a Postfix daemon's name
\newcommand{\daemon}[1]{%
    \texttt{postfix/#1}%
}

% This is ridiculous, but I can't put @ in glossary entries, so . . .
\newcommand{\at}[0]{%
    @%
}

\newcommand{\tab}[0]{%
    \hspace*{2em}%
}

% This may keep hyperref happy about the Abstract entry in the table of
% contents.
\newcounter{dummy}

\begin{document}

\title{Using Decision Trees for optimisation rather than classification
XXX BETTER TITLE}
\author{John Tobin \& Carl Vogel \\ School of Computer Science and Statistics \\
Trinity College \\ Dublin 2 \\ Ireland \\ tobinjt@cs.tcd.ie \&
vogel@cs.tcd.ie }
\maketitle

\refstepcounter{dummy}
\addcontentsline{toc}{section}{Abstract}
\begin{abstract}

    Classification performed by evaluating user-defined sequences of tests
    can be optimised by applying a Decision Tree algorithm to data gathered
    from running the tests, providing the user with a more efficient or
    more accurate sequence of tests.  This usage of Decision Trees is
    illustrated by optimising Postfix~\cite{postfix} anti-spam
    restrictions~\cite{smtpd_access_readme}, a task complicated by the
    division of the tests into groups, restricting the Decision Tree
    algorithm to examining subsets of attributes at a time.

\end{abstract}

\section{Introduction}

Motivation, background on Postfix restrictions (appendix maybe?)

\section{Algorithm}

Standard CART Decision Tree with modifications: 

\begin{itemize}

    \item groups of attributes to examine

    \item result already known --- non-rejections will be weeded out into
        the false branches, eventually ending up in the bottom-left-most
        leaf node.

    \item scoring functions

\end{itemize}

\section{Results}

\begin{itemize}

    \item duplicate rows collapsed; histograms of distribution

    \item optimised ordering --- primary component analysis

    \item image

    \item different trees from different scoring functions

\end{itemize}

\section{Conclusion}

Useful for optimisation and visualisation.

\appendix

\bibliographystyle{../common/bibliography-style}
\bibliography{../common/bibliography}
\label{bibliography}

\end{document}
