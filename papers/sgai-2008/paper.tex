% $Id: logparser.tex 545 2008-05-01 20:00:22Z tobinjt $
%\documentclass[twocolumn,a4paper,11pt,draft]{article}
\documentclass[]{svmult}

% Useful stuff for math mode.
\usepackage{amstext}
% The name of the program, so I only have to change it in one place.
\newcommand{\parsername}{\PLP{}}
\newcommand{\parsernames}{\PLP{}'s}
% Include images
\usepackage[final]{graphicx}
\renewcommand{\refname}{Bibliography}
% Add the bibliography into the table of contents.
\usepackage[section,numbib]{tocbibind}
% Provides commands to distinguish between pdf and dvi output.
\usepackage{ifpdf}

% This is necessary for URLs in the bibliography; I dunno why the url
% package doesn't work; don't include it when generating \PDF{} output.
\ifpdf{}
\else{}
    \usepackage{breakurl}
\fi{}
\usepackage{lastpage}

% Extra footnote functionality, including references to earlier footnotes.
\usepackage[bottom]{footmisc}

% Extra packages recommended by Springer.
\usepackage{mathptmx}
\usepackage{helvet}
\usepackage{courier}
\usepackage{makeidx}
\usepackage{multicol}


% \showgraph{filename}{caption}{label}
\newcommand{\showgraph}[3]{
    \begin{figure}[hbt!]
        \caption{#2}\label{#3}
        \includegraphics{#1}
    \end{figure}
}

%\showtable{filename}{caption}{label}
\newcommand{\showtable}[3]{
    \begin{table}[ht]
        \caption{#2}\label{#3}
        \input{#1}
    \end{table}
}

% Replacement for \ref{}, adds the page number too.
\newcommand{\refwithpage}[1]{%
    \empty{}\ref{#1} [page~\pageref{#1}]%
}
% section references, automatically add \textsection
\newcommand{\sectionref}[1]{%
    \textsection{}\refwithpage{#1}%
}

% A command to format a Postfix daemon's name
\newcommand{\daemon}[1]{%
    \texttt{postfix/#1}%
}

% This is ridiculous, but I can't put @ in glossary entries, so . . .
\newcommand{\at}[0]{%
    @%
}

\newcommand{\tab}[0]{%
    \hspace*{2em}%
}

\begin{document}

\title*{A user-extensible and adaptable parser architecture}
%\author{John Tobin \& Carl Vogel \\ School of Computer Science and Statistics \\
%Trinity College \\ Dublin 2 \\ Ireland \\ tobinjt@cs.tcd.ie \& vogel@cs.tcd.ie }
\author{John Tobin and Carl Vogel}
\institute{John Tobin \& Carl Vogel \\ School of Computer Science and Statistics \\
Trinity College \\ Dublin 2 \\ Ireland \\ tobinjt@cs.tcd.ie \& vogel@cs.tcd.ie }
\maketitle

\abstract{%
    Some parsers need to be very precise and restrictive in what they
    parse, yet allow users to easily adapt existing rules or add new rules
    to parse new inputs, without requiring an in-depth knowledge of the
    parser's internal workings.  This paper presents a novel parsing
    architecture which aims to make the process of parsing new inputs as
    simple as possible, enabling users to trivially add new rules (to parse
    variants of existing inputs) and relatively easily add new actions (to
    do something useful with a previously unknown class of input).
}

XXX DUE MONDAY JUNE 2$^{ND}$ 2008\@.

PRESENTATION PAPERS\@: MAX 14 PAGES\@.

POSTER PAPERS\@: MAX 6 PAGES\@.

BLACK AND WHITE ONLY\@.

\section{Introduction}

Background of the project --- emphasise the variety in the log lines.
Motivation for creating the parser.  At some point I want to compare this
architecture to recursive decent parsers, Lex/Yacc parsers, XXX parsers.

\section{Algorithm}

Split into framework, actions, \& rules.  Explain adding new rules and
actions.  Cut down rules to the minimum possible.  Framework provides
support functions, parsing loop, manages data --- boring stuff users don't
need to concern themselves with.

\section{Results}

Scalability, ordering, caching regexes.  Interesting to run with all rules
versus running with the minimum number necessary to parse the test log
files.  Then try that again with different orderings?

\section{Conclusion}

Makes adding new rules trivial, new actions tractable.  Knowledge of the
framework is rarely necessary.

\appendix

\cite{postfix}

\bibliographystyle{../common/bibliography-style}
\bibliography{../common/bibliography}
\label{bibliography}

\end{document}
