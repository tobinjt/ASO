% $Id$
\documentclass[]{svmult}

% Useful stuff for math mode.
\usepackage{amstext}
% The name of the program, so I only have to change it in one place.
\newcommand{\parsername}{\PLP{}}
\newcommand{\parsernames}{\PLP{}'s}
% Include images
\usepackage[final]{graphicx}
\renewcommand{\refname}{Bibliography}
% Add the bibliography into the table of contents.
\usepackage[section,numbib]{tocbibind}
% Provides commands to distinguish between pdf and dvi output.
\usepackage{ifpdf}
\usepackage{url}

\usepackage[acronym=true,style=altlist,number=none,toc=true]{glossary}
\makeglossary{}
\makeacronym{}

% This is necessary for URLs in the bibliography; I dunno why the url
% package doesn't work; don't include it when generating \PDF{} output.
\ifpdf{}
\else{}
    \usepackage{breakurl}
\fi{}
\usepackage{lastpage}

% Extra footnote functionality, including references to earlier footnotes.
\usepackage[bottom]{footmisc}

% Extra packages recommended by Springer.
\usepackage{mathptmx}
\usepackage{helvet}
\usepackage{courier}
\usepackage{makeidx}
\usepackage{multicol}
\usepackage{cite}


% \showgraph{filename}{caption}{label}
\newcommand{\showgraph}[3]{
    \begin{figure}[hbt!]
        \caption{#2}\label{#3}
        \includegraphics{#1}
    \end{figure}
}

%\showtable{filename}{caption}{label}
\newcommand{\showtable}[3]{
    \begin{table}[ht]
        \caption{#2}\label{#3}
        \input{#1}
    \end{table}
}

% Replacement for \ref{}, adds the page number too.
\newcommand{\refwithpage}[1]{%
    \empty{}\ref{#1} [page~\pageref{#1}]%
}
% section references, automatically add \textsection
\newcommand{\sectionref}[1]{%
    \textsection{}\refwithpage{#1}%
}

% A command to format a Postfix daemon's name
\newcommand{\daemon}[1]{%
    \texttt{postfix/#1}%
}

% This is ridiculous, but I can't put @ in glossary entries, so . . .
\newcommand{\at}[0]{%
    @%
}

\newcommand{\tab}[0]{%
    \hspace*{2em}%
}

\begin{document}

\title*{A user-extensible and adaptable parser architecture}
\author{John Tobin and Carl Vogel}
\institute{John Tobin \at{} School of Computer Science and Statistics,
Trinity College, Dublin 2, Ireland. \newline{} \email{tobinjt@cs.tcd.ie} \and{}
Carl Vogel \at{} School of Computer Science and Statistics,
Trinity College, Dublin 2, Ireland. \newline{} \email{vogel@cs.tcd.ie}}
\maketitle

\abstract{%
    Some parsers need to be very precise and strict in what they parse, yet
    must allow users to easily adapt existing rules or add new rules to
    parse new inputs, without requiring the user to have an in-depth
    knowledge and understanding of the parser's internal workings.  This
    paper presents a novel parsing architecture which aims to make the
    process of parsing new inputs as simple as possible, enabling users to
    trivially add new rules or adapt existing rules (to parse variants of
    existing inputs) and relatively easily add new actions (to do something
    useful with a previously unknown class of input).  The architecture
    provides a framework which actions and rules fit into: this framework
    manages the parsing process, provides support functions for the
    actions, and automatically performs some optimisations which improve
    the speed of parsing.
}

XXX DUE MONDAY JUNE 2$^{ND}$ 2008\@.

PRESENTATION PAPERS\@: MAX 14 PAGES\@.

POSTER PAPERS\@: MAX 6 PAGES\@.

BLACK AND WHITE ONLY\@.

XXX FORMAT OF REFERENCES\@.

% % vim: set textwidth=75 spell :
% Warning: don't use acronyms within definitions, they don't work properly.

\newacronym{RBL}{Real-time Black List}{description={
    Real-time Black Lists are a simple collaborative anti-spam technique
    used to reject or penalise email sent from mail servers reported to
    have sent large volumes of spam.  To use an RBL the mail server makes a
    DNS request incorporating the IP address of the currently connected
    client; if the requested hostname exists the client is on the RBL, and
    the mail server can decide what course of action to take.
}}

\newacronym{API}{Application Programming Interface}{description={
    One of the fundamental concepts when writing programs is the reuse of
    existing code, so that each program does not reinvent the wheel.  When
    a body of code is intended to be reused repeatedly, the user of this
    code needs to be informed of the functionality provided by the code.
    An API defines the interface provided to the user, and acts a contract
    between the user and the provider: if the user adheres to the API the
    provider guarantees it will work, while the provider is free to change
    the implementation as long as the API is preserved.
}}

\newacronym{SMTP}{Simple Mail Transfer Protocol}{description={
    SMTP is the protocol which transfers mail from the sender to the
    recipient across the Internet.  A brief introduction to SMTP is
    provided in section~\refwithpage{SMTP background}.
}}

\newacronym{MTA}{Mail Transfer Agent}{description={
    A Mail Transfer Agent sends and/or receives mail via SMTP\@.
}}

\newacronym{RFC}{Request For Comments}{description={
    The Request For Comments series is a series of proposals defining
    various protocols and file formats, e.g. SMTP\@.  The name is somewhat
    misleading: initially the authors were asking for peer review, but
    these documents are now the de facto standards the Internet runs on.
}}


\section{Introduction}

\label{Introduction}

The architecture described herein was developed as part of a larger project to
improve anti-spam techniques by analysing the performance of the set of filters
currently in use, optimising the order and membership of the set based on that
analysis, and developing new measures to supplement the existing filters where
there is a deficiency.  The approach chosen was to analyse the log files
produced by the mail server, rather than modifying the mail server to generate
performance statistics; this approach improves the chances of other sites
testing and utilising the software.  The need arose for a parser capable of
dealing with the huge variety of log lines produced by Postfix~\cite{postfix},
the \MTA{}: 

\begin{itemize}

    \item Log lines differ between versions of Postfix.
        
    \item The mail administrator can define custom rejection messages, and
        thus custom log lines.

    \item External resources utilised by Postfix (e.g.\ policy
        servers~\cite{policy-servers} or \RBL{} lookups) can also change
        their messages, without warning.

\end{itemize}

Existing parsers were considered, but their parsing coverage was too low,
their parsing too inexact, and/or they did not extract sufficient data for
the purposes of this project.\footnote{A review of the existing parsers
considered for this project will be included in the thesis.} The effort
required to adapt and improve an existing parser was judged to be greater
than the effort to write one from scratch, as the techniques used by the
existing parsers severely limited their potential: some ignored the
majority of log lines, parsing specific log lines with great accuracy, but
in an inextensible fashion; others sloppily parsed the majority of log
lines, but were incapable of distinguishing between log lines of the same
type, e.g.\ rejections, and the data they extracted was constrained to the
\LCD{}.

The solution developed is conceptually simple: provide a few generic functions,
each capable of dealing with an entire \textit{category\/} of log lines (e.g.\
rejections), accompanied by a multitude of precise patterns, each of which
matches all log lines of a specific type, and only that type (e.g.\ rejection
by a specific \RBL{}).




XXX WHERE SHOULD I DISCUSS COMPLICATIONS\@?

During development of the parser it became apparent that in addition to the
obvious variety in log lines, there were a large number of complications to
be overcome

\section{Architecture}

\label{Architecture}

Explain adding new rules and actions.  Cut down rule attributes to the
minimum possible.

Thoughts from Carl: 

\begin{itemize}

    \item context-free rules

    \item partially context-aware actions, though not really

    \item transduction: input log lines $\rightarrow$ database

    \item Closer to NLP than fixed grammar.

\end{itemize}

XXX BEGIN\@:

The architecture is split into three sections: framework, actions and
rules.  Each will be discussed separately, but first an overview:

\begin{description}

    \item [Framework]  The framework is the structure which actions and
        rules fit into.  The framework provides the parsing loop, loading
        and validation of rules, shared data storage, storage of results,
        and other support functions.

    \item [Actions]  The actions perform the work required to deal with a
        category of log lines, e.g.\ creating a new data structure when a
        log line indicating a remote client connected is processed.

    \item [Rules]  The rules are responsible for matching log lines,
        specifying the action to invoke and the data to be extracted from
        the log line.

\end{description}

Decoupling the parsing rules from the associated actions and framework allows
new rules to be written and tested without requiring modifications to the
parser source code (significantly lowering the barrier to entry for new or
casual users who need to parse new log lines), and greatly simplifies
framework, actions and rules. Decoupling the actions from the
framework simplifies both framework and actions: the framework provides
services to the actions, and doesn't need to deal with the complications which
arise, or the task of reconstructing a mail's journey through Postfix; actions
benefit from having services provided by the framework, freeing them to
concentrate on the task of accurately reconstructing each mail's journey
through Postfix and dealing with the complications arising (XXX REFERENCE).

Decoupling also creates a clear separation of functionality: rules handle low
level details of identifying log lines and extracting data from a log line;
actions handle the higher level details of following the path a mail takes
through Postfix, assembling the required data, dealing with complications
arising, etc; the framework provides services to actions and stores data.  

There is some similarity between the parser's design and William Wood's
\ATN{}~\cite{atns, nlpip}, a tool used in Computational Linguistics for
creating grammars to parse or generate sentences.  The resemblance between
\ATN{} and the parser is accidental, but it is interesting how two
different approaches have a similar division of responsibilities, while
having completely different implementations and semantics.

% Do Not Reformat!

\begin{table}[ht]
    \caption{Similarities with ATN}\label{Similarities with ATN}
    % The \smallskip{} below stop the text hitting the lines.
    \begin{tabular}[]{lll}
        \hline
        \noalign{\smallskip}
        \ATN{}        & Parser    & Similarity                          \\
        \noalign{\smallskip}
        \hline
        \noalign{\smallskip}
        Networks      & Framework & Determines the sequence of
                                    transitions or actions which        \\
                      &           & constitutes a valid input.          \\
        Transitions   & Actions   & Save data and impose conditions the
                                    input must meet to be               \\
                      &           & considered valid.                   \\
        Abbreviations & Rules     & Responsible for classifying input.  \\
        \noalign{\smallskip}
        \hline
        \noalign{\smallskip}
    \end{tabular}
\end{table}

\subsection{Framework}

\label{Framework}

The framework provides utility and support functions for the actions and
rules.  It provides shared storage to pass data between actions, most
notably the data structures used to represent each connection and mail,
though there are some others.  It loads and validates rules, controls
parsing of log files, invoking actions and tracking performance data, and
stores results on request.

The function which finds the rule matching the input line and invokes the
requested action can be expressed in pseudo-code (with indentation denoting
flow-of-control) as:

\begin{verbatim}

for each line in the input files:
    for each rule defined by the user:
        if this rule matches the input line:
            perform the action specified by the rule
            skip the remaining rules
            process the next input line
    warn the user that the input line was not parsed

\end{verbatim}

\subsection{Actions}

\label{Actions}

Actions contain the code required to deal with log lines from a particular
category, e.g.\ rejections.  


\subsection{Rules}

\label{Rules}

\section{Results}

Scalability, ordering, caching regexes.  Interesting to run with all rules
versus running with the minimum number necessary to parse the test log
files.  Then try that again with different orderings?  Coverage.

\section{Conclusion}

Makes adding new rules trivial, new actions tractable.  Knowledge of the
framework is rarely necessary.

\cite{postfix}

\bibliographystyle{../common/bibliography-style}
\bibliography{../common/bibliography}
\label{bibliography}

% Redefine the command used to produce the glossary title, because the
% default command produces an unnumbered section whereas I want a numbered
% section.
\renewcommand{\glossarytitle}{\section{Glossary}\label{Glossary}}
\printglossary{}
% Redefine the command a second time to produce acronyms instead of a
% glossary.
\renewcommand{\glossarytitle}{\section{Acronyms}\label{Acronyms}}
\printacronym{}
\end{document}
